% Análisis

\capitulo{Análisis}{analisis}{
Este capítulo está centrado en la especificación de la plataforma 
RLF, cómo se concivió y todas las funcionalidades que ofrece. Además 
de los requisitos establecidos que debe cumplir.
}

\section{Introducción}
\label{sec:introduccion}
El análisis de este producto ha sido guiado por los estándares 
recogidos por la Agencia Espacial Europea para los proyectos de 
Ingeniería del Software \cite{esa}. La estructura de contenidos se ha 
mantenido aunque se han modificado algunos de los apartados 
considerados innecesarios ya que se especifican para un conjunto de 
documentos que describan un producto, y no para uno solo, como ocurre 
en este caso.

Para el desarrollo de RLF Prototype se ha contado con dos clientes del ámbito 
educativo, cada uno especializado en un área que han determinado las 
funciones que contiene el sistema. Así, se cuenta con un cliente 
ingeniero industrial (\textbf{Cliente A}) y con un cliente ingeniero 
informático (\textbf{Cliente B}).

En primera instancia, se especifican las condiciones generales del 
proyecto, así como sus capacidades más notables. A continuación, se 
listan todos los requisitos que fueron impuestos para la creación del 
producto, así como en conjunto de acciones a realizar por los 
distintos usuarios. Por último, se tratarán los requisitos desde el 
punto de vista del diseño.

Una vez todos los aspectos del análisis estén recogidos y 
establecidos, se pasará a diseñar desde el punto de vista técnico el 
sistema RLF.

\section{Descripción general}
\subsection{Capacidades generales}
El producto RLF cumple la finalidad de ofrecer servicios a los 
usuarios. La lista de capacidades que se muestra a continuación es el 
resultado de un primer análisis, que posteriormente serán 
desarrollados para una mayor comprensión:

\begin{itemize}
\item RLF provee de herramientas, generalmente con componentes 
\emph{hardware}, a los usuarios, los cuales pueden utilizarlas de 
forma exclusiva durante un determinado tiempo.
\item Las herramientas ofrecen un conjunto de acciones a realizar. 
Estas acciones son configuradas por parte del usuario para adecuarse a 
sus necesidades.
\item El usuario está informado en todo momento del estado del 
sistema y de las acciones comandadas, así como de las herramientas a 
las que tiene acceso.
\item Cada acción tiene la libertad de utilizar todo su potencial 
mediante servicios externos a la plataforma RLF.
\item Los usuarios pueden acceder al sistema en tiempo real sin limitaciones 
horarias, siendo posible utilizarlo de manera remota.
\item Se establecen un conjunto de normas, o maneras de trabajar, para 
desarrollar nuevas herramientas, que se comportan de manera estándar 
para su inmediato uso.
\item Proporciona mecanismos para asegurar la integridad tanto del 
\emph{software} como \emph{hardware} que se ofrece mediante la 
plataforma RLF.
\end{itemize}

\subsection{Restricciones generales}
Como cada producto \emph{software}, RLF cumple unas determinadas 
restricciones que se aclararon también en el análisis inicial. Son 
las que han condicionado el diseño de la plataforma:

\begin{itemize}
\item Ningún componente de la RLF está limitado por el 
sistema que lo contiene, ya que forma parte de un sistema 
multiplataforma, siendo las herramientas la única excepción.
\item La plataforma RLF responde de igual manera independientemente de 
dónde se encuentren los distintos componentes, siendo posible 
contenerla en una misma máquina, o en una red de computadores.
\item Se asegura la integridad de cada componente por separado, sin 
permitir que por un mal uso o por conjestión aparezcan errores 
inesperados.
\end{itemize}

\subsection{Características de los usuarios}
Se identifican principalmente tres tipos de usuario para los que va 
destinada el sistema. Cada uno desempeña un rol distinto, y por lo 
tanto, se determinan distintas características y responsabilidades 
para cada uno. Cabe esclarecer que dentro de cada categoría de 
usuario, puede haber distintos niveles, pero que estos no son 
determinantes en el uso de la plataforma. Los roles que se especifican 
a continuación no son jerárquicos ni comparten acciones:

\begin{itemize}
\item \textbf{Administradores:} Personal técnico que gestiona y 
mantiene la plataforma RLF en constante funcionamiento. Realizan 
tareas necesarias para asegurar la disponibilidad completa del 
producto, así como gestionar todos los datos que componen el conjunto 
de información manejado por el sistema. Es por ello que deben poseer 
conocimientos medios de utilización de bases de datos y diversos 
sistemas operativos.
\item \textbf{Desarrolladores:} Profesionales encargados de crear las 
herramientas para su posterior utilización por parte de los clientes. 
Deben cumplir un conjunto de normas aplicadas al \emph{framework} para 
desarrollar el \emph{software} necesario que la plataforma RLF pueda 
manejar. Tienen una interacción muy limitada con el sistema.
\item \textbf{Clientes:} Son los usuarios finales de la plataforma, 
que mediante las distintas aplicaciones cliente (aquellas que dan 
acceso a RLF) pueden utilizar las herramientas asignadas.
\end{itemize}

Se entiende que una misma persona puede poseer varios roles. No se 
tienen en cuenta los usuarios indirectos de la plataforma, que no 
acceden a ella, como son los profesores en un entorno educativo que 
definen el conjunto de herramientas a la cual un usuario tiene acceso, 
o los administradores de red que se encargan del buen funcionamiento 
de la misma. Estas acciones recaerán directamente en el usuario 
administrador. 

\subsection{Entorno operacional}
En este punto se analizan las necesidades tecnológicas del sistema 
desde el punto de vista de los usuarios. Se debe distinguir entre las 
necesidades de cada uno de los roles anteriormente comentadas:

\begin{table}[H]
\begin{center}
\begin{tabular}{|| p{12cm} ||}
	\hline
	\hline
	\multicolumn{1}{|| c ||}{\textsc{Administrador}}\\
	\hline
	\hline
	Ordenador portatil con acceso a redes \emph{Ethernet} e 
	inalámbricas. No es necesario que disponga de interfaz gráfica. Se 
	requiere Java JRE y MySQL Query Browser (véase el apéndice 
	\ref{cap:despliegue}).\\
	\hline
	\hline
\end{tabular}
\end{center}
	\caption{Entorno operacional del administrador.}
	\label{tab:entornoadmin}
\end{table}

\begin{table}[H]
\begin{center}
\begin{tabular}{|| p{12cm} ||}
	\hline
	\hline
	\multicolumn{1}{|| c ||}{\textsc{Desarrollador}}\\
	\hline
	\hline
	Ordenador con acceso a redes. Requiere los elementos para 
	poder desarrollar las herramientas en el lenguaje deseado. 
	Dependiendo de la configuración de la plataforma puede ser 
	necesario un cliente FTP.\\
	\hline
	\hline
\end{tabular}
\end{center}
	\caption{Entorno operacional del desarrollador.}
	\label{tab:entornodes}
\end{table}

\begin{table}[H]
\begin{center}
\begin{tabular}{|| p{12cm} ||}
	\hline
	\hline
	\multicolumn{1}{|| c ||}{\textsc{Cliente}}\\
	\hline
	\hline
	Cualquier tipo de máquina que contenga una acceso a la red y la 
	máquina virtual Java.\\
	\hline
	Dispositivo portatil tipo \emph{smartphone} o \emph{tablet} con 
	acceso a redes inalámbricas y navegador de Internet compatible con 
	Javascript.\\
	\hline
	\hline
\end{tabular}
\end{center}
	\caption{Entorno operacional del cliente.}
	\label{tab:entornoclient}
\end{table}

\section{Requisitos del usuario}
A continuación se listan todos los requisitos impuestos por los 
clientes A y B (ver sección \ref{sec:introduccion})
para el diseño de la plataforma RLF. Están clasificados dependiendo 
de su tipo, si aportan funcionalidad o si registren, y contienen una 
subclasificación dependiendo de a qué componente se refieran. Las 
características de un requisito vienen dadas por su identificador, 
prioridad en cuanto a implantación, fuente, necesidad y descripción.

% Requisitos del usuario.
% Requisitos del usuario

\newcounter{rcap}
\newcounter{rres}

\subsection{Requisitos de capacidad}

\reqcapacidad{General: \emph{Hardware} remoto}
{Cliente A}{Alta}{Esencial}
{
El \emph{hardware} será controlado desde la plataforma sin necesidad de 
configurarlo en el lugar físico donde se encuentra.
}

\reqcapacidad{General: \emph{Hardware} distribuido}
{Cliente A}{Alta}{Esencial}
{
El \emph{hardware} que se pondrá a disposición de los clientes estará 
distribuido en varios ordenadores.
}

\reqcapacidad{General: \emph{Hardware} exclusivo}
{Cliente A}{Alta}{Esencial}
{
Todo \emph{hardware} sólo podrá ser usado por un único cliente, sin 
que el uso de otros pueda influirle.
}

\reqcapacidad{General: \emph{Hardware} multidisciplinar}
{Cliente A}{Alta}{Esencial}
{
La plataforma RLF contendrá una interfaz de acceso genérica para 
cualquier tipo de \emph{hardware}.
}

\reqcapacidad{General: Plataforma accesible}
{Cliente A}{Alta}{Esencial}
{
La plataforma será accesible para los clientes desde cualquier 
ordenador conectado a Internet, siendo posible la configuración para 
limitar este acceso.
}

\reqcapacidad{General: Acceso central}
{Cliente A}{Alta}{Esencial}
{
Independientemente de dónde se encuentre el \emph{hardware} el que el 
cliente quiere utilizar, se accederá desde una misma dirección para 
todo el sistema.
}

\reqcapacidad{General: Plataforma modular}
{Cliente B}{Alta}{Esencial}
{
La plataforma RLF estará compuesta por módulos que interactuarán entre 
ellos, al menos dos, los servidores locales o \textbf{laboratorios} 
\index{laboratorio} y el servidor central o \textbf{proveedor} 
\index{proveedor}.
}

\reqcapacidad{General: Cliente del la plataforma}
{Cliente B}{Baja}{Opcional}
{
Para acceder a la plataforma se dispondrá de un \textbf{cliente} 
\index{cliente} que no limitado por el sistema operativo donde se 
encuentre.
}

\reqcapacidad{General: Servicios como \emph{hardware}}
{Cliente B}{Media}{Esencial}
{
El \emph{hardware} deberá ser encapsulado como un tipo de servicio 
genérico, llamado \textbf{herramienta}\index{herramienta}.
}

\reqcapacidad{Proveedor: Base de datos central}
{Cliente A}{Alta}{Esencial}
{
El proveedor contendrá la información en una base de datos central a la 
que accederá cada vez que se requiera utilizar dicha información.
}

\reqcapacidad{Proveedor: Usuarios clientes}
{Cliente A}{Alta}{Esencial}
{
Los clientes poseerán un nombre de usuario, una contraseña, un 
email y un rol específico para acceder a las herramientas.
}

\reqcapacidad{Proveedor: Usuarios administradores}
{Cliente B}{Media}{Esencial}
{
Los administradores serán autentificados en la base de datos del 
proveedor para controlar el acceso a sistemas críticos.
}

\reqcapacidad{Proveedor: Servicio Web}
{Cliente B}{Alta}{Esencial}
{
El proveedor contendrá una interfaz de acceso para la conexión a 
través de Internet y así permitirá a los clientes utilizar los 
servicios de la plataforma.
}

\reqcapacidad{Proveedor: Información en tiempo real}
{Cliente B}{Alta}{Esencial}
{
Los clientes obtendrán la información del proveedor actualizada, tanto 
del estado de las herramientas como el de los laboratorios.
}

\reqcapacidad{Proveedor: Información de las herramientas}
{Cliente A}{Alta}{Esencial}
{
La información de las herramientas estará contenida en la base de datos 
del proveedor para obtener una mayor respuesta cuando los clientes la 
soliciten.
}

\reqcapacidad{Proveedor: Información de los laboratorios}
{Cliente A}{Alta}{Esencial}
{
Al ser un sistema con un componente central, la información de cada 
laboratorio, es decir, dónde se localiza y cómo puede contarse a él, 
se encontrará en la base de datos del proveedor.
}

\reqcapacidad{Proveedor: Acciones}
{Cliente A}{Alta}{Esencial}
{
Las acciones que se podrán llevar a cabo mediante la interfaz de 
acceso al proveedor serán conexión o desconexión de un cliente, 
obtención de la información de las herramientas y su estado, y 
reserva de las mismas.
}

\reqcapacidad{Proveedor: Monitor}
{Cliente B}{Baja}{Opcional}
{
El proveedor tendrá otra interfaz de acceso limitada únicamente para 
conectar y desconectar del sistema, y además obtener el estado actual 
de las herramientas.
}

\reqcapacidad{Proveedor: Seguridad en el acceso}
{Cliente B}{Media}{Opcional}
{
Todas las comunicaciones del proveedor con los clientes estarán 
debidamente cifradas ya que pueden atravesar redes inseguras.
}

\reqcapacidad{Proveedor: Registro de herramientas}
{Cliente B}{Media}{Esencial}
{
El proveedor dará acceso a los laboratorios para que registren 
herramientas, generando un identificador único para cada herramienta y 
una clave de uso.
}

\reqcapacidad{Proveedor: Tiempo de acceso máximo}
{Cliente A}{Media}{Esencial}
{
A cada usuario se le asignará un tiempo de acceso máximo contado en 
minutos, que una vez sobrepasado se expulsa del sistema.
}

\reqcapacidad{Proveedor: Reserva de herramientas}
{Cliente A}{Alta}{Esencial}
{
El proveedor se encargará de bloquear las herramientas reservadas por 
los clientes para que no puedan ser usadas.
}

\reqcapacidad{Laboratorios: Comunicación}
{Cliente B}{Alta}{Esencial}
{
Toda la comunicación entrante y saliente de un laboratorio se 
realizará mediante red.
}

\reqcapacidad{Laboratorios: Base de datos}
{Cliente B}{Alta}{Esencial}
{
Cada laboratorio contendrá su propia base de datos, que elimina el tráfico 
en la red y permite hacer operaciones locales más rápidas. En ellas 
se almacenará la información de las herramientas.
}

\reqcapacidad{Laboratorios: Administración}
{Cliente A}{Alta}{Esencial}
{
Los laboratorios contarán con una interfaz de acceso para poder 
administrarlos sin necesidad de una parada total del sistema.
}

\reqcapacidad{Laboratorios: Registro de herramientas}
{Cliente A}{Alta}{Esencial}
{
Cada laboratorio estará al cargo de varias herramientas, que deberá 
gestionar y administrar.
}

\reqcapacidad{Laboratorios: Acceso a las herramientas}
{Cliente B}{Alta}{Esencial}
{
Para eliminar el tráfico del proveedor, los laboratorios aceptarán
peticiones de los clientes que previamente se hayan conectado al 
sistema y hayan reservado las herramientas necesarias.
}

\reqcapacidad{Laboratorios: Eliminación de herramientas}
{Cliente A}{Baja}{Esencial}
{
Se podrá eliminar una herramienta concreta de un laboratorio si así se 
requiere.
}

\reqcapacidad{Laboratorios: Mantenimiento}
{Cliente A}{Baja}{Esencial}
{
Para aportar seguridad, los laboratorios tendrán un estado de 
``mantenimiento'' donde no se permite la comunicación con los 
clientes, en el cual se podrá realizar tareas de administración del 
mismo.
}

\reqcapacidad{Laboratorios: Ejecuciones de las herramientas}
{Cliente A}{Alta}{Esencial}
{
Los laboratorios sólo permitirán una instancia de ejecución por 
herramienta. Esto conlleva la creación de una cola de peticiones para 
la ejecución de estas.
}

\reqcapacidad{Laboratorios: Comunicación con las herramientas}
{Cliente A}{Alta}{Esencial}
{
Un laboratorio se comunicará con las herramientas que tiene asignadas 
mediante un mecanismo asíncrono para que una herramienta con fallos no 
bloquee la ejecución.
}

\reqcapacidad{Laboratorios: Parada de emergencia}
{Cliente A}{Baja}{Opcional}
{
Cada laboratorio podrá ser parado de manera urgente por parte de un 
administrador, que cierra todas las ejecuciones activas y elimina 
las pendientes. Después, el laboratorio se desactivará.
}

\reqcapacidad{Laboratorios: \emph{Logs}}
{Cliente B}{Baja}{Esencial}
{
Cada laboratorio poseerá su propio mecanismo de registro de eventos y 
errores, que puede ser consultado por un administrador.
}

\reqcapacidad{Laboratorios: Información en tiempo real}
{Cliente A}{Media}{Esencial}
{
Cada vez que un laboratorio cambie su estado, el proveedor será 
informado, incluso cuando se realizan paradas de emergencia.
}

\reqcapacidad{Herramientas: Acciones}
{Cliente A}{Alta}{Esencial}
{
Cada herramienta podrá realizar varias acciones, todas relacionadas 
con el mismo \emph{hardware}. El usuario elegirá una y la 
configurará. No será posible ejecutar a la vez dos acciones de la misma 
herramienta.
}

\reqcapacidad{Herramientas: Configuración de las acciones}
{Cliente B}{Alta}{Esencial}
{
Cada acción tendrá asociada unos parámetros de salida y de entrada, 
con un tipo de datos establecido y una descripción. Mediante la 
definición de los valores de estos parámetros la acción se configura 
para su ejecución. Los parámetros de salida sólo se leerán por parte 
del cliente cuando la acción ha terminado.
}

\reqcapacidad{Herramientas: Excepciones y estado final de la acción}
{Cliente A}{Media}{Esencial}
{
Las acciones, durante su ejecución, podrán lanzar excepciones que 
informan al usuario de un error, además, cada finalización tendrá 
asociada un estado incluido por el desarrollador.
}

\reqcapacidad{Herramientas: Constantes}
{Cliente A}{Baja}{Esencial}
{
Las herramientas podrán contener varias constantes con un tipo de datos 
concreto y un valor que no varía desde que se registra la herramienta.
}

\reqcapacidad{Herramientas: Atributos}
{Cliente A}{Baja}{Esencial}
{
Los laboratorios definirán los atributos propios de cada herramienta, 
donde se incluyen el identificador, la clave, nombre, descripción, rol, 
versión y administrador responsable de la herramienta.
}

\reqcapacidad{Herramientas: Entrada y salida}
{Cliente B}{Alta}{Esencial}
{
Las herramientas dispondrán de una entrada y salida textual para poderse 
comunicar con el usuario.
}

\reqcapacidad{Herramientas: Servicios externos}
{Cliente A}{Alta}{Esencial}
{
Para dar funcionalidad extra a cada herramienta, las acciones podrán 
tener asociadas servicios externos, que se activarán en la ejecución y 
que pueden ser usadas por los clientes.
}

\reqcapacidad{Herramientas: Archivo de descripción}
{Cliente A}{Alta}{Esencial}
{
Para definir una herramienta por completo y registrarla en la 
plataforma RLF, se dispondrá de un fichero con un formato establecido 
que será leído en el momento del registro en el laboratorio.
}

\reqcapacidad{Herramientas: Librerías \emph{libtool}}
{Cliente B}{Alta}{Opcional}
{
\index{\emph{libtool}}Como forma de abstracción, se incluirán
librerías para la comunicación de la herramienta con el laboratorio 
correspondiente. Estas librerías proveerán las funciones de acceso y 
desconexión, lectura y escritura de parámetros, lectura de constantes 
y atributos, lanzamiento y lanzamiento de excepciones.
}

\reqcapacidad{Herramientas: Herramientas de datos}
{Cliente A}{Media}{Opcional}
{
Esta categoría especial de herramientas permitirá acceder a varios 
clientes a la vez, pero sin interactuar con el \emph{hardware}. No 
disponen de entrada textual, ni de parámetros de configuración. Sólo 
tendrán una acción asignada.
}

\reqcapacidad{Herramientas: Limpiadores}
{Cliente B}{Media}{Esencial}
{
Existirá por cada herramienta una acción especial que establece el 
\emph{hardware} de la herramienta a su estado original. No podrá ser 
utilizada por los clientes y de forma automática se lanzará cuando se 
detecta errores en alguna acción.
}

\reqcapacidad{Herramientas: Tiempo máximo de ejecución}
{Cliente B}{Media}{Esencial}
{
Cada acción poseerá un tiempo máximo de ejecución que el laboratorio 
monitoriza. Cuando ese tiempo es sobrepasado, se parará la ejecución por 
completo y se establece el estado de acción fallida.
}

\reqcapacidad{Cliente: Visualización simultánea}
{Cliente A}{Alta}{Esencial}
{
La interfaz del cliente será capaz de visualizar a la vez varias 
acciones de distintas herramientas.
}

\reqcapacidad{Cliente: Uso de las herramientas}
{Cliente B}{Alta}{Esencial}
{
Se proveerá al usuario de una ``consola'' (con entrada y salida textual) 
para el manejo de cada una de las acciones.
}

\reqcapacidad{Cliente: Acciones}
{Cliente A}{Alta}{Esencial}
{
La interfaz permitirá al cliente conectarse y desconectarse del sistema, 
reservar herramientas, comprobar la información de las mismas y su 
estado en tiempo real, y la ejecución de acciones.
}

\reqcapacidad{Cliente: Reserva múltiple}
{Cliente A}{Alta}{Optativa}
{
Un cliente podrá reservar de manera simultánea varias herramientas a 
las que tiene acceso.
}

\reqcapacidad{Cliente: Monitor}
{Cliente A}{Baja}{Optativa}
{
La aplicación cliente dispondrá con una aplicación auxiliar por la que 
acceder la monitor central.
}

\reqcapacidad{Cliente: Información}
{Cliente A}{Media}{Esencial}
{
Todos los datos públicos de las herramientas serán mostrados al cliente 
por medio de la interfaz, añadiendo las descripciones a los parámetros.
}

\subsection{Requisitos de restricción}

\reqrestriccion{General: Cantidad de comunicación}
{Cliente A}{Media}{Esencial}
{
La información trasmitida en las comunicaciones, así como el número 
de estas deberán ser reducidas al mínimo para no saturar cada componente.
}

\reqrestriccion{General: Sistema escalable}
{Cliente B}{Alta}{Esencial}
{
Todo el sistema será escalable, es decir, que no influya en su 
funcionamiento el hecho de aumentar o disminuir el número de 
laboratorios o clientes.
}

\reqrestriccion{Proveedor: Control de seguridad}
{Cliente B}{Media}{Esencial}
{
Los clientes que no se desconecten de forma válida en el sistema o que 
sufran errores, el proveedor no les permitirá volver a conectarse hasta 
que un administrador dé el visto bueno.
}

\reqrestriccion{Proveedor: Bloqueos de peticiones}
{Cliente B}{Alta}{Esencial}
{
El proveedor contará con mecanismos para atender varias peticiones a la 
vez sin dejar ninguna a la espera.
}

\reqrestriccion{Laboratorios: Portables}
{Cliente B}{Alta}{Esencial}
{
Los laboratorios funcionarán sin problemas en diversos sistemas operativos con 
una única implementación.
}

\reqrestriccion{Laboratorios: Acciones asíncronas}
{Cliente A}{Media}{Esencial}
{
Todas las acciones que se llevan a cabo en los laboratorios serán
asíncronas, para no detener ni bloquear ningún componente que 
interfiera con ellos.
}

\reqrestriccion{Laboratorios: Ejecuciones externas}
{Cliente A}{Alta}{Esencial}
{
Las acciones se ejecutarán en los laboratorios como programas externos 
para no añadir carga a la plataforma.
}

\reqrestriccion{Laboratorios: Máximo número de ejecuciones}
{Cliente A}{Alta}{Esencial}
{
Cada laboratorio poseerá un número máximo de acciones en ejecución 
(configurado por el propio administrador). Cuando ocurre esto, las 
acciones que están pendientes tomarán el estado de espera.
}

\reqrestriccion{Herramientas: Parada}
{Cliente A}{Alta}{Esencial}
{
Las herramientas deberán estar preparadas para sufrir paradas sin 
previo aviso, aunque no es necesario que devuelvan el estado original 
al \emph{hardware}.
}

\reqrestriccion{Cliente: Control de seguridad web}
{Cliente A}{Media}{Esencial}
{
El control de seguridad aplicado a los usuarios para el cliente de 
escritorio no bloqueará la cuenta cuando ocurran errores en el cliente 
web.
}


\section{Casos de uso}
Un caso de uso define una serie de pasos o actividades que deben 
realizarse para llevar a cabo un proceso. Las entidades o personajes 
que participan en un caso de uso se denominan actores.

Este apartado se separa en dos partes. La primera es una 
representación gráfica de los distintos casos de uso ordenados por 
actores (administrador, desarrollador y cliente). En la segunda parte 
se explicarán textualmente algunos de los más influyentes, que 
representan las funcionalidades más importantes; \emph{Publicar 
herramienta} y \emph{Ejecutar acción}.

\begin{figure}[h]
	\centering
	\includegraphics[scale=0.6]{images/casos1.png}
	\caption{Casos de uso: Administrador y desarrollador}
	\label{fig:casos1}
\end{figure}

Como se aprecia en la figura \ref{fig:casos1} el administrador tiene 
responsabilidades de mantenimiento del sistema, y todas sus 
funcionalidades están relacionadas con esta tarea. El desarrollador en 
cambio, dentro del sistema sólo podrá realizar parte de la función 
\emph{Publicar herramienta}. Se explicará más adelante este caso 
particular, el cual requiere de dos actores para poder completarse. En 
la figura \ref{fig:casos2} se comprueba que el usuario necesita haber 
iniciado sesión para realizar el resto de funciones, además de que 
para ejecutar una acción sea necesario reservar la herramienta que la 
contiene.

\begin{figure}[h]
	\centering
	\includegraphics[scale=0.6]{images/casos2.png}
	\caption{Casos de uso: Cliente}
	\label{fig:casos2}
\end{figure}

A continuación se muestra la información de los dos casos de uso 
anteriormente señalados, donde se incluye los prerequisitos para 
llevarlos a cabo, los posibles escenarios secundarios y su descripción 
detallada.

\newcounter{casos}

\begin{table}
\begin{center}
	\stepcounter{casos}
\begin{tabular}{|| c | p{10cm} ||}
	\hline
	\multicolumn{2}{|| c ||}{\textbf{CU-\arabic{casos}: Publicar herramienta}}\\
	\hline
	\textsc{Actores} & Administrador, desarrollador \\
	\hline
	\textsc{Precondiciones} &
	\begin{itemize}
		\item El desarrollador debe haber implementado la herramienta 
		de acuerdo con las normas del \emph{framework}.
		\item El laboratorio debe estar desarmado.
	\end{itemize}\\
	\hline
	\textsc{Postcondiciones} &
	\begin{itemize}
		\item La herramienta estará a disposición de los usuarios para poder 
		utilizarla.
	\end{itemize}\\
	\hline
	\multicolumn{2}{|| c ||}{\textsc{Descripción}}\\
	\hline
	\multicolumn{2}{|| p{14cm} ||}{
	El desarrollador crea una herramienta para incluirla en la 
	plataforma. El administrador obtiene los datos necesarios para 
	registrarla y se lo indica al desarrollador.
	}\\
	\hline
	\multicolumn{2}{|| c ||}{\textsc{Escenario principal}}\\
	\hline
	\multicolumn{2}{|| p{14cm} ||}{
	\begin{enumerate}
		\item El administrador introduce el fichero de configuración 
		en el laboratorio mediante la aplicación de mantenimiento.
		\item El laboratorio pide una nueva clave e identificador al 
		proveedor.
		\item El laboratorio valida la configuración y crea las 
		estructuras en su base de datos y la generada para la propia 
		herramienta.
		\item El laboratorio envía al proveedor la información sobre 
		la nueva herramienta y establece su estado como ``no 
		disponible''.
		\item El administrador recibe la nueva clave y el 
		identificador.
		\item El desarrollador modifica el código para incluir la 
		nueva clave.
		\item El administrador arma el laboratorio.
	\end{enumerate}
	}\\
	\hline
	\multicolumn{2}{|| c ||}{\textsc{Escenario secundario}}\\
	\hline
	\multicolumn{2}{|| p{14cm} ||}{
	\begin{enumerate}
		\item El administrador introduce el fichero de configuración 
		en el laboratorio mediante la aplicación de mantenimiento.
		\item El laboratorio pide una nueva clave e identificador al 
		proveedor.
		\item El laboratorio determina que la configuración no está 
		bien construida y tiene errores.
		\item El administrador recibe el error encontrado.
	\end{enumerate}
	}\\
	\hline
	\hline
\end{tabular}
\end{center}
	\label{caso:CU-\arabic{casos}}
\end{table}

\begin{table}
\begin{center}
	\stepcounter{casos}
\begin{tabular}{|| c | p{10cm} ||}
	\hline
	\multicolumn{2}{|| c ||}{\textbf{CU-\arabic{casos}: Ejecutar acción}}\\
	\hline
	\textsc{Actores} & Cliente \\
	\hline
	\textsc{Precondiciones} &
	\begin{itemize}
		\item La herramienta debe haber sido reservada por el usuario.
	\end{itemize}\\
	\hline
	\textsc{Postcondiciones} &
	\begin{itemize}
		\item La acción se ejecuta en el laboratorio e interactua con 
		el cliente.
	\end{itemize}\\
	\hline
	\multicolumn{2}{|| c ||}{\textsc{Descripción}}\\
	\hline
	\multicolumn{2}{|| p{14cm} ||}{
	El cliente desea ejecutar una acción de una herramienta concreta.
	}\\
	\hline
	\multicolumn{2}{|| c ||}{\textsc{Escenario principal}}\\
	\hline
	\multicolumn{2}{|| p{14cm} ||}{
	\begin{enumerate}
		\item El cliente selecciona la acción a ejecutar y pulsa el 
		botón.
		\item La interfaz muestra los parámetros de entrada y los 
		servicios externos de los que dispone la herramienta.
		\item El cliente introduce el valor de esos parámetros y pulsa 
		``Execute''.
		\item El cliente abre los servicios externos indicados con 
		anterioridad.
		\item La aplicación cliente envía al laboratorio la petición 
		de ejecución y se pone a la espera.
		\item Cuando el laboratorio lo dispone, envía a la aplicación 
		cliente la confirmación de ejecución y la información de 
		conexión para el envío y recepción de datos.
		\item La aplicación cliente se conecta con los nuevos 
		parámetros al laboratorio y empieza a recibir los datos de la 
		ejecución.
		\item El cliente interacciona con la ejecución.
	\end{enumerate}
	}\\
	\hline
	\multicolumn{2}{|| c ||}{\textsc{Escenario secundario}}\\
	\hline
	\multicolumn{2}{|| p{14cm} ||}{
	\begin{enumerate}
		\item El cliente selecciona la acción a ejecutar y pulsa el 
		botón.
		\item La interfaz muestra los parámetros de entrada y los 
		servicios externos de los que dispone la herramienta.
		\item El cliente introduce el valor de esos parámetros y pulsa 
		``Execute''.
		\item El cliente abre los servicios externos indicados con 
		anterioridad.
		\item La aplicación cliente envía al laboratorio la petición 
		de ejecución y se pone a la espera.
		\item El laboratorio devuelve un fallo por no poder ejecutar 
		la herramienta.
		\item La aplicación cliente muestra el error por pantalla.
	\end{enumerate}
	}\\
	\hline
	\hline
\end{tabular}
\end{center}
	\label{caso:CU-\arabic{casos}}
\end{table}

\clearpage
\section{Requisitos del \emph{software}}
Los requisitos que aquí se muestran son la respuesta desde el punto de 
vista del diseño del \emph{software} a los requisitos de usuario 
(tanto de capacidad como de restricción). La estructura es idéntica a 
los anteriores excepto que el tipo cambia a funcional y no funcional.

% Requisitos del software.
% Requisitos del software

\newcounter{rfun}
\newcounter{rnof}

\subsection{Requisitos funcionales}

\reqfuncional{General: Aplicaciones}
{RUC-1, RUC-9}{Alta}{Esencial}
{
El \emph{hardware} será controlado completamente por aplicaciones 
que formarán parte de las herramientas y serán ejecutadas mediante 
peticiones de las aplicaciones cliente.
}

\reqfuncional{\emph{Framework}: Desarrollo versátil}
{RUC-4}{Alta}{Esencial}
{
El \emph{framework} de diseño de herramientas no impondrá 
restricciones en cuanto a lenguaje de programación, plataforma ni 
estructura del código.
}

\reqfuncional{\emph{Framework}: Desarrollo cómodo}
{RUC-41}{Alta}{Esencial}
{
Se permitirán utilizar las funciones del sistema de entrada y salida 
por teclado para la comunicación de las herramientas con el cliente. 
Aunque la información que pasa a través de ellas sea enviada por la red.
}

\reqfuncional{Comunicaciones: Modelo TCP/IP}
{RUC-5, RUC-24}{Alta}{Esencial}
{
Todas las comunicaciones serán basadas en el modelo estándar TCP/IP y 
utilizarán un protocolo especialmente diseñado para RLF. Cada nodo 
contendrá una IP de acceso, y la plataforma se identificará con la IP 
del proveedor.
}

\reqfuncional{Comunicaciones: Dirección de la plataforma}
{RUC-6, RUC-7}{Alta}{Esencial}
{
La plataforma se identificará con la IP del proveedor, que será por 
la que se podrá acceder.
}

\reqfuncional{Comunicaciones: Protocolo de comunicaciones}
{RUC-5, RUC-19}{Media}{Esencial}
{
El protocolo de comunicaciones será utilizado por todos los módulos y 
no variará su implementación.
}

\reqfuncional{Comunicaciones: Protocolo síncrono}
{RUC-24}{Media}{Esencial}
{
El protocolo de comunicaciones será un modelo de petición/respuesta 
síncrono, aunque las acciones globales no lo sean.
}

\reqfuncional{General: Vía de acceso}
{RUC-6}{Alta}{Esencial}
{
El proveedor será el encargado de indicar a las aplicaciones cliente 
dónde se encuentran los laboratorios.
}

\reqfuncional{Proveedor: Acceso a la base de datos}
{RUC-10}{Alta}{Esencial}
{
El administrador podrá insertar, eliminar y modificar datos de la base 
de datos central. Será gestionada mediante la aplicación que el 
propio SGBD provea. El proveedor accederá mediante el driver 
correspondiente en la plataforma de desarrollo.
}

\reqfuncional{Proveedor: Autenticación}
{RUC-11, RUC-12}{Media}{Esencial}
{
La autentificación de cualquier usuario, independientemente del tipo 
de rol, será validada por el proveedor.
}

\reqfuncional{Proveedor: Interfaz de acceso}
{RUC-13}{Alta}{Esencial}
{
Las aplicaciones cliente usarán las funciones contenidas en el 
proveedor, las cuales, serán divididas en dos interfaces, 
con una estructura de servicio web definido por los ficheros WSDL y 
así dar acceso a las aplicaciones cliente.
}

\reqfuncional{Proveedor: Actualizaciones de datos}
{RUC-14}{Alta}{Esencial}
{
Los datos almacenados en el proveedor serán los últimos en actualizar. Por
cada actualización correcta de datos realizadas por los distintos 
módulos se deberá considerar como un cambio permanente.
}

\reqfuncional{Proveedor: Información comprimida}
{RUC-15}{Baja}{Opcional}
{
Cuando un cliente solicite información sobre una herramienta, se 
entregará un conjunto de datos previamente comprimidos y calculados 
para una mayor velocidad de respuesta.
}

\reqfuncional{Proveedor: Información de cada laboratorio}
{RUC-16}{Alta}{Esencial}
{
Cuando un cliente reserve una herramienta, se le entregará la 
información de conexión del laboratorio que la contiene, como es la 
IP y el puerto.
}

\reqfuncional{Proveedor: Funciones web}
{RUC-17, RUC-18}{Alta}{Esencial}
{
Las funciones de las dos interfaces de acceso proveerán parámetros 
de entrada y salida, así como excepciones. Estos objetos serán 
capaces de ser serializados, para poder ser enviados mediante el 
protocolo HTTP y SOAP.
}

\reqfuncional{Proveedor: Seguridad en las herramientas}
{RUC-20}{Alta}{Esencial}
{
Las claves y los identificadores de las herramientas serán únicos y 
no reutilizables. Las claves se generarán mediante funciones 
\emph{hash} a partir de la fecha y el identificador obtenido. 
}

\reqfuncional{Proveedor: Tiempo máximo asignado}
{RUC-21}{Alta}{Esencial}
{
Los clientes tendrán preasignado un tiempo almacenado en la base 
de datos, que será cronometrado por los propios laboratorios en los que 
estén registrados.
}

\reqfuncional{Proveedor: Bloqueo de herramientas}
{RUC-3, RUC-22}{Alta}{Esencial}
{
La base de datos contendrá información de quién tiene reservada
cada herramienta, impidiendo que herramientas que no sean de datos se 
encuentren accesibles por dos clientes, por lo se requerirá un bloqueo 
de escritura y lectura en la información cada vez que se requiera reservar.
}

\reqfuncional{Laboratorios: Localización de las aplicaciones}
{RUC-2}{Alta}{Esencial}
{
En cada laboratorio se almacenarán, de forma local, todas las 
aplicaciones implicadas en las ejecuciones de las herramientas que 
tienen asignadas. 
}

\reqfuncional{Laboratorios: Base de datos}
{RUC-25}{Alta}{Esencial}
{
La base de datos de cada laboratorio contendrá la ruta de acceso a la 
herramienta, su clave y su identificador. Será de baja capacidad y 
portable, por lo tanto, el SGBD será SQLite.
}

\reqfuncional{Laboratorios: Comunicación con las herramientas}
{RUC-32}{Alta}{Esencial}
{
El laboratorio creará una base de datos ligera (SQLite) para ofrecer 
comunicaciones con la herramienta. De manera asíncrona obtendrá los 
datos que se origen en la ejecución de la misma.
}

\reqfuncional{Laboratorios: \emph{Kernel}}
{RUC-26}{Alta}{Esencial}
{
Un laboratorio ofrecerá una capa de gestión llamada \emph{Kernel} que 
será independiente de las comunicaciones con los clientes y las 
ejecuciones de herramientas.
}

\reqfuncional{Laboratorios: Funciones del \emph{Kernel}}
{RUC-30, RUC-34}{Alta}{Esencial}
{
El \emph{Kernel} de cada laboratorio será el encargado de enviar y 
recibir comunicaciones con el proveedor, además de añadir y eliminar 
herramientas. También armarán y realizarán las acciones necesarias 
para desarmar el laboratorio.
}

\reqfuncional{Laboratorios: Parada de emergencia en el \emph{Kernel}}
{RUC-30}{Alta}{Esencial}
{
Cuando el \emph{Kernel} reciba una parada de emergencia detendrá los 
gestores anexos inmediatamente, lo que permitirá al administrador 
acceder al \emph{hardware} de forma más rápida.
}

\reqfuncional{Laboratorios: Gestor de comunicaciones}
{RUC-28}{Alta}{Esencial}
{
Capa superior que administrará todas las peticiones recibidas por los 
clientes así como las notificaciones de las ejecuciones.
}

\reqfuncional{Laboratorios: Tiempo máximo de un cliente}
{RUC-21}{Media}{Esencial}
{
Cada gestor de comunicaciones controlará el tiempo que pasa el cliente 
registrado en el laboratorio. Cuando se sobrepasa, informa al 
proveedor y deniega cualquier intento posterior de comunicación.
}

\reqfuncional{Laboratorios: Gestor de ejecución}
{RUC-31, RUC-36}{Alta}{Esencial}
{
Capa superior que administrará todas las ejecuciones de las acciones y 
de los limpiadores.
}

\reqfuncional{Laboratorios: Tiempo máximo de una acción}
{RUC-47}{Media}{Esencial}
{
El gestor de ejecución se encargará de controlar el tiempo máximo de 
cada acción, que será interrumpida en el instante que lo sobrepase.
}

\reqfuncional{Herramientas: Aplicaciones}
{RUC-9, RUC-36}{Alta}{Esencial}
{
Cada herramienta contendrá un conjunto de aplicaciones de escritorio 
que ayudarán a manejar y configurar el \emph{hardware}.
}

\reqfuncional{Herramientas: Base de datos}
{RUC-37}{Alta}{Esencial}
{
Toda la información de las acciones estarán disponible en la base de 
datos creada por el laboratorio en cada herramienta. Y podrá ser 
modificada por el usuario a través del mismo para la configuración de 
las mismas herramientas.
}

\reqfuncional{Herramientas: Base de datos}
{RUC-37, RUC-39}{Alta}{Esencial}
{
Toda la información de las acciones estarán disponible en la base de 
datos creada por el laboratorio en cada herramienta. Y podrá ser 
modificada por el usuario a través del mismo para la configuración de 
las mismas herramientas.
}

\reqfuncional{Herramientas: Generación de datos}
{RUC-38, RUC-40}{Media}{Esencial}
{
Los datos generados en la ejecución de un acción serán guardados en 
la base de datos para su posterior consulta.
}

\reqfuncional{Herramientas: Servicios externos}
{RUC-42}{Alta}{Esencial}
{
La abstracción mediante \emph{sockets} de servicios externos de cada 
herramienta permitirá al usuario utilizarlos como servicios de la 
propia plataforma, con diferentes orígenes.
}

\reqfuncional{Herramientas: Archivo de descripción}
{RUC-43}{Alta}{Esencial}
{
Los archivos de descripción de herramientas seguirán dos esquemas, 
dependiendo del tipo de herramienta, con un formato XML.
}

\reqfuncional{Herramientas: Herramientas de datos}
{RUC-45}{Media}{Opcional}
{
Para los clientes, las herramientas de datos siempre estarán 
disponibles, independientemente del número de usuarios usándolas, y 
dispondrán de un mecanismo autónomo de \emph{broadcast} de información.
}

\reqfuncional{Herramientas: Limpiadores}
{RUC-46}{Media}{Esencial}
{
Estas acciones serán especificadas en el archivo de descripción y 
tendrán preferencia en cuanto a las peticiones de ejecución normales. 
Una herramienta no se podrá ejecutar si ha ocurrido un error y no se 
ha ejecutado con anterioridad el limpiador.
}


\reqfuncional{\emph{Libtool}: Base de datos}
{RUC-44}{Alta}{Esencial}
{
Esta abstracción permitirá la conexión a la base de datos de la 
herramienta para la lectura y escritura de información. El lenguaje de 
programación dependerá de la implementación de la propia herramienta.
}

\reqfuncional{\emph{Libtool}: Base de datos}
{RUC-44}{Alta}{Esencial}
{
Esta abstracción permitirá la conexión a la base de datos de la 
herramienta para la lectura y escritura de información. El lenguaje de 
programación dependerá de la implementación de la propia herramienta.
}

\reqfuncional{Cliente de escritorio: Función}
{RUC-8}{Media}{Esencial}
{
La aplicación cliente principal será ejecutada como aplicación de 
escritorio que utilizará el servicio web proveedor para la 
interacción con la plataforma RLF.
}

\reqfuncional{Cliente de escritorio: Ventanas de ejecución}
{RUC-48, RUC-49}{Alta}{Esencial}
{
Por cada acción en ejecución, el cliente tendrá activa una ventana 
independiente con acceso de entrada y salida por teclado.
}

\reqfuncional{Cliente de escritorio: Herramientas}
{RUC-51, RUC-53}{Alta}{Esencial}
{
Por cada herramienta, la interfaz del cliente poseerá una pestaña que 
contendrá toda su información. Se podrán seleccionar de forma 
independendiente para su posterior reserva.
}

\reqfuncional{Cliente de escritorio: Conexión}
{RUC-50}{Alta}{Esencial}
{
El cliente conservará la información de conexión durante toda la 
ejecución, sin ser necesario volver a conectarse para realizar 
múltiples reservas.
}

\reqfuncional{Cliente de escritorio: Actualización}
{RUC-50}{Baja}{Esencial}
{
Se ofrecerá al cliente la posibilidad de actualizar la información 
del número de herramientas como de su estado sin necesidad de cerrar 
la aplicación.
}

\reqfuncional{Cliente web: Función}
{RUC-18, RUC-52}{Baja}{Opcional}
{
Los clientes podrán acceder a la página web de la plataforma para 
comprobar el estado actual de las herramientas, si están disponibles, 
ocupadas o no conectadas. El formato deberá ser compatible para 
dispositivos con pantalla pequeña.
}

\reqfuncional{Gestor de laboratorios: Función}
{RUC-26}{Baja}{Esencial}
{
Para gestionar los laboratorios se proveerá de una aplicación de 
consolta con acceso por red con las acciones de armar, desarmar, 
parar, registrar y eliminar herramientas, y comprobar el estado del 
laboratorio.
}

\reqfuncional{Registro: \emph{Logs}}
{RUC-34}{Baja}{Esencial}
{
Todo componente en la plataforma se servirá de un conjunto de 
registros para indicar errores y funciones realizadas.
}

\subsection{Requisitos no funcionales}

\reqnofuncional{Comunicaciones: Sistema escalable}
{RUR-2}{Media}{Esencial}
{
La plataforma utilizará su propio sistema de comunicaciones y de altas 
de laboratorios, sin utilizar soluciones ya implementadas como Java 
RMI, CORBA o RPC.
}

\reqnofuncional{Comunicaciones: Protocolo JSON}
{RUR-1}{Media}{Esencial}
{
El protocolo de comunicaciones estará basado en el formato JSON, 
previamente cifrado, que reduce de forma considerable los datos a 
enviar de los objetos a enviar.
}

\reqnofuncional{Comunicaciones: Protocolo JSON avanzado}
{RUR-1}{Media}{Esencial}
{
Cada objeto enviado entre módulos puede contener otros objetos como 
atributos también en formato JSON.
}

\reqnofuncional{Comunicaciones: Peticiones y respuestas}
{RUR-1}{Media}{Esencial}
{
Cada petición o respuesta tendrá asociado un número de operación 
que se incluirá en la cabecera del mensaje.
} 

\reqnofuncional{Proveedor: \emph{Token} de acceso y uso}
{RUR-3}{Media}{Esencial}
{
Cada cliente tiene asociado un atributo con un límite de tiempo que 
será generado cada vez que se conecte para reducir el tráfico de 
comunicaciones y sólo tener que identificarse mediante esa cadena de 
caracteres. Cuando se desconecta será invalidado.
} 

\reqnofuncional{Proveedor: Servicio Web J2EE}
{RUR-4}{Alta}{Esencial}
{
El proveedor será un conjunto de dos servicios web publicados en una 
plataforma Tomcat, además de la base de datos MySQL. El servicio web 
principal llamado RLF\_Provider y el secundario, RLF\_Monitor.
} 

\reqnofuncional{Laboratorios: Máquina virtual de Java}
{RUR-5}{Alta}{Esencial}
{
Los laboratorios estarán implementados en Java y utilizarán drivers 
de conexión a las bases de datos SQLite.
}

\reqnofuncional{Laboratorios: Archivos ejecutables}
{RUR-5}{Alta}{Esencial}
{
Para configurar un laboratorio se proveerá de un fichero de 
configuración junto con el archivo ejecutable.
}

\reqnofuncional{Laboratorios: Ficheros fuente}
{RUR-5}{Alta}{Esencial}
{
Todo recurso necesario para la ejecución de los laboratorios será 
contenida en un directorio que deberá acompañarse con el ejecutable.
}

\reqnofuncional{Laboratorios: Conexiones}
{RUR-6}{Alta}{Esencial}
{
Cada laboratorio posee un puerto de conexión con el proveedor, otro de 
notificaciones para el cliente, uno de recepción de peticiones y por 
último, el puerto de control de mantenimiento.
}

\reqnofuncional{Laboratorios: Asincrónos}
{RUR-6}{Alta}{Esencial}
{
Por cada petición de ejecución se enviará en qué instante empezará 
a funcionar, ya que es posible que no sea instantáneo si hay 
sobrecarga en el laboratorio.
}

\reqnofuncional{Laboratorios: Ejecuciones}
{RUR-7}{Alta}{Esencial}
{
El laboratorio se servirá de peticiones al sistema y señales para 
ejecutar las acciones, así como de un servicio de \emph{pipeling} para 
redirigir la entrada y la salida de la propia acción.
}

\reqnofuncional{Laboratorios: Colas de ejecución}
{RUR-8}{Alta}{Esencial}
{
Cada laboratorio poseerá un contenedor de acciones en ejecución, donde 
está limitado por su configuración inicial. Todas las peticiones de 
ejecución, si los contenedores están completos, serán retrasadas.
}

\reqnofuncional{\emph{Framework}: Parada}
{RUR-9}{Alta}{Esencial}
{
Aunque no se impondrá ninguna norma a los desarrolladores en cuestión 
de bloqueo de ficheros o de programas por el motivo de las paradas 
inesperadas, se instará en la manera de no mantener ficheros abiertos 
ni componentes susceptibles de ocasionar errores en el futuro.
}

\reqnofuncional{Cliente de escritorio: Máquina virtual de Java}
{RUR-4}{Alta}{Esencial}
{
La aplicación de escritorio será implementada en Java para su 
portabilidad y se requerirá configurar mediante un archivo adjuntado 
con el ejecutable.
}

\reqnofuncional{Cliente web: Múltiple conexión}
{RUR-8}{Alta}{Esencial}
{
Mediante el uso de \emph{tokens} almacenados en la base de datos del 
proveedor una desconexión no es obligatoria en el cliente web.
}


\cleardoublepage
