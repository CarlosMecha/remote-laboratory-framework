% Componentes

\subsection{Base de datos central}

Compone el principal almacén de datos de toda la plataforma RLF. 
Contiene un conjunto de tablas que organizan la información para que 
sea accesible mediante el servicio proveedor y los distintos 
laboratorios. Como se puede ver en la figura \ref{fig:erproveedor} 
(modelo entidad-relación), contiene las siguientes estructuras:

\begin{itemize}
\item Una tabla para la información de cada laboratorio. Incluye todos los 
elementos necesarios para la conexión y su estado principal.
\item Tablas de información de usuario, diferenciando entre los 
usuarios clientes y los usuarios administradores.
\item La información principal de cada herramienta está contenida en 
la tabla \emph{tool} junto con su estado y en qué laboratorio se 
encuentra.
\item Un registro temporal por cada herramienta reservada, 
relacionando el cliente con la herramienta, y la fecha de la propia 
reserva.
\item La descripción de cada herramienta, formada por sus constantes, 
parámetros, atributos, acciones y servicios externos están 
almacenados en sus propias tablas.
\end{itemize}

Cada cambio que se realiza en esta base de datos de forma 
automatizada, como son procesos de conexión, desconexión, reserva, 
activación de laboratorios, etc. están agrupados en procedimientos 
almacenados (interfaz \emph{iBase}) \footnote{Conjunto de 
instrucciones en una base de datos donde no hay valor de retorno.} 
para mejorar la seguridad y la fiabilidad del sistema. El resto de 
cambios concretos, como la insercción de una nueva herramienta, se 
realizan mediante el procedimiento normal. Se añaden también un 
conjunto de disparadores e índices para mejorar la velocidad de la 
base de datos.

Como se ve en la figura \ref{fig:Proveedor} la base de datos 
interactua directamente con el servidor web, y además con los 
laboratorios.

\subsection{Servidor Web}

Está compuesto por dos servicios y una página web de acceso a la 
plataforma (figura \ref{fig:Proveedor}) y el componente de 
comunicaciones.

\textbf{NOTA:} El componente de comunicaciones es descrito al final de 
este sección, ya que hay multitud de otros componentes que lo usan a 
pesar de tener la misma implementación (véase sección 
\ref{subsec:comunicaciones}). Aunque sea este componente el que se 
encargue de la recepción y envío de datos a nivel interno, las 
interfaces las proveen el resto de componentes, por lo que sólo se 
considera una capa a bajo nivel en la comunicación y será 
representado como tal.

\begin{figure}[H]
	\centering
	\includegraphics[scale=0.55]{images/erproveedor.png}
	\caption[E-R central]{Estructura de la base de datos central.}
	\label{fig:erproveedor}
\end{figure}

\subsubsection{Servicio Monitor}
Utiliza la información contenida en la base de datos central para que 
mediante la interfaz de acceso \emph{iMonitor} se obtenga el estado de 
las herramientas. Los métodos de esta interfaz son:

\begin{itemize}
\item Autentificación al monitor.
\item Desconexión del monitor.
\item Obtención del estado en forma de lista con el identificador de 
la herramienta, su nombre y su estado.
\end{itemize}

\subsubsection{Página Web}
Presenta una interfaz accesible a través de un navegador para utilizar 
el servicio del monitor. Está preparada para mantener la conexión 
abierta aunque se deje de atender la página. Se compone de una 
sección para la autentificación y una lista de herramientas y sus 
estados.

\begin{figure}[H]
	\centering
	\includegraphics[scale=0.6]{images/Proveedor.png}
	\caption{Servidor Web y Base de datos central.}
	\label{fig:Proveedor}
\end{figure}

\subsubsection{Servicio Proveedor}
Representa la interfaz abstracta de toda la plataforma para el acceso. 
Se puede decir que es el coordinador de todo el sistema, que permite a 
los clientes localizar (con la interfaz \emph{iRLF}) los laboratorios 
e iniciar las conexiones. Se sirve de los datos de la base central 
mediante la interfaz \emph{iDatos} y comunica a los laboratorios 
cuándo serán accedidos por los clientes.

Las acciones de la interfaz \emph{iRLF} se listan a continuación:

\begin{itemize}
\item Conexión al sistema. Cuando un cliente se conecta, genera el 
\emph{token} correspondiente y se lo envía para la identificación de 
todas las operaciones siguientes.
\item Desconexión. Invalida dicho \emph{token} de acceso y libera 
todas las herramientas que pudiera tener reservadas.
\item Obtiene el estado actual de las herramientas a las cuales el 
cliente tiene acceso.
\item Describe las herramientas las cuales pueden ser accedidas por el 
cliente. La descripción es un mensaje codificado en JSON precalculado 
con todos los atributos, constantes, acciones y parámetros de las 
herramientas.
\item Reserva las herramientas seleccionadas. Avisa además a los 
laboratorios para que inicien el contador de tiempo del cliente y 
devuelve a este la información de dónde se encuentra cada herramienta.
\end{itemize}

\subsection{Laboratorio}
\index{laboratorio}
Este complejo componente gestiona todas las peticiones a las herramientas 
por parte de los clientes. Está compuesto por tres subcomponentes 
principales (\emph{Kernel}, gestor de comunicaciones y gestor de 
ejecución) y por otros tres secundarios (comunicaciones, gestor de 
herramientas y base de datos) como se puede ver en la figura 
\ref{fig:Laboratorio}.

\begin{figure}[H]
	\centering
	\includegraphics[scale=0.6]{images/Laboratorio.png}
	\caption{El laboratorio y las herramientas.}
	\label{fig:Laboratorio}
\end{figure}

\subsubsection{\emph{Kernel}}
\index{\emph{Kernel}}
Es el núcleo del laboratorio, que atiende las peticiones del proveedor 
(recibidas por la interfaz \emph{iGestiónClientes}) y las de gestión 
(interfaz \emph{iGestiónLab}) desde el programa administrador. 
También controla la ejecución y el arranque de los otros dos 
subcomponentes principales. Es el único componente que tiene acceso a 
la base de datos central para poder cambiar su estado.

Cuando el laboratorio es armado, el \emph{Kernel} activa a los 
gestores de comunicaciones y de ejecución que se componen de hilos 
independientes. Usa al gestor de herramientas cuando recibe una 
petición de registro o de parada.

Las principales acciones que se concentran en este componente son:
\begin{itemize}
\item Iniciar el laboratorio. Avisa al proveedor que ha sido arrancado.
\item Parar el laboratorio. Sólo se realiza bajo petición de un 
administrador y cuando no está armado.
\item Armar y desarmar. En el estado de armado, se pueden recibir 
peticiones de los clientes y ejecutar las diferentes acciones.
\item Obtener el estado global del laboratorio y sus herramientas. 
Sólo es utilizado por los administradores.
\item Avisar al proveedor (mediante la base de datos) que el tiempo de 
un cliente ha terminado.
\item Iniciar el proceso de registrar o eliminar una herramienta. Sólo 
en el caso de no estar armado.
\item Realizar una parada de emergencia completa.
\end{itemize}

\subsubsection{Gestor de ejecución}
\index{gestor de ejecución}
Contiene todos los elementos necesarios para organizar las peticiones 
(enviadas por el gestor de comunicaciones) de ejecución de las 
diferentes acciones de las herramientas. Controla el tiempo máximo de 
cada ejecución y realiza el envío de datos entrantes y salientes a 
modo de \emph{stream} a los clientes, sin formato establecido. Cuando 
una acción es terminada o interrumpida, genera el mensaje a enviar al 
cliente. También indica al gestor de herramientas qué cambios se 
deben hacer en la base de datos de la herramienta
antes de ejecutar una determinada acción, y recoger estos cuando la 
acción haya terminado.

\subsubsection{Gestor de comunicaciones}
\index{gestor de comunicaciones}
Es el enlace con los clientes. Recibe las peticiones de ejecución por 
la interfaz \emph{iPeticiones} y las encola en el gestor de ejecución. 
Cuando ocurre cualquier evento con esas acciones, se lo comunica al 
cliente mediante el puerto de notificaciones. Con esto se consigue un 
tratamiento asíncrono. Por último, todos los cronómetros de cada 
cliente los contiene este gestor, así como la información de cada 
usuario válido en el sistema.

\subsubsection{Gestor de herramientas}
\index{gestor de herramientas}
Componente encargado en la gestión de las herramientas activas en el 
laboratorio, así como el manejo de su información mediante el 
subcomponente de la base de datos propia. Valida todas las 
configuraciones XML introducidas por los administradores y lee, cuando 
el laboratorio es arrancado, las ya existentes. No funciona como un 
hilo a parte, si no como una biblioteca de métodos.

Además, toda la información en tiempo real es insertada o leída de 
la base de datos de la herramienta como si fuese la propia aplicación.

\subsubsection{Base de datos}
Funciona como un almacén de los datos de acceso de las herramientas. 
Conforma una estructura simple de una única tabla con los elementos 
más importantes, siendo la mostrada en la figura \ref{fig:erlaboratorio}

\begin{figure}[H]
	\centering
	\includegraphics[scale=0.8]{images/labdata.png}
	\caption[E-R local]{Estructura de la base de datos del laboratorio.}
	\label{fig:erlaboratorio}
\end{figure}

\subsection{Datos de la herramienta}
Ha sido concebido como una interfaz entre la aplicación y el 
laboratorio, pero de forma asíncrona. De esta forma, la herramienta es 
tratada como un objeto externo. El contenido es la estructura directa 
de la herramienta (figura \ref{fig:erherramienta}). 

\begin{figure}[h]
	\centering
	\includegraphics[scale=0.65]{images/tool.png}
	\caption[E-R de la herramienta]{Estructura de la base de datos de 
	la herramienta.}
	\label{fig:erherramienta}
\end{figure}

La correspondencia de cada tabla es similar a la base de datos 
central, que contiene la misma información, a excepción de los 
atributos, que conforman las características propias de cada 
herramienta (en el servidor central están representadas como columnas 
de la tabla principal), y las excepciones y estados de las diferentes 
acciones.

Está implementada de forma sencilla para obtener una mayor velocidad 
de escritura y lectura. Es por ello que todas las claves ajenas y otras 
restricciones han sido desactivadas, y son comprobadas de forma 
externa.

\subsection{Herramienta}
Este componente contiene la arquitectura del \emph{framework} de RLF. 
No se definirán las diferentes herramientas entregadas si no la forma 
que deben tener las aplicaciones.

\subsubsection{Aplicaciones}
Son el conjunto de acciones a ejecutar. Están implementadas según las 
normas del \emph{framework} (ver Manual de desarrollo) y sólo son 
dependientes del otro componente \emph{Libtool} y de los servicios 
externos si se requieren.

\subsubsection{Servicios externos}
Poseen un acceso como puerto de comunicaciones donde los clientes 
pueden conectarse, pero no es gestionado por el laboratorio ni por el 
proveedor central.

\subsubsection{\emph{Libtool}}
\index{\emph{libtool}}
Funciona como una biblioteca de funciones para la transformación en 
objeto. Así se pueden obtener los datos escritos por el laboratorio. 
Las funciones que provee este componente son las siguientes:

\begin{itemize}
\item Conexión y desconexión con el laboratorio. Si una aplicación 
requiere lectura o escritura de sus valores, es necesario estar 
conectado.
\item Lectura y escritura de parámetros.
\item Lectura de atributos. Las herramientas no pueden escribir sus 
propios atributos, ya que estos son definidos por el laboratorio 
poseedor.
\item Lectura de constantes.
\item Generación de excepciones.
\item Escritura del estado de la acción.
\end{itemize}

\subsection{Gestión de laboratorios}
\index{gestión de laboratorios}
Este componente forma una aplicación usada por los administradores de 
la plataforma RLF y forma una arquitectura de capas. Cada capa se 
sirve de la interfaz de la anterior, como se puede ver en la figura 
\ref{fig:labconsole}.

\begin{figure}[h]
	\centering
	\includegraphics[scale=0.65]{images/labconsole.png}
	\caption[Gestor de laboratorios por capas]{Estructura de capas del gestor de 
	laboratorios.}
	\label{fig:labconsole}
\end{figure}

\begin{figure}[h]
	\centering
	\includegraphics[scale=0.65]{images/Gestor.png}
	\caption[Gestor de laboratorios]{Componentes del gestor de 
	laboratorios.}
	\label{fig:gestorlab}
\end{figure}

\subsubsection{\emph{LabConsole}}
\index{\emph{LabConsole}}
Actua como interfaz textual para enviar peticiones a los laboratorios. 
Utiliza el conjunto de métodos proveídos por la librería de gestión 
(interfaz \emph{iGestor} en la figura \ref{fig:gestorlab}).

\subsubsection{Librería de gestión}
Incluye los parámetros necesarios introducidos mediante la interfaz 
textual en los mensajes para las peticiones de mantenimiento. Después 
serán enviadas a los laboratorios, obteniendo una respuesta y 
descodificándola.

\subsection{Cliente}
El cliente se muestra como una aplicación de escritorio en Java con 
una interfaz implementada en Swing. Debe estar configurada para 
acceder al servidor central de RLF. Al igual que los laboratorios 
tiene diferentes subcomponentes como hilos independientes (figura 
\ref{fig:cliente}).

\begin{figure}[h]
	\centering
	\includegraphics[scale=0.65]{images/Cliente.png}
	\caption[Comunicaciones RLF]{Componentes del sistema de 
	comunicaciones de la plataforma RLF.}
	\label{fig:cliente}
\end{figure}

\subsubsection{Gestor de eventos}
Es el componente encargado de ``escuchar'' los eventos de la interfaz 
generados por el usuario para llevar a cabo acciones con el proveedor 
o las herramientas. Establece una conexión directa con la interfaz 
\emph{iRLF} generada por el servicio proveedor. Los mensajes enviados 
con este componente no tienen el formato de dato interno, si no que 
utiliza el propio formato HTTP + SOAP que se usa en los servicios web.

\subsubsection{Gestor de notificaciones}
Todas las peticiones que se envían a los laboratorios y las 
notificaciones que se reciben de estos pasan a través de este 
componente, que usa el protocolo definido en le componente 
Comunicaciones. Se sirve de la interfaz \emph{iPeticiones} e 
implementa \emph{iNotificaciones} con una función de inicio de la 
acción y otra notificación de tiempo excedido. Actúa como puerto de 
escucha y posee ejecución paralela al gestor de eventos.

\subsubsection{Interfaz}
Provee al usuario cliente de todos los mecanismos necesarios para la 
realizar el acceso completo a la plataforma.

\subsection{Comunicaciones}
La implementación de este componente, usado en los otros, forma un 
protocolo de comunicación añadido al modelo TCP/IP (véase sección 
\ref{subsec:tcpip}) que es la base de las comunicaciones internas de 
la plataforma en forma de petición/respuesta. Además también tiene 
la función de librería para enviar y recibir datos, 
obteniendo directamente la información importante y \emph{parseada}.

En la figura \ref{fig:comunicaciones} se puede ver la disposición de 
sus subcomponentes.

\begin{figure}[h]
	\centering
	\includegraphics[scale=0.65]{images/Comunicaciones.png}
	\caption[Cliente de escritorio]{Componentes del cliente de 
	escritorio.}
	\label{fig:comunicaciones}
\end{figure}


\subsubsection{Cifrador}
Biblioteca con los métodos necesarios para cifrar datos textuales con 
el algoritmo \emph{base64} \footnote{Base 64 es un sistema de 
numeración posicional que usa 64 como base. Es la mayor potencia de 
dos que puede ser representada usando únicamente los caracteres 
imprimibles de ASCII.\cite{Tanenbaum}} y la obtención de valores 
mediante el algoritmo resumen \emph{SHA-1}.

\subsubsection{Red}
Implementa el protocolo de mensajes RLF. Se basa en un sistema de 
capas que convierte objetos JSON en mensajes reconocibles por el 
sistema. Los mensajes están formados por un identificador que 
corresponde a la operación a realizar y un conjunto de atributos. En 
la primera capa (capa de objeto) se obtienen esos atributos y se 
``aplanan'' (un atributo puede ser a la vez otro objeto JSON). En la 
capa de mensaje se cifra según el algoritmo \emph{base64} para no 
perder caracteres del mensaje y se añade la cabecera, que es el 
tamaño total del mensaje (ver figura \ref{fig:protocolo}). El proceso a 
la inversa también ha sido implementado.

\begin{figure}[h]
	\centering
	\includegraphics[scale=0.65]{images/mensaje.png}
	\caption[Protocolo RLF]{Protocolo RLF.}
	\label{fig:protocolo}
\end{figure}

\textbf{NOTA:} Para obtener más información sobre los identificadores 
de las peticiones, consultar el código adjuntado donde se encuentra la 
lista exhaustiva de los mismos.

Esos mensajes son enviados y recibidos mediante \emph{sockets} que 
provee el sistema operativo. Los métodos son los siguientes:

\begin{itemize}
\item Conexión y desconexión de \emph{sockets} con direcciones remotas.
\item Envío de una petición/respuesta con un \emph{socket} conectado.
\item Recepción de una petición/respuesta por un \emph{socket} a la 
escucha.
\end{itemize}

