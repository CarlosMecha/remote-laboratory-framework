% Conclusión

\capitulo{Conclusiones}{conclusiones}{
``Sólo es posible avanzar cuando se mira lejos. Solo cabe progresar 
cuando se piensa en grande.'' \emph{José Ortega y Gasset}
}

\begin{figure}
	\centering
	\includegraphics[scale=0.5]{images/logo_rlf.png}
\end{figure}

\clearpage

\vspace*{1.5cm}

El avance de las nuevas tecnologías significa el avance de la 
sociedad, y de todos sus aspectos, desde la forma de relacionarse 
hasta la forma de trabajar. Es indudable que Internet revoluciona los 
sistemas actuales, sin que se pueda ignorar. Las empresas se han visto 
obligadas a cambiar su modelo de negocio para adecuarse a estos 
tiempos. Y por supuesto los centros de enseñanza. Se pueden encontrar 
universidades donde no es necesario ir a clase, ya que mediante 
plataformas colaborativas disponibles en Internet, así como la llegada 
de la emisión de sonido y vídeo a través de red el alumno puede 
aprender casi de la misma manera que si se desplazada hasta el centro 
educativo. Pero poco a poco, ese ``casi'' irá desapareciendo, 
aumentando a su vez la posibilidad de acceso a enseñanzas superiores.

La mayor parte de las retribuciones positivas de este proyecto provienen
de haber desarrollado una plataforma desde la capa más baja hasta la 
que está en contacto con el usuario. El uso de las múltiples 
tecnologías y su unión como un único sistema ha demostrado que es 
necesario evolucionar la tecnología desde diferentes puntos de vista, 
y sin ser guiados por un único objetivo.

Este proyecto no sólo ha sido la culminación de seis años de 
estudio, si no también un aspecto importante en la formación para ser 
ingeniero informático, y más aún si la especialidad son los sistemas 
distribuidos, que ahora tanta importancia han adquirido desde la 
aparición de la ``nube'' y los sistemas portátiles.

Así pues, habiendo cumplido con los objetivos marcados al inicio del 
proyecto, la satisfacción de haber realizado este proyecto es plena, 
ya que no sólo se ha conseguido crear una plataforma distribuida 
totalmente funcional, sino que se ha realizado con las mismas 
tecnologías que utilizan las grandes empresas en el mundo industrial y 
educativo. Muchas de las decisiones tomadas durante el desarrollo de 
este proyecto han venido influenciadas por conceptos que actualmente 
los grandes proyectos también se plantean, como es el tema de la 
seguridad. Se han tenido que sortear dificultades muy presentes a la 
hora de desarrollar \emph{software}, que cada vez se hacen mayores 
por la cantidad de dispositivos y sistemas, creados por diferentes 
empresas y organizaciones, que se encuentran en los hogares y centros.

El que en un proyecto de fin de carrera sea utilizado en un escenario 
práctico, real y donde se espera continuar es algo muy poco común, y 
desde luego inmensamente satisfactorio. Se espera que este trabajo 
sirva para futuras ideas que ayuden a la comunidad educativa a formar 
a mejores profesionales e investigar nuevas tecnologías. El esfuerzo 
se ha visto recompensado en forma de nuevas ideas.

\vspace{1cm}

\begin{flushright}
Madrid, 1 de Octubre de 2011.\\
\vspace{2.5cm}
Fdo: Carlos A. Rodríguez Mecha
\end{flushright}

\cleardoublepage
