% Introducción

\capitulo{Introducción}{introduccion}{
La finalidad de este capítulo es dar a conocer el contenido de este 
libro, así como un primer encuentro con el proyecto. Tratado como una 
pequeña introducción, los conceptos novedosos serán explicados en 
capítulos posteriores.
}

\section{Introducción}

Este proyecto se inicia como parte de una solución a problemas 
concretos en la evolución tecnológica. Desde hace años, el mundo 
está inmerso en un constante cambio impulsado por el desarrollo de una 
parte de la ciencia, la cual sirve como base a innumerables 
campos. Tan amplio es el efecto, que no se concibe, en la actualidad, 
la vida sin ella. Ésta es la ciencia de la información.

Todo lo que implica una revolución como ésta se ve reflejado en el 
pensamiento de los nuevos científicos, ingenieros, estudiosos y 
empresarios, los cuales han tenido que adaptar sus ideas a esta nueva 
situación. Éstas son las que ahora, y en el futuro, resolverán 
los problemas que surgen como consecuencia directa del cambio sometido.

La idea (o proyecto), que en las posteriores páginas de este documento es 
explicada, no es más que una de las miles que intentan ayudar a esa 
evolución sin retorno, asegurando que será la base para otras 
soluciones. Se dará como nombre \emph{Remote Laboratory Framework} 
(versión \emph{Prototype}) a este proyecto.

Se empieza, entonces, con los motivos y características que han 
gestado esta idea, así como su análisis y diseño, y, por último, su 
implementación en la vida real.

\section{Motivación del proyecto}

La ciencia de la información ha contribuido al desarrollo de otros 
campos (no sólo en el ambiente tecnológico) y al aumento de 
necesidades concretas. Yendo a las partes más básicas de la vida 
cotidiana, la educación y el trabajo se han visto afectados 
directamente, siendo parte del problema y de la solución.

De tal modo es así que como al cabo de los años de estudio, un alumno se enfrenta a 
numerosos problemas relacionados con ambos campos, independientemente 
del tipo de estudios cursados. Las innumerables horas dedicadas al 
desarrollo y aplicación de los conocimientos adquiridos hacen que las 
ideas surjan como una respuesta lógica a estos mismos problemas.

Pero este proyecto se centrará en dos problemas concretos que hasta hace 
unos años, no tenían aparente solución:

El primero es la \textbf{dependencia} que se tiene en los modelos de educación 
y de trabajo actuales respecto a los lugares donde se desarrollan 
estas actividades. Los traslados pueden llegar a copar gran parte del 
tiempo en un estudiante o trabajador, el cual, la mayor parte es 
improductivo. Cabe decir que se realizan esfuerzos por parte de las 
organizaciones involucradas para intentar eliminar esta gran 
dependencia creando herramientas de comunicación y sistemas 
automatizados remotos. Siendo soluciones válidas, aún quedan muchos 
campos por tratar, como es el caso de las herramientas de trabajo 
electrónicas, desde ordenadores, máquinas y componentes mećanicos 
entre otros, dispuestos en un lugar concreto, donde el personal debe 
desplazarse obligatoriamente para poder usarlos. \emph{El problema de 
la dependencia física de las herramientas de trabajo.}

El segundo problema se centra más en la \textbf{variedad} de estas 
herramientas. Cada empresa vende sus productos a estos centros y 
organizaciones cerrando el paso a la competencia, lo que conlleva a 
que cada uno de estos sea incompatible (en mayor o menor medida) con 
los desarrollados por otras empresas. Esto obliga a los estudiantes y 
trabajadores a aprender como se usa cada producto, incluso cuando su 
funcionalidad es la misma, con el gasto de tiempo y esfuerzo que esto 
requiere. Se acentúa aún más en las herramientas \emph{software}, 
donde, sin seguir ningún estándar, fuerzan al usuario a cambiar su 
forma de trabajo. \emph{El problema de la heterogeneidad de los 
componentes tecnológicos.}

Cualquier intento de solventar el primer problema se enfrenta con el 
segundo de manera irremediable, obligando a que cada solución sea 
específica para un producto en concreto.

Paralelamente, el avance tecnológico desarrolla el concepto de 
``la gran red'', Internet, que se muestra como aliciente para 
solventar dichos problemas. Esta gran tecnología permite lidiar con el 
problema de la dependencia, y las herramientas surgidas de ella, como 
las grandes plataformas de trabajo apuntan al problema de la variedad.

A partir de aquí, se puede extender el concepto de este proyecto a 
otros campos sin aparente relación. En la ingeniería, con sistemas 
complejos compuestos por multitud de componentes electrónicos y 
tecnologías, es donde se invierten grandes fortunas en solventar estos 
dos mismos problemas. Incluso, al aportar los avances tecnologicos a 
campos donde anteriormente no se encontraban, como es el caso de la 
arquitectura y construcción, aparecen dichos problemas.

\section{Objetivos}

Apuntando a unos objetivos concretos y bien definidos, el proyecto 
tiene como finalidad solventar los dos problemas anteriormente 
descritos. Como se verá a lo largo de este documento, se considerará 
crear un sistema que permita principalmente:

\begin{itemize}
\item Acceder a un conjunto de recursos heterogéneos de manera remota, 
es decir, sin necesidad de estar físicamente donde se encuentran. 
\item Conseguir un sistema distribuido que sea visto como un único 
componente por parte del usuario.
\item Incluir la posibilidad de ampliar el sistema de forma escalable 
sin que suponga el desarrollo desde cero.
\end{itemize}

Si se ponen en práctica estos objetivos, se obtendrá una 
\emph{plataforma} que permita la eliminación de muchos sistemas 
dedicados, así como su ahorro, en ambientes de trabajo donde se 
requiera utilizar recursos físicos electrónicos, aplicándose 
también a recursos didácticos en universidades o escuelas. 
De este modo se minimizará el tiempo que se requiere para su uso y el tiempo 
que estas herramientas permanecen \emph{ociosas}. Para cumplir estos 
objetivos, es necesario atacar un conjunto metas más desgranadas que 
atienden a objetivos sencundarios. 

\begin{itemize}
\item Permitir al usuario manejarlos de igual manera que si lo hiciera 
de forma local.
\item Dar cabida a recursos de diferentes tipos, 
transformando esa heterogeneidad en una aparente homogeneidad.
\item Eliminar las dependencias de los recursos en cuanto a requisitos 
propios de la plataforma, es decir, reducir al mínimo las especificaciones 
de funcionamiento de cada recurso.
\item Automatizar todo el proceso de la utilización de los recursos, de 
forma que no requiera de intervención externa para el funcionamiento 
normal del sistema, lo que hará que se pueda utilizar en cualquier 
momento, incluso fuera de horarios de trabajo o lectivos.
\item Ayudar a que las posteriores modificaciones del sistema sean de 
forma controlada y manteniendo, en la medida de lo posible, la 
coherencia del proyecto original.
\end{itemize}

Desde el punto de vista de la creación de nuevos recursos, se 
establecieron unas normas y requisitos para poder salvar esa 
heterogeneidad.

Como conclusión, se dirá que se ha creado un sistema distribuido 
manejado por un \index{\emph{middleware}}\emph{middleware}  
\footnote{Capa de \emph{software} que se ejecuta sobre el sistema operativo 
que permite manejar una red de computadoras como un sistema único 
\cite{Tanenbaum}} que a la vez se comporte como un 
\emph{framework} \footnote{(Plataforma, entorno, marco de trabajo). 
Desde el punto de vista del desarrollo de \emph{software}, es una 
estructura de soporte definida, en la cual otro proyecto de 
\emph{software} puede ser organizado y desarrollado. \cite{Alegsa}} 
para la creación, distribución y uso de nuevos recursos tecnológicos.

\section{Planificación}

Debido a la gran cantidad de trabajo que supone el desarrollo de este 
proyecto, es necesario establecer una planificación \emph{a priori} que 
defina cómo abordar tales problemas.

Siendo la parte más importante el análisis y diseño, se realizará 
en primer lugar la recolección de requisitos definidos en primera 
instancia por el cliente y los usuarios, con su posterior tratamiento y 
desarrollo.

En segundo lugar el diseño completo del sistema, a partir de dichos 
requisitos dará lugar a las especificaciones necesarias para empezar 
el desarrollo del mismo.

\begin{table}[H]
\begin{center}
\begin{tabular}{|| l | c | r ||}
	\hline
	\hline
	Tarea & Etapa & Horas estimadas\\
	\hline
	\hline
	Planteamiento inicial & Análisis y diseño & 16\\
	Aceptación del proyecto & Análisis y diseño & 4\\
	Primera toma de requisitos & Análisis y diseño & 8\\
	Revisión de requisitos & Análisis y diseño & 8\\
	Toma de requisitos & Análisis y diseño & 20\\
	Aceptación de requisitos & Análisis y diseño & 8\\
	Investigación y Diseño & Análisis y diseño & 120\\
	Aceptación del diseño & Análisis y diseño & 4\\
	Desarrollo & Desarrollo & 1100\\
	Verificación y pruebas & Pruebas & 40\\
	Aceptación del sistema & Pruebas & 4\\
	Documentación & Documentación  & 116\\
	Presentación del proyecto & Presentación & 20\\
	\hline
	& \textbf{Total} & 1468\\
	\hline
	\hline
\end{tabular}
\end{center}
	\caption[Planificación del proyecto]{Planificación de las tareas.}
	\label{tab:planificacion}
\end{table}

A continuación, se procederá a poner en práctica el diseño 
obtenido, siendo fieles a lo anteriormente establecido. Esto 
comprenderá desde el desarrollo del código necesario hasta las 
posteriores remodelaciones.

Después, con las primeras versiones del sistema, se procederá a 
realizar los \emph{test} necesarios que verifiquen el buen 
funcionamiento del mismo, en diversas situaciones y sobre múltiples 
estados. Debido a que en esta etapa puede surgir la necesidad de 
modificar el desarrollo del mismo, es posible que se deba retroceder 
de forma cíclica para poder solventar los problemas y errores que 
puedan aparecer.

Además, realizarán múltiples revisiones sobre la documentación 
creada durante todo el proceso para establecer su versión final.

Por último, y con la versión final, tanto del sistema como de la 
documentación, se expondrá el conjunto de los productos en una 
presentación donde asistirá el cliente, los tutores que supervisarán 
el desarrollo de todo el proyecto y un consejo de evaluación.

El recuento de horas se muestra en la tabla \ref{tab:planificacion} y 
calendario propuesto para estas tareas, representado en un diagrama de 
Gantt se encuentra en la figura \ref{fig:calendario} al final del capítulo.

\begin{figure}[h]
	\centering
	\includegraphics[angle=270,scale=0.65]{images/gantt.png}
	\caption[Calendario de tareas]{Calendario por semanas de las 
	tareas realizadas.}
	\label{fig:calendario}
\end{figure}

\section{Contenido}

La documentación que aquí se presenta viene estructurada de forma que 
siga los mismos pasos de la planificación anteriormente descrita:

\begin{description}
\item[Capítulo 1 - Introducción:] Sirve como ayuda para poder 
comprender la motivación de este proyecto, así como los objetivos que 
se desean alcanzar.
\item[Capítulo 2 - Estado arte:] Este capítulo de especial importancia 
da una visión general de las tecnologías usadas, así como la 
evolución que ha permitido el desarrollo del sistema. En las 
diferentes secciones se explicará cada concepto de importancia 
utilizado en este.
\item[Capítulo 3 - Análisis:] Mostrará el conjunto de los requisitos y 
especificaciones concretas que son necesarias para poder llevar a cabo 
correctamente esta idea.
\item[Capítulo 4 - Diseño:] Agrupa todas las soluciones propuestas a 
las anteriores especificaciones, así como la forma de actuar para 
desarrollar el proyecto.
\item[Capítulo 5 - Pruebas:] Especifica las diferentes situaciones 
controladas a las que se someterá el sistema para comprobar el 
correcto funcionamiento, así como los resultados obtenidos.
\item[Capítulo 6 - Problemas y conclusión:] Se especificarán los 
problemas más importantes que se han encontrado en el desarrollo del 
proyecto y una breve conclusión.
\end{description}

Además, se incluyen unos anexos para la correcta utilización y 
comprensión del sistema.

\begin{description}
\item[Anexo A - Presupuesto:] Lista detallada de los costes que supone 
implantar y mantener el sistema.
\item[Anexo B - Guía de despliegue:] Instrucciones que se deberán 
llevar a cabo para poder desplegar completamente el sistema en un 
ambiente de trabajo.
\item[Anexo C - Guía de mantenimiento:] Conjunto de directrices de 
mantenimiento para administraciones de la plataforma.
\item[Anexo D - Guía de desarrollo:] Primeras pautas para crear o 
modificar módulos del propio sistema así como la creación de nuevas 
herramientas.
\item[Anexo E - Guía del usuario:] Destinado al usuario final de la 
aplicación de escritorio y \emph{web}.
\item[Anexo F - Material entregado:] Lista exhaustiva de todos los 
componentes, guías y herramientas entregados a la finalización del 
proyecto en su primer prototipo.
\end{description}

\cleardoublepage


