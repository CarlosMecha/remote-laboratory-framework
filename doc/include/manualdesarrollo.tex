% Apéndice: Manual de desarrollo

\capitulo{Guía de desarrollo de herramientas}{desarrollo}{
Este proyecto basa su versatilidad en la gran cantidad de soluciones 
que se pueden aportar en forma de conjuntos de aplicaciones (llamadas 
herramientas). Aquí se explican los métodos necesarios para 
realizar dichas herramientas.
}

\section*{Estructura de una herramienta}
Las herramientas son conjuntos de aplicaciones sin interfaz gráfica que se 
comunican con la plataforma RLF mediante una librería llamada 
\emph{libtool}. Se caracterizan por:

\begin{itemize}
\item Cada herramienta puede tener una o varias acciones a realizar 
(es decir, una o varias aplicaciones) que sólo se podrá ejecutar una 
instancia de ellas, siendo imposible ejecutar dos o más a la vez.
\item La herramienta puede tener una salida estándar (textual) y una 
entrada también textual que serán las que el usuario visualice.
\item Generalmente estas están asociadas a un \hardware específico, 
como puede ser una cámara de vídeo.
\item Pueden ofrecer uno o varios servicios externos, los cuales 
requieren que el usuario se conecte a ellos de forma independiente a 
la plataforma, llamados \emph{sockets}. Por ejemplo, pueden ser 
servicios externos un servidor FTP o HTML, \emph{streaming} de 
archivos multimedia, conexión a una base de datos, etc.
\item Las herramientas y sus componentes se definen en un fichero XML 
con un formato determinado, y se usará para darlas de alta en la 
plataforma RLF.
\item Aquellas que poseen el tipo ``Herramienta de datos'' no disponen 
de entrada estándar y sólo contienen una acción (o aplicación). 
Además esa acción no tiene parámetros de salida ni de entrada.
\item Cada acción tiene un número determinado de parámetros de 
entrada, salida o ambos que serán definidos antes de ejecutarla, así 
como unas constantes que no variarán su valor. 
Cuando se termine la ejecución, se almacena un resultado, y las 
excepciones que puedan ocurrir.
\end{itemize}

En cuanto a los ficheros que debe contener una herramienta, posee una 
carpeta principal (llamada \emph{root}) donde será la ruta base para 
acceder a todas las aplicaciones.

\section*{Desarrollar una nueva herramienta}
Lo primero es realizar el fichero de configuración que defina el 
comportamiento de la herramienta. Se deberá decidir si la herramienta 
pasa a ser ``de Datos''. Dependiendo del tipo, se tendrá que rellenar 
un fichero XML u otro. Se muestran a continuación dos ejemplos de 
ambos tipos:

\subsection*{Configuración}

\begin{verbatim}
<!-- EJEMPLO DE UNA HERRAMIENTA -->
<?xml version="1.0" encoding="UTF-8"?>
<!DOCTYPE tool SYSTEM
   "/home/rodriguezmecha/Universidad/Proyecto/rlf/tools/tool.dtd"
>
<tool
   path="/home/rodriguezmecha/Universidad/Proyecto/rlf/tools/linux/RLF_DummyTool"
   data="false"
>
    <in-stream/>
    <out-stream/>
    <attributes>
        <name>RLF DummyTool</name>
        <version>0.1</version>
        <description>
           Ejecuciones básicas para realizar pruebas en Linux. 
           Comprende dos acciones que ejecutan operaciones de entrada 
           y salida por teclado y acceso a ficheros locales.
        </description>
        <admin>carlos</admin>
        <role>0</role>
    </attributes>
    <constants>
    </constants>
    <parameters>
        <parameter name="exit_word" data-type="string">
            <description>Palabra con la que salir del programa.</description>
        </parameter>
        <parameter name="nechos" data-type="int">
            <description>Número de echos realizados.</description>
        </parameter>
    </parameters>
    <actions>
        <action name="echo" timeout="5">
            <value>rlfdummytool --echo</value>
            <description>Repite la entrada por teclado.</description>
            <action_parameter name="exit_word" type="in"/>
            <action_parameter name="nechos" type="out"/>
        </action>
        <action name="cpu-info" timeout="7">
            <value>rlfdummytool --cpu</value>
            <description>
               Obtiene información básica sobre el sistema (nombre de 
               la máquina y memoria libre) bajo petición del usuario.
            </description>
        </action>
        <resetter>rlfdummytool --clean</resetter>
    </actions>
</tool>
\end{verbatim}

Son pocas las diferencias que hay entre las dos configuraciones, sólo 
denotar el atributo \emph{data} y la forma de declarar las acciones. 
Cada tipo de herramienta tiene un DTD para comprobar la estructura de 
la configuración (llamados \emph{tool.dtd} y \emph{data-tool.dtd}).

\begin{verbatim}
<!-- EJEMPLO DE UNA HERRAMIENTA DE DATOS -->
<?xml version="1.0" encoding="UTF-8"?>
<!DOCTYPE tool SYSTEM
   "/home/rodriguezmecha/Universidad/Proyecto/rlf/tools/data-tool.dtd"
>
<tool 
   path="/home/rodriguezmecha/Universidad/Proyecto/rlf/tools/linux/RLF_Video"
   data="true"
>
    <out-stream/>
    <attributes>
        <name>RLF Video</name>
        <version>0.1</version>
        <description>
            Streaming de video y sonido. Incluye 3 segundos de delay 
            en el envío. Es necesario utilizar programas externos para 
            poder obtener el flujo de datos.
        </description>
        <admin>carlos</admin>
        <role>0</role>
    </attributes>
    <constants>
        <constant name="devices" data-type="string">
            <value>
               v4l:///dev/video0:input-slave=alsa://:v4l-norm=0
               :v4l-frequency=0:file-caching=300
            </value>
            <description>Dispositivos de captura.</description>
        </constant>
        <constant name="http_ip" data-type="string">
            <value>192.168.1.128</value>
            <description>IP del host.</description>
        </constant>
        <constant name="http_port" data-type="int">
            <value>64000</value>
            <description>Puerto de acceso.</description>
        </constant>
        <constant name="user" data-type="string">
            <value>video</value>
            <description>Usuario para la conexión.</description>
        </constant>
        <constant name="pass" data-type="string">
            <value>video</value>
            <description>Contraseña del usuario.</description>
        </constant>
    </constants>
    <action name="http-streaming" timeout="60">
        <value>rlfvideo --http</value>
        <description>
           Streaming por video mediante el protocolo HMTL. Utilizar un 
           programa como VLC (con recepción de imágenes y sonido por 
           volcado de red) para conectarse al servidor (http://nombre 
           del servidor:puerto/)
        </description>
        <socket port="64000" protocol="http" type="media" mode="r"/>
    </action>
    <resetter>rlfvideo --clean</resetter>
</tool>
\end{verbatim}

En primer lugar se encuentra la cabecera de la configuración, donde se 
indica el tipo de herramienta, su esquema DTD correspondiente 
(incluido en el proyecto) y su carpeta \emph{root} bajo el nombre de 
\emph{path}. A continuación se especifica si se dispone de entrada y 
salida estándar (\emph{in-stream} y \emph{out-stream}).

Después están los atributos de cada herramienta, donde se definen su 
nombre, descripción, administrador (debe estar registrado en la base 
de datos), versión y rol \index{rol}(número donde el 0 es la base, y es más 
restrictivo según se va añadiendo valores).

Las constantes definen valores fijos que no pueden ser cambiados, pero 
que todas las acciones pueden utilizar y obtener. Se componen de un 
tipo de datos, nombre, valor y descripción.

Los parámetros son definidos sólo en las herramientas normales, y se 
componen de un nombre, descripción, un valor máximo, mínimo y por 
defecto (opcional). Más adelante se les relaciona con las acciones.

Las acciones son diferentes comandos con un nombre, una descripción y 
un tiempo máximo de ejecución. A partir de ahí, se definen los 
parámetros que tienen asociados (indicando si son de entrada, salida o 
ambos. También tienen asociadas los \emph{sockets} que utilizará el 
comando para proveer el servicio externo, indicando su puerto, el 
protocolo, el tipo (media, datos o gráficos) y si es de lectura, 
escritura o ambos. Hay que recordar que esos comandos serán llamados a 
partir del directorio root, es decir, en el primer caso, la acción ``cpu-info'' 
se llamará de la forma \texttt{<path>/rlfdummytool --cpu}.

Como acción especial está el \emph{resetter} o limpiador, que no 
podrá ser ejecutado por el usuario y sirve para establecer a un estado 
seguro el \hardware asociado a la herramienta. No tiene parámetros ni 
servicios externos.

\subsection*{Uso de la librería \emph{libtool}}
Como ya se ha comentado, esta librería sirve para poner en contacto a 
la herramienta con la plataforma, y es necesario utilizarla para poder 
registrarla. Se ha añadido a este proyecto tres librerías escritas en 
C/C++, Bash y .NET, compartiendo todas la misma estructura de funciones:

\begin{enumerate}
\item \textbf{RLF\_ Init:} Inicia la conexión con la plataforma. Es 
necesario introducir la clave que se proporcionó al registrar la 
herramienta. Sin esta función no se puede acceder a ninguna otra. Si 
se está ejecutando una acción la cual no ha sido ordenada por la 
plataforma, no se permitirá el acceso a la misma.
\item \textbf{RLF\_ Finalize:} Desconecta la herramienta, indicando un 
código de finalización (establecido por el administrador) y una 
descripción. Es obligatorio realizarlo antes de cerrar la aplicación 
o antes de que no haya una finalización concreta (por ejemplo, antes 
de un bucle infinito). Hay que tener en cuenta que la aplicación puede 
ser detenida en cualquier momento por varios motivos, por lo que debe 
estar preparada para ello.
\item \textbf{RLF\_ GetConst:} Obtiene la constante indicada por el nombre y 
es almacenada en la estructura correspondiente (ver código para más 
información).
\item \textbf{RLF\_ GetAttribute:} Obtiene el atributo correspondiente con el 
nombre introducido. Estos atributos son gestionados por la plataforma.
\item \textbf{RLF\_GetParameter:} Obtiene el valor y la información de 
cualquier tipo de parámetro relacionado con la acción en ejecución 
actual. Estos valores han sido establecidos por el usuario.
\item \textbf{RLF\_ SetParameter:} Establece el valor textual de un parámetro 
de salida que esté asociado a la misma acción.
\item \textbf{RLF\_ ThrowException:} Lanza una excepción al usuario con un 
nombre y una descripción. No interrumpe el ciclo de ejecución.
\end{enumerate}

\textbf{NOTA:} Para obtener más información sobre las librerías 
específicas en cada lenguaje consultar el código entregado.

A continuación se muestra un ejemplo de una acción programada en 
Visual C++ que obtiene la memoria libre del sistema y la envía a la 
salida estándar cada cierto tiempo. Es parte de una herramienta de 
datos:

\begin{verbatim}
// 1. Inicio RLF
RLF_Manager ^ manager = gcnew RLF_Manager();
try {
    manager->init(TOOLKEY);
} catch (RLF_Exception ^ e){
        Console::WriteLine("Error con la base de datos al iniciar. " + e->getMsg());
        stream->Flush();
        return 1;
}

// 2. Obtención de la constante de tiempo.
try {
    time = Convert::ToInt32(manager->getConst("time")->getValue());
} catch (RLF_Exception ^ e){
    Console::WriteLine("Error con la base de datos al obtener datos." + e->getMsg());
    try {
        manager->finalize(1, "Error.");
    } catch (...){
    }
    return 1;
}

try {
    manager->finalize(1, "Error.");
} catch (...){
    Console::WriteLine("Error con la base de datos al finalizar.");
    return 1;
}

// 3. Ejecución.
while(true){
    GlobalMemoryStatusEx (&statex);
    Console::Write("Memory in use: {0:G}% ({1:D} / {2:D} Kbytes)", 
                statex.dwMemoryLoad, statex.ullAvailPhys/DIV,
                statex.ullTotalPhys/DIV);
    Thread::Sleep(1000 * time);
}
\end{verbatim}


\subsection*{Buenas prácticas}
Debido a que las aplicaciones que se programen deben seguir unos 
estándares, es recomendable seguir estos pasos:

\begin{itemize}
\item Las librerías que se utilicen para ejecutar las acciones, 
incluso la librería \emph{libtool} deben estar en la carpeta 
\emph{lib} dentro del directorio \emph{root} de la propia herramienta.
\item Los usuarios sólo verán los valores de los parámetros de 
salida cuando se termine de ejecutar la acción, por lo que escribir 
varias veces un valor no tiene sentido.
\item Los \emph{buffers} de salida son tratados como en los ficheros, 
por lo que si se quiere mostrar en tiempo real al usuario, será 
necesario vaciarlos o reducir su tamaño a cero. (Esto se puede 
conseguir mediante la función \emph{flush} contenida en muchos 
lenguajes de programación.)
\item Si se usa en una misma herramienta distinto \hardware 
dependiendo de la acción a tomar, es mejor separarlas y dividirlas en 
varias herramientas, para que los usuarios puedan disponerlas de 
manera más efectiva.
\end{itemize}

\section*{Crear una nueva \emph{libtool}}
Puede ser necesario crear nuevas librerías para otros lenguajes, como 
Java, Python e incluso el \emph{script} Bat de Windows. Para ello es 
necesario tener instalado la librería de acceso a la base de datos 
SQLite ya que la comunicación se realiza mediante ese formato (se 
puede obtener más información sobre las APIs de SQLite en la 
siguiente dirección \texttt{http://www.sqlite.org/}).

En el desarrollo de la librería hay que tener en cuenta las consultas 
a la base de datos, y la protección de escritura y lectura. Como 
explicar esto puede conllevar un aumento de la documentación bastante 
grande, se puede consultar las librerías ya hechas que están 
profusamente comentadas para ver cómo desarrollar una nueva, ya que la 
estructura es esencialmente la misma y su aplicación es sencilla.

La figura \ref{fig:erlibtool} muestra el modelo que sigue la base de 
datos asociada a cada herramienta.

\cleardoublepage
