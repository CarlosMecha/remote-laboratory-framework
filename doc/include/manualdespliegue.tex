% Apéndice: Manual de despliegue

\capitulo{Guía de despliegue}{despliegue}{
RLF está compuesto de varios módulos que trabajan conjuntamente como 
una única entidad, pero para eso, se requieren un conjunto de pasos 
que los \emph{orquestan} al inicio.
}

\section*{Plataforma Java}
Debido \index{Java} a que es un requisito en \textbf{cualquier} máquina de la 
plataforma RLF, es necesario instalarla antes que cualquier otro 
componente que se describe a continuación. Se pueden obtener de los 
siguientes enlaces:

\textit{Windows, Java 6 JRE}
\begin{itemize}
\item Descarga: \texttt{http://java.com/es/download/}
\item Manual: \texttt{http://java.com/es/download/help/index\_installing.xml}
\end{itemize}

\textit{Linux, Java 6 JDK}
\begin{itemize}
\item Descarga: \texttt{http://www.oracle.com/technetwork/java/javase/downloads/index.html}
\item Manual: \texttt{http://www.guia-ubuntu.org/index.php?title=Java}
\end{itemize}

\section*{Proveedor y monitor}
Estos dos módulos están originalmente creados para entornos Linux, 
aunque debido a las tecnologías usadas, pueden ser instalados en 
Windows. Sólo se explicará cómo ponerlos en funcionamiento en una 
distribución Linux.

Se componen de una base de datos principal en MySQL y de dos servicios 
web contenidos en el mismo paquete para su acceso externo. A 
continuación se muestran los pasos necesarios para incluirlos en el 
sistema.

\subsection*{Base de datos}
Se requerirá instalar los paquetes que proveen la plataforma MySQL 
completa. Para ello se echará mano del gestor de aplicaciones 
incluidos en las distintas distribuciones Linux. Los paquetes a 
instalar son los siguientes:

\textit{MySQL 5.1}

\begin{itemize}
\item \texttt{mysql-common} Paquete general de la plataforma MySQL.
\item \texttt{mysql-server} Contiene todo lo necesario para contener 
un servidor de base de datos.
\item \texttt{mysql-admin} Aplicación para la cómoda creación de 
usuarios y realización de \emph{backups} entre otros.
\item \texttt{mysql-client} Herramientas necesarias para el acceso a 
bases de datos MySQL.
\item \texttt{mysql-query-browser} La solución más cómoda que aporta 
MySQL para el manejo de los datos y creación de tablas en la base de 
datos.
\end{itemize}

\textbf{NOTA:} Es recomendable que la base de datos esté en la misma 
máquina que el servidor web ya que están configurados por defecto 
para acceder a una base de datos local, aunque se incluye en la 
sección de este mismo manual cómo configurar los servicios web para 
poder acceder a la base de datos de forma desacoplada.

Una vez instalados estos paquetes, será necesario configurar el 
servidor para atender las peticiones, en ello se asignará su IP de 
escucha y un puerto que será necesario para que cualquier parte de RLF 
acceda al proveedor. Véase para realizar esta configuración inicial:

\begin{verbatim}
http://dev.mysql.com/doc/refman/5.0/es/unix-post-installation.html
\end{verbatim}

Una vez el servidor esté completo con la base de datos activa, tocará 
el turno de aportarle información. Mediante \emph{MySQL Query Browser} 
se creará un esquema nuevo llamado ``rlf''. Después se desplegarán 
los \emph{scripts} que crean todas las tablas necesarias, se 
encuentran en la carpeta \emph{rlf/src/sql} y son el conjunto de 
archivos indicados en el fichero \emph{makedb.sql}.

\subsubsection*{Creación de usuarios.}
Cuando se pueda disponer de la base de datos, se recomienda crear los 
usuarios de acceso a la misma. Mínimo se requiere un administrador 
general, uno para el servicio web y uno por cada laboratorio que se 
vaya a disponer. Se podrá realizar mediante el programa instalado 
anteriormente \emph{MySQL Administrator}. Los permisos necesarios para 
el usuario del proveedor y de los laboratorios son los siguientes:

En el esquema ``mysql'':
\begin{itemize}
\item \emph{SELECT}
\end{itemize}

En el esquema ``rlf'':
\begin{itemize}
\item \emph{SELECT}
\item \emph{INSERT}
\item \emph{UPDATE}
\item \emph{DELETE}
\item \emph{REFERENCES}
\item \emph{LOCK\_TABLES}
\item \emph{EXECUTE}
\end{itemize}

Mientras que el usuario administrador tendrá todos los permisos en 
todos los esquemas disponibles.

\subsection*{Los servicios web}
Una vez que la base de datos está creada, es necesario instalar el 
servidor que permite ejecutar aplicaciones web. Se ha seleccionado la 
plataforma Tomcat que viene disponible con el IDE NetBeans de 
desarrollo en J2EE.

Antes de poder desplegar la aplicación, se requiere modificar el 
código de acceso a la base de datos del servicio web. Para ello, se 
utilizará el IDE NetBeans que se puede descargar de 
\texttt{http://netbeans.org/}, y se importará el proyecto contenido en 
\emph{rlf/projects/netbeans/RLF\_Provider}. A continuación se modifica la 
clase ``Database.java'':

\begin{verbatim}
/** Localización de la base de datos del proveedor. */
public final static String DATABASE = "jdbc:mysql://<IP>:<PUERTO>/rlf";
/** Usuario de acceso a la base de datos. */
private static String USER = <USUARIO PROVEEDOR>;
/** Contraseña del usuario. */
private static String PASS = <CONTRASEÑA>;
\end{verbatim}

Una vez modificado esto, se compilará el proyecto obteniendo como 
resultado un archivo ``Provider.war''. Después se instalará el 
programa Tomcat mediante los siguientes paquetes:

\emph{Tomcat 6.0} \index{Tomcat}

\begin{itemize}
\item \texttt{tomcat6} Paquete general de la plataforma Tomcat.
\item \texttt{tomcat6-admin} Aplicación para navegador que permite la 
configuración del sistema de manera muy sencilla.
\end{itemize}

Es necesario asignarle una IP y un puerto de escucha. Serán estos 
parámetros los que luego se configurarán en los clientes. Se puede 
obtener información de como configurar Tomcat en el siguiente enlace 
oficial:

\begin{verbatim}
http://tomcat.apache.org/tomcat-6.0-doc/setup.html
\end{verbatim}

Por último se copiará el archivo generado anteriormente en la carpeta 
\emph{webapps} donde se haya instalado Tomcat. A partir de ese momento 
se puede acceder al monitor y al proveedor por medio de web o de los 
clientes, aunque no habrá aún usuarios RLF introducidos. Las rutas 
típicas de acceso a estos dos servicios pueden ser:

\begin{verbatim}
http://<IP>:<PUERTO>/RLF/Provider
http://<IP>:<PUERTO>/RLF/Monitor
\end{verbatim}

\section*{Laboratorios}
Una vez instalada la máquina virtual de Java, no es necesario instalar 
ningún componente más. Conviene copiar todos los archivos contenidos en 
\emph{rlf/bin/lab} (\emph{rlf\textbackslash bin\textbackslash lab} en 
Windows) a otra carpeta y configurar el laboratorio modificando el 
fichero \emph{res/lab.conf} (o \emph{res\textbackslash lab.conf} en 
Windows) de la siguiente forma:

\begin{verbatim}
lab_name=<NOMBRE ÚNICO DEL LABORATORIO>
user=<USUARIO DE LA BASE DE DATOS>
pass=<CONTRASEÑA>
provider_host=<IP DE LA BASE DE DATOS DEL PROVEEDOR>
provider_port=<PUERTO DE LA BASE DE DATOS DEL PROVEEDOR>
labmanager_request_port=<PUERTO GENERAL DEL LABORATORIO. DLF=6400>
client_request_port=<PUERTO DE COMUNICACIONES 1. DFL=6401>
client_notification_port=<PUERTO DE COMUNICACIONES 2. DFL=6402>
provider_request_port=<PUERTO DE COMUNICACIONES 3. DFL=6403>
max_process=<NÚMERO MÁXIMO DE PROCESOS EN EJECUCIÓN. DFL=4>
\end{verbatim}

Es importante apuntar el puerto general del laboratorio ya que es 
necesario para su configuración mediante LabConsole (véase Manual de 
mantenimiento).

\subsection*{Insertar el laboratorio en el proveedor}
Para que los laboratorios puedan utilizar correctamente el proveedor, 
deben ser introducidos en la base de datos de este. Para ello es 
necesario obtener la IP de la máquina donde se ejecuta. Por cada 
laboratorio, se insertará en la base de datos central la siguiente 
sentencia SQL:

\begin{verbatim}
INSERT INTO lab (name, host, description) VALUES ('nombre', 'ip', 'descripción');
\end{verbatim}

\subsection*{Ejecutar el laboratorio}
Para iniciar el laboratorio sólo es necesario acceder a la carpeta 
contenedora y ejecutar el fichero ``lab.jar'' aunque es recomendable 
utilizar la terminal que disponga el sistema operativo para obtener 
posibles errores (aparte de los que aparezcan en el \emph{log}). Se 
podrá realizar esto mediante el siguiente comando, válido tanto en la 
terminal de Linux como en el CMD de Windows:

\begin{verbatim}
java -jar <ruta>/lab.jar
\end{verbatim}

\section*{Herramientas}
El despliegue de las herramientas sólo es referido a las que han sido 
entregadas con este proyecto. Para crear nuevas consultar el Manual de 
desarrollo y el Manual de mantenimiento.

\subsection*{Configuración inicial}
Para que las herramientas funcionen se ha de modificar las rutas 
contenidas en sus ficheros de configuración XML. También puede ser 
necesario cambiar algún parámetro de estas que dependa del sistema 
(como por ejemplo, en el caso de RLF\_Video la IP de acceso). Después, 
se podrán registrar (ver Manual de administrador) y obtener la clave 
única, que se incluirá en el código y posteriormente se compilarán. 
Se ha añadido los proyectos para los distintos IDEs de las 
herramientas para poder ser compilados (y arregladas las dependencias) 
sin dificultad.

\subsection*{Componentes necesarios}
Algunas herramientas necesitan de componentes instalados externos que 
se citan a continuación.

\begin{itemize}
\item \textbf{RLF\_DummyTool:} Ninguno.
\item \textbf{RLF\_Music:}
	\begin{itemize}
	\item Es necesario el programa para linux MPG123 que se puede 
	obtener del paquete con el mismo nombre.
	\item El paquete de desarrollo \emph{libsqlite-dev}.
	\end{itemize}
\item \textbf{RLF\_Video:}
	\begin{itemize}
	\item Utiliza el programa VLC que se obtiene de los repositorios 
	oficiales de la distribución bajo el paquete con el mismo nombre.
	\item El driver necesario para utilizar la cámara web en Linux.
	\end{itemize}
\item \textbf{RLF\_FreeMem:}
	\begin{itemize}
	\item Requiere la librería System.Data.SQLite instalada en el 
	sistema, que se puede obtener de 
	\texttt{http://sqlite.phxsoftware.com/}.
	\end{itemize}
\item \textbf{RLF\_Board:}
	\begin{itemize}
	\item También requiere la librería System.Data.SQLite.
	\item Necesita los drivers para la tarjeta asociada 
	(\texttt{http://www.advantech.com/}). Es recomendable 
	que se recompile resolviendo las dependencias de estas librerías 
	ya que no pueden incluirse en el propio ejecutable.
	\end{itemize}
\end{itemize}

\section*{Usuarios de RLF}
Para que se pueda usar y administrar la plataforma, es necesario dar 
de alta a los usuarios de la misma.

\subsection*{Administradores}
Serán los encargados de mantener las herramientas, así como 
eliminarlas y registrarlas. Además utilizan el programa LabConsole 
para manejar los distintos laboratorios. Para dar de alta a un 
administrador en el sistema es requerido insertar esta sentencia SQL 
en la base de datos de proveedor:

\begin{verbatim}
INSERT INTO user (user, hash_pass, email) 
         VALUES ('nombre', SHA1('contraseña'), 'email');
INSERT INTO admin (name, tlf)
         VALUES ('nombre', 'teléfono');
\end{verbatim}

A partir de aquí ya podrá realizar dichas tareas.

\subsection*{Clientes}
Podrán usar el monitor web y el cliente de escritorio. Son los 
usuarios finales de RLF. Tienen asociado un tiempo máximo de 
reserva de herramientas (a discreción de los administradores, 
generalmente una hora) y un rol que les permite acceder a herramientas 
de mayor nivel o menos (el rol \index{rol} 0 es el más básico y según aumenta es 
más restrictivo). Se deberá añadir estas sentencias SQL al proveedor:

\begin{verbatim}
INSERT INTO user (user, hash_pass, email)
         VALUES ('nombre', SHA1('contraseña'), 'email');
INSERT INTO client (name, timeout, role)
         VALUES ('nombre', tiempo, rol);
\end{verbatim}

\textbf{NOTA:} Un administrador no incluye el rol de cliente, por lo 
que si se requiere un usuario con ambos perfiles, habrá que darlo de 
alta en cada perfil por separado de esta forma:

\begin{verbatim}
INSERT INTO user (user, hash_pass, email)
         VALUES ('nombre', SHA1('contraseña'), 'email');
INSERT INTO admin (name, tlf)
         VALUES ('nombre', 'teléfono');
INSERT INTO client (name, timeout, role)
         VALUES ('nombre', tiempo, rol);
\end{verbatim}
