% Apéndice: Manual de usuario

\capitulo{Guía de usuario}{usuario}{
Este manual está destinado al usuario final de la plataforma RLF. 
Contiene las acciones a realizar con el cliente de escritorio así como 
la utilización del monitor mediante web.
}

\section*{RLF\_ Client}
Este programa da acceso a todas las herramientas incluidas en el 
sistema RLF al que se va a conectar. Funciona tanto en Windows como en 
plataformas Linux. Aunque oficialmente no está soportado, también 
funciona en plataformas Macintosh que sean compatibles con la máquina 
virtual Java.

\subsection*{Instalación}
Para poder ejecutar el cliente, es necesario tener con anterioridad 
unas herramientas instaladas en el ordenador, independientemente del 
sistema operativo y son gratuitas:

\begin{description}
\item[JRE 1.6 de Java Sun:] Contiene la máquina virtual y las 
librerías necesarias para ejecutar programas en el entorno Java. Se 
puede descargar de \texttt{http://java.com/es/download/} aunque para 
sistemas Linux se puede encontrar en los repositorios oficiales con el 
nombre de \texttt{sun-java6-jre}.
\item[(Opcional) Reproductor media VLC:] Sólo es necesario si se 
requieren usar herramientas de vídeo. Se puede encontrar en la página 
oficial (\texttt{http://www.videolan.org/vlc/}) para cualquier 
plataforma. Al igual que antes, los repositorios de la mayoría de las 
distribuciones Linux bajo el nombre de \texttt{vlc}.
\item[(Opcional) Navegador de Internet:] Sólo es necesario para 
algunas herramientas concretas que utilicen servicios como FTP. Debido 
a que existen multitud de navegadores, se puede escoger el que 
quiera, incluso el que venga instalado por defecto en los sistemas 
operativos.
\end{description}

La instalación de estos programas se puede encontrar en Internet sin 
complicación, por lo que no se dedicará tiempo a explicarlo en este 
manual.

Una vez que esto esté listo, se puede pasar a la instalación del 
cliente. Para ello, es necesario obtener los archivos de la carpeta 
\texttt{rlf/bin/client} (\texttt{rlf\textbackslash bin\textbackslash 
client} en Windows) y copiarlos donde se desee. Es importante que las 
carpetas \texttt{res} y \texttt{lib} se copien en la misma ubicación 
donde está el archivo ejecutable \texttt{RLF\_ Client.jar}.

\subsection*{Configuración}
Para que el cliente se conecte a una dirección específica (que será 
dada por el administrador) donde se encuentra el proveedor es 
necesario modificar el archivo de configuración 
\texttt{res/client.conf} (\texttt{res\textbackslash client.conf} en sistemas 
Windows). Este fichero contiene la siguiente línea:

\begin{verbatim}
provider_url=http://[direccion]:[puerto]/[ruta]
\end{verbatim}

Para configurarlo sólo es necesario modificar los parámetros 
\emph{dirección}, \emph{puerto} y \emph{ruta} (eliminando los corchetes). 
Como por ejemplo:

\begin{verbatim}
provider_url=http://163.117.150.85:8084/RLF/Provider
\end{verbatim}

\subsection*{Ejecución}
Para ejecutar el programa sólo es necesario hacer ``doble click'' en el 
archivo \texttt{RLF\_ Client.jar} en cualquier sistema. Si como 
consecuencia se obtiene un mensaje de ``No se reconoce el archivo 
ejecutable'', la instrucción para la línea de comandos es:

\begin{verbatim}
java -jar RLF_Client.jar
\end{verbatim}

\textbf{NOTA:} En sistemas Linux puede haber errores con fallos de 
conexión o de la interfaz gráfica, son externos al programa y vienen 
dados por la máquina virtual:

\begin{itemize}
\item Hay que asegurar que no se está utilizando la versión OpenJDK 
de Java, que es la que viene por defecto en Linux. Esta versión tiene 
problemas (reconocidos oficialmente) con las interfaces gráficas Swing. 
Para arreglar este problema, seguir la guía que se puede obtener en 
esta dirección:
\begin{verbatim}
http://www.e-capy.com/reemplazar-openjdk-por-el-jdk-de-java-en-ubuntu/
\end{verbatim}

\item Siempre es recomendable utilizar el comando anteriormente 
descrito en vez de ejecutar el programa con ``doble click'' ya que 
Java establece sus rutas dependiendo de dónde haya sido ejecutado. Es 
por esto que puede que no encuentre las librerías o el fichero de 
recursos.
\end{itemize}

\clearpage

\subsection*{Interfaz}

\begin{figure}[H]
	\centering
	\includegraphics[scale=0.45]{images/user/screen.png}
\end{figure}

Aquí se listan todos los elementos que componen la interfaz del 
cliente mostrados en la anterior figura. Puede que en determinados 
momentos algunas de estas opciones no estén activas.

\begin{enumerate}
\item Es la barra de menús donde poder acceder a todas las acciones, 
como ruta alternativa a los otros botones. Además se indican los 
atajos de teclado de estas acciones.
\item Con este botón se puede conectar o desconectar el usuario al 
sistema. 
\item Sirve para obtener el estado de cada herramienta antes de 
haberlas reservado.
\item Reconstruye la información de las herramientas por si ha habido 
cambios estructurales en el servidor.
\item Reserva las herramientas seleccionadas.
\item En esta zona se indican las herramientas que se han 
seleccionado, o el tiempo restante de reserva de las mismas. Si se 
posiciona el puntero del ratón encima se pueden obtener el nombre e 
identificador de cada herramienta seleccionada o reservada.
\item Cada pestaña corresponde a una herramienta distinta controlada 
por el sistema.
\item Estado de la herramienta, ver la sección ``Estados de una 
herramienta'' para más información.
\item Nombre e identificador único de cada herramienta.
\item Atributos de cada herramienta, ver la sección ``Atributos de una 
herramienta'' para más información.
\item Descripción de la herramienta.
\item Acción seleccionada de la herramienta. Sólo se pueden ejecutar 
una vez la herramienta haya sido reservada. Si se despliega la lista 
se podrán obtener todas las acciones asignadas.
\item Cuando la herramienta está reservada, con este botón se ejecuta 
la acción seleccionada.
\item Descripción de la acción seleccionada.
\item Sirve para seleccionar la herramienta de la pestaña actual para 
su posterior reserva.
\end{enumerate}

\subsubsection*{Estados de una herramienta}
Cada herramienta tiene asociado un estado con cuatro valores posibles 
que se corresponden a las cuatro imágenes de la siguiente figura:

\begin{itemize}
\item \textbf{Disponible:} Esta herramienta no está siendo usada por nadie y 
puede reservarse. 
\item \textbf{No disponible:} Esta herramienta está siendo usada por otro 
usuario o está desconectada del sistema.
\item \textbf{Reservada:} La herramienta ya está reservada y lista para su 
ejecución.
\item \textbf{En ejecución:} Alguna acción de esta herramienta está 
actualmente en ejecución.
\end{itemize}

\begin{figure}[h]
	\centering
	\includegraphics[scale=1]{images/user/statustools.png}
\end{figure}

\subsubsection*{Atributos de una herramienta}
Cada herramienta posee tres atributos característicos que determinan 
su comportamiento. Se encuentran indicados en la zona 10 explicada en 
el apartado anterior:

\begin{itemize}
\item \textbf{In:} Si está activo, la herramienta acepta interacción con el 
usuario mediante la entrada por teclado. Cuando se ejecute alguna 
acción, se podrán enviar comandos en el visor de ejecuciones.
\item \textbf{Out:} Si está activo indica que la herramienta dará información 
textual al usuario cuando se ejecute una acción.
\item \textbf{Data:} Indica si la herramienta tiene la categoría de 
``Herramienta de datos''. Este tipo de herramientas no tienen entrada 
por teclado, sólo contienen una acción y permiten que varios usuarios 
la utilicen a la vez, por lo tanto, esta herramienta siempre estará 
libre para su uso.
\end{itemize}

\subsection*{Cómo utilizar las herramientas}
El proceso para obtener unas herramientas y realizar acciones con 
ellas se describe a continuación.

\subsubsection*{Conexión}
En primer lugar se ha de conectar con el servidor, para ello se pulsa 
el boton de \textit{Login} y se introduce el usuario y la contraseña 
asignada en el diálogo que aparecerá como se ve en la figura 
a continuación. A partir de ese momento, el resto de las opciones 
estarán disponibles.

\begin{figure}[h]
	\centering
	\includegraphics[scale=0.8]{images/user/login.png}
\end{figure}



\subsubsection*{Selección de las herramientas}
Cuando las herramientas aparezcan, se podrá seleccionar aquellas que 
dispongan del estado \textbf{Disponible} mediante el \emph{checkbox} 
propio de cada pestaña. Debido a que puede haber varios usuarios 
realizando peticiones, es recomendable actualizar (con el botón 
\textit{Refresh}) los estados de las herramientas antes de 
seleccionarlas, ya que no se actualizan en tiempo real. Cuando se 
hayan seleccionado las herramientas, aparecerán en el indicador 
mostrado en la siguiente figura.

\begin{figure}[h]
	\centering
	\includegraphics[scale=0.8]{images/user/select.png}
\end{figure}

\subsubsection*{Reserva}
Para poder utilizar unas es necesario reservarlas antes. Cuando todas 
las herramientas que se deseen estén seleccionadas, se procederá a 
reservarlas con el botón \textit{Take Tools}. A partir de aquí, el 
cronómetro se pondrá en marcha, ya que cada usuario tiene un tiempo 
máximo de uso de las herramientas.

\begin{figure}[h]
	\centering
	\includegraphics[scale=0.8]{images/user/timer.png}
\end{figure}

\textbf{ATENCIÓN:} Cuando el cronómetro está cerca del límite de 
tiempo lanza un aviso mediante un diálogo, después, cuando se ha 
terminado cierra todas las acciones en ejecución y libera las 
herramientas.

\subsubsection*{Ejecutar una acción}
Cada herramienta puede ejecutar a la vez una sola acción, pero se 
pueden tener varias acciones de distintas herramientas ejecutándose en 
el mismo instante. Para ello, selecciona la acción correspondiente y 
se pulsa el botón \textit{Execute}.

Se mostrará un diálogo con los parámetros de entrada de la acción y 
con los \emph{sockets} para servicios externos. En los parámetros se indica 
el tipo y si se pasa el ratón por encima del nombre, se podrá obtener 
la descripción del mismo.

\begin{figure}[H]
	\centering
	\includegraphics[scale=0.4]{images/user/inparams.png}
\end{figure}

En el caso de esta figura, hay un parámetro de entrada llamado 
``volts'' de tipo \emph{double} y un servicio externo de FTP de 
lectura al que se puede acceder mediante la URL 
\texttt{ftp://163.117.150.95:64001/}. Para ello se utiliza un 
navegador o un cliente FTP.

\textbf{ATENCIÓN:} Los \emph{sockets} representan servicios que provee 
la herramienta pero que deben ser atendidos fuera de la interfaz de 
RLF\_ Client. Esto significa que es necesario utilizar un programa 
externo, como puede ser un navegador o un reproductor de vídeo. La 
información que se muestra es la localización o \emph{URL} de ese 
mismo servicio, así como su protocolo y el tipo de datos que obtiene.

\begin{figure}[h]
	\centering
	\includegraphics[scale=0.4]{images/user/video.png}
\end{figure}

Cuando se tengan configurados los parámetros, se podrá proceder a 
ejecutar la acción. Si la máquina donde se encuentra la herramienta 
tiene mucha carga, puede ser necesario esperar hasta que se pueda 
ejecutar. Se sabe que la ejecución está en marcha cuando el icono 
muestra el título ``Running...''.

\begin{figure}[h]
	\centering
	\includegraphics[scale=0.6]{images/user/running.png}
\end{figure}

\subsubsection*{Interactuar con una acción}
Dependiendo del tipo de herramienta, en el diálogo de ejecución se 
podrá obtener la información de salida de la herramienta e introducir 
comandos mediante el campo de texto y el botón \textit{Send}. Se 
podrá terminar la ejecución abortándola (cerrando la ventana) o 
completando la acción, la cual lanzará otra ventana con los 
resultados de la aplicación, que conforman el estado final, los 
parámetros de salida y las excepciones lanzadas, como se ve en la 
siguiente figura.

\begin{figure}[h]
	\centering
	\includegraphics[scale=0.5]{images/user/stopdialog.png}
\end{figure}

\subsubsection*{Liberar las herramientas}
Tanto si se ha terminado de utilizar las herramientas como si se ha 
acabado el tiempo, se ha de desconectar mediante el botón 
\textit{Logout} o cerrando la aplicación.

\subsubsection*{Errores comunes}
Por motivos de seguridad, si ocurriera algún error grave en la 
aplicación RLF\_ Client y el usuario no se pudiera desconectar con 
normalidad, no podrá volverse a conectar hasta que el administrador 
haya dado su aprobación, para ello, póngase en contacto con él e 
indique el motivo de su error.

\clearpage

\section*{RLF\_ Monitor}
Este servicio está disponible para poder comprobar en tiempo real y 
desde cualquier dispositivo el estado de las herramientas. Aunque la 
interfaz esté diseñada especialmente para dispositivos móviles, 
puede ser consultado desde un navegador de un ordenador personal sin 
problemas. Con ello se pretende ahorrar tiempo al usuario para 
informarse si las herramientas que quiere utilizar están en ese 
momento libres.

Para acceder al servicio, se debe ir al a dirección proporcionada por 
el administrador mediante cualquier tipo de navegador de Internet. 
Como por ejemplo:

\begin{verbatim}
http://163.117.150.85:8080/RLF/Monitor/
\end{verbatim}

\begin{figure}[H]
	\centering
	\includegraphics[scale=0.5]{images/user/loginmonitor.png}
\end{figure}

Esa página pedirá el nombre y contraseña del usuario que se le ha 
asignado, una vez esté conectado, podrá obtener información como esta:

\begin{figure}[H]
	\centering
	\includegraphics[scale=0.5]{images/user/monitor.png}
\end{figure}

\subsection*{Estados de una herramienta en el monitor}
La lista de estados que se pueden obtener mediante el servicio del 
monitor es distinta a la que se puede observar con la aplicación 
cliente RLF\_ Client:

\begin{itemize}
\item \textbf{Disponible:} Esta herramienta no está siendo usada por nadie y 
puede reservarse. 
\item \textbf{No disponible:} Esta herramienta está desconectada del sistema.
\item \textbf{En uso:} Algún usuario tiene reservada esta herramienta.
\end{itemize}

\begin{figure}[h]
	\centering
	\includegraphics[scale=1]{images/user/statustoolsmonitor.png}
\end{figure}

\cleardoublepage
