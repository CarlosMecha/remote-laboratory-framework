% Apéndice: Manual de mantenimiento

\capitulo{Guía de mantenimiento}{mantenimiento}{
Este manual está dedicado a los administradores para realizar un 
correcto mantenimiento de la plataforma RLF mientras esta esté en 
ejecución. Además se incluyen algunos errores externos para su 
tratamiento.
}

\section*{Proveedor y monitor}
La única parte del proveedor y monitor que necesita mantenimiento es 
la base de datos global. Aunque la mayoría del tratamiento de datos es 
automático, en la versión \version entregada hay que realizar algunas 
tareas a mano.

\subsection*{Usuarios}
Debido a motivos de seguridad, si un usuario no se desconecta 
correctamente del sistema (en el cliente, en el monitor no ocurre 
esto) su cuenta se queda bloqueada para su revisión. Para permitir 
otra vez el uso de esta cuenta, hay que eliminar el \emph{token} de 
acceso del usuario, es decir, establecer su valor a \emph{NULL}. Se 
puede utilizar un programa de gestión de la base de datos como MySQL 
Browser (descargable en \texttt{http://dev.mysql.com/doc/query-browser/es/index.html}).

\subsection*{\emph{Backups}}
El proceso de duplicar \index{\emph{backup}} la base de datos es necesario 
cada cierto tiempo (a discreción de los administradores) para 
salvaguardar los datos de los usuarios. Realmente, este proceso es 
requerido sólo para estos datos, ya que la forma de desplegar el 
sistema hace que no sea muy complejo volver a añadir los datos. Para 
obtener información de cómo hacer un \emph{backup} en una base de 
datos MySQL consultar \texttt{http://dev.mysql.com/doc/refman/5.1/en/backup-methods.html}.

\subsection*{\emph{Logs}}
Es recomendable que se revisen los \emph{logs} \index{\emph{log}} del 
proveedor y monitor por si ha habido algún problema durante las 
ejecuciones. Se encuentran en la carpeta del mismo nombre y pueden ser 
consultados mediante un visor de texto genérico. También puede ser 
necesario borrarlos cada cierto tiempo para no saturar el disco.

\section*{Laboratorios}
El mantenimiento de los laboratorios se realiza mediante el programa 
\emph{LabConsole} incluido en este proyecto. Los laboratorios tienen 
tres estados principales:

\begin{itemize}
\item \textbf{Iniciado:} Es el estado por defecto de un laboratorio. 
En él se pueden realizar la mayoría de las tareas de mantenimiento, 
como registrar una nueva herramienta y eliminarla, y también armar el 
propio laboratorio. Cuando se desarma el laboratorio o se arranca por 
primera vez se alcanza este estado.
\item \textbf{Armado:} En este estado el laboratorio está preparado 
para escuchar las peticiones de los usuarios y comunicarse con el 
proveedor. No se puede realizar tareas de mantenimiento, sólo obtener 
el estado de cada una de las herramientas en tiempo real.
\item \textbf{Parado:} La ejecución del laboratorio se cierra. Para 
una parada controlada es necesario desarmarlo antes. También se puede 
llegar a este estado con una parada de emergencia.
\end{itemize}

\subsection*{\emph{LabConsole}}
Esta herramienta permite acceder a cualquier laboratorio de la red. 
Para utilizarlo se requiere un nombre de administrador y su 
contraseña. Las acciones que se pueden realizar son las siguientes:

\begin{verbatim}
java -jar LabConsole.jar [-h <IP> -p <Puerto>] -user <Usuario>
                         -pass <Contraseña> <Comando> <Parámetros>
\end{verbatim}

\begin{itemize}
\item \textbf{Armar:} Arma el laboratorio al que se accede. Antes de 
escuchar peticiones, se ejecutarán todos los limpiadores de las 
herramientas.
\begin{verbatim}
<Comando>: arm
<Parámetros>: ninguno
\end{verbatim}
\item \textbf{Desarmar:} Desarma el laboratorio. Es recomendable que 
se compruebe el estado antes de las herramientas, ya que si hay 
usuarios usándolas se desconectarán.
\begin{verbatim}
<Comando>: disarm
<Parámetros>: ninguno
\end{verbatim}
\item \textbf{Parar:} Para por completo el laboratorio. Si está armado 
lanzará un error.
\begin{verbatim}
<Comando>: stop
<Parámetros>: ninguno
\end{verbatim}
\item \textbf{Emergencia:} Para por completo el laboratorio y dejará 
las herramientas en el estado actual. Todos los usuarios conectados 
serán expulsados y no se aceptarán nuevas conexiones. La clave de 
emergencia por defecto es ``Emergency!''.
\begin{verbatim}
<Comando>: emergency
<Parámetros>: <Clave de emergencia>
\end{verbatim}
\item \textbf{Estado:} Obtiene el estado de las herramientas (en 
ejecución o parada) y del propio laboratorio. Esta acción se pude 
realizar cuando el laboratorio está iniciado o armado.
\begin{verbatim}
<Comando>: status
<Parámetros>: ninguno
\end{verbatim}
\item \textbf{Registrar:} Registra una herramienta. Es necesario 
indicar dónde se encuentra el fichero \emph{.xml} de configuración 
(ruta completa local). Cuando se registre se obtendrá el identificador 
único de la herramienta y su clave. Esta acción sólo puede llevarse 
a cabo cuando el laboratorio no está armado. Véase el Manual de Despliegue.
\begin{verbatim}
<Comando>: registry
<Parámetros>: <Ruta local del fichero XML>
\end{verbatim}
\item \textbf{Eliminar:} Elimina una herramienta. Es necesario 
indicar el identificador de la herramienta y su clave. También será 
borrada de la base de datos. Esta acción sólo puede llevarse a cabo 
cuando el laboratorio no está armado.
\begin{verbatim}
<Comando>: drop
<Parámetros>: <ID de la herramienta> <Clave de la herramienta>
\end{verbatim}
\end{itemize}

\subsection*{Limpiadores}
Cada herramienta posee una acción especial para \emph{resetear} el 
\hardware asociado. Esta acción se ejecuta automáticamente cuando se 
arma el laboratorio y cuando ha ocurrido fallo de ejecución. Es 
recomendable que se realice cada día un desarmado y armado de todos 
los laboratorios para que se ejecuten los limpiadores al menos una vez 
cada 24 horas.

\subsection*{\emph{Logs}}
Al igual que el proveedor y el monitor, los laboratorios tienen su 
propio registro de sucesos, que es recomendable observar. También 
se pueden eliminar cada cierto tiempo.

\section*{Herramientas}
No necesitan un mantenimiento concreto, aunque por motivos de 
seguridad, puede ser necesario registrar y eliminar las herramientas 
cada mes o periodo similar y así puedan cambiar de clave.

\subsection*{\emph{Hardware}}
A pesar de que el mantenimiento del \hardware sea automático, el 
administrador debe realizar revisiones periódicas al mismo, 
independientemente del uso que se le ha dado.
