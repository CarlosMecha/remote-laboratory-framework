% Apéndice: Material Entregado

\capitulo{Material entregado}{material}{
Debido a que este producto contiene múltiples módulos y componentes, 
se listan a continuación todo lo se ha entregado junto con este 
documento, y su disposición.
}

\section*{Componentes}
\subsection*{Proveedor \index{proveedor}}
Compone el servicio web de acceso a la plataforma y los \emph{scripts} de la 
base de datos para poder crearla. Este servicio web debe ser 
desplegado en alguna plataforma como Tomcat o JBoss que permita 
aplicaciones Java. Los \emph{scripts} están realizados para el SGBD MySQL.

\subsection*{Monitor \index{monitor}}
Sirve para obtener información en tiempo real de las herramientas 
registradas en el sistema. Está compuesto por un servicio y página 
web con formato para dispositivos móviles o pantallas pequeñas.

\subsection*{Laboratorio \index{laboratorio}}
Módulo de gestión de los distintos servidores locales que contienen 
herramientas. Puede instalarse en cualquier plataforma y toma la 
función de un proceso \emph{demonio} \index{demonio}.

\subsection*{Gestor de laboratorios}
Es la herramienta para poder administrar todos los laboratorios 
ejecutados en la red. Aplicación por consola que se recomienda usar en 
sistemas Unix.

\subsection*{Cliente \index{cliente}}
Permite acceder a la plataforma RLF. Puede ser utilizado en cualquier 
sistema operativo con soporte para Java. Utiliza el servicio web 
propio del proveedor.

\subsection*{Componentes RLF}
Son un conjunto de herramientas que utilizan todos los módulos. 
Dispone de un cifrador \index{cifrador} de datos y la capa de 
comunicación de toda la plataforma mediante sockets \index{socket}.

\subsection*{Bibliotecas \emph{libtool} \index{\emph{libtool}}}
Conjunto de funciones y métodos específicos para conectar la 
herramienta con el laboratorio. Están hechas en varios lenguajes de 
programación.

\subsection*{Herramienta 1: RLF\_ Video}
Emite video y sonido mediante un servidor HTTP. Está destinada para 
sistemas Unix fue desarrollada en Shell Bash. Su capa base es el 
programa/servidor VLC. Es una herramienta multiusuario.

\subsection*{Herramienta 2: RLF\_ Music}
Reproduce canciones seleccionadas por el usuario en el sistema remoto, 
por lo que no pueden ser oidas desde el cliente. Pensado para sistemas 
Unix con el programa MPG123.

\subsection*{Herramienta 3: RLF\_ DummyTool}
Conjunto de pruebas básicas en consola Shell Bash para Unix. Comprueba 
la información del sistema en tiempo real.

\subsection*{Herramienta 4: RLF\_ FreeMem}
Obtiene la memoria en uso de un sistema Windows. Esta herramienta es 
multiusuario y la información también es en tiempo real.

\subsection*{Herramienta 5: RLF\_ Board}
Acciones para utilizar la tarjeta Advantec PCI1711BE. Desarrollado para 
sistemas Windows. Pensado para realizar prácticas de laboratorio de 
electrónica y automática. Permite enviar un pulso de 5 segundos con 
el voltaje introducido, obteniendo la respuesta mediante un servicio 
FTP.

\subsection*{Documentación}
Todo el código entregado está profusamente comentado para ayudar a 
desarrollar nuevos módulos a partir de código antiguo. Además, para 
el código escrito en Java, se ha entregado la documentación del API 
completa (realizada mediante \emph{javadoc}).

\clearpage

\section*{Distribución}
La disposición de directorios entregados es la siguiente:
\begin{verbatim}
|-- bin
|   |-- client
|   |   |-- lib
|   |   |-- log
|   |   `-- res
|   |-- lab
|   |   |-- log
|   |   `-- res
|   |-- labconsole
|   `-- provider
|-- doc
|   |-- api
|   |   |-- index-files
|   |   |-- org
|   |   `-- resources
|   |-- images
|   |   `-- user
|   |-- include
|   |-- lib
|   `-- template
|-- img
|-- lib
|-- projects
|   |-- eclipse
|   |   |-- RLF_Lab
|   |   |-- RLF_LabConsole
|   |   |-- RLF_LabManager
|   |   |-- RLF_Log
|   |   `-- RLF_Utils
|   |-- netbeans
|   |   |-- RLF_Client
|   |   `-- RLF_Provider
|   `-- visual studio
|       |-- LibTool
|       |-- RLF_Board
|       `-- RLF_FreeMem
|-- src
|   `-- sql
`-- tools
    |-- lib
    |   |-- Bash
    |   |-- C
    |   `-- NET
    |-- linux
    |   |-- RLF_DummyTool
    |   |-- RLF_Music
    |   `-- RLF_Video
    `-- windows
        |-- RLF_Board
        `-- RLF_FreeMem
\end{verbatim}

\clearpage

\subsection*{Directorio \emph{bin}}
Contiene todos los archivos ejecutables de los distintos componentes 
organizados por carpetas. Si se desea ejecutar esos archivos binarios 
en otra localización, es necesario copiar todo lo que contienen sus 
carpetas, ya que es necesario para un correcto funcionamiento.

\subsection*{Directorio \emph{doc}}
En este directiorio está toda la documentación aportada para el 
proyecto, desde la descripción de los APIs hasta este propio 
documento, en \emph{latex} con todos los recursos necesarios para la 
maquetación del mismo.

\subsection*{Directorio \emph{img}}
Se agrupan todas las imágenes que se han utilizado para el desarrollo 
del proyecto. Son las usadas en los componentes aunque no son 
necesarias para su ejecución, ya que han sido introducidas en el mismo 
código.

\subsection*{Directorio \emph{lib}}
Aquí están todas las librerías requeridas para la compilación de 
los distintos módulos. Cada proyecto tiene su referencia en este 
directorio. Además, se puede encontrar la propia librería de RLF.

\subsection*{Directorio \emph{projects}}
Se ha considerado que es más sencillo realizar nuevos módulos o 
modificar los entregados mediante los proyectos originales de los 
distintos IDEs, como son Eclipse, NetBeans y Visual Studio. Aquí está 
contenido de forma organizada haciendo referencia al tipo de IDE.

\subsection*{Directorio \emph{src}}
Se encuentran los archivos fuente de los componentes que no tienen 
establecido su propio proyecto. Es el caso de los \emph{scripts} en 
SQL de las distintas bases de datos.

\subsection*{Directorio \emph{tools}}
Contiene todas las herramientas entregadas (sus archivos ejecutables) 
separadas por plataforma y las propias \emph{libtools} en los 
distintos lenguajes. Antes de mover estos archivos consúltese el 
Manual del Desarrollador.

\cleardoublepage
