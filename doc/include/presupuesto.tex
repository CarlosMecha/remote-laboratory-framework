% Apéndice: Presupuesto

\capitulo{Presupuesto}{presupuesto}{
La versatilidad aportada a este proyecto reduce en gran medida 
los gastos necesarios para el despliegue del mismo. Se adjunta a 
continuación un presupuesto aproximado dependiendo de las necesidades 
actuales del lugar de trabajo o enseñanza. Todos los precios aquí 
recogidos datan de Septiembre de 2011.
}

\section*{Costes de desarrollo}
No son necesarios para el despliegue pero cuantifican la cantidad de 
trabajo empleada para la creación de la plataforma, así con su 
diseño y documentación. 

\subsection*{Coste de personal}
Este cálculo se realiza atendiendo a la planificación que se incluyó 
en la tabla \ref{tab:planificacion} del capítulo introductorio. El 
precio por hora está estimado según los sueldos en Septiembre de 2011 
en Madrid:

\begin{table}[H]
\begin{center}
\begin{tabular*}{12cm}{|| p{8.5cm} @{\extracolsep{\fill}} | r ||}
	\hline
	\hline
	Concepto & Coste en \euro\\
	\hline
	\hline
	Ing. Informático por hora & 30,00\\
	Corrector de documentación por hora & 17,00\\
	\hline
	\textbf{TOTAL} & 44.210,00\\
	\hline
	\hline
\end{tabular*}
\end{center}
	\caption{Coste de personal}
	\label{coste:personal}
\end{table}


\subsection*{Costes de material para el desarrollo}
Algunos de estos precios no son por la compra del producto (señalados 
con \# ), si no por el gasto aproximado en el uso del material.

\begin{table}[H]
\begin{center}
\begin{tabular*}{12cm}{|| p{8.5cm} @{\extracolsep{\fill}} | r ||}
	\hline
	\hline
	Concepto & Coste en \euro\\
	\hline
	\hline
	Servidor primario \# & 180,00\\
	Servidor secundario \# & 120,00\\
	Ordenador portátil \# & 120,00\\
	Dispositivo móvil & 169,00\\
	Sistema operativo Windows 7 Professional & 0,00\\
	Sistema operativo Windows XP Professional & 0,00\\
	Sistema operativo Unix & 0,00\\
	Base de datos MySQL & 0,00\\
	Base de datos SQLite & 0,00\\
	Servidor de aplicaciones Tomcat o JBoss & 0,00\\
	Cámara web & 13,50\\
	Micrófono & 6,99\\
	Advantech PCI-1711-BE \# & 60,00\\
	Altavoces doble canal & 29,50\\
	\hline
	\textbf{TOTAL} & 698,99\\
	\hline
	\hline
\end{tabular*}
\end{center}
	\caption{Coste material para el desarrollo}
	\label{coste:matdesarrollo}
\end{table}

Deduciéndose como coste total del desarrollo:

\begin{table}[H]
\begin{center}
\begin{tabular*}{12cm}{|| p{8.5cm} @{\extracolsep{\fill}} | r ||}
	\hline
	\hline
	Concepto & Coste en \euro\\
	\hline
	\hline
	Coste personal & 44.210,00\\
	Coste material & 698,99\\
	\hline
	\textbf{TOTAL} & 44.908,99\\
	\hline
	\hline
\end{tabular*}
\end{center}
	\caption{Coste total de desarrollo}
	\label{coste:totaldesarrollo}
\end{table}

\section*{Costes generales de mantenimiento}
Estos costes están asumidos para el despliegue de la plataforma, 
contando que su distribución se encuentre en la misma red local, 
aunque el número de nodos es indiferente.

\begin{table}[H]
\begin{center}
\begin{tabular*}{12cm}{|| p{8.5cm} @{\extracolsep{\fill}} | r ||}
	\hline
	\hline
	Concepto & Coste anual en \euro\\
	\hline
	\hline
	Línea de alta velocidad empresarial, \emph{Movistar} & 840,00\\
	Suministro eléctrico, \emph{Endesa} & 1.200,00\\
	Seguridad física y de datos, \emph{Movistar} & 600,00\\
	Coste estimado en reparaciones & 400,00\\
	\hline
	\textbf{TOTAL} & 3.040,00\\
	\hline
	\hline
\end{tabular*}
\end{center}
	\caption[Costes generales]{Costes aproximados de servicios 
	requeridos.}
	\label{coste:general}
\end{table}

Debido a que los componentes \hardware tienen más uso que aquellos con 
un limitado tiempo de acceso diario que se pueden encontrar en las 
universidades o puestos de trabajo, se asume un incremento en el coste 
de reparación anual.

\section*{Coste material por componentes}
A pesar de que todos los componentes de RLF pueden estar contenidos en 
la misma máquina, se especifican los costes de cada uno por separado, 
después se incluyen configuraciones estándar con su inversión 
correspondiente.

\subsection*{Proveedor y monitor}
Servidor \index{servidor} central de toda la plataforma, es la máquina con más 
prestaciones de todo el conjunto. Sus sistema es Unix y no requiere de 
ningún programa de pago debeido a su configuración estándar. Puede 
desdoblarse si se requiere para obtener una mayor velocidad y 
potencia, quedando un servidor con mayor capacidad para el proveedor y 
uno con menos para el monitor.

\begin{table}[H]
\begin{center}
\begin{tabular*}{12cm}{|| p{8.5cm} @{\extracolsep{\fill}} | r ||}
	\hline
	\hline
	Concepto & Coste en \euro\\
	\hline
	\hline
	PowerEdge T410 Tower Server, \emph{Dell} & 1.217,00\\
	\{Latitude 13 (Terminal portatil de acceso), \emph{Dell}\} & \{619,00\}\\
	Router (Incluido en el contrato de la línea) & 0,00\\
	Sistema de alimentación ininterrumpida (SAI) & 79,90\\
	Sistema operativo Unix & 0,00\\
	Base de datos MySQL & 0,00\\
	Servidor de aplicaciones Tomcat o JBoss & 0,00\\
	\hline
	\textbf{TOTAL} & 1.296,90\\
	\textbf{TOTAL \{con terminal de acceso\}} & 1.915,90\\
	\hline
	\hline
\end{tabular*}
\end{center}
	\caption[Coste del proveedor y monitor]{Costes aproximados de un 
	proveedor con monitor estándar.}
	\label{coste:proveedor}
\end{table}

Si se desea cobrar por los servicios de esta plataforma, se deberá 
pagar la licencia de MySQL y J2EE a Oracle. La terminal de acceso es 
un ordenador portatil de prestaciones medias con sistema operativo 
Unix. Es opcional y sólo se requiere uno por sistema RLF.

\subsection*{Laboratorio}
Por cada laboratorio en la plataforma se tiene que aplicar este coste. 
Dependiendo del sistema se deberá pagar la licencia de Windows.

\begin{table}[H]
\begin{center}
\begin{tabular*}{12cm}{|| p{8.5cm} @{\extracolsep{\fill}} | r ||}
	\hline
	\hline
	Concepto & Coste en \euro\\
	\hline
	\hline
	PowerEdge T110, \emph{Dell} & 378,00\\
	Sistema operativo Unix & 0,00\\
	\{Sistema operativo Windows 7 Professional\} & \{149,99\}\\
	Base de datos SQLite & 0,00\\
	\hline
	\textbf{TOTAL} & 378,00\\
	\textbf{TOTAL \{con Windows\}} & 527,99\\
	\hline
	\hline
\end{tabular*}
\end{center}
	\caption[Coste por laboratorio]{Costes aproximados por cada 
	laboratorio.}
	\label{coste:laboratorio}
\end{table}

\clearpage

\subsection*{Herramientas}

\begin{table}[H]
\begin{center}
\begin{tabular*}{12cm}{|| p{8.5cm} @{\extracolsep{\fill}} | r ||}
	\hline
	\hline
	Concepto & Coste en \euro\\
	\hline
	\hline
	eLight HD 720p Webcam, \emph{Trust} & 79,00\\
	Advantech PCI-1711-BE & 499,00\\
	Altavoces doble canal & 29,50\\
	\hline
	\textbf{TOTAL} & 607,50\\
	\hline
	\hline
\end{tabular*}
\end{center}
	\caption[Costes de herramientas]{Costes aproximados de las 
	herramientas entregadas.}
	\label{coste:herramientas}
\end{table}

\textbf{NOTA:} Aquí sólo se recogen los gastos de las herramientas 
entregadas junto con este proyecto.

\section*{Ejemplo de un sistema estándar}

Se compone de un proveedor central, un laboratorio de Unix y otro de 
Windows. Todos en diferentes máquinas y con todas las herramientas 
entregadas en uso. No es necesario una terminal de acceso, ya que el 
proveedor cuenta con ella.

\begin{table}[H]
\begin{center}
\begin{tabular*}{12cm}{|| p{8.5cm} @{\extracolsep{\fill}} | r ||}
	\hline
	\hline
	Concepto & Coste en \euro\\
	\hline
	\hline
	Gastos generales (al año) & 3.040,00\\
	Proveedor & 1.296,90\\
	Laboratorio Windows & 527,99\\
	Laboratorio Linux & 378,00\\
	Herramientas & 607,50\\
	\hline
	\textbf{TOTAL} & 5850,39\\
	\hline
	\hline
\end{tabular*}
\end{center}
	\caption[Presupuesto de un sistema estándar]{Presupuesto 
	aproximado de un sistema estándar.}
	\label{coste:sistema}
\end{table}

\cleardoublepage
