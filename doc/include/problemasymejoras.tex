% Problemas y mejoras

\capitulo{Problemas y mejoras}{problemas}{
Se recoge en este capítulo los problemas derivados de la 
implementación de la plataforma RLF así como los trabajos futuros que 
se deberán tener en cuenta para posteriores versiones.
}

\section{Problemas encontrados en el desarrollo}
Se listan a continuación los problemas referentes a la implementación 
y a las tecnologías usadas. Sólo han sido incluidos aquellos más importantes 
que han determinado el desarrollo de la plataforma RLF.

\subsection{La sincronización}
Siendo un problema que arrastran muchas de las plataformas que existen 
actualmente en el mercado, fue arrastrado desde el principio de la 
implementación. Se ha invertido mucho tiempo en conseguir un sistema 
que es asíncrono a partir de muchos componentes síncronos.

Los sistemas síncronos \cite{SistemasOperativos} son aquellos que se 
bloquean a la espera de una comunicación concreta por parte del otro 
interlocutor. En cambio, los sistemas asíncronos, pueden realizar 
otras tareas mientras el otro interlocutor genera la información. Los 
distintos componentes síncronos que se aprecian, como la entrada y 
salida estándar de todas las herramientas, han sido modificados para 
poder permitir no bloquear al resto de componentes.

Esto se ha conseguido con la sustitución de los mecanismos de lectura 
y escritura (de \emph{sockets}, ficheros y teclado) tradicionales, que 
se presentan en la máquina virtual de Java como \emph{streams} 
pertenecientes a las librerías ``java.io''. A partir de la versión 
1.4.2 de Java, Sun Microsystems añadió las librerías ``java.nio'' 
\index{Java!nio} \cite{JavaNIO} que aportaban una nueva forma de ver 
la entrada y salida para la máquina virtual. El conjunto fue llamado 
Java New I/O, en referencia al sistema antiguo, Java I/O. La 
estructura de estas librerías es muy parecida a los sistemas Unix, 
puediendo utilizar funciones no bloqueantes, y algunas herramientas 
muy útiles para no malgastar tiempo de cálculo. Además, incluyen 
\emph{buffers} gestionados por el propio sistema operativo anfitrión 
que dotan de una mayor velocidad a la, de por si lenta, máquina de 
Java.

\begin{figure}[h]
	\centering
	\includegraphics[scale=0.9]{images/io-vs-nio.png}
	\caption[Java.io VS Java.nio]{Comparación de las librerías 
	``java.io'' y su nueva versión ``java.nio''.}
	\label{fig:niovsio}
\end{figure}

En la figura \ref{fig:niovsio} se puede comprobar como con las 
distintas versiones de estos dos conjuntos librerías se mejora la 
velocidad de lectura y escritura. Se realizó el estudio con varios 
ficheros de distinto tamaño (coordenada X) y la velocidad de lectura y 
posterior escritura en KB/s (coordenada Y). Las series corresponden a 
dos experimentos, A y B, con la versión 2.2 de ``java.io'' y la 
versión 2.4 de ``java.nio'' \cite{Javaiovsnio}.

\subsection{La portabilidad de Java}
Aunque es una plataforma que puede ejecutarse en múltiples sistemas, 
Java necesita ``una ayuda'' para poder realizar bien su cometido en 
Linux y Windows por igual. El primer problema que se encuentra es en 
el acceso a archivos del sistema de ficheros. El árbol de rutas es 
distinto para \emph{ext4} \footnote{Sistema de ficheros que las distribuciones 
Linux usan en la actualidad}. que para \emph{ntfs} \footnote{Sistema 
de ficheros moderno para las últimas versiones de Windows, como XP, 
Vista y Windows 7.}. Esto conlleva a que el código de implementación 
debe ser lo suficientemente genérico como para que no surjan problemas 
a la hora de portar los distintos componentes.

Aunque en el desarrollo de RLF no se ha optado en ningún momento por 
la división de código, es decir, incluir un código para cada 
sistema operativo, se han realizado múltiples cambios en el diseño 
para adaptarlo a un único código. Se puede ver a continuación, un 
ejemplo de división de código atendiendo al sistema operativo 
\cite{java2}:

\begin{verbatim}
String osName = System.getProperty("os.name");
if ((osName.equals("Windows NT") 
    || osName.equals("Windows 7")
    || osName.equals("Windows XP")) {
        // Do something...
} else if ((osName.equals("Linux")
    || osName.equals("Mac")) {
        // Do another thing...
} else {
        // Cry.
}
\end{verbatim}

\subsection{\emph{Streaming} de vídeo y Java}
Fueron muchos días los que se intentó, sin éxito, utilizar la 
cámara web de la herramienta RLF\_Video a través de Java. Para ello 
se utilizó el \emph{framework} aportado por Sun Microsystems para el 
control de sistemas multimedia llamado JMF (\emph{Java Media 
Framework}).

A pesar de que el nuevo dueño de las tecnologías Java, Oracle, 
indique en su página que sigue en activo y se está desarrollando 
actualmente aplicaciones con él, la última versión data de 2001, con 
la versión de Java anterior a la 1.4.2 (la primera considerada 
``moderna''). Además, la documentación para el desarrollo ya no es 
accesible desde las páginas oficiales.

Se optó por utilizar el servidor de VLC, incluido en las 
distribuciones Linux y adecuarlo mediante BASH para poder ejecutarlo 
desde la plataforma RLF.

\subsection{La tarjeta PCI-1711-BE}
\index{PCI-1711-BE}
Uno de los elementos más importante que se ha aportado a las 
herramientas presentadas, es la interactividad con un \emph{hardware} 
donde su acceso era en el mismo lugar donde se encontraba. Se utilizó 
una tarjeta que provee de entradas y salidas electrónicas, llamada 
Advantech PCI-1711-BE que disponía de un conjunto de librerías para 
interactuar por medio del \emph{software}, que podían ser utilizadas 
en sistemas Windows y Linux.

Todo intento por utilizar la documentación (escrita en 1996) y las 
herramientas aportadas fue un fracaso hasta que se consiguieron unos 
ejemplos que se podían utilizar en el IDE Visual Studio 2005. Siendo 
aún incompatibles con los sistemas actuales, mediante sustitución de 
código antiguo y de librerías que ya no existen en Windows, se pudo 
adecuar a la plataforma .NET, y con ello, a la plataforma RLF.

Se puede ver a continuación el código original de algunos ejemplos de 
dicha tarjeta:

\begin{verbatim}
/*
 (...)
 * Revision       : 1.00                                           *
 * Date           : 7/1/2003                   Advantech Co., Ltd. *
 (...)
 */
 
 (...)
 
// Estos tipos de datos no son compatibles con .NET
DWORD  dwErrCde;
ULONG  lDevNum;
long   lDriverHandle;
USHORT usChan;
 (...)
 
// Funciones no soportadas por Windows
getch();
 (...)


\end{verbatim}

\clearpage

\section{Próximos pasos}
No cabe duda que la plataforma RLF aquí presentada necesita más 
desarrollo para poder afianzarla como un producto comercial. Estando 
aún en la versión \emph{Prototype}, requiere de determinadas tareas 
para poder ser implantada en entornos de trabajo. Se recopilan a 
continuación los próximos trabajos propuestos:

\begin{itemize}
\item Se pueden aplicar pruebas de estrés a la plataforma mayores de 
las que se incluyen en la sección \ref{sec:pruebasestres}, contando 
con varias decedas de laboratorios, y varios cientos de usuarios.
\item Acoplamiento del sistema de cifrado de comunicaciones con 
túneles SSH que aportarían blindaje a la plataforma, de la misma 
forma que se muestra en la figura \ref{fig:ssh}.
\item Diseño de tareas automatizadas para la base de datos del 
proveedor, como comprobación de nodos de la red o de usuarios con 
estados erróneos.
\item Creación de ``paquetes'' de aplicaciones para alumnos, que 
permitan obtener todo el \emph{software} necesario en un solo 
instalador (como por ejemplo, reproductores de video, clientes FTP, 
etc) pudiendo además reservar las herramientas en conjunto con 
anterioridad.
\end{itemize}

\begin{figure}[H]
	\centering
	\includegraphics[scale=0.7]{images/ssh.png}
	\caption[Túnel SSH]{Ejemplo de la arquitectura de un túnel SSH.}
	\label{fig:ssh}
\end{figure}

\subsection{Próximas versiones de RLF}
Dada la base de RLF Prototype, cabe proponer otras funciones que, con unos 
determinados cambios, pueden llevarse a cabo, aprovechando las 
características principales del mismo. Las líneas de desarrollo 
pueden variarse e incluso crear nuevas, todo depende de las necesidades 
del centro que utilice RLF. Se muestra en la figura \ref{fig:versiones} 
continuación una idea de las próximas versiones y sus cambios.

\begin{figure}[H]
	\centering
	\includegraphics[scale=0.7]{images/versiones.png}
	\caption[Próximas líneas RLF]{Próximas posibles líneas RLF.}
	\label{fig:versiones}
\end{figure}

\begin{description}
\item[RLF.org] Corresponde a la evolución natural del proyecto, con 
mejoras de estabilidad y seguridad, y una fase completa de 
\emph{tests} de estrés. Sería la candidata para salir al 
mercado y adecuarse a varias líneas de trabajo. 
\item[RLF.edu] Línea especializada en entornos educativos, donde se 
pueden establecer configuraciones estándar para la realización de 
prácticas, como por ejemplo, que todos los laboratorios contengan una 
herramienta de vídeo, y que obligue a la interfaz a contener un 
reproductor de \emph{streaming} embebido.
\item[RLF@home] Cambiando el sistema de reserva de herramientas, y 
liberando de carga a los laboratorios, se puede convertir en una 
plataforma para edificios o casas inteligentes. Así se podrían 
gestionar distintos dispositivos desde el cliente.
\item[RLF Science] Al igual que algunos centros de investigación, se 
pueden alquilar por determinado tiempo elementos \emph{hardware} a 
usuarios de Internet. Como por ejemplo, en el observatorio astronómico 
de Chile, se puede utilizar el telescopio si se paga una cuota. 
También puede servir como plataforma para sistemas \emph{grid}, 
proporcionando capacidad de cálculo y de almacenamiento sustituyendo las 
herramientas \emph{hardware} por terminales de acceso a distintos 
nodos de una red de computación.
\end{description}


