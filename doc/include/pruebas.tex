% Pruebas realizadas

\newcounter{prueba}
\newcounter{pfuncion}
\newcounter{perror}
\newcounter{pestres}

\capitulo{Pruebas}{pruebas}{
Las pruebas de verificación del sistema implementado es una parte 
importante del desarrollo \emph{software}. En este capítulo se 
incluyen las más destacadas que validan el conjunto de la plataforma 
RLF.
}

Las pruebas que se muestran a continuación han sido seleccionadas del 
conjunto global que se realizaron en la implementación del proyecto. 
Debido al gran número de estas no se ha podido incluir todas.

\section{Configuraciones}
Se han establecido un conjunto de configuraciones de la arquitectura 
para las distintas pruebas, con lo que se consigue verificar que 
independientemente de la estructura del sistema, los resultados sean 
los mismos. Estas configuraciones tienen un nombre asociado que se 
especificará en cada una de las pruebas. A continuación se muestra 
cada configuración, con el número de componentes que intervienen:

\begin{table}[H]
\begin{center}
\begin{tabular}{|| l | c | c | c | p{10cm} ||}
	\hline
	\hline
	Nombre & L & CE & CW & Disposición\\
	\hline
	\hline
	Básica & 1 & 2 & 1 & Todos los componentes en la misma máquina, 
	excepto un CE y el CW que está en otra máquina en red.\\
	\hline
	Común & 2 & 1 & 1 & El P y un L se disponen en 
	una máquina, el otro L en un sistema Windows en red 
	local con el CE. El CW en un \emph{smartphone} con 3G.\\
	\hline
	Remota & 2 & 2 & 2 & Cada componente está en una máquina distinta, 
	pero el P y los L están conectados en red local, mientras que los 
	CE y CW se conectan mediante Internet.\\
	\hline
	Educativa & 5 & 2 & 1 & P se encuentra en una red externa, los L 
	están conectados en red local, y los CE y CW en otra red externa. Algunas 
	herramientas han sido duplicadas.\\
	\hline
	Industrial & 5 & 5 & 2 & Cada componente se encuentra en un nodo 
	distinto, distribuidos entre redes locales e Internet. Algunas 
	herramientas han sido duplicadas.\\
	\hline
	\hline
\end{tabular}
\end{center}
	\caption{Configuraciones de las pruebas.}
	\label{tab:configuraciones}
\end{table}

\textbf{NOTA 1:} Los \emph{laboratorios} (L) son numerados secuencialmente 
de forma \emph{1}, \emph{2}, \emph{3}\ldots Los \emph{clientes de 
escritorio} (CE) se corresponden con el alfabeto en mayúsculas: \emph{A}, 
\emph{B}, \emph{C}\ldots Y por último, los \emph{clientes web} (CW) 
corresponden con el alfabeto en minúsculas: \emph{a}, \emph{b}, 
\emph{c}\ldots

\textbf{NOTA 2:} Todas las configuraciones se componen de un solo 
proveedor y monitor (P), y de las cinco herramientas entregadas con el 
proyecto, además de un número concreto de laboratorio, clientes de 
escritorio y clientes web.

\section{Pruebas de verificación de funcionalidad}

Estas pruebas han sido concebidas para verificar todas las funciones y 
opciones que aporta la plataforma RLF. Han sido realizadas en un 
entorno controlado y comprobando que la respuesta del sistema a 
determinadas acciones era la esperada.

Durante el desarrollo del proyecto, estas pruebas se repitieron 
obteniendo diferentes errores que fueron corregidos hasta obtener los 
resultados esperados.

\begin{table}[H]
\begin{center}
	\stepcounter{prueba}
	\stepcounter{pfuncion}
\begin{tabular}{|| m{1cm} m{2cm} | m{2.5cm} m{2.5cm} | l @{\extracolsep{\fill}} c ||}
	\hline
	\hline
	\multicolumn{6}{|| c ||}{\textsc{Prueba \arabic{prueba}}}\\
	\hline
	\textbf{ID} & PF-\Alph{pfuncion} &
	\textbf{Configuración} & Básica & 
	\textbf{Resultado} & Válido\\
	\hline
	\hline
	\multicolumn{6}{|| c ||}{\textbf{Descripción}}\\
	\hline
	\multicolumn{6}{|| p{14cm} ||}{
	Se accede desde la red externa mediante el CE para reservar todos 
	las herramientas disponibles, ejecutando a la vez dos de ellas. 
	Después se comprueban el estado de las herramientas mediante el 
	CW, donde todas están ocupadas.
	}\\
	\hline
	\multicolumn{6}{|| c ||}{\textbf{Acciones}}\\
	\hline
	\multicolumn{6}{|| p{14cm} ||}{
		\begin{enumerate}
		\item Reserva de todas las herramientas.
		\item Ejecución de dos de ellas.
		\item Comprobación por la web.
		\item Obtención de las herramientas ocupadas.
		\end{enumerate}
	}\\
	\hline
	\hline
\end{tabular}
\end{center}
\end{table}

\begin{table}[H]
\begin{center}
	\stepcounter{prueba}
	\stepcounter{pfuncion}
\begin{tabular}{|| m{1cm} m{2cm} | m{2.5cm} m{2.5cm} | l @{\extracolsep{\fill}} c ||}
	\hline
	\hline
	\multicolumn{6}{|| c ||}{\textsc{Prueba \arabic{prueba}}}\\
	\hline
	\textbf{ID} & PF-\Alph{pfuncion} &
	\textbf{Configuración} & Común & 
	\textbf{Resultado} & Válido\\
	\hline
	\hline
	\multicolumn{6}{|| c ||}{\textbf{Descripción}}\\
	\hline
	\multicolumn{6}{|| p{14cm} ||}{
	Se comprueba ejecución de herramientas en diferentes L desde el 
	mismo CE.
	}\\
	\hline
	\multicolumn{6}{|| c ||}{\textbf{Acciones}}\\
	\hline
	\multicolumn{6}{|| p{14cm} ||}{
		\begin{enumerate}
		\item Reserva de una herramienta de cada laboratorio.
		\item Ejecución de las acciones de las herramientas.
		\item Desconexión.
		\end{enumerate}
	}\\
	\hline
	\hline
\end{tabular}
\end{center}
\end{table}

\begin{table}[H]
\begin{center}
	\stepcounter{prueba}
	\stepcounter{pfuncion}
\begin{tabular}{|| m{1cm} m{2cm} | m{2.5cm} m{2.5cm} | l @{\extracolsep{\fill}} c ||}
	\hline
	\hline
	\multicolumn{6}{|| c ||}{\textsc{Prueba \arabic{prueba}}}\\
	\hline
	\textbf{ID} & PF-\Alph{pfuncion} &
	\textbf{Configuración} & Básica & 
	\textbf{Resultado} & Válido\\
	\hline
	\hline
	\multicolumn{6}{|| c ||}{\textbf{Descripción}}\\
	\hline
	\multicolumn{6}{|| p{14cm} ||}{
	El objetivo es la comprobación de los limpiadores de cada 
	herramienta. Para ello es requerido un cierre forzado de una 
	acción.
	}\\
	\hline
	\multicolumn{6}{|| c ||}{\textbf{Acciones}}\\
	\hline
	\multicolumn{6}{|| p{14cm} ||}{
		\begin{enumerate}
		\item Reserva de una herramienta.
		\item Ejecución de una acción.
		\item Forzado de cierre desde el CE.
		\item Comprobación del \emph{log} del laboratorio responsable.
		\item El limpiador se ejecuta a los pocos segundos de cerrar 
		la acción.
		\end{enumerate}
	}\\
	\hline
	\hline
\end{tabular}
\end{center}
\end{table}

\begin{table}[H]
\begin{center}
	\stepcounter{prueba}
	\stepcounter{pfuncion}
\begin{tabular}{|| m{1cm} m{2cm} | m{2.5cm} m{2.5cm} | l @{\extracolsep{\fill}} c ||}
	\hline
	\hline
	\multicolumn{6}{|| c ||}{\textsc{Prueba \arabic{prueba}}}\\
	\hline
	\textbf{ID} & PF-\Alph{pfuncion} &
	\textbf{Configuración} & Básica & 
	\textbf{Resultado} & Válido\\
	\hline
	\hline
	\multicolumn{6}{|| c ||}{\textbf{Descripción}}\\
	\hline
	\multicolumn{6}{|| p{14cm} ||}{
	Se verifica que las herramientas de datos se ejecuten por varios 
	usuarios y que sólo se cierren cuando todos se han desconectado.
	}\\
	\hline
	\multicolumn{6}{|| c ||}{\textbf{Acciones}}\\
	\hline
	\multicolumn{6}{|| p{14cm} ||}{
		\begin{enumerate}
		\item Reserva de una herramienta con el CE \emph{A}.
		\item Reserva de la misma herramienta con el CE \emph{B}.
		\item El CE \emph{B} ejecuta la acción de la herramienta.
		\item Al cabo de un tiempo, el CE a también.
		\item El CE \emph{B} cierra la ejecución y se desconecta.
		\item Por último, CE a para la acción.
		\item Se comprueba el \emph{log} de L y se verifica que la 
		acción se ha detenido correctamente cuando el CE \emph{A} se 
		desconectó.
		\end{enumerate}
	}\\
	\hline
	\hline
\end{tabular}
\end{center}
\end{table}

\begin{table}[H]
\begin{center}
	\stepcounter{prueba}
	\stepcounter{pfuncion}
\begin{tabular}{|| m{1cm} m{2cm} | m{2.5cm} m{2.5cm} | l @{\extracolsep{\fill}} c ||}
	\hline
	\hline
	\multicolumn{6}{|| c ||}{\textsc{Prueba \arabic{prueba}}}\\
	\hline
	\textbf{ID} & PF-\Alph{pfuncion} &
	\textbf{Configuración} & Básica & 
	\textbf{Resultado} & Válido\\
	\hline
	\hline
	\multicolumn{6}{|| c ||}{\textbf{Descripción}}\\
	\hline
	\multicolumn{6}{|| p{14cm} ||}{
	Comprobación de la liberación de herramientas una vez que un 
	usuario se ha desconectado, y su posterior uso por parte de otro 
	usuario.
	}\\
	\hline
	\multicolumn{6}{|| c ||}{\textbf{Acciones}}\\
	\hline
	\multicolumn{6}{|| p{14cm} ||}{
		\begin{enumerate}
		\item Reserva de un conjunto de herramientas con el CE \emph{A}.
		\item Conexión del CE \emph{B}.
		\item El CE \emph{B} comprueba el estado de las herramientas en uso 
		por CE \emph{A}.
		\item CE \emph{A} cierra algunas acciones forzándolas y otras con la 
		finalización normal, después se desconecta.
		\item El CE \emph{B} reserva las herramientas que CE a ha liberado.
		\item Ejecuta las acciones de esas herramientas.
		\item Debe esperar para el inicio de algunas acciones, 
		correspondientes con las que se ha forzado su cierre por CE 
		\emph{A}, ya que se están ejecutando sus limpiadores.
		\item Se comprueba el estado de las herramientas en el CW.
		\item CE \emph{B} libera sus herramientas después de utilizarlas.
		\item Se vuelve a comprobar el estado de las herramientas 
		mediante el CW y se verifica que todas están libres.
		\end{enumerate}
	}\\
	\hline
	\hline
\end{tabular}
\end{center}
\end{table}

\begin{table}[H]
\begin{center}
	\stepcounter{prueba}
	\stepcounter{pfuncion}
\begin{tabular}{|| m{1cm} m{2cm} | m{2.5cm} m{2.5cm} | l @{\extracolsep{\fill}} c ||}
	\hline
	\hline
	\multicolumn{6}{|| c ||}{\textsc{Prueba \arabic{prueba}}}\\
	\hline
	\textbf{ID} & PF-\Alph{pfuncion} &
	\textbf{Configuración} & Básica & 
	\textbf{Resultado} & Válido\\
	\hline
	\hline
	\multicolumn{6}{|| c ||}{\textbf{Descripción}}\\
	\hline
	\multicolumn{6}{|| p{14cm} ||}{
	Verificación de los roles establecidos para los distintos usuarios 
	y las herramientas que pueden utilizar.
	}\\
	\hline
	\multicolumn{6}{|| c ||}{\textbf{Acciones}}\\
	\hline
	\multicolumn{6}{|| p{14cm} ||}{
		\begin{enumerate}
		\item Se registran las herramientas con distintos roles.
		\item Se crea un usuario con un rol específico.
		\item El usuario, mediante el CE y el CW comprueba que sólo 
		tiene acceso a las herramientas con su rol o inferior.
		\end{enumerate}
	}\\
	\hline
	\hline
\end{tabular}
\end{center}
\end{table}

\begin{table}[H]
\begin{center}
	\stepcounter{prueba}
	\stepcounter{pfuncion}
\begin{tabular}{|| m{1cm} m{2cm} | m{2.5cm} m{2.5cm} | l @{\extracolsep{\fill}} c ||}
	\hline
	\hline
	\multicolumn{6}{|| c ||}{\textsc{Prueba \arabic{prueba}}}\\
	\hline
	\textbf{ID} & PF-\Alph{pfuncion} &
	\textbf{Configuración} & Básica & 
	\textbf{Resultado} & Válido\\
	\hline
	\hline
	\multicolumn{6}{|| c ||}{\textbf{Descripción}}\\
	\hline
	\multicolumn{6}{|| p{14cm} ||}{
	Se pretende comprobar la correcta liberación de herramientas 
	después de que el usuario haya sido expulsado del sistema por 
	exceso de tiempo.
	}\\
	\hline
	\multicolumn{6}{|| c ||}{\textbf{Acciones}}\\
	\hline
	\multicolumn{6}{|| p{14cm} ||}{
		\begin{enumerate}
		\item El CE reserva varias herramientas.
		\item Ejecuta una acción hasta que se le acaba el tiempo.
		\item El sistema expulsa al usuario, cerrando las acciones en 
		ejecución y liberando las herramientas.
		\item Por último, se ejecutan las acciones.
		\end{enumerate}
	}\\
	\hline
	\hline
\end{tabular}
\end{center}
\end{table}

\begin{table}[H]
\begin{center}
	\stepcounter{prueba}
	\stepcounter{pfuncion}
\begin{tabular}{|| m{1cm} m{2cm} | m{2.5cm} m{2.5cm} | l @{\extracolsep{\fill}} c ||}
	\hline
	\hline
	\multicolumn{6}{|| c ||}{\textsc{Prueba \arabic{prueba}}}\\
	\hline
	\textbf{ID} & PF-\Alph{pfuncion} &
	\textbf{Configuración} & Común & 
	\textbf{Resultado} & Válido\\
	\hline
	\hline
	\multicolumn{6}{|| c ||}{\textbf{Descripción}}\\
	\hline
	\multicolumn{6}{|| p{14cm} ||}{
	Esta prueba fue diseñada para utilizar los servicios externos de 
	cada una de las herramientas.
	}\\
	\hline
	\multicolumn{6}{|| c ||}{\textbf{Acciones}}\\
	\hline
	\multicolumn{6}{|| p{14cm} ||}{
		\begin{enumerate}
		\item Los CE \emph{A} y \emph{B} reservan varias herramientas, 
		una de ellas de datos para acceder a su servicio externo.
		\item Ambos ejecutan las herramientas utilizando a la vez 
		programas externos para el manejo de los servicios.
		\item Todos los servicios y las acciones en ejecución 
		responden correctamente, tanto en el envío de datos como en la 
		recepción.
		\end{enumerate}
	}\\
	\hline
	\hline
\end{tabular}
\end{center}
\end{table}

\begin{table}[H]
\begin{center}
	\stepcounter{prueba}
	\stepcounter{pfuncion}
\begin{tabular}{|| m{1cm} m{2cm} | m{2.5cm} m{2.5cm} | l @{\extracolsep{\fill}} c ||}
	\hline
	\hline
	\multicolumn{6}{|| c ||}{\textsc{Prueba \arabic{prueba}}}\\
	\hline
	\textbf{ID} & PF-\Alph{pfuncion} &
	\textbf{Configuración} & Común & 
	\textbf{Resultado} & Válido\\
	\hline
	\hline
	\multicolumn{6}{|| c ||}{\textbf{Descripción}}\\
	\hline
	\multicolumn{6}{|| p{14cm} ||}{
	Prueba la capacidad de insertar herramientas \emph{en 
	caliente}, es decir, mientras el sistema está en pleno 
	funcionamiento, atendidendo peticiones de los usuarios.
	}\\
	\hline
	\multicolumn{6}{|| c ||}{\textbf{Acciones}}\\
	\hline
	\multicolumn{6}{|| p{14cm} ||}{
		\begin{enumerate}
		\item El CE \emph{A} reserva algunas herramientas.
		\item El CE \emph{B} se conecta al sistema pero aún no reserva nada.
		\item Se inserta en el L \emph{1} una herramienta y se arma.
		\item El CE \emph{B} puede reservar desde ese momento la herramienta.
		\end{enumerate}
	}\\
	\hline
	\hline
\end{tabular}
\end{center}
\end{table}

\begin{table}[H]
\begin{center}
	\stepcounter{prueba}
	\stepcounter{pfuncion}
\begin{tabular}{|| m{1cm} m{2cm} | m{2.5cm} m{2.5cm} | l @{\extracolsep{\fill}} c ||}
	\hline
	\hline
	\multicolumn{6}{|| c ||}{\textsc{Prueba \arabic{prueba}}}\\
	\hline
	\textbf{ID} & PF-\Alph{pfuncion} &
	\textbf{Configuración} & Remoto & 
	\textbf{Resultado} & Válido\\
	\hline
	\hline
	\multicolumn{6}{|| c ||}{\textbf{Descripción}}\\
	\hline
	\multicolumn{6}{|| p{14cm} ||}{
	La última prueba de funciones corresponde a la respuesta de un L 
	cuando está sobrecargado de acciones en ejecución.
	}\\
	\hline
	\multicolumn{6}{|| c ||}{\textbf{Acciones}}\\
	\hline
	\multicolumn{6}{|| p{14cm} ||}{
		\begin{enumerate}
		\item El L \emph{1} se configura para que sólo acepte 4 
		procesos en ejecución.
		\item Los CE \emph{A} y \emph{B} reservan varias herramientas 
		que pertenecen a L \emph{1} y ejecutan todas las acciones 
		correspondientes.
		\item Las ejecuciones se ven bloqueadas con el estado de 
		espera hasta que alguna con estado de ejecución termine.
		\item Las peticiones de ejecución se ejecutan con el orden de 
		llegada.
		\end{enumerate}
	}\\
	\hline
	\hline
\end{tabular}
\end{center}
\end{table}

\section{Pruebas de recuperación de errores}
Las siguientes pruebas comprobaron que bajo ciertas situaciones de 
error o incontroladas, el sistema respondía tal y como corresponde a 
la recuperación de errores implementada.

Muchos de los errores que pueden surgir en una plataforma RLF son 
externos a la misma, por lo que su control se resume en una parada 
controlada del componente o componentes afectados.

\begin{table}[H]
\begin{center}
	\stepcounter{prueba}
	\stepcounter{perror}
\begin{tabular}{|| m{1cm} m{2cm} | m{2.5cm} m{2.5cm} | l @{\extracolsep{\fill}} c ||}
	\hline
	\hline
	\multicolumn{6}{|| c ||}{\textsc{Prueba \arabic{prueba}}}\\
	\hline
	\textbf{ID} & PE-\Alph{perror} &
	\textbf{Configuración} & Remoto & 
	\textbf{Resultado} & Válido\\
	\hline
	\hline
	\multicolumn{6}{|| c ||}{\textbf{Descripción del error}}\\
	\hline
	\multicolumn{6}{|| p{14cm} ||}{
	La herramienta en ejecución sufre un fallo bloqueante.
	}\\
	\hline
	\multicolumn{6}{|| c ||}{\textbf{Respuesta}}\\
	\hline
	\multicolumn{6}{|| p{14cm} ||}{
		\begin{enumerate}
		\item El L encargado de la herramienta espera hasta que se 
		agote el tiempo de la acción. Dado ese caso, se considera un 
		error bloqueante.
		\item L establece el estado de error para esa acción, para por 
		completo la ejecución de la acción y añade una petición para 
		ejecutar el limpiador.
		\item El CE recibe la notificación de que la acción actual ha 
		sido parada por un error surgido.
		\item Se muestra las excepciones enviadas por la herramienta y 
		el estado final de la misma.
		\end{enumerate}
	}\\
	\hline
	\hline
\end{tabular}
\end{center}
\end{table}

\begin{table}[H]
\begin{center}
	\stepcounter{prueba}
	\stepcounter{perror}
\begin{tabular}{|| m{1cm} m{2cm} | m{2.5cm} m{2.5cm} | l @{\extracolsep{\fill}} c ||}
	\hline
	\hline
	\multicolumn{6}{|| c ||}{\textsc{Prueba \arabic{prueba}}}\\
	\hline
	\textbf{ID} & PE-\Alph{perror} &
	\textbf{Configuración} & Remoto & 
	\textbf{Resultado} & Válido\\
	\hline
	\hline
	\multicolumn{6}{|| c ||}{\textbf{Descripción del error}}\\
	\hline
	\multicolumn{6}{|| p{14cm} ||}{
	Uno de los laboratorios es desarmado por el administrador o sufre 
	una parada por un error.
	}\\
	\hline
	\multicolumn{6}{|| c ||}{\textbf{Respuesta}}\\
	\hline
	\multicolumn{6}{|| p{14cm} ||}{
		\begin{enumerate}
		\item El CE recibe el mensaje de error y comprueba las 
		herramientas reservadas de ese laboratorio.
		\item Se muestra el diálogo de aviso por cierre inesperado de 
		L.
		\end{enumerate}
	}\\
	\hline
	\hline
\end{tabular}
\end{center}
\end{table}

\begin{table}[H]
\begin{center}
	\stepcounter{prueba}
	\stepcounter{perror}
\begin{tabular}{|| m{1cm} m{2cm} | m{2.5cm} m{2.5cm} | l @{\extracolsep{\fill}} c ||}
	\hline
	\hline
	\multicolumn{6}{|| c ||}{\textsc{Prueba \arabic{prueba}}}\\
	\hline
	\textbf{ID} & PE-\Alph{perror} &
	\textbf{Configuración} & Remoto & 
	\textbf{Resultado} & Válido\\
	\hline
	\hline
	\multicolumn{6}{|| c ||}{\textbf{Descripción del error}}\\
	\hline
	\multicolumn{6}{|| p{14cm} ||}{
	El P sufre un error o su nodo es desconectado de la red.
	}\\
	\hline
	\multicolumn{6}{|| c ||}{\textbf{Respuesta}}\\
	\hline
	\multicolumn{6}{|| p{14cm} ||}{
		\begin{enumerate}
		\item Cuando los CE o CW se intentan conectar al P, o 
		realizar una acción si ya lo estaban, se muestra el diálogo de 
		problema con el servidor central.
		\end{enumerate}
	}\\
	\hline
	\hline
\end{tabular}
\end{center}
\end{table}

\begin{table}[H]
\begin{center}
	\stepcounter{prueba}
	\stepcounter{perror}
\begin{tabular}{|| m{1cm} m{2cm} | m{2.5cm} m{2.5cm} | l @{\extracolsep{\fill}} c ||}
	\hline
	\hline
	\multicolumn{6}{|| c ||}{\textsc{Prueba \arabic{prueba}}}\\
	\hline
	\textbf{ID} & PE-\Alph{perror} &
	\textbf{Configuración} & Remoto & 
	\textbf{Resultado} & Válido\\
	\hline
	\hline
	\multicolumn{6}{|| c ||}{\textbf{Descripción del error}}\\
	\hline
	\multicolumn{6}{|| p{14cm} ||}{
	El CE sufre un error o una desconexión del sistema.
	}\\
	\hline
	\multicolumn{6}{|| c ||}{\textbf{Respuesta}}\\
	\hline
	\multicolumn{6}{|| p{14cm} ||}{
		\begin{enumerate}
		\item El CE no consigue desconectarse del sistema de manera 
		normal.
		\item El P bloquea la cuenta de usuario hasta que se ponga en 
		contacto con el administrador para que la desbloquee.
		\end{enumerate}
	}\\
	\hline
	\hline
\end{tabular}
\end{center}
\end{table}

\begin{table}[H]
\begin{center}
	\stepcounter{prueba}
	\stepcounter{perror}
\begin{tabular}{|| m{1cm} m{2cm} | m{2.5cm} m{2.5cm} | l @{\extracolsep{\fill}} c ||}
	\hline
	\hline
	\multicolumn{6}{|| c ||}{\textsc{Prueba \arabic{prueba}}}\\
	\hline
	\textbf{ID} & PE-\Alph{perror} &
	\textbf{Configuración} & Remoto & 
	\textbf{Resultado} & Válido\\
	\hline
	\hline
	\multicolumn{6}{|| c ||}{\textbf{Descripción del error}}\\
	\hline
	\multicolumn{6}{|| p{14cm} ||}{
	El CW sufre un error o una desconexión del sistema.
	}\\
	\hline
	\multicolumn{6}{|| c ||}{\textbf{Respuesta}}\\
	\hline
	\multicolumn{6}{|| p{14cm} ||}{
		\begin{enumerate}
		\item El CW no consigue desconectarse del monitor de manera 
		normal.
		\item P no bloquea la cuenta de usuario, por lo que se puede 
		volver a conectar desde un CW o un CE.
		\end{enumerate}
	}\\
	\hline
	\hline
\end{tabular}
\end{center}
\end{table}


\begin{figure}
	\centering
	\includegraphics[scale=0.3]{images/mac.png}
	\caption[RLF en Mac]{En estas pruebas también se han incluido 
	nodos con el sistema Macintosh.}
	\label{fig:mac}
\end{figure}

\section{Pruebas de estrés}
\label{sec:pruebasestres}
Por último, estas pruebas están recogidas para verificar el correcto 
funcionamiento en ambientes con gran carga o baja potencia. El número 
de componentes y clientes es aumentado.

\begin{table}[H]
\begin{center}
	\stepcounter{prueba}
	\stepcounter{pestres}
\begin{tabular}{|| m{1cm} m{2cm} | m{2.5cm} m{2.5cm} | l @{\extracolsep{\fill}} c ||}
	\hline
	\hline
	\multicolumn{6}{|| c ||}{\textsc{Prueba \arabic{prueba}}}\\
	\hline
	\textbf{ID} & PX-\Alph{pestres} &
	\textbf{Configuración} & Básico & 
	\textbf{Resultado} & Válido\\
	\hline
	\hline
	\multicolumn{6}{|| c ||}{\textbf{Descripción del conjunto}}\\
	\hline
	\multicolumn{6}{|| p{14cm} ||}{
	Todos los componentes se encuentran en un \emph{netbook}, con un 
	procesador Atom 450 y 1 GB de memoria RAM. Acceden dos CE a 
	él, uno de ellos ejecuta la herramienta de vídeo por 
	\emph{streaming}. Acceden otros dos CW mediante dispositivos 
	\emph{Android}.
	}\\
	\hline
	\multicolumn{6}{|| c ||}{\textbf{Respuesta}}\\
	\hline
	\multicolumn{6}{|| p{14cm} ||}{
	El sistema responde correctamente ocupando el 95\% del procesador 
	y 150 MBs de RAM. El vídeo se recibe de manera fluida y la 
	interfaz del CE \emph{A} funciona correctamente.
	}\\
	\hline
	\hline
\end{tabular}
\end{center}
\end{table}

\begin{table}[H]
\begin{center}
	\stepcounter{prueba}
	\stepcounter{pestres}
\begin{tabular}{|| m{1cm} m{2cm} | m{2.5cm} m{2.5cm} | l @{\extracolsep{\fill}} c ||}
	\hline
	\hline
	\multicolumn{6}{|| c ||}{\textsc{Prueba \arabic{prueba}}}\\
	\hline
	\textbf{ID} & PX-\Alph{pestres} &
	\textbf{Configuración} & Remoto & 
	\textbf{Resultado} & No válido\\
	\hline
	\hline
	\multicolumn{6}{|| c ||}{\textbf{Descripción del conjunto}}\\
	\hline
	\multicolumn{6}{|| p{14cm} ||}{
	En un entorno donde las conexiones entre las redes son de muy baja 
	velocidad, se prueba el acceso a las diferentes herramientas. El P 
	y L \emph{1} están conectados a la red mediante \emph{tethering} con 
	un teléfono móvil con acceso 3G (alrededor de 200 kB/s de bajada y 
	64 kB/s de subida en Madrid \cite{tresg}), así como los dos CE y 
	otro CW conectados a redes normales.
	}\\
	\hline
	\multicolumn{6}{|| c ||}{\textbf{Respuesta}}\\
	\hline
	\multicolumn{6}{|| p{14cm} ||}{
	Los accesos al proveedor se ven ralentizados. Cuando se ejecuta 
	una acción, a pesar de que la interfaz no se bloquea, el flujo de 
	datos es demasiado lento como para poder tener la continuidad 
	necesaria para un correcto funcionamiento. 
	}\\
	\hline
	\hline
\end{tabular}
\end{center}
\end{table}

\begin{table}[H]
\begin{center}
	\stepcounter{prueba}
	\stepcounter{pestres}
\begin{tabular}{|| m{1cm} m{2cm} | m{2.5cm} m{2.5cm} | l @{\extracolsep{\fill}} c ||}
	\hline
	\hline
	\multicolumn{6}{|| c ||}{\textsc{Prueba \arabic{prueba}}}\\
	\hline
	\textbf{ID} & PX-\Alph{pestres} &
	\textbf{Configuración} & Educativa & 
	\textbf{Resultado} & Válido\\
	\hline
	\hline
	\multicolumn{6}{|| c ||}{\textbf{Descripción del conjunto}}\\
	\hline
	\multicolumn{6}{|| p{14cm} ||}{
	Se accede desde 15 terminales en los laboratorios 
	informáticos de la universidad hacia el sistema que se encuentra 
	en el taller de Automática. Las redes de conexión son las propias 
	de la universidad.
	}\\
	\hline
	\multicolumn{6}{|| c ||}{\textbf{Respuesta}}\\
	\hline
	\multicolumn{6}{|| p{14cm} ||}{
	La plataforma RLF funciona sin ningún retardo.
	}\\
	\hline
	\hline
\end{tabular}
\end{center}
\end{table}

\begin{table}[H]
\begin{center}
	\stepcounter{prueba}
	\stepcounter{pestres}
\begin{tabular}{|| m{1cm} m{2cm} | m{2.5cm} m{2.5cm} | l @{\extracolsep{\fill}} c ||}
	\hline
	\hline
	\multicolumn{6}{|| c ||}{\textsc{Prueba \arabic{prueba}}}\\
	\hline
	\textbf{ID} & PX-\Alph{pestres} &
	\textbf{Configuración} & Industrial & 
	\textbf{Resultado} & Válido\\
	\hline
	\hline
	\multicolumn{6}{|| c ||}{\textbf{Descripción del conjunto}}\\
	\hline
	\multicolumn{6}{|| p{14cm} ||}{
	Los componentes se encuentran distribuidos en distintas 
	localidades separadas por varios kilómetros de distancia. Además,
	uno de los laboratorios y un CE se situaron en Osaka, Japón.
	}\\
	\hline
	\multicolumn{6}{|| c ||}{\textbf{Respuesta}}\\
	\hline
	\multicolumn{6}{|| p{14cm} ||}{
	La plataforma RLF funciona sin ningún retardo.
	}\\
	\hline
	\hline
\end{tabular}
\end{center}
\end{table}

Como se ha comprobado en las numerosas pruebas, la plataforma RLF 
\emph{Prototype 1} ha superado los objetivos de estabilidad y 
seguridad. Por supuesto, para obtener una versión completa para el 
mercado, ha de someterse a pruebas de estrés mucho más severas. Al 
estar tratando con un sistema crítico donde los errores pueden suponer 
un gasto en reparaciones bastante importante, es necesario un control 
de calidad exhaustivo y de gran precisión.

\cleardoublepage

