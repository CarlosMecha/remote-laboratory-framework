% Requisitos del software

\newcounter{rfun}
\newcounter{rnof}

\subsection{Requisitos funcionales}

\reqfuncional{General: Aplicaciones}
{RUC-1, RUC-9}{Alta}{Esencial}
{
El \emph{hardware} será controlado completamente por aplicaciones 
que formarán parte de las herramientas y serán ejecutadas mediante 
peticiones de las aplicaciones cliente.
}

\reqfuncional{\emph{Framework}: Desarrollo versátil}
{RUC-4}{Alta}{Esencial}
{
El \emph{framework} de diseño de herramientas no impondrá 
restricciones en cuanto a lenguaje de programación, plataforma ni 
estructura del código.
}

\reqfuncional{\emph{Framework}: Desarrollo cómodo}
{RUC-41}{Alta}{Esencial}
{
Se permitirán utilizar las funciones del sistema de entrada y salida 
por teclado para la comunicación de las herramientas con el cliente. 
Aunque la información que pasa a través de ellas sea enviada por la red.
}

\reqfuncional{Comunicaciones: Modelo TCP/IP}
{RUC-5, RUC-24}{Alta}{Esencial}
{
Todas las comunicaciones serán basadas en el modelo estándar TCP/IP y 
utilizarán un protocolo especialmente diseñado para RLF. Cada nodo 
contendrá una IP de acceso, y la plataforma se identificará con la IP 
del proveedor.
}

\reqfuncional{Comunicaciones: Dirección de la plataforma}
{RUC-6, RUC-7}{Alta}{Esencial}
{
La plataforma se identificará con la IP del proveedor, que será por 
la que se podrá acceder.
}

\reqfuncional{Comunicaciones: Protocolo de comunicaciones}
{RUC-5, RUC-19}{Media}{Esencial}
{
El protocolo de comunicaciones será utilizado por todos los módulos y 
no variará su implementación.
}

\reqfuncional{Comunicaciones: Protocolo síncrono}
{RUC-24}{Media}{Esencial}
{
El protocolo de comunicaciones será un modelo de petición/respuesta 
síncrono, aunque las acciones globales no lo sean.
}

\reqfuncional{General: Vía de acceso}
{RUC-6}{Alta}{Esencial}
{
El proveedor será el encargado de indicar a las aplicaciones cliente 
dónde se encuentran los laboratorios.
}

\reqfuncional{Proveedor: Acceso a la base de datos}
{RUC-10}{Alta}{Esencial}
{
El administrador podrá insertar, eliminar y modificar datos de la base 
de datos central. Será gestionada mediante la aplicación que el 
propio SGBD provea. El proveedor accederá mediante el driver 
correspondiente en la plataforma de desarrollo.
}

\reqfuncional{Proveedor: Autenticación}
{RUC-11, RUC-12}{Media}{Esencial}
{
La autentificación de cualquier usuario, independientemente del tipo 
de rol, será validada por el proveedor.
}

\reqfuncional{Proveedor: Interfaz de acceso}
{RUC-13}{Alta}{Esencial}
{
Las aplicaciones cliente usarán las funciones contenidas en el 
proveedor, las cuales, serán divididas en dos interfaces, 
con una estructura de servicio web definido por los ficheros WSDL y 
así dar acceso a las aplicaciones cliente.
}

\reqfuncional{Proveedor: Actualizaciones de datos}
{RUC-14}{Alta}{Esencial}
{
Los datos almacenados en el proveedor serán los últimos en actualizar. Por
cada actualización correcta de datos realizadas por los distintos 
módulos se deberá considerar como un cambio permanente.
}

\reqfuncional{Proveedor: Información comprimida}
{RUC-15}{Baja}{Opcional}
{
Cuando un cliente solicite información sobre una herramienta, se 
entregará un conjunto de datos previamente comprimidos y calculados 
para una mayor velocidad de respuesta.
}

\reqfuncional{Proveedor: Información de cada laboratorio}
{RUC-16}{Alta}{Esencial}
{
Cuando un cliente reserve una herramienta, se le entregará la 
información de conexión del laboratorio que la contiene, como es la 
IP y el puerto.
}

\reqfuncional{Proveedor: Funciones web}
{RUC-17, RUC-18}{Alta}{Esencial}
{
Las funciones de las dos interfaces de acceso proveerán parámetros 
de entrada y salida, así como excepciones. Estos objetos serán 
capaces de ser serializados, para poder ser enviados mediante el 
protocolo HTTP y SOAP.
}

\reqfuncional{Proveedor: Seguridad en las herramientas}
{RUC-20}{Alta}{Esencial}
{
Las claves y los identificadores de las herramientas serán únicos y 
no reutilizables. Las claves se generarán mediante funciones 
\emph{hash} a partir de la fecha y el identificador obtenido. 
}

\reqfuncional{Proveedor: Tiempo máximo asignado}
{RUC-21}{Alta}{Esencial}
{
Los clientes tendrán preasignado un tiempo almacenado en la base 
de datos, que será cronometrado por los propios laboratorios en los que 
estén registrados.
}

\reqfuncional{Proveedor: Bloqueo de herramientas}
{RUC-3, RUC-22}{Alta}{Esencial}
{
La base de datos contendrá información de quién tiene reservada
cada herramienta, impidiendo que herramientas que no sean de datos se 
encuentren accesibles por dos clientes, por lo se requerirá un bloqueo 
de escritura y lectura en la información cada vez que se requiera reservar.
}

\reqfuncional{Laboratorios: Localización de las aplicaciones}
{RUC-2}{Alta}{Esencial}
{
En cada laboratorio se almacenarán, de forma local, todas las 
aplicaciones implicadas en las ejecuciones de las herramientas que 
tienen asignadas. 
}

\reqfuncional{Laboratorios: Base de datos}
{RUC-25}{Alta}{Esencial}
{
La base de datos de cada laboratorio contendrá la ruta de acceso a la 
herramienta, su clave y su identificador. Será de baja capacidad y 
portable, por lo tanto, el SGBD será SQLite.
}

\reqfuncional{Laboratorios: Comunicación con las herramientas}
{RUC-32}{Alta}{Esencial}
{
El laboratorio creará una base de datos ligera (SQLite) para ofrecer 
comunicaciones con la herramienta. De manera asíncrona obtendrá los 
datos que se origen en la ejecución de la misma.
}

\reqfuncional{Laboratorios: \emph{Kernel}}
{RUC-26}{Alta}{Esencial}
{
Un laboratorio ofrecerá una capa de gestión llamada \emph{Kernel} que 
será independiente de las comunicaciones con los clientes y las 
ejecuciones de herramientas.
}

\reqfuncional{Laboratorios: Funciones del \emph{Kernel}}
{RUC-30, RUC-34}{Alta}{Esencial}
{
El \emph{Kernel} de cada laboratorio será el encargado de enviar y 
recibir comunicaciones con el proveedor, además de añadir y eliminar 
herramientas. También armarán y realizarán las acciones necesarias 
para desarmar el laboratorio.
}

\reqfuncional{Laboratorios: Parada de emergencia en el \emph{Kernel}}
{RUC-30}{Alta}{Esencial}
{
Cuando el \emph{Kernel} reciba una parada de emergencia detendrá los 
gestores anexos inmediatamente, lo que permitirá al administrador 
acceder al \emph{hardware} de forma más rápida.
}

\reqfuncional{Laboratorios: Gestor de comunicaciones}
{RUC-28}{Alta}{Esencial}
{
Capa superior que administrará todas las peticiones recibidas por los 
clientes así como las notificaciones de las ejecuciones.
}

\reqfuncional{Laboratorios: Tiempo máximo de un cliente}
{RUC-21}{Media}{Esencial}
{
Cada gestor de comunicaciones controlará el tiempo que pasa el cliente 
registrado en el laboratorio. Cuando se sobrepasa, informa al 
proveedor y deniega cualquier intento posterior de comunicación.
}

\reqfuncional{Laboratorios: Gestor de ejecución}
{RUC-31, RUC-36}{Alta}{Esencial}
{
Capa superior que administrará todas las ejecuciones de las acciones y 
de los limpiadores.
}

\reqfuncional{Laboratorios: Tiempo máximo de una acción}
{RUC-47}{Media}{Esencial}
{
El gestor de ejecución se encargará de controlar el tiempo máximo de 
cada acción, que será interrumpida en el instante que lo sobrepase.
}

\reqfuncional{Herramientas: Aplicaciones}
{RUC-9, RUC-36}{Alta}{Esencial}
{
Cada herramienta contendrá un conjunto de aplicaciones de escritorio 
que ayudarán a manejar y configurar el \emph{hardware}.
}

\reqfuncional{Herramientas: Base de datos}
{RUC-37}{Alta}{Esencial}
{
Toda la información de las acciones estarán disponible en la base de 
datos creada por el laboratorio en cada herramienta. Y podrá ser 
modificada por el usuario a través del mismo para la configuración de 
las mismas herramientas.
}

\reqfuncional{Herramientas: Base de datos}
{RUC-37, RUC-39}{Alta}{Esencial}
{
Toda la información de las acciones estarán disponible en la base de 
datos creada por el laboratorio en cada herramienta. Y podrá ser 
modificada por el usuario a través del mismo para la configuración de 
las mismas herramientas.
}

\reqfuncional{Herramientas: Generación de datos}
{RUC-38, RUC-40}{Media}{Esencial}
{
Los datos generados en la ejecución de un acción serán guardados en 
la base de datos para su posterior consulta.
}

\reqfuncional{Herramientas: Servicios externos}
{RUC-42}{Alta}{Esencial}
{
La abstracción mediante \emph{sockets} de servicios externos de cada 
herramienta permitirá al usuario utilizarlos como servicios de la 
propia plataforma, con diferentes orígenes.
}

\reqfuncional{Herramientas: Archivo de descripción}
{RUC-43}{Alta}{Esencial}
{
Los archivos de descripción de herramientas seguirán dos esquemas, 
dependiendo del tipo de herramienta, con un formato XML.
}

\reqfuncional{Herramientas: Herramientas de datos}
{RUC-45}{Media}{Opcional}
{
Para los clientes, las herramientas de datos siempre estarán 
disponibles, independientemente del número de usuarios usándolas, y 
dispondrán de un mecanismo autónomo de \emph{broadcast} de información.
}

\reqfuncional{Herramientas: Limpiadores}
{RUC-46}{Media}{Esencial}
{
Estas acciones serán especificadas en el archivo de descripción y 
tendrán preferencia en cuanto a las peticiones de ejecución normales. 
Una herramienta no se podrá ejecutar si ha ocurrido un error y no se 
ha ejecutado con anterioridad el limpiador.
}


\reqfuncional{\emph{Libtool}: Base de datos}
{RUC-44}{Alta}{Esencial}
{
Esta abstracción permitirá la conexión a la base de datos de la 
herramienta para la lectura y escritura de información. El lenguaje de 
programación dependerá de la implementación de la propia herramienta.
}

\reqfuncional{\emph{Libtool}: Base de datos}
{RUC-44}{Alta}{Esencial}
{
Esta abstracción permitirá la conexión a la base de datos de la 
herramienta para la lectura y escritura de información. El lenguaje de 
programación dependerá de la implementación de la propia herramienta.
}

\reqfuncional{Cliente de escritorio: Función}
{RUC-8}{Media}{Esencial}
{
La aplicación cliente principal será ejecutada como aplicación de 
escritorio que utilizará el servicio web proveedor para la 
interacción con la plataforma RLF.
}

\reqfuncional{Cliente de escritorio: Ventanas de ejecución}
{RUC-48, RUC-49}{Alta}{Esencial}
{
Por cada acción en ejecución, el cliente tendrá activa una ventana 
independiente con acceso de entrada y salida por teclado.
}

\reqfuncional{Cliente de escritorio: Herramientas}
{RUC-51, RUC-53}{Alta}{Esencial}
{
Por cada herramienta, la interfaz del cliente poseerá una pestaña que 
contendrá toda su información. Se podrán seleccionar de forma 
independendiente para su posterior reserva.
}

\reqfuncional{Cliente de escritorio: Conexión}
{RUC-50}{Alta}{Esencial}
{
El cliente conservará la información de conexión durante toda la 
ejecución, sin ser necesario volver a conectarse para realizar 
múltiples reservas.
}

\reqfuncional{Cliente de escritorio: Actualización}
{RUC-50}{Baja}{Esencial}
{
Se ofrecerá al cliente la posibilidad de actualizar la información 
del número de herramientas como de su estado sin necesidad de cerrar 
la aplicación.
}

\reqfuncional{Cliente web: Función}
{RUC-18, RUC-52}{Baja}{Opcional}
{
Los clientes podrán acceder a la página web de la plataforma para 
comprobar el estado actual de las herramientas, si están disponibles, 
ocupadas o no conectadas. El formato deberá ser compatible para 
dispositivos con pantalla pequeña.
}

\reqfuncional{Gestor de laboratorios: Función}
{RUC-26}{Baja}{Esencial}
{
Para gestionar los laboratorios se proveerá de una aplicación de 
consolta con acceso por red con las acciones de armar, desarmar, 
parar, registrar y eliminar herramientas, y comprobar el estado del 
laboratorio.
}

\reqfuncional{Registro: \emph{Logs}}
{RUC-34}{Baja}{Esencial}
{
Todo componente en la plataforma se servirá de un conjunto de 
registros para indicar errores y funciones realizadas.
}

\subsection{Requisitos no funcionales}

\reqnofuncional{Comunicaciones: Sistema escalable}
{RUR-2}{Media}{Esencial}
{
La plataforma utilizará su propio sistema de comunicaciones y de altas 
de laboratorios, sin utilizar soluciones ya implementadas como Java 
RMI, CORBA o RPC.
}

\reqnofuncional{Comunicaciones: Protocolo JSON}
{RUR-1}{Media}{Esencial}
{
El protocolo de comunicaciones estará basado en el formato JSON, 
previamente cifrado, que reduce de forma considerable los datos a 
enviar de los objetos a enviar.
}

\reqnofuncional{Comunicaciones: Protocolo JSON avanzado}
{RUR-1}{Media}{Esencial}
{
Cada objeto enviado entre módulos puede contener otros objetos como 
atributos también en formato JSON.
}

\reqnofuncional{Comunicaciones: Peticiones y respuestas}
{RUR-1}{Media}{Esencial}
{
Cada petición o respuesta tendrá asociado un número de operación 
que se incluirá en la cabecera del mensaje.
} 

\reqnofuncional{Proveedor: \emph{Token} de acceso y uso}
{RUR-3}{Media}{Esencial}
{
Cada cliente tiene asociado un atributo con un límite de tiempo que 
será generado cada vez que se conecte para reducir el tráfico de 
comunicaciones y sólo tener que identificarse mediante esa cadena de 
caracteres. Cuando se desconecta será invalidado.
} 

\reqnofuncional{Proveedor: Servicio Web J2EE}
{RUR-4}{Alta}{Esencial}
{
El proveedor será un conjunto de dos servicios web publicados en una 
plataforma Tomcat, además de la base de datos MySQL. El servicio web 
principal llamado RLF\_Provider y el secundario, RLF\_Monitor.
} 

\reqnofuncional{Laboratorios: Máquina virtual de Java}
{RUR-5}{Alta}{Esencial}
{
Los laboratorios estarán implementados en Java y utilizarán drivers 
de conexión a las bases de datos SQLite.
}

\reqnofuncional{Laboratorios: Archivos ejecutables}
{RUR-5}{Alta}{Esencial}
{
Para configurar un laboratorio se proveerá de un fichero de 
configuración junto con el archivo ejecutable.
}

\reqnofuncional{Laboratorios: Ficheros fuente}
{RUR-5}{Alta}{Esencial}
{
Todo recurso necesario para la ejecución de los laboratorios será 
contenida en un directorio que deberá acompañarse con el ejecutable.
}

\reqnofuncional{Laboratorios: Conexiones}
{RUR-6}{Alta}{Esencial}
{
Cada laboratorio posee un puerto de conexión con el proveedor, otro de 
notificaciones para el cliente, uno de recepción de peticiones y por 
último, el puerto de control de mantenimiento.
}

\reqnofuncional{Laboratorios: Asincrónos}
{RUR-6}{Alta}{Esencial}
{
Por cada petición de ejecución se enviará en qué instante empezará 
a funcionar, ya que es posible que no sea instantáneo si hay 
sobrecarga en el laboratorio.
}

\reqnofuncional{Laboratorios: Ejecuciones}
{RUR-7}{Alta}{Esencial}
{
El laboratorio se servirá de peticiones al sistema y señales para 
ejecutar las acciones, así como de un servicio de \emph{pipeling} para 
redirigir la entrada y la salida de la propia acción.
}

\reqnofuncional{Laboratorios: Colas de ejecución}
{RUR-8}{Alta}{Esencial}
{
Cada laboratorio poseerá un contenedor de acciones en ejecución, donde 
está limitado por su configuración inicial. Todas las peticiones de 
ejecución, si los contenedores están completos, serán retrasadas.
}

\reqnofuncional{\emph{Framework}: Parada}
{RUR-9}{Alta}{Esencial}
{
Aunque no se impondrá ninguna norma a los desarrolladores en cuestión 
de bloqueo de ficheros o de programas por el motivo de las paradas 
inesperadas, se instará en la manera de no mantener ficheros abiertos 
ni componentes susceptibles de ocasionar errores en el futuro.
}

\reqnofuncional{Cliente de escritorio: Máquina virtual de Java}
{RUR-4}{Alta}{Esencial}
{
La aplicación de escritorio será implementada en Java para su 
portabilidad y se requerirá configurar mediante un archivo adjuntado 
con el ejecutable.
}

\reqnofuncional{Cliente web: Múltiple conexión}
{RUR-8}{Alta}{Esencial}
{
Mediante el uso de \emph{tokens} almacenados en la base de datos del 
proveedor una desconexión no es obligatoria en el cliente web.
}
