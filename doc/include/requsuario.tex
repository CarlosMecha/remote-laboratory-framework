% Requisitos del usuario

\newcounter{rcap}
\newcounter{rres}

\subsection{Requisitos de capacidad}

\reqcapacidad{General: \emph{Hardware} remoto}
{Cliente A}{Alta}{Esencial}
{
El \emph{hardware} será controlado desde la plataforma sin necesidad de 
configurarlo en el lugar físico donde se encuentra.
}

\reqcapacidad{General: \emph{Hardware} distribuido}
{Cliente A}{Alta}{Esencial}
{
El \emph{hardware} que se pondrá a disposición de los clientes estará 
distribuido en varios ordenadores.
}

\reqcapacidad{General: \emph{Hardware} exclusivo}
{Cliente A}{Alta}{Esencial}
{
Todo \emph{hardware} sólo podrá ser usado por un único cliente, sin 
que el uso de otros pueda influirle.
}

\reqcapacidad{General: \emph{Hardware} multidisciplinar}
{Cliente A}{Alta}{Esencial}
{
La plataforma RLF contendrá una interfaz de acceso genérica para 
cualquier tipo de \emph{hardware}.
}

\reqcapacidad{General: Plataforma accesible}
{Cliente A}{Alta}{Esencial}
{
La plataforma será accesible para los clientes desde cualquier 
ordenador conectado a Internet, siendo posible la configuración para 
limitar este acceso.
}

\reqcapacidad{General: Acceso central}
{Cliente A}{Alta}{Esencial}
{
Independientemente de dónde se encuentre el \emph{hardware} el que el 
cliente quiere utilizar, se accederá desde una misma dirección para 
todo el sistema.
}

\reqcapacidad{General: Plataforma modular}
{Cliente B}{Alta}{Esencial}
{
La plataforma RLF estará compuesta por módulos que interactuarán entre 
ellos, al menos dos, los servidores locales o \textbf{laboratorios} 
\index{laboratorio} y el servidor central o \textbf{proveedor} 
\index{proveedor}.
}

\reqcapacidad{General: Cliente del la plataforma}
{Cliente B}{Baja}{Opcional}
{
Para acceder a la plataforma se dispondrá de un \textbf{cliente} 
\index{cliente} que no limitado por el sistema operativo donde se 
encuentre.
}

\reqcapacidad{General: Servicios como \emph{hardware}}
{Cliente B}{Media}{Esencial}
{
El \emph{hardware} deberá ser encapsulado como un tipo de servicio 
genérico, llamado \textbf{herramienta}\index{herramienta}.
}

\reqcapacidad{Proveedor: Base de datos central}
{Cliente A}{Alta}{Esencial}
{
El proveedor contendrá la información en una base de datos central a la 
que accederá cada vez que se requiera utilizar dicha información.
}

\reqcapacidad{Proveedor: Usuarios clientes}
{Cliente A}{Alta}{Esencial}
{
Los clientes poseerán un nombre de usuario, una contraseña, un 
email y un rol específico para acceder a las herramientas.
}

\reqcapacidad{Proveedor: Usuarios administradores}
{Cliente B}{Media}{Esencial}
{
Los administradores serán autentificados en la base de datos del 
proveedor para controlar el acceso a sistemas críticos.
}

\reqcapacidad{Proveedor: Servicio Web}
{Cliente B}{Alta}{Esencial}
{
El proveedor contendrá una interfaz de acceso para la conexión a 
través de Internet y así permitirá a los clientes utilizar los 
servicios de la plataforma.
}

\reqcapacidad{Proveedor: Información en tiempo real}
{Cliente B}{Alta}{Esencial}
{
Los clientes obtendrán la información del proveedor actualizada, tanto 
del estado de las herramientas como el de los laboratorios.
}

\reqcapacidad{Proveedor: Información de las herramientas}
{Cliente A}{Alta}{Esencial}
{
La información de las herramientas estará contenida en la base de datos 
del proveedor para obtener una mayor respuesta cuando los clientes la 
soliciten.
}

\reqcapacidad{Proveedor: Información de los laboratorios}
{Cliente A}{Alta}{Esencial}
{
Al ser un sistema con un componente central, la información de cada 
laboratorio, es decir, dónde se localiza y cómo puede contarse a él, 
se encontrará en la base de datos del proveedor.
}

\reqcapacidad{Proveedor: Acciones}
{Cliente A}{Alta}{Esencial}
{
Las acciones que se podrán llevar a cabo mediante la interfaz de 
acceso al proveedor serán conexión o desconexión de un cliente, 
obtención de la información de las herramientas y su estado, y 
reserva de las mismas.
}

\reqcapacidad{Proveedor: Monitor}
{Cliente B}{Baja}{Opcional}
{
El proveedor tendrá otra interfaz de acceso limitada únicamente para 
conectar y desconectar del sistema, y además obtener el estado actual 
de las herramientas.
}

\reqcapacidad{Proveedor: Seguridad en el acceso}
{Cliente B}{Media}{Opcional}
{
Todas las comunicaciones del proveedor con los clientes estarán 
debidamente cifradas ya que pueden atravesar redes inseguras.
}

\reqcapacidad{Proveedor: Registro de herramientas}
{Cliente B}{Media}{Esencial}
{
El proveedor dará acceso a los laboratorios para que registren 
herramientas, generando un identificador único para cada herramienta y 
una clave de uso.
}

\reqcapacidad{Proveedor: Tiempo de acceso máximo}
{Cliente A}{Media}{Esencial}
{
A cada usuario se le asignará un tiempo de acceso máximo contado en 
minutos, que una vez sobrepasado se expulsa del sistema.
}

\reqcapacidad{Proveedor: Reserva de herramientas}
{Cliente A}{Alta}{Esencial}
{
El proveedor se encargará de bloquear las herramientas reservadas por 
los clientes para que no puedan ser usadas.
}

\reqcapacidad{Laboratorios: Comunicación}
{Cliente B}{Alta}{Esencial}
{
Toda la comunicación entrante y saliente de un laboratorio se 
realizará mediante red.
}

\reqcapacidad{Laboratorios: Base de datos}
{Cliente B}{Alta}{Esencial}
{
Cada laboratorio contendrá su propia base de datos, que elimina el tráfico 
en la red y permite hacer operaciones locales más rápidas. En ellas 
se almacenará la información de las herramientas.
}

\reqcapacidad{Laboratorios: Administración}
{Cliente A}{Alta}{Esencial}
{
Los laboratorios contarán con una interfaz de acceso para poder 
administrarlos sin necesidad de una parada total del sistema.
}

\reqcapacidad{Laboratorios: Registro de herramientas}
{Cliente A}{Alta}{Esencial}
{
Cada laboratorio estará al cargo de varias herramientas, que deberá 
gestionar y administrar.
}

\reqcapacidad{Laboratorios: Acceso a las herramientas}
{Cliente B}{Alta}{Esencial}
{
Para eliminar el tráfico del proveedor, los laboratorios aceptarán
peticiones de los clientes que previamente se hayan conectado al 
sistema y hayan reservado las herramientas necesarias.
}

\reqcapacidad{Laboratorios: Eliminación de herramientas}
{Cliente A}{Baja}{Esencial}
{
Se podrá eliminar una herramienta concreta de un laboratorio si así se 
requiere.
}

\reqcapacidad{Laboratorios: Mantenimiento}
{Cliente A}{Baja}{Esencial}
{
Para aportar seguridad, los laboratorios tendrán un estado de 
``mantenimiento'' donde no se permite la comunicación con los 
clientes, en el cual se podrá realizar tareas de administración del 
mismo.
}

\reqcapacidad{Laboratorios: Ejecuciones de las herramientas}
{Cliente A}{Alta}{Esencial}
{
Los laboratorios sólo permitirán una instancia de ejecución por 
herramienta. Esto conlleva la creación de una cola de peticiones para 
la ejecución de estas.
}

\reqcapacidad{Laboratorios: Comunicación con las herramientas}
{Cliente A}{Alta}{Esencial}
{
Un laboratorio se comunicará con las herramientas que tiene asignadas 
mediante un mecanismo asíncrono para que una herramienta con fallos no 
bloquee la ejecución.
}

\reqcapacidad{Laboratorios: Parada de emergencia}
{Cliente A}{Baja}{Opcional}
{
Cada laboratorio podrá ser parado de manera urgente por parte de un 
administrador, que cierra todas las ejecuciones activas y elimina 
las pendientes. Después, el laboratorio se desactivará.
}

\reqcapacidad{Laboratorios: \emph{Logs}}
{Cliente B}{Baja}{Esencial}
{
Cada laboratorio poseerá su propio mecanismo de registro de eventos y 
errores, que puede ser consultado por un administrador.
}

\reqcapacidad{Laboratorios: Información en tiempo real}
{Cliente A}{Media}{Esencial}
{
Cada vez que un laboratorio cambie su estado, el proveedor será 
informado, incluso cuando se realizan paradas de emergencia.
}

\reqcapacidad{Herramientas: Acciones}
{Cliente A}{Alta}{Esencial}
{
Cada herramienta podrá realizar varias acciones, todas relacionadas 
con el mismo \emph{hardware}. El usuario elegirá una y la 
configurará. No será posible ejecutar a la vez dos acciones de la misma 
herramienta.
}

\reqcapacidad{Herramientas: Configuración de las acciones}
{Cliente B}{Alta}{Esencial}
{
Cada acción tendrá asociada unos parámetros de salida y de entrada, 
con un tipo de datos establecido y una descripción. Mediante la 
definición de los valores de estos parámetros la acción se configura 
para su ejecución. Los parámetros de salida sólo se leerán por parte 
del cliente cuando la acción ha terminado.
}

\reqcapacidad{Herramientas: Excepciones y estado final de la acción}
{Cliente A}{Media}{Esencial}
{
Las acciones, durante su ejecución, podrán lanzar excepciones que 
informan al usuario de un error, además, cada finalización tendrá 
asociada un estado incluido por el desarrollador.
}

\reqcapacidad{Herramientas: Constantes}
{Cliente A}{Baja}{Esencial}
{
Las herramientas podrán contener varias constantes con un tipo de datos 
concreto y un valor que no varía desde que se registra la herramienta.
}

\reqcapacidad{Herramientas: Atributos}
{Cliente A}{Baja}{Esencial}
{
Los laboratorios definirán los atributos propios de cada herramienta, 
donde se incluyen el identificador, la clave, nombre, descripción, rol, 
versión y administrador responsable de la herramienta.
}

\reqcapacidad{Herramientas: Entrada y salida}
{Cliente B}{Alta}{Esencial}
{
Las herramientas dispondrán de una entrada y salida textual para poderse 
comunicar con el usuario.
}

\reqcapacidad{Herramientas: Servicios externos}
{Cliente A}{Alta}{Esencial}
{
Para dar funcionalidad extra a cada herramienta, las acciones podrán 
tener asociadas servicios externos, que se activarán en la ejecución y 
que pueden ser usadas por los clientes.
}

\reqcapacidad{Herramientas: Archivo de descripción}
{Cliente A}{Alta}{Esencial}
{
Para definir una herramienta por completo y registrarla en la 
plataforma RLF, se dispondrá de un fichero con un formato establecido 
que será leído en el momento del registro en el laboratorio.
}

\reqcapacidad{Herramientas: Librerías \emph{libtool}}
{Cliente B}{Alta}{Opcional}
{
\index{\emph{libtool}}Como forma de abstracción, se incluirán
librerías para la comunicación de la herramienta con el laboratorio 
correspondiente. Estas librerías proveerán las funciones de acceso y 
desconexión, lectura y escritura de parámetros, lectura de constantes 
y atributos, lanzamiento y lanzamiento de excepciones.
}

\reqcapacidad{Herramientas: Herramientas de datos}
{Cliente A}{Media}{Opcional}
{
Esta categoría especial de herramientas permitirá acceder a varios 
clientes a la vez, pero sin interactuar con el \emph{hardware}. No 
disponen de entrada textual, ni de parámetros de configuración. Sólo 
tendrán una acción asignada.
}

\reqcapacidad{Herramientas: Limpiadores}
{Cliente B}{Media}{Esencial}
{
Existirá por cada herramienta una acción especial que establece el 
\emph{hardware} de la herramienta a su estado original. No podrá ser 
utilizada por los clientes y de forma automática se lanzará cuando se 
detecta errores en alguna acción.
}

\reqcapacidad{Herramientas: Tiempo máximo de ejecución}
{Cliente B}{Media}{Esencial}
{
Cada acción poseerá un tiempo máximo de ejecución que el laboratorio 
monitoriza. Cuando ese tiempo es sobrepasado, se parará la ejecución por 
completo y se establece el estado de acción fallida.
}

\reqcapacidad{Cliente: Visualización simultánea}
{Cliente A}{Alta}{Esencial}
{
La interfaz del cliente será capaz de visualizar a la vez varias 
acciones de distintas herramientas.
}

\reqcapacidad{Cliente: Uso de las herramientas}
{Cliente B}{Alta}{Esencial}
{
Se proveerá al usuario de una ``consola'' (con entrada y salida textual) 
para el manejo de cada una de las acciones.
}

\reqcapacidad{Cliente: Acciones}
{Cliente A}{Alta}{Esencial}
{
La interfaz permitirá al cliente conectarse y desconectarse del sistema, 
reservar herramientas, comprobar la información de las mismas y su 
estado en tiempo real, y la ejecución de acciones.
}

\reqcapacidad{Cliente: Reserva múltiple}
{Cliente A}{Alta}{Optativa}
{
Un cliente podrá reservar de manera simultánea varias herramientas a 
las que tiene acceso.
}

\reqcapacidad{Cliente: Monitor}
{Cliente A}{Baja}{Optativa}
{
La aplicación cliente dispondrá con una aplicación auxiliar por la que 
acceder la monitor central.
}

\reqcapacidad{Cliente: Información}
{Cliente A}{Media}{Esencial}
{
Todos los datos públicos de las herramientas serán mostrados al cliente 
por medio de la interfaz, añadiendo las descripciones a los parámetros.
}

\subsection{Requisitos de restricción}

\reqrestriccion{General: Cantidad de comunicación}
{Cliente A}{Media}{Esencial}
{
La información trasmitida en las comunicaciones, así como el número 
de estas deberán ser reducidas al mínimo para no saturar cada componente.
}

\reqrestriccion{General: Sistema escalable}
{Cliente B}{Alta}{Esencial}
{
Todo el sistema será escalable, es decir, que no influya en su 
funcionamiento el hecho de aumentar o disminuir el número de 
laboratorios o clientes.
}

\reqrestriccion{Proveedor: Control de seguridad}
{Cliente B}{Media}{Esencial}
{
Los clientes que no se desconecten de forma válida en el sistema o que 
sufran errores, el proveedor no les permitirá volver a conectarse hasta 
que un administrador dé el visto bueno.
}

\reqrestriccion{Proveedor: Bloqueos de peticiones}
{Cliente B}{Alta}{Esencial}
{
El proveedor contará con mecanismos para atender varias peticiones a la 
vez sin dejar ninguna a la espera.
}

\reqrestriccion{Laboratorios: Portables}
{Cliente B}{Alta}{Esencial}
{
Los laboratorios funcionarán sin problemas en diversos sistemas operativos con 
una única implementación.
}

\reqrestriccion{Laboratorios: Acciones asíncronas}
{Cliente A}{Media}{Esencial}
{
Todas las acciones que se llevan a cabo en los laboratorios serán
asíncronas, para no detener ni bloquear ningún componente que 
interfiera con ellos.
}

\reqrestriccion{Laboratorios: Ejecuciones externas}
{Cliente A}{Alta}{Esencial}
{
Las acciones se ejecutarán en los laboratorios como programas externos 
para no añadir carga a la plataforma.
}

\reqrestriccion{Laboratorios: Máximo número de ejecuciones}
{Cliente A}{Alta}{Esencial}
{
Cada laboratorio poseerá un número máximo de acciones en ejecución 
(configurado por el propio administrador). Cuando ocurre esto, las 
acciones que están pendientes tomarán el estado de espera.
}

\reqrestriccion{Herramientas: Parada}
{Cliente A}{Alta}{Esencial}
{
Las herramientas deberán estar preparadas para sufrir paradas sin 
previo aviso, aunque no es necesario que devuelvan el estado original 
al \emph{hardware}.
}

\reqrestriccion{Cliente: Control de seguridad web}
{Cliente A}{Media}{Esencial}
{
El control de seguridad aplicado a los usuarios para el cliente de 
escritorio no bloqueará la cuenta cuando ocurran errores en el cliente 
web.
}
