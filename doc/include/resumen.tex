% Resumen

\vspace*{2cm}
\section*{Resumen}

En el último medio siglo, la sociedad se ha visto abrumada por el 
avance de tecnologías que ya forman parte de la vida cotidiana. Es por 
esto que es imposible dar cabida a largo plazo a los modelos de 
trabajo e industria tradicionales. El cambio se está produciendo en la 
actualidad y por ese motivo surgen nuevas ideas para intentar adecuar 
estos modelos a la ``sociedad de la información''. Las empresas y 
centros de investigación intentan dar ``pasos de gigante'' para 
innovar y mantenerse en la cima.

La evolución de estas tecnologías también implica que los problemas 
a los que se enfrentan estos modelos se modifiquen, desaparezcan e 
incluso se creen nuevos. Algunos como la capacidad de las herramientas 
han desaparecido por completo para dar paso a soluciones que aporten 
velocidad y potencia. Otros siguen aún sin resolverse, 
como las limitaciones físicas de localización que se presentan en la 
industria contemporanea.

El proyecto que se detalla a continuación pretende dar una solución 
concreta a este problema aplicado a la forma de enseñanza en centros 
tecnológicos y universidades, uniendo y utilizando los avances punteros 
que son accesibles para cualquier estudiante, desarrollador o 
ingeniero. Se trata de una solución versátil que permite salvar otros 
tipos de problemas derivados, así como ser utilizada para varios 
usos no contenidos en este documento, como es la domótica o 
investigación. En el caso que se ocupa se centrará en desarrollar una plataforma 
\emph{online} que permita acceder y administrar de forma remota a un 
conjunto de recursos \emph{hardware} y \emph{software} distribuidos de forma 
transparente para el usuario final. Para ello, se manejarán conceptos 
como \emph{comunicación}, \emph{virtualización} y \emph{bases de datos}.

A lo largo de las siguientes páginas se dará un repaso a las 
tecnologías que se han usado así como la evolución histórica de las 
mismas. También se hará incapié en los requisitos para el desarrollo 
y diseño del mismo.

\cleardoublepage

\vspace*{2cm}
\section*{Abstract}

In the last half century, the society has been overwhelmed by the 
progress of technologies that are already part of everyday life. This 
is why it is impossible to accommodate long-term work patterns and 
traditional industry. The change is happening now and that is why new 
ideas to try to adapt these models to the ``information society''. The 
companies and research centers are trying to ``giant steps'' to innovate 
and stay on top.

The evolution of these technologies also means that the problems faced 
by these models change, disappear and even create new. Some, like the 
ability of the tools have altogether disappeared to make way for 
solutions that provide speed and power. Others aren’t completely 
settled yet, as the physical limitations of location presented in 
contemporary industry.

The project described below is to provide a concrete solution to this problem
applied to the form of teaching in universities and technological centers,
connecting and using the pointer advances that are accessible to any student,
developer or engineer. It is a versatile solution that can save other 
types of problems and be used for various purposes not contained in 
this document, such as home automation or research. In the present case 
will focus on developing a platform \emph{online} to allow access and 
remotely manage a set of \emph{hardware} and \emph{software} resources distributed 
transparently to the end user. To this end, concepts such as 
\emph{communication}, \emph{virtualization} and \emph{databases} will 
be handled.

Throughout the following pages will give an overview of the 
technologies used, as well as the historical evolution of the same. 
Emphasis will also be on the requirements for the design and development.

\cleardoublepage
