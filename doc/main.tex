\documentclass{template/rlftemplate}

\usepackage{lib/rlfvocabulary}

\begin{document}

% Configuración varia
\renewcommand{\tablename}{Tabla} 
\renewcommand{\listtablename}{Índice de tablas}
\setcounter{page}{1}
\pagenumbering{roman}

% Portada
% Portada

\title{\RLF_normal} 
\author{\Carlos}
\date{\Fecha}

\begin{titlepage}

\begin{figure}[ht]
	\begin{center}
	\vspace{0.5cm}
	% UC3M Logo
	\includegraphics[width=2.5cm]{images/logo_uc3m.png}
	\vspace{-1cm}
	\end{center}
\end{figure}

\begin{center}
	UNIVERSIDAD CARLOS III DE MADRID \\
	Ingeniería Informática\\
	\vspace{3cm} {\huge \textbf{Diseño y desarrollo de un prototipo de 
	Framework para laboratorios remotos}}\\
	\vspace{0.5cm}{\Large \textbf{Implementación de las primeras 
	herramientas y funcionalidades}}\\
	
	\begin{figure}[ht]
		\begin{center}
			\vspace{1.5cm}
			% RLF Logo
			\includegraphics[scale=0.2]{images/logo_rlf.png}
			
		\end{center}
	\end{figure}
	
	%\vspace{1cm} {\Large \textsc{\RLFnormal}}\\
	%\vspace{0.5cm}{\large \textsc{Prototype}}\\

	\vspace{2cm} {\large PROYECTO DE FIN DE CARRERA} \\
	\vspace{0.5cm} \Fecha
	\vspace{1.5cm}
\end{center}

% Autores.
\begin{parhmargin}{8cm}{0cm}
	
	Autor: \textbf{\textit{\Carlos}}\\
	Tutores: \textbf{\textit{\Ramon}}\\
	\hspace*{1.4cm}\textbf{\textit{\Javier}}\\
\end{parhmargin}

\end{titlepage}



\emptypage

% Contraportada
% Contraportada
\thispagestyle{empty}

\begin{center}
		\vspace*{6cm} 
		{\huge \textbf{Diseño y desarrollo de un prototipo de 
		Framework para laboratorios remotos}}\\
		\vspace{0.5cm}{\Large \textbf{Implementación de las primeras herramientas y 
		funcionalidades}}\\
		\vspace{2cm}
		{\huge \textsc{\RLFnormal}}\\
		\vspace{0.5cm}
		{\large \textsc{Prototype}}\\
		\vspace{4cm}
		{\large \Carlos}
		\vfill
		\Copyright{}
\end{center}



\emptypage

% Agradecimientos
% Agradecimientos
\thispagestyle{empty}

\begin{flushright}
		\vspace*{7cm}
		\textit{A mis padres, pues sin ellos nunca lo hubiera conseguido.}\\
		\textit{A Aitor, Antonio, Diego, Jose, Juan y Sergio, por aguantarme.}\\
		\textit{A Mayu, por estar siempre ahí.}\\
		\textit{A mis tutores, Ramón y Javier, por haberme guiado.}\\
		\textit{A Anabel, por contestar cada una de mis incesantes preguntas.}\\
		\textit{y al resto de mis amigos y compañeros por alegrarme estos seis años.}\\
		\vspace{4cm}
		``Si tú crees que no puedes conseguirlo, no te preocupes, no lo harás.''\\
		\textbf{Vancouver, 2010}
\end{flushright}

\emptypage

% Resumen
% Resumen

\vspace*{2cm}
\section*{Resumen}

En el último medio siglo, la sociedad se ha visto abrumada por el 
avance de tecnologías que ya forman parte de la vida cotidiana. Es por 
esto que es imposible dar cabida a largo plazo a los modelos de 
trabajo e industria tradicionales. El cambio se está produciendo en la 
actualidad y por ese motivo surgen nuevas ideas para intentar adecuar 
estos modelos a la ``sociedad de la información''. Las empresas y 
centros de investigación intentan dar ``pasos de gigante'' para 
innovar y mantenerse en la cima.

La evolución de estas tecnologías también implica que los problemas 
a los que se enfrentan estos modelos se modifiquen, desaparezcan e 
incluso se creen nuevos. Algunos como la capacidad de las herramientas 
han desaparecido por completo para dar paso a soluciones que aporten 
velocidad y potencia. Otros siguen aún sin resolverse, 
como las limitaciones físicas de localización que se presentan en la 
industria contemporanea.

El proyecto que se detalla a continuación pretende dar una solución 
concreta a este problema aplicado a la forma de enseñanza en centros 
tecnológicos y universidades, uniendo y utilizando los avances punteros 
que son accesibles para cualquier estudiante, desarrollador o 
ingeniero. Se trata de una solución versátil que permite salvar otros 
tipos de problemas derivados, así como ser utilizada para varios 
usos no contenidos en este documento, como es la domótica o 
investigación. En el caso que se ocupa se centrará en desarrollar una plataforma 
\emph{online} que permita acceder y administrar de forma remota a un 
conjunto de recursos \emph{hardware} y \emph{software} distribuidos de forma 
transparente para el usuario final. Para ello, se manejarán conceptos 
como \emph{comunicación}, \emph{virtualización} y \emph{bases de datos}.

A lo largo de las siguientes páginas se dará un repaso a las 
tecnologías que se han usado así como la evolución histórica de las 
mismas. También se hará incapié en los requisitos para el desarrollo 
y diseño del mismo.

\cleardoublepage

\vspace*{2cm}
\section*{Abstract}

In the last half century, the society has been overwhelmed by the 
progress of technologies that are already part of everyday life. This 
is why it is impossible to accommodate long-term work patterns and 
traditional industry. The change is happening now and that is why new 
ideas to try to adapt these models to the ``information society''. The 
companies and research centers are trying to ``giant steps'' to innovate 
and stay on top.

The evolution of these technologies also means that the problems faced 
by these models change, disappear and even create new. Some, like the 
ability of the tools have altogether disappeared to make way for 
solutions that provide speed and power. Others aren’t completely 
settled yet, as the physical limitations of location presented in 
contemporary industry.

The project described below is to provide a concrete solution to this problem
applied to the form of teaching in universities and technological centers,
connecting and using the pointer advances that are accessible to any student,
developer or engineer. It is a versatile solution that can save other 
types of problems and be used for various purposes not contained in 
this document, such as home automation or research. In the present case 
will focus on developing a platform \emph{online} to allow access and 
remotely manage a set of \emph{hardware} and \emph{software} resources distributed 
transparently to the end user. To this end, concepts such as 
\emph{communication}, \emph{virtualization} and \emph{databases} will 
be handled.

Throughout the following pages will give an overview of the 
technologies used, as well as the historical evolution of the same. 
Emphasis will also be on the requirements for the design and development.

\cleardoublepage


% Índice
% Índice

\tableofcontents

\cleardoublepage

\listoffigures

\cleardoublepage

\listoftables

\cleardoublepage


\setcounter{page}{1}
\pagenumbering{arabic}

% Introducción
% Introducción

\capitulo{Introducción}{introduccion}{
La finalidad de este capítulo es dar a conocer el contenido de este 
libro, así como un primer encuentro con el proyecto. Tratado como una 
pequeña introducción, los conceptos novedosos serán explicados en 
capítulos posteriores.
}

\section{Introducción}

Este proyecto se inicia como parte de una solución a problemas 
concretos en la evolución tecnológica. Desde hace años, el mundo 
está inmerso en un constante cambio impulsado por el desarrollo de una 
parte de la ciencia, la cual sirve como base a innumerables 
campos. Tan amplio es el efecto, que no se concibe, en la actualidad, 
la vida sin ella. Ésta es la ciencia de la información.

Todo lo que implica una revolución como ésta se ve reflejado en el 
pensamiento de los nuevos científicos, ingenieros, estudiosos y 
empresarios, los cuales han tenido que adaptar sus ideas a esta nueva 
situación. Éstas son las que ahora, y en el futuro, resolverán 
los problemas que surgen como consecuencia directa del cambio sometido.

La idea (o proyecto), que en las posteriores páginas de este documento es 
explicada, no es más que una de las miles que intentan ayudar a esa 
evolución sin retorno, asegurando que será la base para otras 
soluciones. Se dará como nombre \emph{Remote Laboratory Framework} 
(versión \emph{Prototype}) a este proyecto.

Se empieza, entonces, con los motivos y características que han 
gestado esta idea, así como su análisis y diseño, y, por último, su 
implementación en la vida real.

\section{Motivación del proyecto}

La ciencia de la información ha contribuido al desarrollo de otros 
campos (no sólo en el ambiente tecnológico) y al aumento de 
necesidades concretas. Yendo a las partes más básicas de la vida 
cotidiana, la educación y el trabajo se han visto afectados 
directamente, siendo parte del problema y de la solución.

De tal modo es así que como al cabo de los años de estudio, un alumno se enfrenta a 
numerosos problemas relacionados con ambos campos, independientemente 
del tipo de estudios cursados. Las innumerables horas dedicadas al 
desarrollo y aplicación de los conocimientos adquiridos hacen que las 
ideas surjan como una respuesta lógica a estos mismos problemas.

Pero este proyecto se centrará en dos problemas concretos que hasta hace 
unos años, no tenían aparente solución:

El primero es la \textbf{dependencia} que se tiene en los modelos de educación 
y de trabajo actuales respecto a los lugares donde se desarrollan 
estas actividades. Los traslados pueden llegar a copar gran parte del 
tiempo en un estudiante o trabajador, el cual, la mayor parte es 
improductivo. Cabe decir que se realizan esfuerzos por parte de las 
organizaciones involucradas para intentar eliminar esta gran 
dependencia creando herramientas de comunicación y sistemas 
automatizados remotos. Siendo soluciones válidas, aún quedan muchos 
campos por tratar, como es el caso de las herramientas de trabajo 
electrónicas, desde ordenadores, máquinas y componentes mećanicos 
entre otros, dispuestos en un lugar concreto, donde el personal debe 
desplazarse obligatoriamente para poder usarlos. \emph{El problema de 
la dependencia física de las herramientas de trabajo.}

El segundo problema se centra más en la \textbf{variedad} de estas 
herramientas. Cada empresa vende sus productos a estos centros y 
organizaciones cerrando el paso a la competencia, lo que conlleva a 
que cada uno de estos sea incompatible (en mayor o menor medida) con 
los desarrollados por otras empresas. Esto obliga a los estudiantes y 
trabajadores a aprender como se usa cada producto, incluso cuando su 
funcionalidad es la misma, con el gasto de tiempo y esfuerzo que esto 
requiere. Se acentúa aún más en las herramientas \emph{software}, 
donde, sin seguir ningún estándar, fuerzan al usuario a cambiar su 
forma de trabajo. \emph{El problema de la heterogeneidad de los 
componentes tecnológicos.}

Cualquier intento de solventar el primer problema se enfrenta con el 
segundo de manera irremediable, obligando a que cada solución sea 
específica para un producto en concreto.

Paralelamente, el avance tecnológico desarrolla el concepto de 
``la gran red'', Internet, que se muestra como aliciente para 
solventar dichos problemas. Esta gran tecnología permite lidiar con el 
problema de la dependencia, y las herramientas surgidas de ella, como 
las grandes plataformas de trabajo apuntan al problema de la variedad.

A partir de aquí, se puede extender el concepto de este proyecto a 
otros campos sin aparente relación. En la ingeniería, con sistemas 
complejos compuestos por multitud de componentes electrónicos y 
tecnologías, es donde se invierten grandes fortunas en solventar estos 
dos mismos problemas. Incluso, al aportar los avances tecnologicos a 
campos donde anteriormente no se encontraban, como es el caso de la 
arquitectura y construcción, aparecen dichos problemas.

\section{Objetivos}

Apuntando a unos objetivos concretos y bien definidos, el proyecto 
tiene como finalidad solventar los dos problemas anteriormente 
descritos. Como se verá a lo largo de este documento, se considerará 
crear un sistema que permita principalmente:

\begin{itemize}
\item Acceder a un conjunto de recursos heterogéneos de manera remota, 
es decir, sin necesidad de estar físicamente donde se encuentran. 
\item Conseguir un sistema distribuido que sea visto como un único 
componente por parte del usuario.
\item Incluir la posibilidad de ampliar el sistema de forma escalable 
sin que suponga el desarrollo desde cero.
\end{itemize}

Si se ponen en práctica estos objetivos, se obtendrá una 
\emph{plataforma} que permita la eliminación de muchos sistemas 
dedicados, así como su ahorro, en ambientes de trabajo donde se 
requiera utilizar recursos físicos electrónicos, aplicándose 
también a recursos didácticos en universidades o escuelas. 
De este modo se minimizará el tiempo que se requiere para su uso y el tiempo 
que estas herramientas permanecen \emph{ociosas}. Para cumplir estos 
objetivos, es necesario atacar un conjunto metas más desgranadas que 
atienden a objetivos sencundarios. 

\begin{itemize}
\item Permitir al usuario manejarlos de igual manera que si lo hiciera 
de forma local.
\item Dar cabida a recursos de diferentes tipos, 
transformando esa heterogeneidad en una aparente homogeneidad.
\item Eliminar las dependencias de los recursos en cuanto a requisitos 
propios de la plataforma, es decir, reducir al mínimo las especificaciones 
de funcionamiento de cada recurso.
\item Automatizar todo el proceso de la utilización de los recursos, de 
forma que no requiera de intervención externa para el funcionamiento 
normal del sistema, lo que hará que se pueda utilizar en cualquier 
momento, incluso fuera de horarios de trabajo o lectivos.
\item Ayudar a que las posteriores modificaciones del sistema sean de 
forma controlada y manteniendo, en la medida de lo posible, la 
coherencia del proyecto original.
\end{itemize}

Desde el punto de vista de la creación de nuevos recursos, se 
establecieron unas normas y requisitos para poder salvar esa 
heterogeneidad.

Como conclusión, se dirá que se ha creado un sistema distribuido 
manejado por un \index{\emph{middleware}}\emph{middleware}  
\footnote{Capa de \emph{software} que se ejecuta sobre el sistema operativo 
que permite manejar una red de computadoras como un sistema único 
\cite{Tanenbaum}} que a la vez se comporte como un 
\emph{framework} \footnote{(Plataforma, entorno, marco de trabajo). 
Desde el punto de vista del desarrollo de \emph{software}, es una 
estructura de soporte definida, en la cual otro proyecto de 
\emph{software} puede ser organizado y desarrollado. \cite{Alegsa}} 
para la creación, distribución y uso de nuevos recursos tecnológicos.

\section{Planificación}

Debido a la gran cantidad de trabajo que supone el desarrollo de este 
proyecto, es necesario establecer una planificación \emph{a priori} que 
defina cómo abordar tales problemas.

Siendo la parte más importante el análisis y diseño, se realizará 
en primer lugar la recolección de requisitos definidos en primera 
instancia por el cliente y los usuarios, con su posterior tratamiento y 
desarrollo.

En segundo lugar el diseño completo del sistema, a partir de dichos 
requisitos dará lugar a las especificaciones necesarias para empezar 
el desarrollo del mismo.

\begin{table}[H]
\begin{center}
\begin{tabular}{|| l | c | r ||}
	\hline
	\hline
	Tarea & Etapa & Horas estimadas\\
	\hline
	\hline
	Planteamiento inicial & Análisis y diseño & 16\\
	Aceptación del proyecto & Análisis y diseño & 4\\
	Primera toma de requisitos & Análisis y diseño & 8\\
	Revisión de requisitos & Análisis y diseño & 8\\
	Toma de requisitos & Análisis y diseño & 20\\
	Aceptación de requisitos & Análisis y diseño & 8\\
	Investigación y Diseño & Análisis y diseño & 120\\
	Aceptación del diseño & Análisis y diseño & 4\\
	Desarrollo & Desarrollo & 1100\\
	Verificación y pruebas & Pruebas & 40\\
	Aceptación del sistema & Pruebas & 4\\
	Documentación & Documentación  & 116\\
	Presentación del proyecto & Presentación & 20\\
	\hline
	& \textbf{Total} & 1468\\
	\hline
	\hline
\end{tabular}
\end{center}
	\caption[Planificación del proyecto]{Planificación de las tareas.}
	\label{tab:planificacion}
\end{table}

A continuación, se procederá a poner en práctica el diseño 
obtenido, siendo fieles a lo anteriormente establecido. Esto 
comprenderá desde el desarrollo del código necesario hasta las 
posteriores remodelaciones.

Después, con las primeras versiones del sistema, se procederá a 
realizar los \emph{test} necesarios que verifiquen el buen 
funcionamiento del mismo, en diversas situaciones y sobre múltiples 
estados. Debido a que en esta etapa puede surgir la necesidad de 
modificar el desarrollo del mismo, es posible que se deba retroceder 
de forma cíclica para poder solventar los problemas y errores que 
puedan aparecer.

Además, realizarán múltiples revisiones sobre la documentación 
creada durante todo el proceso para establecer su versión final.

Por último, y con la versión final, tanto del sistema como de la 
documentación, se expondrá el conjunto de los productos en una 
presentación donde asistirá el cliente, los tutores que supervisarán 
el desarrollo de todo el proyecto y un consejo de evaluación.

El recuento de horas se muestra en la tabla \ref{tab:planificacion} y 
calendario propuesto para estas tareas, representado en un diagrama de 
Gantt se encuentra en la figura \ref{fig:calendario} al final del capítulo.

\begin{figure}[h]
	\centering
	\includegraphics[angle=270,scale=0.65]{images/gantt.png}
	\caption[Calendario de tareas]{Calendario por semanas de las 
	tareas realizadas.}
	\label{fig:calendario}
\end{figure}

\section{Contenido}

La documentación que aquí se presenta viene estructurada de forma que 
siga los mismos pasos de la planificación anteriormente descrita:

\begin{description}
\item[Capítulo 1 - Introducción:] Sirve como ayuda para poder 
comprender la motivación de este proyecto, así como los objetivos que 
se desean alcanzar.
\item[Capítulo 2 - Estado arte:] Este capítulo de especial importancia 
da una visión general de las tecnologías usadas, así como la 
evolución que ha permitido el desarrollo del sistema. En las 
diferentes secciones se explicará cada concepto de importancia 
utilizado en este.
\item[Capítulo 3 - Análisis:] Mostrará el conjunto de los requisitos y 
especificaciones concretas que son necesarias para poder llevar a cabo 
correctamente esta idea.
\item[Capítulo 4 - Diseño:] Agrupa todas las soluciones propuestas a 
las anteriores especificaciones, así como la forma de actuar para 
desarrollar el proyecto.
\item[Capítulo 5 - Pruebas:] Especifica las diferentes situaciones 
controladas a las que se someterá el sistema para comprobar el 
correcto funcionamiento, así como los resultados obtenidos.
\item[Capítulo 6 - Problemas y conclusión:] Se especificarán los 
problemas más importantes que se han encontrado en el desarrollo del 
proyecto y una breve conclusión.
\end{description}

Además, se incluyen unos anexos para la correcta utilización y 
comprensión del sistema.

\begin{description}
\item[Anexo A - Presupuesto:] Lista detallada de los costes que supone 
implantar y mantener el sistema.
\item[Anexo B - Guía de despliegue:] Instrucciones que se deberán 
llevar a cabo para poder desplegar completamente el sistema en un 
ambiente de trabajo.
\item[Anexo C - Guía de mantenimiento:] Conjunto de directrices de 
mantenimiento para administraciones de la plataforma.
\item[Anexo D - Guía de desarrollo:] Primeras pautas para crear o 
modificar módulos del propio sistema así como la creación de nuevas 
herramientas.
\item[Anexo E - Guía del usuario:] Destinado al usuario final de la 
aplicación de escritorio y \emph{web}.
\item[Anexo F - Material entregado:] Lista exhaustiva de todos los 
componentes, guías y herramientas entregados a la finalización del 
proyecto en su primer prototipo.
\end{description}

\cleardoublepage




% Estado del arte
% Estado del arte

\capitulo{Estado del arte}{arte}{
Para entender las técnicas y diseño que se van a utilizar en todo el 
proyecto, se incluye un repaso por la historia de las redes de 
ordenadores y las bases de datos, así como la evolución de los 
sistemas de trabajo y de información. También se mostrarán algunas 
de las características más importantes de todos los recursos 
utilizados para el desarrollo.
}

\section{El pasado}
Desde que Alan Turing (1912-1954) definiera la máquina con su propio nombre 
\cite{Turing} en 1936, el mundo de la informática o ciencia de la 
información, ha sufrido un crecimiento a marchas forzadas de una forma 
súbita comparada con el resto de las ciencias. En el momento que se 
concreta la definición de algoritmo \footnote{Conjunto de reglas que 
expresa la manera de resolver un problema o un tipo específico de 
problemas, en un número finito de pasos o cómputos de 
expresiones \cite{DiccEncTec}.} comienza la base de la era de la 
denominada ``sociedad de la información''.

\subsection{Hasta 1930: La prehistoria}
Sin que se definiera aún esta ciencia, toda la atención estaba 
centrada en resolver sencillos problemas matemáticos de forma 
automática. Mediante válvulas y tubos de vacío, se realizaban 
experimentos con ondas de radio que emulaban sistemas 
\cite{IEEEhistoria}.

Operaciones tan básicas como las puertas lógicas que sirven como base 
a los paradigmas informáticos se crearon en 1924 \footnote{Walther 
Bothe construyó una puerta lógica AND para usarla en experimentos 
físicos, por lo cual recibió el premio Nobel de Física en 
1954\cite{IEEEhistoria}.}. Con estas herramientas empezó a surgir el 
concepto de lenguajes de programación (aún no con ese nombre) que 
eran necesarios para traducir un lenguaje formal matemático a un 
conjunto de instrucciones o componentes que definieran una 
\emph{estructura} electrónica que desempeñara una determinada función. 
Los estudios se centraban en solventar los costes de crear máquinas 
especializadas en una única tarea. Se registran los laboratorios Bell, 
con la idea de buscar soluciones electrónicas para cálculos básicos 
matemáticos. Gracias a esto, Vannervar Bush pudo completar su máquina 
que resolvió por primera vez ecuaciones diferenciales de manera 
autónoma.

\subsection{1930 - 1945: El inicio}
Kurt Gödel se adentró en los lenguajes formales anteriormente comentados 
\cite{OxfordReference}, dando paso a que el propio Turing inventara la primera 
máquina programable capaz de realizar diferentes acciones 
según las instrucciones administradas. Con ello, el matemático pudo realizar 
el primer juego de ajedrez donde el adversario era una máquina 
completamente autónoma. El cálculo de cada jugada podía durar varias 
horas.

En 1941, se inventa, por parte de Konrad Zuse, la computadora 
programable y completamente automática. Se llamó ``K3'' (véase 
figura \ref{fig:z3} y los ordenadores actuales tienen muchas 
similitudes con ella. A partir de ese año, las sucesiones de 
computadoras creadas por parte de grupos de científicos 
y matemáticos aumentaron en capacidad y redujeron sus limitaciones, 
con lo que se encontraron nuevos problemas que solventar, aunque 
seguían siendo máquinas que no se podían comunicar unas con otras.

\begin{figure}[h]
	\centering
	\includegraphics[scale=2]{images/z3.png}
	\caption[El ordenador ``K3'']{El equipo de Konrad Zuse podía almacenar hasta 64 palabras.}
	\label{fig:z3}
\end{figure}


\subsection{1945 - 1960: La agencia \emph{ARPA} \index{ARPA|see{\emph{ARPANET}}}}
En estos años Turing realiza un estudio sobre la comunicación en 
las computadoras, pero en este caso, atendiendo a la interacción con 
los asistentes humanos. Fue uno de los primeros pasos para entender el 
avance que ocurrió como consecuencia de la guerra fría, la creación 
de ARPA \footnote{Agencia de Proyectos de Investigación Avanzada. 
Organización creada para investigación de defensa que originalmente 
sólo disponía de una pequeña oficina, sin investigadores ni 
científicos \cite{Tanenbaum}.}, la organización que más adelante 
diera a conocer uno de los inventos más importantes del siglo XX, la 
red de ordenadores.

La empresa IBM empezó a crear las computadoras de forma industrial, ya 
programables, y estandarizó el lenguaje ensamblador \cite{Ensamblador} 
como forma de programación para sus máquinas. A la vez, lenguajes 
como Fortran y Cobol, que aún se pueden utilizar, salieron a la luz 
como sustitutos para este engorroso lenguaje, que no era escalable y mucho menos 
portable, ya que cada \software estaba realizado para una máquina con 
unas especificaciones físicas muy concretas. Estos lenguajes 
utilizan un compilador \footnote{Programa especial que convierte 
instrucciones a alto nivel en código máquina o ensamblador 
\cite{Ensamblador}.} desarrollado generalmente por la misma empresa 
que provee el conjunto de herramientas para el desarrollo de 
aplicaciones.

\subsection{1960 - 1970: Creación de \emph{ARPANET} \index{Internet!\emph{ARPANET}}}
Los inventos fueron sucediéndose de año en año, aumentando la 
capacidad de cada máquina y reduciendo sus costes. Aún era pronto 
para que aparecieran los ordenadores personales, pero el número de 
computadoras creció, y con ello algunas necesidades como la interacción 
entre las mismas. Es aquí cuando por parte de la organización ARPA 
se aglutinaron algunos de los primeros conceptos que iban apareciendo sobre 
las redes de ordenadores. También se aplica, en 1969, el primer 
protocolo de comunicación en redes, complejo conocido como NCP 
\footnote{\emph{Net Control Protocol}. Conjunto de primitivas que 
permiten la realización de tareas de comunicación a alto nivel 
\cite{Tanenbaum}.}. Al final de la década, la fibra óptica se 
empieza a utilizar para comunicaciones en redes militares que aumenta 
de forma considerable la velocidad de las mismas.

Por otro lado, se dan las primeras pinceladas a los conceptos de 
programación estructurada, en la que actualmente se basan los 
lenguajes más modernos y potentes. El mismo grupo que creó años 
antes el lenguaje Cobol, introduce en el mercado la primera solución 
al almacenamiento masivo de datos en formato electrónico, el IDS 
(Almacén de Datos Integrado \cite{WDatabase}). Desde ese momento, las 
bases de datos y las redes estuvieron estrechamente unidas.

\subsection{1970 - 1980: Evolución de la comunicación}
\label{subsec:Los70}
Una vez la base de las comunicaciones fue creada, los distintos grupos 
de investigación aplicaron todas las técnicas aprendidas a los 
sistemas en red. Esto suponía que los datos en bruto no era lo único 
que se compartía, sino conjuntos de datos coherentes y estructurados. 
Empezaron a enviarse ficheros enteros con protocolos como FTP 
(\emph{File Transfer Protocol}) y se crearon almacenes de datos a los 
que se accedía por medio de una red. Modelos tan importantes 
actualmente como TCP/IP \footnote{Arquitectura que se cimenta en los 
dos protocolos TCP e IP, que hace posible la conexión entre múltiples 
redes heterogéneas \cite{Tanenbaum}. (Ver página 
\pageref{subsec:tcpip}).} en los que se basa Internet aparecieron como 
respuesta al problema que se presentaba a los desarrolladores para 
implementar comunicaciones entre ordenadores. El primer correo 
electrónico llegó a su destino gracias a Ray Tomlison en 1971.

Por parte de IBM, en esta década se crearon las bases de datos 
relacionales, que, al contrario que sus predecesoras, permiten la 
navegación por medio de enlaces entre los mismos, como se muestra en 
la figura \ref{fig:relationalDB}. Después se estandarizó el lenguaje 
SQL\index{base de datos!SQL}, \footnote{\emph{Structured Query 
Language}. Lenguaje creado para el manejo y la búsqueda de datos en 
los gestores de bases de datos relacionales.} que se conserva hasta 
nuestros días con ligeros cambios.

\begin{figure}[H]
	\centering
	\includegraphics[scale=0.6]{images/relational.png}
	\caption[Esquema relacional de una base de datos]{Los esquemas relacionales permiten enlazar unos datos con otros.}
	\label{fig:relationalDB}
\end{figure}

Los ordenadores se podía comunicar sin dificultades gracias a 
la aparición de \emph{Ethernet} que utilizaba un cable único para 
unir varias computadoras de forma local. También, Microsoft y Apple se 
crearon con el fin de proveer los primeros ordenadores personales. El 
número de máquinas creció de forma imprevista hasta para estas 
empresas \cite{NuevasRedes}.

\subsection{1980 - 1990: La era de los ordenadores personales}
Desde el punto de vista de las redes, esta década fue la época del 
afianzamiento de las tecnologías anteriormente desarrolladas. 
Aparecieron sistemas que posibilitaban una mayor transparencia en las 
comunicaciones, como los servidores DNS \footnote{Sistema de Nombres 
de Dominio. Organiza las máquinas dentro de dominios y hace una 
correspondencia de un nombre con la dirección de un 
\emph{host}\cite{Tanenbaum}.} que permitían localizar otras máquinas 
a  partir de un identificador más ``amigable'' que un número de 
128 \emph{bits}.

En 1983, ARPANET se separa de los fines militares para pasar las redes al 
ámbito civil. Para muchos, esta fue la creación de Internet\index{Internet}. La 
aparición de lenguajes y tecnologías sufrió un crecimiento 
exponencial, apareciendo organizaciones como GNU \footnote{\emph{GNU 
is Not Unix.} Proyecto para crear un sistema operativo completamente 
libre bajo licencia GPL \cite{LinuxKernel}.}, Cisco Systems, Adobe, 
etc. Surgen, también, los estándares que servirán para el desarrollo 
de las interfaces de comunicación actuales, como XML (\emph{eXtended Markup 
Language}).

Con el auge de la nueva forma de programación orientada a objetos, 
se crean las nuevas bases de datos, que establecieron todo su 
contenido como entidades coherentes. Esto sirvió también para una optimización 
en la utilización de los enlaces o claves entre datos.

\subsection{1990 - 2000: La aparición de las \emph{puntocom}}
\label{subsec:Los90}
La inversión necesaria para poder tener un ordenador personal se fue
reduciendo al cabo del tiempo, haciendo posible que las familias pudieran 
disponer de una o varias máquinas en su propio hogar.

El \emph{Word Wide Web} (www\index{Internet!www}) se crea como una nueva forma de entender 
\index{Internet}, donde usuarios y empresas pueden publicar su información de 
manera sencilla y accesible desde cualquier parte del mundo con acceso 
a una línea telefónica. Es aquí cuando el modelo de trabajo de las empresas 
y la forma de enseñanza empieza a modificarse. Aparecen por primera 
vez (aunque de forma rudimentaria) las plataformas 
\emph{online}, que permiten acceder a información y procesos de forma 
remota desde un ordenador personal, como son \emph{RPC}, y CORBA\index{CORBA} 
de las que se hablará en la sección \ref{sec:comunicacion}.

Muchas empresas aprovechan este ``boom'' para modificar sus modelos de 
negocio y aparecer en el escaparate mundial. Se publican portales 
donde compartir información de forma anónima, con soporte para 
realizar preguntas y contestarlas sobre todos los aspectos del 
conocimiento (web social). La información pasa a ser parte de la sociedad de manera 
pública y donde se acuñarán términos como ``la nube'' y los 
``sistemas distribuidos''. La cantidad de información contenida en 
Internet empieza a ser imposible de tratar con el formato actual y 
se plantea hacer cambios estructurales para incorporar la Web 
Semántica, y la Web 2.0. Sun Microsystems crea \index{Java} su máquina 
virtual en 1995.

Durante este periodo, las limitaciones de velocidad y capacidad se van 
reduciendo de manera exponencial (como se muestra en la figura 
\ref{fig:modem-growth}), lo que permite que la información que se puede 
compartir aumente y varíe, en forma de vídeos, música, etc. Las bases 
de datos se convierten en almacenes gestionados automáticamente, 
llamados \emph{Data Warehouse} que supone otro escalón más para el 
negocio y las transacciones monetarias por Internet.

\begin{figure}
	\centering
	\includegraphics[scale=0.8]{images/modem-growth.png}
	\caption[Velocidad de conexión]{Velocidad de conexión comparada con la ley de Moore \cite{CloudComputing}.}
	\label{fig:modem-growth}
\end{figure}

Además, aparecen nuevos dispositivos de comunicación inalámbrica, como el 
\emph{Bluetooth} \footnote{Red inalámbrica de corto alcance creada por 
el consorcio de L.M. Ericsson, IBM, Intel, Nokia y Toshiba 
\cite{Tanenbaum}.}, \emph{Wifi} \footnote{Red inalámbrica LAN que 
utiliza protocolos derivados del 802.11 \cite{Tanenbaum}.} y GPRS, 
\footnote{Servicio de Radio de Paquetes Generales. Es la generación 
2.5 de los sistemas inalámbricos en móviles. Los paquetes van por 
encima de las redes GSM \cite{Tanenbaum}.} que permiten ampliar aún 
más la difusión de Internet y solventar más limitaciones físicas. 
Podemos ver en la figura \ref{fig:internet-growth} la evolución en la 
década de los noventa del número de entradas DNS que era accesible a 
través de Internet.

\begin{figure}[h]
	\centering
	\includegraphics[scale=0.68]{images/internet-growth.png}
	\caption[Entradas en los DNS]{Número de entradas en los DNS públicos \cite{CloudComputing}.}
	\label{fig:internet-growth}
\end{figure}

\subsection{2000 - 2010: El crecimiento exponencial}
No sólo ha crecido el número de nodos de red en Internet, sino 
también la forma de concebir los sistemas de computación y 
almacenamiento de datos. Es cuando se crean las grandes granjas de 
ordenadores, se empiezan a sustituir los supercomputadores por 
múltiples ordenadores personales conectados en red que abaratan los 
costes de mantenimiento y escalabilidad. Aparece la nueva estructura 
de bases de datos en la ``nube'' para compartir información 
remotamente y de forma transparente. Con ellas se desarrollan 
\emph{APIs} \footnote{Interfaz de Programación de Aplicaciones. 
Conjunto de funciones y procedimientos que ofrece cierta biblioteca 
para ser utilizada por otro \emph{software} \cite{Tanenbaum}.} de código 
libre para que las distintas plataformas puedan acceder a esas bases de 
datos desde cualquier parte del mundo.

Además, evolucionan los pequeños dispositivos portátiles, con gran 
capacidad de cálculo y almacenamiento, que, mediante baterías, permiten 
una autonomía de muchas horas, sin necesidad de cables ni otras 
limitaciones físicas. Para muchos, estos dispositivos podrán llegar a 
sustituir a los ordenadores como se conocen en la actualidad, 
reduciendo su tamaño y aumentando su velocidad y potencia.

Una nueva forma de trabajo a distancia se empieza a poner en práctica 
en algunas administraciones y empresas, el ``teletrabajo''. Para ello 
se utilizan sistemas de acceso remoto, de los que se ha hablado antes. 
En el \emph{Valle del Silicio}, en California (Estados Unidos), muchas 
empresas tecnológicas apuestan por dar a sus empleados un día a la 
semana, en el que pueden trabajar a distancia desde sus casas, aumentando 
la satisfacción de los trabajadores.

Como una consecuencia lógica del aumento de máquinas, tanto 
portátiles como fijas, de diferentes empresas y con diferentes 
soluciones, se crea un nuevo problema, la heterogeneidad de la forma 
de trabajar de estos dispositivos.

\section{El presente}
\label{sec:presente}

Todas las miradas están centradas en la infraestructura del 
\emph{cloud computing} \index{cloud computing|see{``nube''}} o 
``nube'' \index{sistema distribuido!``nube''}. Ningún 
organismo oficial, negocio, centro de enseñanza o de investigación se 
puede concebir sin la ayuda de esta tecnología, que ahorra cantidades 
muy importantes de capital en personal y recursos físicos. Por otro 
lado, se invierten grandes fortunas en el desarrollo e implantación de 
nuevas soluciones empleando este sistema.

Aparecen lo que se conoce como los sistemas SOI (Infraestructuras 
Orientadas a Servicio) que solucionan las nuevas exigencias de 
flexibilidad y cambio que necesitan los modelos de negocio. 
Implementan arquitecturas orientadas a servicio (SOA) que contienen 
componentes (tanto físicos como lógicos) fácilmente reemplazables y 
que aportan homogeneización y escalabilidad. Partiendo de una 
plataforma rígida, como se encuentran en los sistemas tradicionales, 
se debe conseguir una respuesta ``on demand'' situando el servicio 
como cliente.

La desventaja más importante de estos sistemas es la definición de un mapa 
claro de dependencias entre componentes (debido a su complejidad), lo 
que aporta la necesidad de predecir de forma más concreta la carga que 
tiene cada componente. Es por esto que entra en juego el reparto de 
carga dinámico que viene siendo uno de los grandes problemas a los que 
se enfrentan los desarrolladores e investigadores en esta época. Como 
solución parcial, se añaden sistemas de monitorización para obtener 
información en tiempo real del conjunto del sistema. Se verá más 
adelante que en este mismo proyecto, se han utilizado estas 
herramientas.

Es por esto que casi todas las grandes compañías de informática están
poniendo cada vez más empeño en ofrecer más y mejores recursos 
\emph{cloud computing}: SalesForge.com (Forge.com) Amazon (Amazon EC2 y 
S3), Microsoft(Windows Azure), Google (Google Apps Engine), IBM 
(SmartCloud)\ldots Se puede ver en la figura \ref{fig:eCloudComputing} 
la tendencia por adaptar las nuevas tecnologías surgidas de la 
``nube''.

\begin{figure}[h]
	\centering
	\includegraphics[scale=0.6]{images/eCloudComputing.png}
	\caption[Evolución de la ``nube'']{El desarrollo de la ``nube'' es la gran apuesta de esta 
	década.}
	\label{fig:eCloudComputing}
\end{figure}

Es aquí donde toma importancia el concepto de \emph{virtualización} \index{virtualización}. 
En los sistemas tradicionales, el \hardware limitaba los componentes 
\software a utilizar, pero gracias a la virtualización, la parte 
física de los sistemas pasa a un segundo plano, siendo invisible para 
los desarrolladores de soluciones. Aunque a primera vista suponga una 
gran carga para estos sistemas, ya que \emph{aplana} las características 
de cada \hardware , a la larga aporta una gran ventaja a la hora de 
modificarlos. Siendo un concepto muy importante en este 
proyecto, sólo se podrá tratar por encima, ya que el sistema de 
virtualización usado en el mismo es sencillo debido a la finalidad del 
mismo.

El dinero invertido en estas nuevas tecnologías suele venir de 
capitales privados con el apoyo de grandes empresas, pero se está 
viendo el aumento de los productos \emph{open source} 
\footnote{``Código abierto es el término con el que se conoce al 
\software distribuido y desarrollado libremente.''\cite{LinuxKernel}} 
que son mantenidos en parte por universidades y 
desarrolladores anónimos. Es decir, la ``nube'' es accesible incluso a 
empresas y comunidades de poco capital. Muchos de los servicios se 
pueden obtener (en versión reducida generalmente) gratis, como es el 
caso del almacenamiento en \emph{Dropbox}, la gestión de proyectos en 
\emph{GitHub}, publicación web en \emph{Godaddy}, etc. Por tanto, los 
servicios son tanto B2C (de negocio para clientes) y B2B (de negocio 
para negocios, término referido también al \emph{outsourcing}).

Uno de los proyectos más ambiciosos \emph{open source} para la 
creación de una plataforma \emph{cloud computing} completamente gratis se 
llama OpenNebula \index{Internet!OpenNebula}, desarrollado en parte por investigadores de la 
Universidad Complutense de Madrid \cite{openNebula}. Este sistema 
provee al usuario, mediante un conjunto de herramientas (tanto 
comerciales como libres) de su propia ``nube'' accesible desde 
cualquier parte del mundo (véase figura \ref{fig:openNebula}).

Haciendo referencia a este proyecto, en la actualidad no existe ningún 
producto que atienda todos los requisitos y objetivos propuestos en 
este documento. Las plataformas más parecidas, desarrolladas por 
empresas para su propio uso, como es el caso de la plataforma de 
programas de trabajo de Telefónica sólo proveen un entorno 
\emph{software}.

\begin{figure}
	\centering
	\includegraphics[scale=0.6]{images/opennebula.png}
	\caption[OpenNebula]{OpenNebula utiliza servicios comerciales como soporte.}
	\label{fig:openNebula}
\end{figure}

\section{El futuro}

Es difícil predecir el futuro teniendo en cuenta la velocidad que ha 
tomado la evolución de la tecnología en este último medio siglo, 
aunque está claro que el desarrollo de Internet será la pieza clave 
que mueva la sociedad de la información.

Las limitaciones físicas que ahora existen serán un mero recuerdo y 
los nuevos tipos de computadores, como los biológicos o los cuánticos 
sustituirán a los ordenadores personales actuales, reducidos en 
tamaño y aumentados en capacidad y potencia. La cantidad de 
información que se pueda almacenar no será más un problema y los 
esfuerzos se centrarán en la velocidad y optimización.

Ya ha empezado la era de la domótica\index{domótica}, donde las propias casas son 
elementos integrados en sistemas complejos conectados a Internet, 
automáticos e independientes. En realidad, la domótica no es más que 
manejar \hardware de forma remota. En la figura \ref{fig:domotica} se 
muestra un dispositivo de control de \emph{hardware} acoplado a una 
casa.

\begin{figure}[h]
	\centering
	\includegraphics[scale=0.6]{images/domotica.png}
	\caption[Componente domótico]{La domótica puede ser un componente más en la 
	construcción inmobiliaria.}
	\label{fig:domotica}
\end{figure}


\section{Los sistemas y modelos de trabajo}
Al igual que las arquitecturas de las redes de ordenadores, las 
plataformas de trabajo convencional se pueden clasificar dependiendo 
de su distribución y sus tipos de comunicación. A continuación, se 
establecen los tipos básicos de sistemas que serán clave para el 
entendimiento de este proyecto.

\subsection{Sistema individual}
Se puede considerar como una red de un solo nodo como se puede ver en 
la figura \ref{fig:sindividual}. Es aquel sistema 
donde no existe comunicación y todo depende de un mismo elemento. Es 
así como se ha trabajado hasta antes de la revolución de la sociedad 
de la información de forma tecnológica.

\begin{figure}[h]
	\centering
	\includegraphics[scale=0.5]{images/sindividuales.png}
	\caption[Sistema individual]{Máquinas en un sistema individual.}
	\label{fig:sindividual}
\end{figure}

Los dispositivos portátiles son un caso especial de un sistema 
individual, sólo que permite el desplazamiento del propio nodo entre 
otras redes, por lo que puede salir de esta categoría y aplicarse a 
las siguientes. Actualmente se consideran sistemas portátiles también 
a los \emph{Smartphones} \footnote{Teléfono móvil con funciones 
avanzadas de gestión de programas y acceso a Internet.}, 
\emph{Tablets} \footnote{Dispositivo portátil de 
mayor capacidad y tamaño que un móvil, generalmente para uso 
multimedia e Internet, aunque cada vez más para acciones más 
complejas.} y \emph{Netbooks} \footnote{Ordenadores de tamaño reducido 
con una gran autonomía debido a la batería que poseen, pensados para 
trabajos ligeros y el acceso a Internet}. En la última década, las empresas que 
tradicionalmente se dedicaban a los ordenadores personales, han 
cambiado su modelo de producción a estos dispositivos portátiles, 
desarrollando plataformas completas y \software específico para ellas.

\begin{figure}[h]
	\centering
	\includegraphics[scale=0.5]{images/android.png}
	\caption[Android]{El sistema operativo Android está 
	disponible para \emph{tablets} y \emph{smartphones}.}
	\label{fig:android}
\end{figure}

Las grandes empresas se han abierto camino en estos dispositivos 
gracias a los nuevos sistemas operativos Android (figura 
\ref{fig:android}), Windows Phone 7 e 
iOS, de Google, Microsoft y Windows respectivamente. El resto se ha 
ido desplazando para ocupar una cuota mínima en el mercado actual, 
debido a su falta de apoyos por la comunidad de desarrolladores, como 
son Symbian de Nokia, Palm de ahora HP y el Sistema RIM de los 
dispositivos Blackberry.

\subsection{Sistemas centralizados}
Son aquellos en los que los nodos dependen de otros especiales que 
administran y manejan toda la comunicación. En terminología técnica 
se consideran como los paradigmas de Cliente/Servidor (figura 
\ref{fig:client-server}, varios nodos o 
clientes se conectan a un servidor para solicitar servicios)
\cite{Tanenbaum}. Generalmente, los nodos centrales son de mayor 
tamaño o capacidad, y también tienen una mayor responsabilidad. Sin 
ellos, la comunicación no existe y, por lo tanto, el desarrollo es 
imposible. Es aquí donde se permite compartir, además de datos, 
\emph{hardware} y recursos que se ponen a disposición de los nodos 
del sistema.

\begin{figure}
	\centering
	\includegraphics[scale=0.6]{images/scentralizado.png}
	\caption[Sistema centralizado]{Esquema de un sistema centralizado.}
	\label{fig:client-server}
\end{figure}

También es donde aparecen los sistemas por niveles, que son la 
evolución propia de los sistemas centralizados, donde un nodo 
central de un sistema es un nodo cliente más en un nivel superior, 
como podemos ver en la figura \ref{fig:multilevel-server}.

\begin{figure}
	\centering
	\includegraphics[scale=0.3]{images/sjerarquico.png}
	\caption[Sistema jerárquico]{Representación de un sistema centralizado por niveles.}
	\label{fig:multilevel-server}
\end{figure}

El problema más típico de estos sistemas es su falta de robustez, 
siendo dependiente de un sistema central. Además, la escalabilidad 
está limitada por la potencia de estos nodos centrales, que tienen que 
atender todas las peticiones.

\subsection{Sistemas distribuidos \index{sistema distribuido}}
Suponen la evolución a los sistemas centralizados. Son un conjunto de 
nodos donde, de forma transparente, funcionan como uno único 
\cite{Tanenbaum}. Su escalabilidad es mucho mayor que los sistemas 
centralizados, debido a la homogeneidad de roles. Para poder controlar 
estas arquitecturas, aparecen el \index{\emph{middleware}}. Estas 
plataformas no pueden concebirse sin los términos 
\emph{virtualización} y \emph{comunicación}.

Estos sistemas ofrecen unas características que se listan a 
continuación \cite{DAD}.

Ventajas:
\begin{itemize}
	\item Reducción del coste del computador y del acceso a la red.
	\item Compartición de recursos.
	\item Escalabilidad.
	\item Tolerancia a fallos. Al no depender de un sólo sistema 
	central, existe la posibilidad que después de que ocurra un fallo, 
	el conjunto del sistema no se vea afectado.
\end{itemize}

Inconvenientes:
\begin{itemize}
	\item Múltiples puntos de fallo.
	\item Seguridad. Considerado como el gran inconveniente en estos 
	sistemas debido a los innumerables puntos de acceso.
\end{itemize}

\subsection{Internet y la ``nube''}
Comprende múltiples sistemas centralizados y distribuidos, organizados 
de manera jerárquica que ofrecen servicios a los nodos de la red. Desde la 
descarga de información, hasta la compartición, uso y publicación de 
la misma. El término ``nube'' \index{sistema distribuido!``nube''} 
aparece como referencia a los servicios 
(generalmente almacenaje de información) en Internet, que se proveen, 
siendo transparente al usuario donde se encuentran, aunque 
disponible para su acceso en todo momento. La limitación de capacidad 
pasa a un segundo plano ya que entran en juego las granjas de 
ordenadores anteriormente mencionadas \cite{CloudComputing}.

\begin{figure}[h]
	\centering
	\includegraphics[scale=0.6]{images/cloudComputing.png}
	\caption[La ``nube'']{La ``nube'' provee servicios a multitud de dispositivos.}
	\label{fig:cloudcomputing}
\end{figure}

Esta ``nube'' es accesible desde cualquier dispositivo con conexión a 
Internet (figura \ref{fig:cloudcomputing}, lo que hace que sea 
necesario disponer de plataformas que 
acepten varios tipos de formatos, tanto estándar como privativos. Esto 
ha supuesto un gran avance para homogeneizarlas y poder obtener 
servicios sin preocuparse de restricciones y limitaciones. Está 
compuesta por capas que son intermediarios para acceder a las máquinas 
finales.

\subsection{Sistemas \emph{clusters} y \emph{grid}}
Son casos especiales de sistemas centralizados, en los que el servicio 
principal es la capacidad de cálculo. Están formados por nodos 
genéricos baratos, fácilmente reemplazables. Los \emph{grids} son 
agrupaciones de \emph{clusters} (como los de la figura 
\ref{fig:cluster}) accesibles a través de Internet, 
poniéndolos a disposición para que realicen una tarea por nodo.

\begin{figure}[h]
	\centering
	\includegraphics[scale=0.9]{images/cluster.png}
	\caption[\emph{Cluster}]{Varios ordenadores personales forman un \emph{cluster}.}
	\label{fig:cluster}
\end{figure}

Actualmente muchas empresas ofrecen servicios de \emph{clusters} y 
\emph{grid} para otras empresas, que suponen un gran ahorro. Está 
estimado que los negocios que contratan estos servicios pueden 
ahorrarse hasta un millón de dólares al año \cite{amazon}.

\section{Las plataformas virtuales \index{virtualización}}
Debido a la extensión que supondría comentar todas las plataformas de 
desarrollo actuales (o las usadas en el proyecto), sólo se pondrá 
atención a aquellas que formen parte del sistema a emplear en el 
proyecto, es decir, las plataformas virtuales.

La estrategia de la virtualización está compuesta por múltiples 
campos. Desde la virtualización de un sistema operativo completo, es 
decir, emularlo en otro sistema, por lo que se podría tener varios 
sistemas operativos en ejecución a la vez en una única máquina, hasta 
la emulación de almacenamiento, que conforma un grupo de 
``contenedores'' de datos, incompatibles entre ellos y heterogéneos, 
formando una única unidad. En los servidores actuales se utiliza esta 
técnica para conseguir una serie de ventajas explicadas a 
continuación \cite{vmware}:

\begin{itemize}
	\item Rápida incorporación de nuevos recursos para los servidores virtualizados.
	\item Reducción de los costes de espacio, de \hardware y consumo necesario.
	\item Administración global centralizada y simplificada.
	\item Permite gestionar los centros de cálculo como un \emph{pool} 
	de recursos o agrupación de toda la capacidad de procesamiento, 
	memoria, red y almacenamiento disponible en nuestra 
	infraestructura.
	\item Mejora en los procesos de réplica de sistemas. Tanto para 
	pruebas como para \emph{backups}.
	\item Aislamiento: Un fallo general de sistema de una máquina 
	virtual no afecta al resto de máquinas virtuales.
	\item Migración en caliente de máquinas virtuales (sin pérdida de 
	servicio) de un servidor físico a otro, eliminando la necesidad de 
	paradas planificadas por mantenimiento de los servidores físicos.
	\item Balanceo dinámico de máquinas virtuales entre los servidores 
	físicos, como se ha comentado en \ref{sec:presente}.
\end{itemize}

Existe un caso específico en el que sólo se virtualiza una parte 
concreta del sistema. Esto también supone muchas ventajas en cuanto al 
desarrollo de aplicaciones y el uso de determinados servicios en 
cualquier tipo de máquina.

\subsection{Java y su máquina virtual}
\label{subsec:Java}
Java \index{Java} se puede denominar como un conjunto de herramientas (bastante 
amplio) que ha obtenido mucha fama desde su creación. Aparte de ser 
un lenguaje de programación orientado a objetos, tiene por detrás 
muchos servicios que lo hacen el lenguaje más utilizado del mundo en 
la creación de aplicaciones. Consta de \cite{java2}:

\begin{enumerate}
	\item El lenguaje de programación orientado a objetos.
	\item Bibliotecas estándar para todas las plataformas.
	\item Compilador a \emph{bytecode}, entendible por la máquina 
	virtual.
	\item La máquina virtual de Java que ejecuta ese código. Hace que 
	cualquier programa escrito en este lenguaje funcione en cualquier 
	plataforma independientemente del sistema instalado.
\end{enumerate}

Es una plataforma que compila e interpreta código. Pero se pueden 
incluir también todas las otras herramientas que permiten añadir 
versatilidad a este conjunto:

\begin{enumerate}
	\item La edición estándar de Java contiene bibliotecas para la 
	realización de interfaces gráficas.
	\item También dispone de \emph{applets} que permiten añadir 
	aplicaciones a entornos web de manera sencilla.
	\item La edición de empresa (o J2EE) dispone de herramientas web 
	para la creación de páginas y servicios web con la sintaxis 
	normal. Estas herramientas son JSP, Servlets y Filtros.
	\item Compilador a \emph{bytecode}, entendible por la máquina 
	virtual.
	\item Dispone de conectores para cualquier base de datos del 
	mercado, por lo que es sencillo utilizar estos productos en las 
	aplicaciones.
	\item Aunque no es propio de la plataforma de Java, Google ha 
	desarrollado un SDK \footnote{\emph{Software Development Kit.} 
	Conjunto de bibliotecas para el desarrollo de aplicaciones en una 
	determinada plataforma.} a disposición de los usuarios, para poder 
	crear aplicaciones para su sistema operativo Android en este 
	lenguaje.
\end{enumerate}

Todo no son ventajas. Java tiene mala fama porque al ser un sistema 
virtualizado, utiliza más recursos y es más lento que los lenguajes 
completamente compilados, como lo son C o C++. Aún así, es 
indiscutible que debida a la gran versatilidad de la que dispone, es 
una herramienta indispensable para un programador.

\subsubsection*{J2EE: Java 2 Enterprise Edition}
\index{Java!J2EE}
\label{subsubsec:j2ee}

Esta versión más amplia, como ya se ha comentado, dispone de varias 
herramientas para la implementación de aplicaciones con un alto grado 
de comunicaciones, desde páginas web simples a servicios web 
sustentados en servidores. También se ha desarrollado gracias a Java 
programas como GlassFish y Tomcat, que permiten incluir estos 
servicios mediante un único archivo y son accesibles por red. La 
versión de empresas contiene todo lo necesario para crear servidores 
de aplicaciones, en diversas capas distribuidas, flexible y escalable.

A partir de ahí, se han creado multitud de \framework s y otras 
herramientas para un desarrollo más sencillo y más estructurado.

Actualmente el futuro de Java es un poco incierto debido a que la 
empresa Oracle compró Sun Microsystem, y ha optado por políticas que 
impiden el desarrollo libre de esta plataforma.

\section{El almacenamiento}
\label{sec:almacenamiento}
Se obtiene una correspondencia entre las plataformas de 
almacenaje masivo de información con los distintos tipos de sistemas 
ya que están basadas en ellos. Se puede decir que 
el almacenamiento es la segunda parte más importante de la 
informática, después del tratamiento de esos datos. Es por ello que 
la vía de las bases de datos ha ido en constante evolución de forma 
paralela a las comunicaciones entre ordenadores.

No se explicará la capa más baja del almacenaje de la información, 
como son los sistemas de ficheros ya que no compete a este proyecto. 
Los sistemas de almacenaje pasan a ser grandes administradores de 
información que no sólo se preocupan por contener los datos, sino 
también por ofrecer soluciones para la búsqueda y el mantenimiento de los 
mismos.

\subsection{Sistemas Gestores de Bases de Datos}
Son conocidos como el producto final de la base de datos \index{base de datos} en sí. 
Engloban al conjunto de herramientas necesarias para utilizar todo lo 
referente a esas bases de datos. Desde el lenguaje de consulta SQL 
(ver página \pageref{subsec:Los70}) hasta los motores de optimización 
y búsqueda, administración de usuarios y seguridad, comunicaciones, 
incluso algunas proveen herramientas específicas para acceder a la 
información desde otras plataformas mediante conectores.

Existen en la actualidad multitud de soluciones para almacenamiento, 
adecuadas al uso que se les vaya hacer, con gran cantidad de datos o 
plataformas especialmente ligeras para una rápida utilización.

En este proyecto se van a centrar en usar dos específicas para dos 
necesidades distintas, como son MySQL y SQLite.

\subsection{SQLite \index{base de datos!SQLite}}
Este gestor tiene una característica muy importante y es que es 
extremadamente ligero. Tan ligero que se usa en dispositivos de baja 
capacidad como los anteriormente mencionados \emph{smartphones}. No 
está preparado para manejar grandes cantidades de información, ni de 
tratamiento avanzado de datos, como, por ejemplo, las claves ajenas de 
las tablas. Dispone de una interfaz muy sencilla y accesible desde 
cualquier lenguaje de programación, sacando el máximo rendimiento 
desde C. A costa de su flexibilidad, se han sacrificado las tareas de 
mantenimiento y de optimización, por lo que no cuenta con procesos que 
ayuden a obtener la información de manera más rápida. No dispone de 
acceso remoto a la información y se comporta como un fichero binario 
en el sistema operativo.

Se puede ver en la figura \ref{fig:sqlite} que dispone de una 
estructura mucho más sencilla que el resto de soluciones que existen 
en la actualidad. SQLite es código libre y se puede obtener de manera 
totalmente gratuita.

\begin{figure}
	\centering
	\includegraphics[scale=0.9]{images/sqlite.png}
	\caption[Estructura de SQLite]{SQLite no dispone de procesos de optimización.}
	\label{fig:sqlite}
\end{figure}

\subsection{MySQL \index{base de datos!MySQL}}
Muy popularizado en Internet, este gestor gratuito pero con licencia 
doble (comercial y libre) es una solución media para la mayoría de 
los proyectos existentes. Contiene todas las opciones de un gran 
gestor empresarial, pero con un tamaño más reducido y menos 
optimizado. Es perfecto para cantidades moderadas de información (como 
las utilizadas en pequeñas y medianas empresas), y se han realizado 
multitud de versiones para optimizar sus capacidades, como Google, que 
la modificó para aumentar su capacidad de mantenimiento (por motivos 
estructurales), o Facebook, que actualmente sigue utilizando los 
servidores con MySQL.

Posee múltiples procesos de mantenimiento que necesitan estar en 
constante ejecución, además les da acceso de forma remota, por 
lo que es idóneo para sistemas con un alto grado de comunicación.

Existen multitud más de soluciones, como son Oracle Database, la 
\emph{open source} por excelencia, PostgreSQL, también empresariales como 
las proporcionadas por IBM y los grandes almacenes de datos.

\section{La comunicación}
\label{sec:comunicacion}
Todas las comunicaciones en los ordenadores actuales poseen una misma 
estructura definida por los protocolos que intervienen en ellas. 
Aunque cada protocolo tenga una función, debido a que el sistema está 
definido por capas, comparten muchos de los elementos. Para empezar 
esta sección, se definirá el modelo actual que se aplica a Internet, 
ya que es el que se utilizará en este proyecto. Después, se dará un 
repaso a las distintas tecnologías que están por encima de ese 
modelo, y las utilidades y características más importantes de cada una.

\subsection{El modelo TCP/IP\index{TCP/IP}}
\label{subsec:tcpip}
Es un modelo de referencia que se usó para ARPANET, a pesar de existir 
otro estándar definido en 1983 desarrollado por la ISO (Organización 
Internacional de Estándares), que es el que se ha mantenido hasta nuestros 
días. Se basa en una pila de capas, cada cual usa sus capas inferiores 
y les proporciona más funcionalidad. Así, la capa de menor nivel es 
la más básica.

Estas capas tienen unos protocolos que actúan con la información que 
le llega de una capa superior, como se ve en la figura \ref{fig:tcpip}.

\begin{figure}
	\centering
	\includegraphics[scale=0.6]{images/tcpip.png}
	\caption[Modelo TCP/IP]{Modelo TCP/IP y modelo OSI desarrollado en 1983.}
	\label{fig:tcpip}
\end{figure}

\subsubsection*{Capa de red}
Es la capa de menor nivel y se encarga de las tareas más básicas de 
comunicación. Realmente, localiza una dirección MAC 
\footnote{Dirección única asignada a cada interfaz de red, 
correspondiente con la subcapa de Control de Acceso al Medio 
\cite{Tanenbaum}.} correspondiente a una máquina y envía bit a bit 
toda la información suministrada por capas superiores. Además añade 
comprobaciones de errores a nivel de bit. Esta es dependiente de 
la estructura de red, y existen versiones para redes LAN, ARPANET, 
Radio de paquete y SATNET.

\subsubsection*{Capa de interred}
Permite mantener la transparencia entre distintas redes, por lo 
que es el centro de unión entre toda la arquitectura. Su trabajo es 
permitir que los equipos inyecten datos dentro de cualquier red y que 
éstos viajen a su destino de manera independiente.

Su protocolo principal es IP (Protocolo de Internet) y da nombre al 
modelo.

\subsubsection*{Capa de transporte}
Permite a una máquina origen y destino mantener una conversación 
completa, independientemente de cuantos ``saltos entre redes'' tengan 
que realizar los datos.

Los protocolos de esta capa son TCP (Protocolo de Control de 
Transporte), orientado a conexión y seguro que también da nombre al modelo, y 
UPD (Protocolo de Datagrama de Usuario) para aplicaciones que no 
deseen control de flujo.

\subsubsection*{Capa de aplicación}
Es la última capa que está en contacto con el usuario. Es sobre ella 
donde el \emph{software} de las máquinas puede crear sus propios 
protocolos, teniendo por debajo todos los anteriormente descritos. Los 
protocolos más famosos comprenden:

\begin{description}
	\item[HTTP:] Protocolo de Transferencia de Hipertexto 
	\cite{CommYRedes}. Es base para la WWW (\ref{subsec:Los90}). Puede 
	transferir cualquier tipo de archivo a través de Internet. Está 
	estructurado de forma Cliente/Servidor orientado a transacciones.
	\item[FTP:] Es el protocolo estándar de Internet para envío de 
	archivos. Tiene la característica de usar dos tipos de conexión, 
	una de control y otra de datos \cite{TCPIllustrated}.
	\item[SMTP:] Protocolo Simple de Transferencia de Correos. Es 
	parecido al protocolo FTP pero su finalidad es enviar y recibir 
	mensajes enviados de un usuario a otro a través de la red 
	\cite{TCPIllustrated}.
	\item[DNS:] Sistema de Nombres de Dominio. Se compone de una base 
	de datos distribuida que hace la correspondencia entre nombres de 
	máquinas y sus direcciones.
\end{description}

Los protocolos deben ser implementados sobre un sistema de envío, 
desde los más simples, como los \emph{sockets}, hasta plataformas 
complejas como \emph{.NET} \index{.NET}. A continuación se describirán los más 
importantes.

\subsection{Sockets}
\index{socket}
Son la abstracción básica que contienen los diferentes sistemas 
operativos para enviar datos mediante los distintos protocolos de 
nivel de transporte anteriormente detallados. A bajo nivel, funcionan 
como posiciones de memoria donde se escriben y se leen datos, conectan 
dos procesos de forma bidireccional entre dos máquinas 
\cite{SistemasOperativos}.

Su funcionamiento es sencillo:

\begin{enumerate}
	\item Se crea el \socket asignándole un protocolo de transporte 
	(generalmente TCP o UDP).
	\item Al \socket se le asigna un puerto de la máquina al que 
	conectarse o al que va a escuchar (dependiendo si es del lado del 
	cliente o del servidor). Desde ese momento, este está a la espera 
	y preparado para la comunicación.
	\item Mediante operaciones de lectura y escritura (similares a la 
	escritura y lectura de ficheros) se envían y reciben datos entre 
	los dos extremos de la comunicación.
	\item Se cierran los \emph{sockets} de conexión. 
\end{enumerate}

Con este mecanismo se han creado los protocolos complejos y las 
plataformas que se describen a continuación.

\subsection{Llamadas a procedimientos remotos}
Se trata de una implementación por medio de \emph{sockets} que permite 
invocaciones remotas de funciones. De forma transparente al 
programador, se realiza la comunicación, se procesa en la otra 
máquina y se devuelve el resultado. Son la base de las plataformas 
distribuidas, basadas en objetos \cite{SistemasOperativos}. Su 
evolución directa son las \emph{Invocaciones a Procedimientos Remotos}.

\subsection{Java RMI}
Compone el conjunto de herramientas implementadas para la máquina 
virtual \index{Java!máquina virtual} de Java (ver sección 
\ref{subsec:Java}), que permiten realizar aplicaciones distribuidas, 
ocultando las comunicaciones para que el desarrollador programe como si 
todo el contenido estuviera en la misma máquina.

Se basa en una estructura de servicios a los que, mediante un registro de 
nombres, los clientes acceden y obtienen la localización del servicio 
requerido \cite{JavaRMI}. Estos pueden invocar métodos que no se encuentran en su 
propia máquina. Cada aplicación contiene una interfaz donde se 
especifican los métodos remotos y los parámetros de los mismos.

Como se puede ver en la figura \ref{fig:rmi}, el \emph{RMI Registry} 
debe ser accesible desde todos los nodos de la red. Estos, mediante 
una etiqueta, obtienen la localización del objeto dentro de la red. 
Después son conectados a la máquina que lo posee y los objetos son 
enviados serializados \footnote{Un objeto serializado es su 
representación íntegra de forma textual, que se puede escribir o leer 
de un fichero (al igual que enviar como un dato más). Generalmente es 
usado para guardar objetos en el estado actual.} para su posterior uso.

\begin{figure}
	\centering
	\includegraphics[scale=0.4]{images/rmi.png}
	\caption[Estructura de Java RMI]{Arquitectura de Java RMI \cite{DAD}.}
	\label{fig:rmi}
\end{figure}

El inconveniente más importante es que esta plataforma sólo se puede 
utilizar en \emph{software} para se ejecute en la \index{Java!máquina virtual} de Java, 
limitando el número de programas que pueden interactuar con ella.

\subsection{CORBA}

Con una arquitectura muy parecida a Java RMI, es una plataforma más 
compleja que permite interactuar independientemente del lenguaje de 
programación utilizado en el \emph{software}. Esto hace posible que se 
realicen comunicaciones entre distintas arquitecturas de forma 
transparente al usuario y al desarrollador. A pesar de los lenguajes 
utilizados, CORBA\index{CORBA} está basado en el paradigma orientado a objetos, por 
lo que, aunque se estén utilizando lenguajes como C o Matlab, se 
realizan invocaciones a métodos.

Se caracteriza por definir las interfaces de acceso con un lenguaje 
llamado IDL (Lenguaje de Definición de Interfaces) que, al igual que 
en Java RMI, todos los nodos deben poseer. Está considerada como una 
de las arquitecturas más difíciles de utilizar debido a su gran 
complejidad y al número de configuraciones posibles. Actualmente está 
siendo utilizada en proyectos de gran envergadura donde se requiere 
una alta interoperabilidad.

En la figura \ref{fig:corba} se puede apreciar las distintas capas de 
la arquitectura.

\begin{figure}[h]
	\centering
	\includegraphics[scale=0.3]{images/corba.png}
	\caption[Estructura de CORBA]{Componentes de un sistema CORBA \cite{DAD}.}
	\label{fig:corba}
\end{figure}

\subsection{Framework .NET\index{.NET}}
Esta plataforma fue desarrollada por Microsoft para dar salida a los 
lenguajes en los que se puede desarrollar \emph{software} en Windows, 
aunque también dispone de plataforma en sistemas Unix. Al igual que 
Java, utiliza una ``máquina virtual'' para ejecutar el código 
compilado (CLR o \emph{Common Language Runtime}, figura \ref{fig:dotnet} \cite{dotNet}). 
Gracias a ello, se aglutinaron todos los lenguajes de la plataforma 
(C\#, J\#, Visual C++, Visual Basic y .NET \index{.NET!ASP.NET}) 
creando una solución muy versátil. Además, permite la comunicación, 
al igual que Java RMI\index{Java!Java RMI}, transparente 
al usuario entre estos programas. La potencia es mayor cuando se 
pueden incluir servicios web nativamente.

Para desarrollar en esta plataforma se dispone de un entorno 
especializado, Visual Studio, que añade más posibilidades a .NET.

\begin{figure}
	\centering
	\includegraphics[scale=0.35]{images/dotnet.png}
	\caption[Estructura de .NET]{Arquitectura de .NET \cite{DAD}.}
	\label{fig:dotnet}
\end{figure}

En los últimos años, ha surgido un auge ya que Microsoft permitió 
añadir compatibilidad con .NET a su sistema operativo para 
dispositivos móviles, \emph{Windows Phone 7}.

\subsection{Servicios Web\index{servicio web}}
Como ya se comentó en la página \pageref{sec:presente}, la tendencia 
pasa a ser la creación de servicios destinados tanto a otras empresas 
como a clientes finales. Estos servicios se pueden considerar como 
aplicaciones completas a las que se accede remotamente.

Por motivos de seguridad, se aceptó que la forma de comunicar estos 
servicios fuera a través de los métodos tradicionales de Internet, es 
decir, mediante los protocolos ya establecidos, como es HTTP. Este 
protocolo se quedó pequeño pronto y se plantearon añadir más 
funcionalidades por encima sin perder esa base. Es así como se crean 
los protocolos de intercambio de objetos (más complejos que simple 
información), como SOAP, que utiliza un formato XML para el envío e 
identificación de los objetos (véase figura \ref{fig:serviciosweb}).

Como en las anteriores plataformas descritas, los servicios web se 
definen mediante una interfaz común llamada WSDL\index{servicio 
web!WSDL} (\emph{Web Service Description Language}), donde los clientes 
pueden obtener toda la información de los métodos a utilizar, sus 
parámetros y sus excepciones. Surgen vías alternativas que aprovechan 
los métodos de comunicación actuales de los 
navegadores\index{Internet!navegador} de Internet, para obtener el 
máximo beneficio de estos servicios. Un ejemplo es REST, cada vez más 
extendido, el cual realiza las peticiones mediante la URL de la propia 
página, introduciendo como parámetros el método a ejecutar. Este 
sistema es usado en sitios web como Facebook y son de gran ayuda a los 
desarrolladores ya que no necesitan ningún tipo de librería externa 
para poder utilizar por completo los distintos servicios que provee la 
empresa.

\begin{figure}
	\centering
	\includegraphics[scale=0.55]{images/serviciosweb.png}
	\caption[Servicios web]{Estructura típica de un sistema con varios servicios web.}
	\label{fig:serviciosweb}
\end{figure}

\section{Las limitaciones}
A lo largo de este capítulo se ha demostrado cómo se han ido 
superponiendo unas soluciones a otras, al igual que aparecían nuevos 
problemas y nuevas necesidades surgidas a partir de las propias 
soluciones. Aún queda mucho camino para resolver las grandes 
limitaciones de potencia y capacidad, pero otras pueden ser 
solventadas mediante las ideas surgidas de los investigadores, 
estudiantes e ingenieros.

\subsection{La interoperabilidad \index{Java!interoperabilidad} 
\index{.NET!interoperabilidad}}
Desde que se creó el ``K3'', cada empresa, investigador o científico 
ha implantado sus soluciones sin preocuparse de la interacción con su 
entorno, pero en estos tiempos, ha sido una de las necesidades más 
urgentes debido a la gran cantidad de dispositivos y productos que 
existen en el mercado. Las empresas que sólo miran por su producto se 
condenan al fracaso a largo plazo. La competencia es mayor y 
las guerras de productos incomodan cada vez más al cliente, que, por lo 
general, no tiene una compañía como exclusiva.

Gracias a que el desarrollo de las distintas soluciones en cuanto a 
comunicación y almacenaje se ha mantenido al margen de productos muy 
concretos, parece ser que la solución al problema de interoperabilidad 
pasa obligatoriamente por estos campos. Con la aparición de la 
``nube'' y los sistemas virtuales, se ha salvado este gran obstáculo 
(en mayor o menor medida). Pero sigue faltando soluciones más 
genéricas para \software . Aparte de Java y .NET, no existen otras 
plataformas que permitan ejecutar programas independientemente de la 
máquina ni del sistema operativo. Es lógico entonces que si se desea 
solventar un problema se necesite utilizar cualquiera de esos dos sistemas.

\subsection{Acceso remoto}
Se ha avanzado mucho desde que cada máquina era independiente una de 
otra, y todo lo necesario estaba contenido en la misma. Ahora la 
sustitución de una de estas en una red o un sistema complejo es casi 
inmediato, sin tener que realizar \emph{backups} concretos ni 
alterando la estructura de dicha red. Cada nuevo producto que se crea 
en la actualidad deja de verse como algo instalado en una máquina, 
sino más bien como un conjunto de programas que necesitan estar 
comunicados en servidores, clientes, terminales, etc. Por supuesto, 
sigue habiendo \hardware que aún es imposible su acceso remoto, y es 
este proyecto el que pretende solucionar en gran medida este problema, 
usando las tecnologías anteriormente mencionadas, y aprendiendo de la 
evolución de los sistemas.

\cleardoublepage


% Análisis
% Análisis

\capitulo{Análisis}{analisis}{
Este capítulo está centrado en la especificación de la plataforma 
RLF, cómo se concivió y todas las funcionalidades que ofrece. Además 
de los requisitos establecidos que debe cumplir.
}

\section{Introducción}
\label{sec:introduccion}
El análisis de este producto ha sido guiado por los estándares 
recogidos por la Agencia Espacial Europea para los proyectos de 
Ingeniería del Software \cite{esa}. La estructura de contenidos se ha 
mantenido aunque se han modificado algunos de los apartados 
considerados innecesarios ya que se especifican para un conjunto de 
documentos que describan un producto, y no para uno solo, como ocurre 
en este caso.

Para el desarrollo de RLF Prototype se ha contado con dos clientes del ámbito 
educativo, cada uno especializado en un área que han determinado las 
funciones que contiene el sistema. Así, se cuenta con un cliente 
ingeniero industrial (\textbf{Cliente A}) y con un cliente ingeniero 
informático (\textbf{Cliente B}).

En primera instancia, se especifican las condiciones generales del 
proyecto, así como sus capacidades más notables. A continuación, se 
listan todos los requisitos que fueron impuestos para la creación del 
producto, así como en conjunto de acciones a realizar por los 
distintos usuarios. Por último, se tratarán los requisitos desde el 
punto de vista del diseño.

Una vez todos los aspectos del análisis estén recogidos y 
establecidos, se pasará a diseñar desde el punto de vista técnico el 
sistema RLF.

\section{Descripción general}
\subsection{Capacidades generales}
El producto RLF cumple la finalidad de ofrecer servicios a los 
usuarios. La lista de capacidades que se muestra a continuación es el 
resultado de un primer análisis, que posteriormente serán 
desarrollados para una mayor comprensión:

\begin{itemize}
\item RLF provee de herramientas, generalmente con componentes 
\emph{hardware}, a los usuarios, los cuales pueden utilizarlas de 
forma exclusiva durante un determinado tiempo.
\item Las herramientas ofrecen un conjunto de acciones a realizar. 
Estas acciones son configuradas por parte del usuario para adecuarse a 
sus necesidades.
\item El usuario está informado en todo momento del estado del 
sistema y de las acciones comandadas, así como de las herramientas a 
las que tiene acceso.
\item Cada acción tiene la libertad de utilizar todo su potencial 
mediante servicios externos a la plataforma RLF.
\item Los usuarios pueden acceder al sistema en tiempo real sin limitaciones 
horarias, siendo posible utilizarlo de manera remota.
\item Se establecen un conjunto de normas, o maneras de trabajar, para 
desarrollar nuevas herramientas, que se comportan de manera estándar 
para su inmediato uso.
\item Proporciona mecanismos para asegurar la integridad tanto del 
\emph{software} como \emph{hardware} que se ofrece mediante la 
plataforma RLF.
\end{itemize}

\subsection{Restricciones generales}
Como cada producto \emph{software}, RLF cumple unas determinadas 
restricciones que se aclararon también en el análisis inicial. Son 
las que han condicionado el diseño de la plataforma:

\begin{itemize}
\item Ningún componente de la RLF está limitado por el 
sistema que lo contiene, ya que forma parte de un sistema 
multiplataforma, siendo las herramientas la única excepción.
\item La plataforma RLF responde de igual manera independientemente de 
dónde se encuentren los distintos componentes, siendo posible 
contenerla en una misma máquina, o en una red de computadores.
\item Se asegura la integridad de cada componente por separado, sin 
permitir que por un mal uso o por conjestión aparezcan errores 
inesperados.
\end{itemize}

\subsection{Características de los usuarios}
Se identifican principalmente tres tipos de usuario para los que va 
destinada el sistema. Cada uno desempeña un rol distinto, y por lo 
tanto, se determinan distintas características y responsabilidades 
para cada uno. Cabe esclarecer que dentro de cada categoría de 
usuario, puede haber distintos niveles, pero que estos no son 
determinantes en el uso de la plataforma. Los roles que se especifican 
a continuación no son jerárquicos ni comparten acciones:

\begin{itemize}
\item \textbf{Administradores:} Personal técnico que gestiona y 
mantiene la plataforma RLF en constante funcionamiento. Realizan 
tareas necesarias para asegurar la disponibilidad completa del 
producto, así como gestionar todos los datos que componen el conjunto 
de información manejado por el sistema. Es por ello que deben poseer 
conocimientos medios de utilización de bases de datos y diversos 
sistemas operativos.
\item \textbf{Desarrolladores:} Profesionales encargados de crear las 
herramientas para su posterior utilización por parte de los clientes. 
Deben cumplir un conjunto de normas aplicadas al \emph{framework} para 
desarrollar el \emph{software} necesario que la plataforma RLF pueda 
manejar. Tienen una interacción muy limitada con el sistema.
\item \textbf{Clientes:} Son los usuarios finales de la plataforma, 
que mediante las distintas aplicaciones cliente (aquellas que dan 
acceso a RLF) pueden utilizar las herramientas asignadas.
\end{itemize}

Se entiende que una misma persona puede poseer varios roles. No se 
tienen en cuenta los usuarios indirectos de la plataforma, que no 
acceden a ella, como son los profesores en un entorno educativo que 
definen el conjunto de herramientas a la cual un usuario tiene acceso, 
o los administradores de red que se encargan del buen funcionamiento 
de la misma. Estas acciones recaerán directamente en el usuario 
administrador. 

\subsection{Entorno operacional}
En este punto se analizan las necesidades tecnológicas del sistema 
desde el punto de vista de los usuarios. Se debe distinguir entre las 
necesidades de cada uno de los roles anteriormente comentadas:

\begin{table}[H]
\begin{center}
\begin{tabular}{|| p{12cm} ||}
	\hline
	\hline
	\multicolumn{1}{|| c ||}{\textsc{Administrador}}\\
	\hline
	\hline
	Ordenador portatil con acceso a redes \emph{Ethernet} e 
	inalámbricas. No es necesario que disponga de interfaz gráfica. Se 
	requiere Java JRE y MySQL Query Browser (véase el apéndice 
	\ref{cap:despliegue}).\\
	\hline
	\hline
\end{tabular}
\end{center}
	\caption{Entorno operacional del administrador.}
	\label{tab:entornoadmin}
\end{table}

\begin{table}[H]
\begin{center}
\begin{tabular}{|| p{12cm} ||}
	\hline
	\hline
	\multicolumn{1}{|| c ||}{\textsc{Desarrollador}}\\
	\hline
	\hline
	Ordenador con acceso a redes. Requiere los elementos para 
	poder desarrollar las herramientas en el lenguaje deseado. 
	Dependiendo de la configuración de la plataforma puede ser 
	necesario un cliente FTP.\\
	\hline
	\hline
\end{tabular}
\end{center}
	\caption{Entorno operacional del desarrollador.}
	\label{tab:entornodes}
\end{table}

\begin{table}[H]
\begin{center}
\begin{tabular}{|| p{12cm} ||}
	\hline
	\hline
	\multicolumn{1}{|| c ||}{\textsc{Cliente}}\\
	\hline
	\hline
	Cualquier tipo de máquina que contenga una acceso a la red y la 
	máquina virtual Java.\\
	\hline
	Dispositivo portatil tipo \emph{smartphone} o \emph{tablet} con 
	acceso a redes inalámbricas y navegador de Internet compatible con 
	Javascript.\\
	\hline
	\hline
\end{tabular}
\end{center}
	\caption{Entorno operacional del cliente.}
	\label{tab:entornoclient}
\end{table}

\section{Requisitos del usuario}
A continuación se listan todos los requisitos impuestos por los 
clientes A y B (ver sección \ref{sec:introduccion})
para el diseño de la plataforma RLF. Están clasificados dependiendo 
de su tipo, si aportan funcionalidad o si registren, y contienen una 
subclasificación dependiendo de a qué componente se refieran. Las 
características de un requisito vienen dadas por su identificador, 
prioridad en cuanto a implantación, fuente, necesidad y descripción.

% Requisitos del usuario.
% Requisitos del usuario

\newcounter{rcap}
\newcounter{rres}

\subsection{Requisitos de capacidad}

\reqcapacidad{General: \emph{Hardware} remoto}
{Cliente A}{Alta}{Esencial}
{
El \emph{hardware} será controlado desde la plataforma sin necesidad de 
configurarlo en el lugar físico donde se encuentra.
}

\reqcapacidad{General: \emph{Hardware} distribuido}
{Cliente A}{Alta}{Esencial}
{
El \emph{hardware} que se pondrá a disposición de los clientes estará 
distribuido en varios ordenadores.
}

\reqcapacidad{General: \emph{Hardware} exclusivo}
{Cliente A}{Alta}{Esencial}
{
Todo \emph{hardware} sólo podrá ser usado por un único cliente, sin 
que el uso de otros pueda influirle.
}

\reqcapacidad{General: \emph{Hardware} multidisciplinar}
{Cliente A}{Alta}{Esencial}
{
La plataforma RLF contendrá una interfaz de acceso genérica para 
cualquier tipo de \emph{hardware}.
}

\reqcapacidad{General: Plataforma accesible}
{Cliente A}{Alta}{Esencial}
{
La plataforma será accesible para los clientes desde cualquier 
ordenador conectado a Internet, siendo posible la configuración para 
limitar este acceso.
}

\reqcapacidad{General: Acceso central}
{Cliente A}{Alta}{Esencial}
{
Independientemente de dónde se encuentre el \emph{hardware} el que el 
cliente quiere utilizar, se accederá desde una misma dirección para 
todo el sistema.
}

\reqcapacidad{General: Plataforma modular}
{Cliente B}{Alta}{Esencial}
{
La plataforma RLF estará compuesta por módulos que interactuarán entre 
ellos, al menos dos, los servidores locales o \textbf{laboratorios} 
\index{laboratorio} y el servidor central o \textbf{proveedor} 
\index{proveedor}.
}

\reqcapacidad{General: Cliente del la plataforma}
{Cliente B}{Baja}{Opcional}
{
Para acceder a la plataforma se dispondrá de un \textbf{cliente} 
\index{cliente} que no limitado por el sistema operativo donde se 
encuentre.
}

\reqcapacidad{General: Servicios como \emph{hardware}}
{Cliente B}{Media}{Esencial}
{
El \emph{hardware} deberá ser encapsulado como un tipo de servicio 
genérico, llamado \textbf{herramienta}\index{herramienta}.
}

\reqcapacidad{Proveedor: Base de datos central}
{Cliente A}{Alta}{Esencial}
{
El proveedor contendrá la información en una base de datos central a la 
que accederá cada vez que se requiera utilizar dicha información.
}

\reqcapacidad{Proveedor: Usuarios clientes}
{Cliente A}{Alta}{Esencial}
{
Los clientes poseerán un nombre de usuario, una contraseña, un 
email y un rol específico para acceder a las herramientas.
}

\reqcapacidad{Proveedor: Usuarios administradores}
{Cliente B}{Media}{Esencial}
{
Los administradores serán autentificados en la base de datos del 
proveedor para controlar el acceso a sistemas críticos.
}

\reqcapacidad{Proveedor: Servicio Web}
{Cliente B}{Alta}{Esencial}
{
El proveedor contendrá una interfaz de acceso para la conexión a 
través de Internet y así permitirá a los clientes utilizar los 
servicios de la plataforma.
}

\reqcapacidad{Proveedor: Información en tiempo real}
{Cliente B}{Alta}{Esencial}
{
Los clientes obtendrán la información del proveedor actualizada, tanto 
del estado de las herramientas como el de los laboratorios.
}

\reqcapacidad{Proveedor: Información de las herramientas}
{Cliente A}{Alta}{Esencial}
{
La información de las herramientas estará contenida en la base de datos 
del proveedor para obtener una mayor respuesta cuando los clientes la 
soliciten.
}

\reqcapacidad{Proveedor: Información de los laboratorios}
{Cliente A}{Alta}{Esencial}
{
Al ser un sistema con un componente central, la información de cada 
laboratorio, es decir, dónde se localiza y cómo puede contarse a él, 
se encontrará en la base de datos del proveedor.
}

\reqcapacidad{Proveedor: Acciones}
{Cliente A}{Alta}{Esencial}
{
Las acciones que se podrán llevar a cabo mediante la interfaz de 
acceso al proveedor serán conexión o desconexión de un cliente, 
obtención de la información de las herramientas y su estado, y 
reserva de las mismas.
}

\reqcapacidad{Proveedor: Monitor}
{Cliente B}{Baja}{Opcional}
{
El proveedor tendrá otra interfaz de acceso limitada únicamente para 
conectar y desconectar del sistema, y además obtener el estado actual 
de las herramientas.
}

\reqcapacidad{Proveedor: Seguridad en el acceso}
{Cliente B}{Media}{Opcional}
{
Todas las comunicaciones del proveedor con los clientes estarán 
debidamente cifradas ya que pueden atravesar redes inseguras.
}

\reqcapacidad{Proveedor: Registro de herramientas}
{Cliente B}{Media}{Esencial}
{
El proveedor dará acceso a los laboratorios para que registren 
herramientas, generando un identificador único para cada herramienta y 
una clave de uso.
}

\reqcapacidad{Proveedor: Tiempo de acceso máximo}
{Cliente A}{Media}{Esencial}
{
A cada usuario se le asignará un tiempo de acceso máximo contado en 
minutos, que una vez sobrepasado se expulsa del sistema.
}

\reqcapacidad{Proveedor: Reserva de herramientas}
{Cliente A}{Alta}{Esencial}
{
El proveedor se encargará de bloquear las herramientas reservadas por 
los clientes para que no puedan ser usadas.
}

\reqcapacidad{Laboratorios: Comunicación}
{Cliente B}{Alta}{Esencial}
{
Toda la comunicación entrante y saliente de un laboratorio se 
realizará mediante red.
}

\reqcapacidad{Laboratorios: Base de datos}
{Cliente B}{Alta}{Esencial}
{
Cada laboratorio contendrá su propia base de datos, que elimina el tráfico 
en la red y permite hacer operaciones locales más rápidas. En ellas 
se almacenará la información de las herramientas.
}

\reqcapacidad{Laboratorios: Administración}
{Cliente A}{Alta}{Esencial}
{
Los laboratorios contarán con una interfaz de acceso para poder 
administrarlos sin necesidad de una parada total del sistema.
}

\reqcapacidad{Laboratorios: Registro de herramientas}
{Cliente A}{Alta}{Esencial}
{
Cada laboratorio estará al cargo de varias herramientas, que deberá 
gestionar y administrar.
}

\reqcapacidad{Laboratorios: Acceso a las herramientas}
{Cliente B}{Alta}{Esencial}
{
Para eliminar el tráfico del proveedor, los laboratorios aceptarán
peticiones de los clientes que previamente se hayan conectado al 
sistema y hayan reservado las herramientas necesarias.
}

\reqcapacidad{Laboratorios: Eliminación de herramientas}
{Cliente A}{Baja}{Esencial}
{
Se podrá eliminar una herramienta concreta de un laboratorio si así se 
requiere.
}

\reqcapacidad{Laboratorios: Mantenimiento}
{Cliente A}{Baja}{Esencial}
{
Para aportar seguridad, los laboratorios tendrán un estado de 
``mantenimiento'' donde no se permite la comunicación con los 
clientes, en el cual se podrá realizar tareas de administración del 
mismo.
}

\reqcapacidad{Laboratorios: Ejecuciones de las herramientas}
{Cliente A}{Alta}{Esencial}
{
Los laboratorios sólo permitirán una instancia de ejecución por 
herramienta. Esto conlleva la creación de una cola de peticiones para 
la ejecución de estas.
}

\reqcapacidad{Laboratorios: Comunicación con las herramientas}
{Cliente A}{Alta}{Esencial}
{
Un laboratorio se comunicará con las herramientas que tiene asignadas 
mediante un mecanismo asíncrono para que una herramienta con fallos no 
bloquee la ejecución.
}

\reqcapacidad{Laboratorios: Parada de emergencia}
{Cliente A}{Baja}{Opcional}
{
Cada laboratorio podrá ser parado de manera urgente por parte de un 
administrador, que cierra todas las ejecuciones activas y elimina 
las pendientes. Después, el laboratorio se desactivará.
}

\reqcapacidad{Laboratorios: \emph{Logs}}
{Cliente B}{Baja}{Esencial}
{
Cada laboratorio poseerá su propio mecanismo de registro de eventos y 
errores, que puede ser consultado por un administrador.
}

\reqcapacidad{Laboratorios: Información en tiempo real}
{Cliente A}{Media}{Esencial}
{
Cada vez que un laboratorio cambie su estado, el proveedor será 
informado, incluso cuando se realizan paradas de emergencia.
}

\reqcapacidad{Herramientas: Acciones}
{Cliente A}{Alta}{Esencial}
{
Cada herramienta podrá realizar varias acciones, todas relacionadas 
con el mismo \emph{hardware}. El usuario elegirá una y la 
configurará. No será posible ejecutar a la vez dos acciones de la misma 
herramienta.
}

\reqcapacidad{Herramientas: Configuración de las acciones}
{Cliente B}{Alta}{Esencial}
{
Cada acción tendrá asociada unos parámetros de salida y de entrada, 
con un tipo de datos establecido y una descripción. Mediante la 
definición de los valores de estos parámetros la acción se configura 
para su ejecución. Los parámetros de salida sólo se leerán por parte 
del cliente cuando la acción ha terminado.
}

\reqcapacidad{Herramientas: Excepciones y estado final de la acción}
{Cliente A}{Media}{Esencial}
{
Las acciones, durante su ejecución, podrán lanzar excepciones que 
informan al usuario de un error, además, cada finalización tendrá 
asociada un estado incluido por el desarrollador.
}

\reqcapacidad{Herramientas: Constantes}
{Cliente A}{Baja}{Esencial}
{
Las herramientas podrán contener varias constantes con un tipo de datos 
concreto y un valor que no varía desde que se registra la herramienta.
}

\reqcapacidad{Herramientas: Atributos}
{Cliente A}{Baja}{Esencial}
{
Los laboratorios definirán los atributos propios de cada herramienta, 
donde se incluyen el identificador, la clave, nombre, descripción, rol, 
versión y administrador responsable de la herramienta.
}

\reqcapacidad{Herramientas: Entrada y salida}
{Cliente B}{Alta}{Esencial}
{
Las herramientas dispondrán de una entrada y salida textual para poderse 
comunicar con el usuario.
}

\reqcapacidad{Herramientas: Servicios externos}
{Cliente A}{Alta}{Esencial}
{
Para dar funcionalidad extra a cada herramienta, las acciones podrán 
tener asociadas servicios externos, que se activarán en la ejecución y 
que pueden ser usadas por los clientes.
}

\reqcapacidad{Herramientas: Archivo de descripción}
{Cliente A}{Alta}{Esencial}
{
Para definir una herramienta por completo y registrarla en la 
plataforma RLF, se dispondrá de un fichero con un formato establecido 
que será leído en el momento del registro en el laboratorio.
}

\reqcapacidad{Herramientas: Librerías \emph{libtool}}
{Cliente B}{Alta}{Opcional}
{
\index{\emph{libtool}}Como forma de abstracción, se incluirán
librerías para la comunicación de la herramienta con el laboratorio 
correspondiente. Estas librerías proveerán las funciones de acceso y 
desconexión, lectura y escritura de parámetros, lectura de constantes 
y atributos, lanzamiento y lanzamiento de excepciones.
}

\reqcapacidad{Herramientas: Herramientas de datos}
{Cliente A}{Media}{Opcional}
{
Esta categoría especial de herramientas permitirá acceder a varios 
clientes a la vez, pero sin interactuar con el \emph{hardware}. No 
disponen de entrada textual, ni de parámetros de configuración. Sólo 
tendrán una acción asignada.
}

\reqcapacidad{Herramientas: Limpiadores}
{Cliente B}{Media}{Esencial}
{
Existirá por cada herramienta una acción especial que establece el 
\emph{hardware} de la herramienta a su estado original. No podrá ser 
utilizada por los clientes y de forma automática se lanzará cuando se 
detecta errores en alguna acción.
}

\reqcapacidad{Herramientas: Tiempo máximo de ejecución}
{Cliente B}{Media}{Esencial}
{
Cada acción poseerá un tiempo máximo de ejecución que el laboratorio 
monitoriza. Cuando ese tiempo es sobrepasado, se parará la ejecución por 
completo y se establece el estado de acción fallida.
}

\reqcapacidad{Cliente: Visualización simultánea}
{Cliente A}{Alta}{Esencial}
{
La interfaz del cliente será capaz de visualizar a la vez varias 
acciones de distintas herramientas.
}

\reqcapacidad{Cliente: Uso de las herramientas}
{Cliente B}{Alta}{Esencial}
{
Se proveerá al usuario de una ``consola'' (con entrada y salida textual) 
para el manejo de cada una de las acciones.
}

\reqcapacidad{Cliente: Acciones}
{Cliente A}{Alta}{Esencial}
{
La interfaz permitirá al cliente conectarse y desconectarse del sistema, 
reservar herramientas, comprobar la información de las mismas y su 
estado en tiempo real, y la ejecución de acciones.
}

\reqcapacidad{Cliente: Reserva múltiple}
{Cliente A}{Alta}{Optativa}
{
Un cliente podrá reservar de manera simultánea varias herramientas a 
las que tiene acceso.
}

\reqcapacidad{Cliente: Monitor}
{Cliente A}{Baja}{Optativa}
{
La aplicación cliente dispondrá con una aplicación auxiliar por la que 
acceder la monitor central.
}

\reqcapacidad{Cliente: Información}
{Cliente A}{Media}{Esencial}
{
Todos los datos públicos de las herramientas serán mostrados al cliente 
por medio de la interfaz, añadiendo las descripciones a los parámetros.
}

\subsection{Requisitos de restricción}

\reqrestriccion{General: Cantidad de comunicación}
{Cliente A}{Media}{Esencial}
{
La información trasmitida en las comunicaciones, así como el número 
de estas deberán ser reducidas al mínimo para no saturar cada componente.
}

\reqrestriccion{General: Sistema escalable}
{Cliente B}{Alta}{Esencial}
{
Todo el sistema será escalable, es decir, que no influya en su 
funcionamiento el hecho de aumentar o disminuir el número de 
laboratorios o clientes.
}

\reqrestriccion{Proveedor: Control de seguridad}
{Cliente B}{Media}{Esencial}
{
Los clientes que no se desconecten de forma válida en el sistema o que 
sufran errores, el proveedor no les permitirá volver a conectarse hasta 
que un administrador dé el visto bueno.
}

\reqrestriccion{Proveedor: Bloqueos de peticiones}
{Cliente B}{Alta}{Esencial}
{
El proveedor contará con mecanismos para atender varias peticiones a la 
vez sin dejar ninguna a la espera.
}

\reqrestriccion{Laboratorios: Portables}
{Cliente B}{Alta}{Esencial}
{
Los laboratorios funcionarán sin problemas en diversos sistemas operativos con 
una única implementación.
}

\reqrestriccion{Laboratorios: Acciones asíncronas}
{Cliente A}{Media}{Esencial}
{
Todas las acciones que se llevan a cabo en los laboratorios serán
asíncronas, para no detener ni bloquear ningún componente que 
interfiera con ellos.
}

\reqrestriccion{Laboratorios: Ejecuciones externas}
{Cliente A}{Alta}{Esencial}
{
Las acciones se ejecutarán en los laboratorios como programas externos 
para no añadir carga a la plataforma.
}

\reqrestriccion{Laboratorios: Máximo número de ejecuciones}
{Cliente A}{Alta}{Esencial}
{
Cada laboratorio poseerá un número máximo de acciones en ejecución 
(configurado por el propio administrador). Cuando ocurre esto, las 
acciones que están pendientes tomarán el estado de espera.
}

\reqrestriccion{Herramientas: Parada}
{Cliente A}{Alta}{Esencial}
{
Las herramientas deberán estar preparadas para sufrir paradas sin 
previo aviso, aunque no es necesario que devuelvan el estado original 
al \emph{hardware}.
}

\reqrestriccion{Cliente: Control de seguridad web}
{Cliente A}{Media}{Esencial}
{
El control de seguridad aplicado a los usuarios para el cliente de 
escritorio no bloqueará la cuenta cuando ocurran errores en el cliente 
web.
}


\section{Casos de uso}
Un caso de uso define una serie de pasos o actividades que deben 
realizarse para llevar a cabo un proceso. Las entidades o personajes 
que participan en un caso de uso se denominan actores.

Este apartado se separa en dos partes. La primera es una 
representación gráfica de los distintos casos de uso ordenados por 
actores (administrador, desarrollador y cliente). En la segunda parte 
se explicarán textualmente algunos de los más influyentes, que 
representan las funcionalidades más importantes; \emph{Publicar 
herramienta} y \emph{Ejecutar acción}.

\begin{figure}[h]
	\centering
	\includegraphics[scale=0.6]{images/casos1.png}
	\caption{Casos de uso: Administrador y desarrollador}
	\label{fig:casos1}
\end{figure}

Como se aprecia en la figura \ref{fig:casos1} el administrador tiene 
responsabilidades de mantenimiento del sistema, y todas sus 
funcionalidades están relacionadas con esta tarea. El desarrollador en 
cambio, dentro del sistema sólo podrá realizar parte de la función 
\emph{Publicar herramienta}. Se explicará más adelante este caso 
particular, el cual requiere de dos actores para poder completarse. En 
la figura \ref{fig:casos2} se comprueba que el usuario necesita haber 
iniciado sesión para realizar el resto de funciones, además de que 
para ejecutar una acción sea necesario reservar la herramienta que la 
contiene.

\begin{figure}[h]
	\centering
	\includegraphics[scale=0.6]{images/casos2.png}
	\caption{Casos de uso: Cliente}
	\label{fig:casos2}
\end{figure}

A continuación se muestra la información de los dos casos de uso 
anteriormente señalados, donde se incluye los prerequisitos para 
llevarlos a cabo, los posibles escenarios secundarios y su descripción 
detallada.

\newcounter{casos}

\begin{table}
\begin{center}
	\stepcounter{casos}
\begin{tabular}{|| c | p{10cm} ||}
	\hline
	\multicolumn{2}{|| c ||}{\textbf{CU-\arabic{casos}: Publicar herramienta}}\\
	\hline
	\textsc{Actores} & Administrador, desarrollador \\
	\hline
	\textsc{Precondiciones} &
	\begin{itemize}
		\item El desarrollador debe haber implementado la herramienta 
		de acuerdo con las normas del \emph{framework}.
		\item El laboratorio debe estar desarmado.
	\end{itemize}\\
	\hline
	\textsc{Postcondiciones} &
	\begin{itemize}
		\item La herramienta estará a disposición de los usuarios para poder 
		utilizarla.
	\end{itemize}\\
	\hline
	\multicolumn{2}{|| c ||}{\textsc{Descripción}}\\
	\hline
	\multicolumn{2}{|| p{14cm} ||}{
	El desarrollador crea una herramienta para incluirla en la 
	plataforma. El administrador obtiene los datos necesarios para 
	registrarla y se lo indica al desarrollador.
	}\\
	\hline
	\multicolumn{2}{|| c ||}{\textsc{Escenario principal}}\\
	\hline
	\multicolumn{2}{|| p{14cm} ||}{
	\begin{enumerate}
		\item El administrador introduce el fichero de configuración 
		en el laboratorio mediante la aplicación de mantenimiento.
		\item El laboratorio pide una nueva clave e identificador al 
		proveedor.
		\item El laboratorio valida la configuración y crea las 
		estructuras en su base de datos y la generada para la propia 
		herramienta.
		\item El laboratorio envía al proveedor la información sobre 
		la nueva herramienta y establece su estado como ``no 
		disponible''.
		\item El administrador recibe la nueva clave y el 
		identificador.
		\item El desarrollador modifica el código para incluir la 
		nueva clave.
		\item El administrador arma el laboratorio.
	\end{enumerate}
	}\\
	\hline
	\multicolumn{2}{|| c ||}{\textsc{Escenario secundario}}\\
	\hline
	\multicolumn{2}{|| p{14cm} ||}{
	\begin{enumerate}
		\item El administrador introduce el fichero de configuración 
		en el laboratorio mediante la aplicación de mantenimiento.
		\item El laboratorio pide una nueva clave e identificador al 
		proveedor.
		\item El laboratorio determina que la configuración no está 
		bien construida y tiene errores.
		\item El administrador recibe el error encontrado.
	\end{enumerate}
	}\\
	\hline
	\hline
\end{tabular}
\end{center}
	\label{caso:CU-\arabic{casos}}
\end{table}

\begin{table}
\begin{center}
	\stepcounter{casos}
\begin{tabular}{|| c | p{10cm} ||}
	\hline
	\multicolumn{2}{|| c ||}{\textbf{CU-\arabic{casos}: Ejecutar acción}}\\
	\hline
	\textsc{Actores} & Cliente \\
	\hline
	\textsc{Precondiciones} &
	\begin{itemize}
		\item La herramienta debe haber sido reservada por el usuario.
	\end{itemize}\\
	\hline
	\textsc{Postcondiciones} &
	\begin{itemize}
		\item La acción se ejecuta en el laboratorio e interactua con 
		el cliente.
	\end{itemize}\\
	\hline
	\multicolumn{2}{|| c ||}{\textsc{Descripción}}\\
	\hline
	\multicolumn{2}{|| p{14cm} ||}{
	El cliente desea ejecutar una acción de una herramienta concreta.
	}\\
	\hline
	\multicolumn{2}{|| c ||}{\textsc{Escenario principal}}\\
	\hline
	\multicolumn{2}{|| p{14cm} ||}{
	\begin{enumerate}
		\item El cliente selecciona la acción a ejecutar y pulsa el 
		botón.
		\item La interfaz muestra los parámetros de entrada y los 
		servicios externos de los que dispone la herramienta.
		\item El cliente introduce el valor de esos parámetros y pulsa 
		``Execute''.
		\item El cliente abre los servicios externos indicados con 
		anterioridad.
		\item La aplicación cliente envía al laboratorio la petición 
		de ejecución y se pone a la espera.
		\item Cuando el laboratorio lo dispone, envía a la aplicación 
		cliente la confirmación de ejecución y la información de 
		conexión para el envío y recepción de datos.
		\item La aplicación cliente se conecta con los nuevos 
		parámetros al laboratorio y empieza a recibir los datos de la 
		ejecución.
		\item El cliente interacciona con la ejecución.
	\end{enumerate}
	}\\
	\hline
	\multicolumn{2}{|| c ||}{\textsc{Escenario secundario}}\\
	\hline
	\multicolumn{2}{|| p{14cm} ||}{
	\begin{enumerate}
		\item El cliente selecciona la acción a ejecutar y pulsa el 
		botón.
		\item La interfaz muestra los parámetros de entrada y los 
		servicios externos de los que dispone la herramienta.
		\item El cliente introduce el valor de esos parámetros y pulsa 
		``Execute''.
		\item El cliente abre los servicios externos indicados con 
		anterioridad.
		\item La aplicación cliente envía al laboratorio la petición 
		de ejecución y se pone a la espera.
		\item El laboratorio devuelve un fallo por no poder ejecutar 
		la herramienta.
		\item La aplicación cliente muestra el error por pantalla.
	\end{enumerate}
	}\\
	\hline
	\hline
\end{tabular}
\end{center}
	\label{caso:CU-\arabic{casos}}
\end{table}

\clearpage
\section{Requisitos del \emph{software}}
Los requisitos que aquí se muestran son la respuesta desde el punto de 
vista del diseño del \emph{software} a los requisitos de usuario 
(tanto de capacidad como de restricción). La estructura es idéntica a 
los anteriores excepto que el tipo cambia a funcional y no funcional.

% Requisitos del software.
% Requisitos del software

\newcounter{rfun}
\newcounter{rnof}

\subsection{Requisitos funcionales}

\reqfuncional{General: Aplicaciones}
{RUC-1, RUC-9}{Alta}{Esencial}
{
El \emph{hardware} será controlado completamente por aplicaciones 
que formarán parte de las herramientas y serán ejecutadas mediante 
peticiones de las aplicaciones cliente.
}

\reqfuncional{\emph{Framework}: Desarrollo versátil}
{RUC-4}{Alta}{Esencial}
{
El \emph{framework} de diseño de herramientas no impondrá 
restricciones en cuanto a lenguaje de programación, plataforma ni 
estructura del código.
}

\reqfuncional{\emph{Framework}: Desarrollo cómodo}
{RUC-41}{Alta}{Esencial}
{
Se permitirán utilizar las funciones del sistema de entrada y salida 
por teclado para la comunicación de las herramientas con el cliente. 
Aunque la información que pasa a través de ellas sea enviada por la red.
}

\reqfuncional{Comunicaciones: Modelo TCP/IP}
{RUC-5, RUC-24}{Alta}{Esencial}
{
Todas las comunicaciones serán basadas en el modelo estándar TCP/IP y 
utilizarán un protocolo especialmente diseñado para RLF. Cada nodo 
contendrá una IP de acceso, y la plataforma se identificará con la IP 
del proveedor.
}

\reqfuncional{Comunicaciones: Dirección de la plataforma}
{RUC-6, RUC-7}{Alta}{Esencial}
{
La plataforma se identificará con la IP del proveedor, que será por 
la que se podrá acceder.
}

\reqfuncional{Comunicaciones: Protocolo de comunicaciones}
{RUC-5, RUC-19}{Media}{Esencial}
{
El protocolo de comunicaciones será utilizado por todos los módulos y 
no variará su implementación.
}

\reqfuncional{Comunicaciones: Protocolo síncrono}
{RUC-24}{Media}{Esencial}
{
El protocolo de comunicaciones será un modelo de petición/respuesta 
síncrono, aunque las acciones globales no lo sean.
}

\reqfuncional{General: Vía de acceso}
{RUC-6}{Alta}{Esencial}
{
El proveedor será el encargado de indicar a las aplicaciones cliente 
dónde se encuentran los laboratorios.
}

\reqfuncional{Proveedor: Acceso a la base de datos}
{RUC-10}{Alta}{Esencial}
{
El administrador podrá insertar, eliminar y modificar datos de la base 
de datos central. Será gestionada mediante la aplicación que el 
propio SGBD provea. El proveedor accederá mediante el driver 
correspondiente en la plataforma de desarrollo.
}

\reqfuncional{Proveedor: Autenticación}
{RUC-11, RUC-12}{Media}{Esencial}
{
La autentificación de cualquier usuario, independientemente del tipo 
de rol, será validada por el proveedor.
}

\reqfuncional{Proveedor: Interfaz de acceso}
{RUC-13}{Alta}{Esencial}
{
Las aplicaciones cliente usarán las funciones contenidas en el 
proveedor, las cuales, serán divididas en dos interfaces, 
con una estructura de servicio web definido por los ficheros WSDL y 
así dar acceso a las aplicaciones cliente.
}

\reqfuncional{Proveedor: Actualizaciones de datos}
{RUC-14}{Alta}{Esencial}
{
Los datos almacenados en el proveedor serán los últimos en actualizar. Por
cada actualización correcta de datos realizadas por los distintos 
módulos se deberá considerar como un cambio permanente.
}

\reqfuncional{Proveedor: Información comprimida}
{RUC-15}{Baja}{Opcional}
{
Cuando un cliente solicite información sobre una herramienta, se 
entregará un conjunto de datos previamente comprimidos y calculados 
para una mayor velocidad de respuesta.
}

\reqfuncional{Proveedor: Información de cada laboratorio}
{RUC-16}{Alta}{Esencial}
{
Cuando un cliente reserve una herramienta, se le entregará la 
información de conexión del laboratorio que la contiene, como es la 
IP y el puerto.
}

\reqfuncional{Proveedor: Funciones web}
{RUC-17, RUC-18}{Alta}{Esencial}
{
Las funciones de las dos interfaces de acceso proveerán parámetros 
de entrada y salida, así como excepciones. Estos objetos serán 
capaces de ser serializados, para poder ser enviados mediante el 
protocolo HTTP y SOAP.
}

\reqfuncional{Proveedor: Seguridad en las herramientas}
{RUC-20}{Alta}{Esencial}
{
Las claves y los identificadores de las herramientas serán únicos y 
no reutilizables. Las claves se generarán mediante funciones 
\emph{hash} a partir de la fecha y el identificador obtenido. 
}

\reqfuncional{Proveedor: Tiempo máximo asignado}
{RUC-21}{Alta}{Esencial}
{
Los clientes tendrán preasignado un tiempo almacenado en la base 
de datos, que será cronometrado por los propios laboratorios en los que 
estén registrados.
}

\reqfuncional{Proveedor: Bloqueo de herramientas}
{RUC-3, RUC-22}{Alta}{Esencial}
{
La base de datos contendrá información de quién tiene reservada
cada herramienta, impidiendo que herramientas que no sean de datos se 
encuentren accesibles por dos clientes, por lo se requerirá un bloqueo 
de escritura y lectura en la información cada vez que se requiera reservar.
}

\reqfuncional{Laboratorios: Localización de las aplicaciones}
{RUC-2}{Alta}{Esencial}
{
En cada laboratorio se almacenarán, de forma local, todas las 
aplicaciones implicadas en las ejecuciones de las herramientas que 
tienen asignadas. 
}

\reqfuncional{Laboratorios: Base de datos}
{RUC-25}{Alta}{Esencial}
{
La base de datos de cada laboratorio contendrá la ruta de acceso a la 
herramienta, su clave y su identificador. Será de baja capacidad y 
portable, por lo tanto, el SGBD será SQLite.
}

\reqfuncional{Laboratorios: Comunicación con las herramientas}
{RUC-32}{Alta}{Esencial}
{
El laboratorio creará una base de datos ligera (SQLite) para ofrecer 
comunicaciones con la herramienta. De manera asíncrona obtendrá los 
datos que se origen en la ejecución de la misma.
}

\reqfuncional{Laboratorios: \emph{Kernel}}
{RUC-26}{Alta}{Esencial}
{
Un laboratorio ofrecerá una capa de gestión llamada \emph{Kernel} que 
será independiente de las comunicaciones con los clientes y las 
ejecuciones de herramientas.
}

\reqfuncional{Laboratorios: Funciones del \emph{Kernel}}
{RUC-30, RUC-34}{Alta}{Esencial}
{
El \emph{Kernel} de cada laboratorio será el encargado de enviar y 
recibir comunicaciones con el proveedor, además de añadir y eliminar 
herramientas. También armarán y realizarán las acciones necesarias 
para desarmar el laboratorio.
}

\reqfuncional{Laboratorios: Parada de emergencia en el \emph{Kernel}}
{RUC-30}{Alta}{Esencial}
{
Cuando el \emph{Kernel} reciba una parada de emergencia detendrá los 
gestores anexos inmediatamente, lo que permitirá al administrador 
acceder al \emph{hardware} de forma más rápida.
}

\reqfuncional{Laboratorios: Gestor de comunicaciones}
{RUC-28}{Alta}{Esencial}
{
Capa superior que administrará todas las peticiones recibidas por los 
clientes así como las notificaciones de las ejecuciones.
}

\reqfuncional{Laboratorios: Tiempo máximo de un cliente}
{RUC-21}{Media}{Esencial}
{
Cada gestor de comunicaciones controlará el tiempo que pasa el cliente 
registrado en el laboratorio. Cuando se sobrepasa, informa al 
proveedor y deniega cualquier intento posterior de comunicación.
}

\reqfuncional{Laboratorios: Gestor de ejecución}
{RUC-31, RUC-36}{Alta}{Esencial}
{
Capa superior que administrará todas las ejecuciones de las acciones y 
de los limpiadores.
}

\reqfuncional{Laboratorios: Tiempo máximo de una acción}
{RUC-47}{Media}{Esencial}
{
El gestor de ejecución se encargará de controlar el tiempo máximo de 
cada acción, que será interrumpida en el instante que lo sobrepase.
}

\reqfuncional{Herramientas: Aplicaciones}
{RUC-9, RUC-36}{Alta}{Esencial}
{
Cada herramienta contendrá un conjunto de aplicaciones de escritorio 
que ayudarán a manejar y configurar el \emph{hardware}.
}

\reqfuncional{Herramientas: Base de datos}
{RUC-37}{Alta}{Esencial}
{
Toda la información de las acciones estarán disponible en la base de 
datos creada por el laboratorio en cada herramienta. Y podrá ser 
modificada por el usuario a través del mismo para la configuración de 
las mismas herramientas.
}

\reqfuncional{Herramientas: Base de datos}
{RUC-37, RUC-39}{Alta}{Esencial}
{
Toda la información de las acciones estarán disponible en la base de 
datos creada por el laboratorio en cada herramienta. Y podrá ser 
modificada por el usuario a través del mismo para la configuración de 
las mismas herramientas.
}

\reqfuncional{Herramientas: Generación de datos}
{RUC-38, RUC-40}{Media}{Esencial}
{
Los datos generados en la ejecución de un acción serán guardados en 
la base de datos para su posterior consulta.
}

\reqfuncional{Herramientas: Servicios externos}
{RUC-42}{Alta}{Esencial}
{
La abstracción mediante \emph{sockets} de servicios externos de cada 
herramienta permitirá al usuario utilizarlos como servicios de la 
propia plataforma, con diferentes orígenes.
}

\reqfuncional{Herramientas: Archivo de descripción}
{RUC-43}{Alta}{Esencial}
{
Los archivos de descripción de herramientas seguirán dos esquemas, 
dependiendo del tipo de herramienta, con un formato XML.
}

\reqfuncional{Herramientas: Herramientas de datos}
{RUC-45}{Media}{Opcional}
{
Para los clientes, las herramientas de datos siempre estarán 
disponibles, independientemente del número de usuarios usándolas, y 
dispondrán de un mecanismo autónomo de \emph{broadcast} de información.
}

\reqfuncional{Herramientas: Limpiadores}
{RUC-46}{Media}{Esencial}
{
Estas acciones serán especificadas en el archivo de descripción y 
tendrán preferencia en cuanto a las peticiones de ejecución normales. 
Una herramienta no se podrá ejecutar si ha ocurrido un error y no se 
ha ejecutado con anterioridad el limpiador.
}


\reqfuncional{\emph{Libtool}: Base de datos}
{RUC-44}{Alta}{Esencial}
{
Esta abstracción permitirá la conexión a la base de datos de la 
herramienta para la lectura y escritura de información. El lenguaje de 
programación dependerá de la implementación de la propia herramienta.
}

\reqfuncional{\emph{Libtool}: Base de datos}
{RUC-44}{Alta}{Esencial}
{
Esta abstracción permitirá la conexión a la base de datos de la 
herramienta para la lectura y escritura de información. El lenguaje de 
programación dependerá de la implementación de la propia herramienta.
}

\reqfuncional{Cliente de escritorio: Función}
{RUC-8}{Media}{Esencial}
{
La aplicación cliente principal será ejecutada como aplicación de 
escritorio que utilizará el servicio web proveedor para la 
interacción con la plataforma RLF.
}

\reqfuncional{Cliente de escritorio: Ventanas de ejecución}
{RUC-48, RUC-49}{Alta}{Esencial}
{
Por cada acción en ejecución, el cliente tendrá activa una ventana 
independiente con acceso de entrada y salida por teclado.
}

\reqfuncional{Cliente de escritorio: Herramientas}
{RUC-51, RUC-53}{Alta}{Esencial}
{
Por cada herramienta, la interfaz del cliente poseerá una pestaña que 
contendrá toda su información. Se podrán seleccionar de forma 
independendiente para su posterior reserva.
}

\reqfuncional{Cliente de escritorio: Conexión}
{RUC-50}{Alta}{Esencial}
{
El cliente conservará la información de conexión durante toda la 
ejecución, sin ser necesario volver a conectarse para realizar 
múltiples reservas.
}

\reqfuncional{Cliente de escritorio: Actualización}
{RUC-50}{Baja}{Esencial}
{
Se ofrecerá al cliente la posibilidad de actualizar la información 
del número de herramientas como de su estado sin necesidad de cerrar 
la aplicación.
}

\reqfuncional{Cliente web: Función}
{RUC-18, RUC-52}{Baja}{Opcional}
{
Los clientes podrán acceder a la página web de la plataforma para 
comprobar el estado actual de las herramientas, si están disponibles, 
ocupadas o no conectadas. El formato deberá ser compatible para 
dispositivos con pantalla pequeña.
}

\reqfuncional{Gestor de laboratorios: Función}
{RUC-26}{Baja}{Esencial}
{
Para gestionar los laboratorios se proveerá de una aplicación de 
consolta con acceso por red con las acciones de armar, desarmar, 
parar, registrar y eliminar herramientas, y comprobar el estado del 
laboratorio.
}

\reqfuncional{Registro: \emph{Logs}}
{RUC-34}{Baja}{Esencial}
{
Todo componente en la plataforma se servirá de un conjunto de 
registros para indicar errores y funciones realizadas.
}

\subsection{Requisitos no funcionales}

\reqnofuncional{Comunicaciones: Sistema escalable}
{RUR-2}{Media}{Esencial}
{
La plataforma utilizará su propio sistema de comunicaciones y de altas 
de laboratorios, sin utilizar soluciones ya implementadas como Java 
RMI, CORBA o RPC.
}

\reqnofuncional{Comunicaciones: Protocolo JSON}
{RUR-1}{Media}{Esencial}
{
El protocolo de comunicaciones estará basado en el formato JSON, 
previamente cifrado, que reduce de forma considerable los datos a 
enviar de los objetos a enviar.
}

\reqnofuncional{Comunicaciones: Protocolo JSON avanzado}
{RUR-1}{Media}{Esencial}
{
Cada objeto enviado entre módulos puede contener otros objetos como 
atributos también en formato JSON.
}

\reqnofuncional{Comunicaciones: Peticiones y respuestas}
{RUR-1}{Media}{Esencial}
{
Cada petición o respuesta tendrá asociado un número de operación 
que se incluirá en la cabecera del mensaje.
} 

\reqnofuncional{Proveedor: \emph{Token} de acceso y uso}
{RUR-3}{Media}{Esencial}
{
Cada cliente tiene asociado un atributo con un límite de tiempo que 
será generado cada vez que se conecte para reducir el tráfico de 
comunicaciones y sólo tener que identificarse mediante esa cadena de 
caracteres. Cuando se desconecta será invalidado.
} 

\reqnofuncional{Proveedor: Servicio Web J2EE}
{RUR-4}{Alta}{Esencial}
{
El proveedor será un conjunto de dos servicios web publicados en una 
plataforma Tomcat, además de la base de datos MySQL. El servicio web 
principal llamado RLF\_Provider y el secundario, RLF\_Monitor.
} 

\reqnofuncional{Laboratorios: Máquina virtual de Java}
{RUR-5}{Alta}{Esencial}
{
Los laboratorios estarán implementados en Java y utilizarán drivers 
de conexión a las bases de datos SQLite.
}

\reqnofuncional{Laboratorios: Archivos ejecutables}
{RUR-5}{Alta}{Esencial}
{
Para configurar un laboratorio se proveerá de un fichero de 
configuración junto con el archivo ejecutable.
}

\reqnofuncional{Laboratorios: Ficheros fuente}
{RUR-5}{Alta}{Esencial}
{
Todo recurso necesario para la ejecución de los laboratorios será 
contenida en un directorio que deberá acompañarse con el ejecutable.
}

\reqnofuncional{Laboratorios: Conexiones}
{RUR-6}{Alta}{Esencial}
{
Cada laboratorio posee un puerto de conexión con el proveedor, otro de 
notificaciones para el cliente, uno de recepción de peticiones y por 
último, el puerto de control de mantenimiento.
}

\reqnofuncional{Laboratorios: Asincrónos}
{RUR-6}{Alta}{Esencial}
{
Por cada petición de ejecución se enviará en qué instante empezará 
a funcionar, ya que es posible que no sea instantáneo si hay 
sobrecarga en el laboratorio.
}

\reqnofuncional{Laboratorios: Ejecuciones}
{RUR-7}{Alta}{Esencial}
{
El laboratorio se servirá de peticiones al sistema y señales para 
ejecutar las acciones, así como de un servicio de \emph{pipeling} para 
redirigir la entrada y la salida de la propia acción.
}

\reqnofuncional{Laboratorios: Colas de ejecución}
{RUR-8}{Alta}{Esencial}
{
Cada laboratorio poseerá un contenedor de acciones en ejecución, donde 
está limitado por su configuración inicial. Todas las peticiones de 
ejecución, si los contenedores están completos, serán retrasadas.
}

\reqnofuncional{\emph{Framework}: Parada}
{RUR-9}{Alta}{Esencial}
{
Aunque no se impondrá ninguna norma a los desarrolladores en cuestión 
de bloqueo de ficheros o de programas por el motivo de las paradas 
inesperadas, se instará en la manera de no mantener ficheros abiertos 
ni componentes susceptibles de ocasionar errores en el futuro.
}

\reqnofuncional{Cliente de escritorio: Máquina virtual de Java}
{RUR-4}{Alta}{Esencial}
{
La aplicación de escritorio será implementada en Java para su 
portabilidad y se requerirá configurar mediante un archivo adjuntado 
con el ejecutable.
}

\reqnofuncional{Cliente web: Múltiple conexión}
{RUR-8}{Alta}{Esencial}
{
Mediante el uso de \emph{tokens} almacenados en la base de datos del 
proveedor una desconexión no es obligatoria en el cliente web.
}


\cleardoublepage


% Diseño
% Diseño

\capitulo{Diseño}{diseno}{
En el siguiente capítulo se tratará el diseño de la plataforma RLF, 
desde los primeros conceptos hasta la implementación. También se describirán 
cada una de las partes involucradas en el funcionamiento del sistema.
}

\section{Arquitectura del sistema}
El primer paso del diseño es, mediante la recopilación de 
información para la anterior parte de análisis, la definición de la 
arquitectura del sistema, es decir, sus componentes principales y la 
forma de interacción entre ellos. Para ello se ha valido de un esquema 
(figura \ref{fig:arquitectura}) que los identifica. Estos componentes 
forman una estructura general de cliente/servidor o servidores 
jerárquicos, aunque cada componente implementa en su interior otro 
tipo de arquitecturas que se definirán más adelante.

Siguiendo el orden de la parte superior a la inferior de dicha figura, 
obtenemos los siguientes componentes:

La plataforma se asienta sobre una base de datos central que contiene 
la información general del sistema. Con ella, los diferentes 
componentes obtienen la información actualizada de las acciones 
tomadas y del estado global. Los laboratorios y el servidor web son 
los componentes directamente relacionados con esta base.

El servidor web actúa como la puerta de acceso a la plataforma RLF. 
Está compuesto por dos servicios web (servicio proveedor y servicio 
monitor) y una página web para el acceso a uno de ellos. El servicio 
proveedor funciona como componente principal, y se encarga de tratar 
las peticiones por parte de los clientes y darles acceso a los 
distintos laboratorios.

Los laboratorios conforman el esqueleto de la plataforma y funcionan 
como gestores de peticiones de ejecución para las herramientas. 
Contienen una base de datos local propia de cada laboratorio, pero 
también se sirven de la base de datos central para informar del estado 
al resto de componentes. Primero, el servidor web indica al 
laboratorio los clientes válidos permitidos en el sistema, después 
son los propios clientes los que contactan con él para que atienda sus 
peticiones y empieza el intercambio de mensajes.

También, laboratorios son gestionados externamente por los administradores 
mediante un componente remoto (gestor de laboratorios) que no tiene 
interacción con otros componentes de la plataforma. Con él se puede 
arrancar los laboratorios, registrar herramientas y otras tareas de 
mantenimiento.

Al otro lado del sistema se encuentra el cliente de escritorio, que 
permite acceder al usuario cliente a la plataforma. Provee de la 
interfaz necesaria para manejar cada herramienta y realizar las 
peticiones. Está relacionado con la interfaz que implementa el 
servicio proveedor ya que realiza el acceso a través de ella.

Como casos especiales, se encuentran las aplicaciones externas y los 
servicios externos. No están directamente relacionados con la 
plataforma ni implementados en la misma, aunque son parte importante 
para el desarrollo de herramientas. Contienen su propias 
comunicaciones y es el usuario cliente el encargado de acceder a los 
servicios mediantes estas aplicaciones y son las herramientas las 
encargadas de activarlos.

Por último se tienen las herramientas y sus datos, que funcionan como 
dos componentes por separados, ya que en realidad, esos datos 
(contenidos en bases de datos locales) son los intermediarios entre 
los laboratorios y las herramientas. Mediante funciones del sistema, 
los laboratorios arrancan las aplicaciones de las herramientas, pero 
es, mediante dichas bases de datos, donde intercambian información de 
forma asíncrona.

En el siguiente apartado se explicarán con detalle cada uno de los 
componentes, indicando sus relaciones y sus funciones.

\begin{figure}[h]
	\centering
	\includegraphics[angle=90,scale=0.7]{images/arquitectura.png}
	\caption[Arquitectura de RLF]{Arquitectura de RLF según sus 
	componentes.}
	\label{fig:arquitectura}
\end{figure}

\clearpage

\section{Diseño de componentes}

Para comprender el conjunto de la plataforma, se dispone a explicar 
cada componente por separado (y sus subcomponentes), incluyendo las 
relaciones que tiene con otros. Las definiciones que se muestran a 
continuación responden al estándar \emph{UML 1.1}. Se establece el mismo 
orden que se siguió en el apartado anterior.

% Componentes

\subsection{Base de datos central}

Compone el principal almacén de datos de toda la plataforma RLF. 
Contiene un conjunto de tablas que organizan la información para que 
sea accesible mediante el servicio proveedor y los distintos 
laboratorios. Como se puede ver en la figura \ref{fig:erproveedor} 
(modelo entidad-relación), contiene las siguientes estructuras:

\begin{itemize}
\item Una tabla para la información de cada laboratorio. Incluye todos los 
elementos necesarios para la conexión y su estado principal.
\item Tablas de información de usuario, diferenciando entre los 
usuarios clientes y los usuarios administradores.
\item La información principal de cada herramienta está contenida en 
la tabla \emph{tool} junto con su estado y en qué laboratorio se 
encuentra.
\item Un registro temporal por cada herramienta reservada, 
relacionando el cliente con la herramienta, y la fecha de la propia 
reserva.
\item La descripción de cada herramienta, formada por sus constantes, 
parámetros, atributos, acciones y servicios externos están 
almacenados en sus propias tablas.
\end{itemize}

Cada cambio que se realiza en esta base de datos de forma 
automatizada, como son procesos de conexión, desconexión, reserva, 
activación de laboratorios, etc. están agrupados en procedimientos 
almacenados (interfaz \emph{iBase}) \footnote{Conjunto de 
instrucciones en una base de datos donde no hay valor de retorno.} 
para mejorar la seguridad y la fiabilidad del sistema. El resto de 
cambios concretos, como la insercción de una nueva herramienta, se 
realizan mediante el procedimiento normal. Se añaden también un 
conjunto de disparadores e índices para mejorar la velocidad de la 
base de datos.

Como se ve en la figura \ref{fig:Proveedor} la base de datos 
interactua directamente con el servidor web, y además con los 
laboratorios.

\subsection{Servidor Web}

Está compuesto por dos servicios y una página web de acceso a la 
plataforma (figura \ref{fig:Proveedor}) y el componente de 
comunicaciones.

\textbf{NOTA:} El componente de comunicaciones es descrito al final de 
este sección, ya que hay multitud de otros componentes que lo usan a 
pesar de tener la misma implementación (véase sección 
\ref{subsec:comunicaciones}). Aunque sea este componente el que se 
encargue de la recepción y envío de datos a nivel interno, las 
interfaces las proveen el resto de componentes, por lo que sólo se 
considera una capa a bajo nivel en la comunicación y será 
representado como tal.

\begin{figure}[H]
	\centering
	\includegraphics[scale=0.55]{images/erproveedor.png}
	\caption[E-R central]{Estructura de la base de datos central.}
	\label{fig:erproveedor}
\end{figure}

\subsubsection{Servicio Monitor}
Utiliza la información contenida en la base de datos central para que 
mediante la interfaz de acceso \emph{iMonitor} se obtenga el estado de 
las herramientas. Los métodos de esta interfaz son:

\begin{itemize}
\item Autentificación al monitor.
\item Desconexión del monitor.
\item Obtención del estado en forma de lista con el identificador de 
la herramienta, su nombre y su estado.
\end{itemize}

\subsubsection{Página Web}
Presenta una interfaz accesible a través de un navegador para utilizar 
el servicio del monitor. Está preparada para mantener la conexión 
abierta aunque se deje de atender la página. Se compone de una 
sección para la autentificación y una lista de herramientas y sus 
estados.

\begin{figure}[H]
	\centering
	\includegraphics[scale=0.6]{images/Proveedor.png}
	\caption{Servidor Web y Base de datos central.}
	\label{fig:Proveedor}
\end{figure}

\subsubsection{Servicio Proveedor}
Representa la interfaz abstracta de toda la plataforma para el acceso. 
Se puede decir que es el coordinador de todo el sistema, que permite a 
los clientes localizar (con la interfaz \emph{iRLF}) los laboratorios 
e iniciar las conexiones. Se sirve de los datos de la base central 
mediante la interfaz \emph{iDatos} y comunica a los laboratorios 
cuándo serán accedidos por los clientes.

Las acciones de la interfaz \emph{iRLF} se listan a continuación:

\begin{itemize}
\item Conexión al sistema. Cuando un cliente se conecta, genera el 
\emph{token} correspondiente y se lo envía para la identificación de 
todas las operaciones siguientes.
\item Desconexión. Invalida dicho \emph{token} de acceso y libera 
todas las herramientas que pudiera tener reservadas.
\item Obtiene el estado actual de las herramientas a las cuales el 
cliente tiene acceso.
\item Describe las herramientas las cuales pueden ser accedidas por el 
cliente. La descripción es un mensaje codificado en JSON precalculado 
con todos los atributos, constantes, acciones y parámetros de las 
herramientas.
\item Reserva las herramientas seleccionadas. Avisa además a los 
laboratorios para que inicien el contador de tiempo del cliente y 
devuelve a este la información de dónde se encuentra cada herramienta.
\end{itemize}

\subsection{Laboratorio}
\index{laboratorio}
Este complejo componente gestiona todas las peticiones a las herramientas 
por parte de los clientes. Está compuesto por tres subcomponentes 
principales (\emph{Kernel}, gestor de comunicaciones y gestor de 
ejecución) y por otros tres secundarios (comunicaciones, gestor de 
herramientas y base de datos) como se puede ver en la figura 
\ref{fig:Laboratorio}.

\begin{figure}[H]
	\centering
	\includegraphics[scale=0.6]{images/Laboratorio.png}
	\caption{El laboratorio y las herramientas.}
	\label{fig:Laboratorio}
\end{figure}

\subsubsection{\emph{Kernel}}
\index{\emph{Kernel}}
Es el núcleo del laboratorio, que atiende las peticiones del proveedor 
(recibidas por la interfaz \emph{iGestiónClientes}) y las de gestión 
(interfaz \emph{iGestiónLab}) desde el programa administrador. 
También controla la ejecución y el arranque de los otros dos 
subcomponentes principales. Es el único componente que tiene acceso a 
la base de datos central para poder cambiar su estado.

Cuando el laboratorio es armado, el \emph{Kernel} activa a los 
gestores de comunicaciones y de ejecución que se componen de hilos 
independientes. Usa al gestor de herramientas cuando recibe una 
petición de registro o de parada.

Las principales acciones que se concentran en este componente son:
\begin{itemize}
\item Iniciar el laboratorio. Avisa al proveedor que ha sido arrancado.
\item Parar el laboratorio. Sólo se realiza bajo petición de un 
administrador y cuando no está armado.
\item Armar y desarmar. En el estado de armado, se pueden recibir 
peticiones de los clientes y ejecutar las diferentes acciones.
\item Obtener el estado global del laboratorio y sus herramientas. 
Sólo es utilizado por los administradores.
\item Avisar al proveedor (mediante la base de datos) que el tiempo de 
un cliente ha terminado.
\item Iniciar el proceso de registrar o eliminar una herramienta. Sólo 
en el caso de no estar armado.
\item Realizar una parada de emergencia completa.
\end{itemize}

\subsubsection{Gestor de ejecución}
\index{gestor de ejecución}
Contiene todos los elementos necesarios para organizar las peticiones 
(enviadas por el gestor de comunicaciones) de ejecución de las 
diferentes acciones de las herramientas. Controla el tiempo máximo de 
cada ejecución y realiza el envío de datos entrantes y salientes a 
modo de \emph{stream} a los clientes, sin formato establecido. Cuando 
una acción es terminada o interrumpida, genera el mensaje a enviar al 
cliente. También indica al gestor de herramientas qué cambios se 
deben hacer en la base de datos de la herramienta
antes de ejecutar una determinada acción, y recoger estos cuando la 
acción haya terminado.

\subsubsection{Gestor de comunicaciones}
\index{gestor de comunicaciones}
Es el enlace con los clientes. Recibe las peticiones de ejecución por 
la interfaz \emph{iPeticiones} y las encola en el gestor de ejecución. 
Cuando ocurre cualquier evento con esas acciones, se lo comunica al 
cliente mediante el puerto de notificaciones. Con esto se consigue un 
tratamiento asíncrono. Por último, todos los cronómetros de cada 
cliente los contiene este gestor, así como la información de cada 
usuario válido en el sistema.

\subsubsection{Gestor de herramientas}
\index{gestor de herramientas}
Componente encargado en la gestión de las herramientas activas en el 
laboratorio, así como el manejo de su información mediante el 
subcomponente de la base de datos propia. Valida todas las 
configuraciones XML introducidas por los administradores y lee, cuando 
el laboratorio es arrancado, las ya existentes. No funciona como un 
hilo a parte, si no como una biblioteca de métodos.

Además, toda la información en tiempo real es insertada o leída de 
la base de datos de la herramienta como si fuese la propia aplicación.

\subsubsection{Base de datos}
Funciona como un almacén de los datos de acceso de las herramientas. 
Conforma una estructura simple de una única tabla con los elementos 
más importantes, siendo la mostrada en la figura \ref{fig:erlaboratorio}

\begin{figure}[H]
	\centering
	\includegraphics[scale=0.8]{images/labdata.png}
	\caption[E-R local]{Estructura de la base de datos del laboratorio.}
	\label{fig:erlaboratorio}
\end{figure}

\subsection{Datos de la herramienta}
Ha sido concebido como una interfaz entre la aplicación y el 
laboratorio, pero de forma asíncrona. De esta forma, la herramienta es 
tratada como un objeto externo. El contenido es la estructura directa 
de la herramienta (figura \ref{fig:erherramienta}). 

\begin{figure}[h]
	\centering
	\includegraphics[scale=0.65]{images/tool.png}
	\caption[E-R de la herramienta]{Estructura de la base de datos de 
	la herramienta.}
	\label{fig:erherramienta}
\end{figure}

La correspondencia de cada tabla es similar a la base de datos 
central, que contiene la misma información, a excepción de los 
atributos, que conforman las características propias de cada 
herramienta (en el servidor central están representadas como columnas 
de la tabla principal), y las excepciones y estados de las diferentes 
acciones.

Está implementada de forma sencilla para obtener una mayor velocidad 
de escritura y lectura. Es por ello que todas las claves ajenas y otras 
restricciones han sido desactivadas, y son comprobadas de forma 
externa.

\subsection{Herramienta}
Este componente contiene la arquitectura del \emph{framework} de RLF. 
No se definirán las diferentes herramientas entregadas si no la forma 
que deben tener las aplicaciones.

\subsubsection{Aplicaciones}
Son el conjunto de acciones a ejecutar. Están implementadas según las 
normas del \emph{framework} (ver Manual de desarrollo) y sólo son 
dependientes del otro componente \emph{Libtool} y de los servicios 
externos si se requieren.

\subsubsection{Servicios externos}
Poseen un acceso como puerto de comunicaciones donde los clientes 
pueden conectarse, pero no es gestionado por el laboratorio ni por el 
proveedor central.

\subsubsection{\emph{Libtool}}
\index{\emph{libtool}}
Funciona como una biblioteca de funciones para la transformación en 
objeto. Así se pueden obtener los datos escritos por el laboratorio. 
Las funciones que provee este componente son las siguientes:

\begin{itemize}
\item Conexión y desconexión con el laboratorio. Si una aplicación 
requiere lectura o escritura de sus valores, es necesario estar 
conectado.
\item Lectura y escritura de parámetros.
\item Lectura de atributos. Las herramientas no pueden escribir sus 
propios atributos, ya que estos son definidos por el laboratorio 
poseedor.
\item Lectura de constantes.
\item Generación de excepciones.
\item Escritura del estado de la acción.
\end{itemize}

\subsection{Gestión de laboratorios}
\index{gestión de laboratorios}
Este componente forma una aplicación usada por los administradores de 
la plataforma RLF y forma una arquitectura de capas. Cada capa se 
sirve de la interfaz de la anterior, como se puede ver en la figura 
\ref{fig:labconsole}.

\begin{figure}[h]
	\centering
	\includegraphics[scale=0.65]{images/labconsole.png}
	\caption[Gestor de laboratorios por capas]{Estructura de capas del gestor de 
	laboratorios.}
	\label{fig:labconsole}
\end{figure}

\begin{figure}[h]
	\centering
	\includegraphics[scale=0.65]{images/Gestor.png}
	\caption[Gestor de laboratorios]{Componentes del gestor de 
	laboratorios.}
	\label{fig:gestorlab}
\end{figure}

\subsubsection{\emph{LabConsole}}
\index{\emph{LabConsole}}
Actua como interfaz textual para enviar peticiones a los laboratorios. 
Utiliza el conjunto de métodos proveídos por la librería de gestión 
(interfaz \emph{iGestor} en la figura \ref{fig:gestorlab}).

\subsubsection{Librería de gestión}
Incluye los parámetros necesarios introducidos mediante la interfaz 
textual en los mensajes para las peticiones de mantenimiento. Después 
serán enviadas a los laboratorios, obteniendo una respuesta y 
descodificándola.

\subsection{Cliente}
El cliente se muestra como una aplicación de escritorio en Java con 
una interfaz implementada en Swing. Debe estar configurada para 
acceder al servidor central de RLF. Al igual que los laboratorios 
tiene diferentes subcomponentes como hilos independientes (figura 
\ref{fig:cliente}).

\begin{figure}[h]
	\centering
	\includegraphics[scale=0.65]{images/Cliente.png}
	\caption[Comunicaciones RLF]{Componentes del sistema de 
	comunicaciones de la plataforma RLF.}
	\label{fig:cliente}
\end{figure}

\subsubsection{Gestor de eventos}
Es el componente encargado de ``escuchar'' los eventos de la interfaz 
generados por el usuario para llevar a cabo acciones con el proveedor 
o las herramientas. Establece una conexión directa con la interfaz 
\emph{iRLF} generada por el servicio proveedor. Los mensajes enviados 
con este componente no tienen el formato de dato interno, si no que 
utiliza el propio formato HTTP + SOAP que se usa en los servicios web.

\subsubsection{Gestor de notificaciones}
Todas las peticiones que se envían a los laboratorios y las 
notificaciones que se reciben de estos pasan a través de este 
componente, que usa el protocolo definido en le componente 
Comunicaciones. Se sirve de la interfaz \emph{iPeticiones} e 
implementa \emph{iNotificaciones} con una función de inicio de la 
acción y otra notificación de tiempo excedido. Actúa como puerto de 
escucha y posee ejecución paralela al gestor de eventos.

\subsubsection{Interfaz}
Provee al usuario cliente de todos los mecanismos necesarios para la 
realizar el acceso completo a la plataforma.

\subsection{Comunicaciones}
La implementación de este componente, usado en los otros, forma un 
protocolo de comunicación añadido al modelo TCP/IP (véase sección 
\ref{subsec:tcpip}) que es la base de las comunicaciones internas de 
la plataforma en forma de petición/respuesta. Además también tiene 
la función de librería para enviar y recibir datos, 
obteniendo directamente la información importante y \emph{parseada}.

En la figura \ref{fig:comunicaciones} se puede ver la disposición de 
sus subcomponentes.

\begin{figure}[h]
	\centering
	\includegraphics[scale=0.65]{images/Comunicaciones.png}
	\caption[Cliente de escritorio]{Componentes del cliente de 
	escritorio.}
	\label{fig:comunicaciones}
\end{figure}


\subsubsection{Cifrador}
Biblioteca con los métodos necesarios para cifrar datos textuales con 
el algoritmo \emph{base64} \footnote{Base 64 es un sistema de 
numeración posicional que usa 64 como base. Es la mayor potencia de 
dos que puede ser representada usando únicamente los caracteres 
imprimibles de ASCII.\cite{Tanenbaum}} y la obtención de valores 
mediante el algoritmo resumen \emph{SHA-1}.

\subsubsection{Red}
Implementa el protocolo de mensajes RLF. Se basa en un sistema de 
capas que convierte objetos JSON en mensajes reconocibles por el 
sistema. Los mensajes están formados por un identificador que 
corresponde a la operación a realizar y un conjunto de atributos. En 
la primera capa (capa de objeto) se obtienen esos atributos y se 
``aplanan'' (un atributo puede ser a la vez otro objeto JSON). En la 
capa de mensaje se cifra según el algoritmo \emph{base64} para no 
perder caracteres del mensaje y se añade la cabecera, que es el 
tamaño total del mensaje (ver figura \ref{fig:protocolo}). El proceso a 
la inversa también ha sido implementado.

\begin{figure}[h]
	\centering
	\includegraphics[scale=0.65]{images/mensaje.png}
	\caption[Protocolo RLF]{Protocolo RLF.}
	\label{fig:protocolo}
\end{figure}

\textbf{NOTA:} Para obtener más información sobre los identificadores 
de las peticiones, consultar el código adjuntado donde se encuentra la 
lista exhaustiva de los mismos.

Esos mensajes son enviados y recibidos mediante \emph{sockets} que 
provee el sistema operativo. Los métodos son los siguientes:

\begin{itemize}
\item Conexión y desconexión de \emph{sockets} con direcciones remotas.
\item Envío de una petición/respuesta con un \emph{socket} conectado.
\item Recepción de una petición/respuesta por un \emph{socket} a la 
escucha.
\end{itemize}



\section{Despliegue de componentes}

Como última sección se presenta el despliegue de los componentes en 
los distintos dispositivos o máquinas que son necesarios para el 
funcionamiento del sistema, también se incluyen los entornos de 
ejecución necesarios. Se puede ver en la figura \ref{fig:despliegue}, 
siguiendo el estándar \emph{UML 1.1} el conjunto de componentes.

Este esquema es orientativo ya que debido a la versatilidad de la 
plataforma, todos los componentes pueden estar en el mismo nodo, o 
convinaciones de ellos. En cambio, la multiplicidad de 
cada relación entre los nodos sí que es obligatoria, haciendo 
imposible tener en una misma plataforma RLF dos servidores centrales, 
o que un mismo cliente esté conectado a dos plataformas con la misma 
aplicación cliente. Para más información consultar el Manual de 
despliegue contenido en esta misma documentación. 

\textbf{NOTA:} Los elementos ``artifact'' representan las aplicaciones 
finales que contienen a los componentes, referenciados mediante la 
relación ``manifest''.

En primera instancia se define el servidor central con la base de 
datos de RLF sobre el entorno MySQL y los servicios en la plataforma 
web Tomcat que soporta J2EE (ver sección \ref{subsubsec:j2ee}). A él 
se pueden conectar dispositivos portátiles con navegadores (como 
entorno de ejecución) para acceder a la página web de monitorización 
que provee el servidor central.

Después se encuentran los nodos que contienen los laboratorios. 
Mediante la máquina virtual de java se ejecuta la aplicación 
\emph{Lab} que contiene el componente principal del laboratorio. Con 
SQLite se gestionan los datos de las herramientas. Y, dependiendo del 
tipo de aplicación, se ejecutan las herramientas en el sistema 
operativo, al igual que los servicios externos. Todos los elementos 
necesarios para la ejecución de las acciones de las herramientas 
tienen que estar contenidos en estos mismos nodos.

El nodo cliente contiene todas las aplicaciones necesarias para poder 
utilizar los servicios externos, además de la máquina virtual Java 
para poder ejecutar el cliente de escritorio, llamado ``RLF\_Client''. 
Este nodo debe tener conexión con la red del servidor central así 
como acceso a los laboratorios.

Por último, el gestor de laboratorios se encuentra en el terminal de 
acceso a la red de los laboratorios. Al ser una aplicación Java, 
necesitará la máquina virtual para que funcione.

Los siguientes pasos en el desarrollo de la plataforma incluyen la 
implementación y las pruebas. No se ha dedicado un capítulo exclusivo 
para la implementación debido a que en el material entregado se 
encuentran todas las referencias y definiciones necesarias para 
comprenderla. Se podrá encontrar en el código, además, los 
algoritmos y estructuras necesarias para el funcionamiento de la 
plataforma RLF.

\begin{figure}[h]
	\centering
	\includegraphics[angle=90,scale=0.7]{images/despliegue.png}
	\caption[Despliegue de RLF]{Despliegue de RLF por componentes.}
	\label{fig:despliegue}
\end{figure}

\cleardoublepage


% Pruebas realizadas
% Pruebas realizadas

\newcounter{prueba}
\newcounter{pfuncion}
\newcounter{perror}
\newcounter{pestres}

\capitulo{Pruebas}{pruebas}{
Las pruebas de verificación del sistema implementado es una parte 
importante del desarrollo \emph{software}. En este capítulo se 
incluyen las más destacadas que validan el conjunto de la plataforma 
RLF.
}

Las pruebas que se muestran a continuación han sido seleccionadas del 
conjunto global que se realizaron en la implementación del proyecto. 
Debido al gran número de estas no se ha podido incluir todas.

\section{Configuraciones}
Se han establecido un conjunto de configuraciones de la arquitectura 
para las distintas pruebas, con lo que se consigue verificar que 
independientemente de la estructura del sistema, los resultados sean 
los mismos. Estas configuraciones tienen un nombre asociado que se 
especificará en cada una de las pruebas. A continuación se muestra 
cada configuración, con el número de componentes que intervienen:

\begin{table}[H]
\begin{center}
\begin{tabular}{|| l | c | c | c | p{10cm} ||}
	\hline
	\hline
	Nombre & L & CE & CW & Disposición\\
	\hline
	\hline
	Básica & 1 & 2 & 1 & Todos los componentes en la misma máquina, 
	excepto un CE y el CW que está en otra máquina en red.\\
	\hline
	Común & 2 & 1 & 1 & El P y un L se disponen en 
	una máquina, el otro L en un sistema Windows en red 
	local con el CE. El CW en un \emph{smartphone} con 3G.\\
	\hline
	Remota & 2 & 2 & 2 & Cada componente está en una máquina distinta, 
	pero el P y los L están conectados en red local, mientras que los 
	CE y CW se conectan mediante Internet.\\
	\hline
	Educativa & 5 & 2 & 1 & P se encuentra en una red externa, los L 
	están conectados en red local, y los CE y CW en otra red externa. Algunas 
	herramientas han sido duplicadas.\\
	\hline
	Industrial & 5 & 5 & 2 & Cada componente se encuentra en un nodo 
	distinto, distribuidos entre redes locales e Internet. Algunas 
	herramientas han sido duplicadas.\\
	\hline
	\hline
\end{tabular}
\end{center}
	\caption{Configuraciones de las pruebas.}
	\label{tab:configuraciones}
\end{table}

\textbf{NOTA 1:} Los \emph{laboratorios} (L) son numerados secuencialmente 
de forma \emph{1}, \emph{2}, \emph{3}\ldots Los \emph{clientes de 
escritorio} (CE) se corresponden con el alfabeto en mayúsculas: \emph{A}, 
\emph{B}, \emph{C}\ldots Y por último, los \emph{clientes web} (CW) 
corresponden con el alfabeto en minúsculas: \emph{a}, \emph{b}, 
\emph{c}\ldots

\textbf{NOTA 2:} Todas las configuraciones se componen de un solo 
proveedor y monitor (P), y de las cinco herramientas entregadas con el 
proyecto, además de un número concreto de laboratorio, clientes de 
escritorio y clientes web.

\section{Pruebas de verificación de funcionalidad}

Estas pruebas han sido concebidas para verificar todas las funciones y 
opciones que aporta la plataforma RLF. Han sido realizadas en un 
entorno controlado y comprobando que la respuesta del sistema a 
determinadas acciones era la esperada.

Durante el desarrollo del proyecto, estas pruebas se repitieron 
obteniendo diferentes errores que fueron corregidos hasta obtener los 
resultados esperados.

\begin{table}[H]
\begin{center}
	\stepcounter{prueba}
	\stepcounter{pfuncion}
\begin{tabular}{|| m{1cm} m{2cm} | m{2.5cm} m{2.5cm} | l @{\extracolsep{\fill}} c ||}
	\hline
	\hline
	\multicolumn{6}{|| c ||}{\textsc{Prueba \arabic{prueba}}}\\
	\hline
	\textbf{ID} & PF-\Alph{pfuncion} &
	\textbf{Configuración} & Básica & 
	\textbf{Resultado} & Válido\\
	\hline
	\hline
	\multicolumn{6}{|| c ||}{\textbf{Descripción}}\\
	\hline
	\multicolumn{6}{|| p{14cm} ||}{
	Se accede desde la red externa mediante el CE para reservar todos 
	las herramientas disponibles, ejecutando a la vez dos de ellas. 
	Después se comprueban el estado de las herramientas mediante el 
	CW, donde todas están ocupadas.
	}\\
	\hline
	\multicolumn{6}{|| c ||}{\textbf{Acciones}}\\
	\hline
	\multicolumn{6}{|| p{14cm} ||}{
		\begin{enumerate}
		\item Reserva de todas las herramientas.
		\item Ejecución de dos de ellas.
		\item Comprobación por la web.
		\item Obtención de las herramientas ocupadas.
		\end{enumerate}
	}\\
	\hline
	\hline
\end{tabular}
\end{center}
\end{table}

\begin{table}[H]
\begin{center}
	\stepcounter{prueba}
	\stepcounter{pfuncion}
\begin{tabular}{|| m{1cm} m{2cm} | m{2.5cm} m{2.5cm} | l @{\extracolsep{\fill}} c ||}
	\hline
	\hline
	\multicolumn{6}{|| c ||}{\textsc{Prueba \arabic{prueba}}}\\
	\hline
	\textbf{ID} & PF-\Alph{pfuncion} &
	\textbf{Configuración} & Común & 
	\textbf{Resultado} & Válido\\
	\hline
	\hline
	\multicolumn{6}{|| c ||}{\textbf{Descripción}}\\
	\hline
	\multicolumn{6}{|| p{14cm} ||}{
	Se comprueba ejecución de herramientas en diferentes L desde el 
	mismo CE.
	}\\
	\hline
	\multicolumn{6}{|| c ||}{\textbf{Acciones}}\\
	\hline
	\multicolumn{6}{|| p{14cm} ||}{
		\begin{enumerate}
		\item Reserva de una herramienta de cada laboratorio.
		\item Ejecución de las acciones de las herramientas.
		\item Desconexión.
		\end{enumerate}
	}\\
	\hline
	\hline
\end{tabular}
\end{center}
\end{table}

\begin{table}[H]
\begin{center}
	\stepcounter{prueba}
	\stepcounter{pfuncion}
\begin{tabular}{|| m{1cm} m{2cm} | m{2.5cm} m{2.5cm} | l @{\extracolsep{\fill}} c ||}
	\hline
	\hline
	\multicolumn{6}{|| c ||}{\textsc{Prueba \arabic{prueba}}}\\
	\hline
	\textbf{ID} & PF-\Alph{pfuncion} &
	\textbf{Configuración} & Básica & 
	\textbf{Resultado} & Válido\\
	\hline
	\hline
	\multicolumn{6}{|| c ||}{\textbf{Descripción}}\\
	\hline
	\multicolumn{6}{|| p{14cm} ||}{
	El objetivo es la comprobación de los limpiadores de cada 
	herramienta. Para ello es requerido un cierre forzado de una 
	acción.
	}\\
	\hline
	\multicolumn{6}{|| c ||}{\textbf{Acciones}}\\
	\hline
	\multicolumn{6}{|| p{14cm} ||}{
		\begin{enumerate}
		\item Reserva de una herramienta.
		\item Ejecución de una acción.
		\item Forzado de cierre desde el CE.
		\item Comprobación del \emph{log} del laboratorio responsable.
		\item El limpiador se ejecuta a los pocos segundos de cerrar 
		la acción.
		\end{enumerate}
	}\\
	\hline
	\hline
\end{tabular}
\end{center}
\end{table}

\begin{table}[H]
\begin{center}
	\stepcounter{prueba}
	\stepcounter{pfuncion}
\begin{tabular}{|| m{1cm} m{2cm} | m{2.5cm} m{2.5cm} | l @{\extracolsep{\fill}} c ||}
	\hline
	\hline
	\multicolumn{6}{|| c ||}{\textsc{Prueba \arabic{prueba}}}\\
	\hline
	\textbf{ID} & PF-\Alph{pfuncion} &
	\textbf{Configuración} & Básica & 
	\textbf{Resultado} & Válido\\
	\hline
	\hline
	\multicolumn{6}{|| c ||}{\textbf{Descripción}}\\
	\hline
	\multicolumn{6}{|| p{14cm} ||}{
	Se verifica que las herramientas de datos se ejecuten por varios 
	usuarios y que sólo se cierren cuando todos se han desconectado.
	}\\
	\hline
	\multicolumn{6}{|| c ||}{\textbf{Acciones}}\\
	\hline
	\multicolumn{6}{|| p{14cm} ||}{
		\begin{enumerate}
		\item Reserva de una herramienta con el CE \emph{A}.
		\item Reserva de la misma herramienta con el CE \emph{B}.
		\item El CE \emph{B} ejecuta la acción de la herramienta.
		\item Al cabo de un tiempo, el CE a también.
		\item El CE \emph{B} cierra la ejecución y se desconecta.
		\item Por último, CE a para la acción.
		\item Se comprueba el \emph{log} de L y se verifica que la 
		acción se ha detenido correctamente cuando el CE \emph{A} se 
		desconectó.
		\end{enumerate}
	}\\
	\hline
	\hline
\end{tabular}
\end{center}
\end{table}

\begin{table}[H]
\begin{center}
	\stepcounter{prueba}
	\stepcounter{pfuncion}
\begin{tabular}{|| m{1cm} m{2cm} | m{2.5cm} m{2.5cm} | l @{\extracolsep{\fill}} c ||}
	\hline
	\hline
	\multicolumn{6}{|| c ||}{\textsc{Prueba \arabic{prueba}}}\\
	\hline
	\textbf{ID} & PF-\Alph{pfuncion} &
	\textbf{Configuración} & Básica & 
	\textbf{Resultado} & Válido\\
	\hline
	\hline
	\multicolumn{6}{|| c ||}{\textbf{Descripción}}\\
	\hline
	\multicolumn{6}{|| p{14cm} ||}{
	Comprobación de la liberación de herramientas una vez que un 
	usuario se ha desconectado, y su posterior uso por parte de otro 
	usuario.
	}\\
	\hline
	\multicolumn{6}{|| c ||}{\textbf{Acciones}}\\
	\hline
	\multicolumn{6}{|| p{14cm} ||}{
		\begin{enumerate}
		\item Reserva de un conjunto de herramientas con el CE \emph{A}.
		\item Conexión del CE \emph{B}.
		\item El CE \emph{B} comprueba el estado de las herramientas en uso 
		por CE \emph{A}.
		\item CE \emph{A} cierra algunas acciones forzándolas y otras con la 
		finalización normal, después se desconecta.
		\item El CE \emph{B} reserva las herramientas que CE a ha liberado.
		\item Ejecuta las acciones de esas herramientas.
		\item Debe esperar para el inicio de algunas acciones, 
		correspondientes con las que se ha forzado su cierre por CE 
		\emph{A}, ya que se están ejecutando sus limpiadores.
		\item Se comprueba el estado de las herramientas en el CW.
		\item CE \emph{B} libera sus herramientas después de utilizarlas.
		\item Se vuelve a comprobar el estado de las herramientas 
		mediante el CW y se verifica que todas están libres.
		\end{enumerate}
	}\\
	\hline
	\hline
\end{tabular}
\end{center}
\end{table}

\begin{table}[H]
\begin{center}
	\stepcounter{prueba}
	\stepcounter{pfuncion}
\begin{tabular}{|| m{1cm} m{2cm} | m{2.5cm} m{2.5cm} | l @{\extracolsep{\fill}} c ||}
	\hline
	\hline
	\multicolumn{6}{|| c ||}{\textsc{Prueba \arabic{prueba}}}\\
	\hline
	\textbf{ID} & PF-\Alph{pfuncion} &
	\textbf{Configuración} & Básica & 
	\textbf{Resultado} & Válido\\
	\hline
	\hline
	\multicolumn{6}{|| c ||}{\textbf{Descripción}}\\
	\hline
	\multicolumn{6}{|| p{14cm} ||}{
	Verificación de los roles establecidos para los distintos usuarios 
	y las herramientas que pueden utilizar.
	}\\
	\hline
	\multicolumn{6}{|| c ||}{\textbf{Acciones}}\\
	\hline
	\multicolumn{6}{|| p{14cm} ||}{
		\begin{enumerate}
		\item Se registran las herramientas con distintos roles.
		\item Se crea un usuario con un rol específico.
		\item El usuario, mediante el CE y el CW comprueba que sólo 
		tiene acceso a las herramientas con su rol o inferior.
		\end{enumerate}
	}\\
	\hline
	\hline
\end{tabular}
\end{center}
\end{table}

\begin{table}[H]
\begin{center}
	\stepcounter{prueba}
	\stepcounter{pfuncion}
\begin{tabular}{|| m{1cm} m{2cm} | m{2.5cm} m{2.5cm} | l @{\extracolsep{\fill}} c ||}
	\hline
	\hline
	\multicolumn{6}{|| c ||}{\textsc{Prueba \arabic{prueba}}}\\
	\hline
	\textbf{ID} & PF-\Alph{pfuncion} &
	\textbf{Configuración} & Básica & 
	\textbf{Resultado} & Válido\\
	\hline
	\hline
	\multicolumn{6}{|| c ||}{\textbf{Descripción}}\\
	\hline
	\multicolumn{6}{|| p{14cm} ||}{
	Se pretende comprobar la correcta liberación de herramientas 
	después de que el usuario haya sido expulsado del sistema por 
	exceso de tiempo.
	}\\
	\hline
	\multicolumn{6}{|| c ||}{\textbf{Acciones}}\\
	\hline
	\multicolumn{6}{|| p{14cm} ||}{
		\begin{enumerate}
		\item El CE reserva varias herramientas.
		\item Ejecuta una acción hasta que se le acaba el tiempo.
		\item El sistema expulsa al usuario, cerrando las acciones en 
		ejecución y liberando las herramientas.
		\item Por último, se ejecutan las acciones.
		\end{enumerate}
	}\\
	\hline
	\hline
\end{tabular}
\end{center}
\end{table}

\begin{table}[H]
\begin{center}
	\stepcounter{prueba}
	\stepcounter{pfuncion}
\begin{tabular}{|| m{1cm} m{2cm} | m{2.5cm} m{2.5cm} | l @{\extracolsep{\fill}} c ||}
	\hline
	\hline
	\multicolumn{6}{|| c ||}{\textsc{Prueba \arabic{prueba}}}\\
	\hline
	\textbf{ID} & PF-\Alph{pfuncion} &
	\textbf{Configuración} & Común & 
	\textbf{Resultado} & Válido\\
	\hline
	\hline
	\multicolumn{6}{|| c ||}{\textbf{Descripción}}\\
	\hline
	\multicolumn{6}{|| p{14cm} ||}{
	Esta prueba fue diseñada para utilizar los servicios externos de 
	cada una de las herramientas.
	}\\
	\hline
	\multicolumn{6}{|| c ||}{\textbf{Acciones}}\\
	\hline
	\multicolumn{6}{|| p{14cm} ||}{
		\begin{enumerate}
		\item Los CE \emph{A} y \emph{B} reservan varias herramientas, 
		una de ellas de datos para acceder a su servicio externo.
		\item Ambos ejecutan las herramientas utilizando a la vez 
		programas externos para el manejo de los servicios.
		\item Todos los servicios y las acciones en ejecución 
		responden correctamente, tanto en el envío de datos como en la 
		recepción.
		\end{enumerate}
	}\\
	\hline
	\hline
\end{tabular}
\end{center}
\end{table}

\begin{table}[H]
\begin{center}
	\stepcounter{prueba}
	\stepcounter{pfuncion}
\begin{tabular}{|| m{1cm} m{2cm} | m{2.5cm} m{2.5cm} | l @{\extracolsep{\fill}} c ||}
	\hline
	\hline
	\multicolumn{6}{|| c ||}{\textsc{Prueba \arabic{prueba}}}\\
	\hline
	\textbf{ID} & PF-\Alph{pfuncion} &
	\textbf{Configuración} & Común & 
	\textbf{Resultado} & Válido\\
	\hline
	\hline
	\multicolumn{6}{|| c ||}{\textbf{Descripción}}\\
	\hline
	\multicolumn{6}{|| p{14cm} ||}{
	Prueba la capacidad de insertar herramientas \emph{en 
	caliente}, es decir, mientras el sistema está en pleno 
	funcionamiento, atendidendo peticiones de los usuarios.
	}\\
	\hline
	\multicolumn{6}{|| c ||}{\textbf{Acciones}}\\
	\hline
	\multicolumn{6}{|| p{14cm} ||}{
		\begin{enumerate}
		\item El CE \emph{A} reserva algunas herramientas.
		\item El CE \emph{B} se conecta al sistema pero aún no reserva nada.
		\item Se inserta en el L \emph{1} una herramienta y se arma.
		\item El CE \emph{B} puede reservar desde ese momento la herramienta.
		\end{enumerate}
	}\\
	\hline
	\hline
\end{tabular}
\end{center}
\end{table}

\begin{table}[H]
\begin{center}
	\stepcounter{prueba}
	\stepcounter{pfuncion}
\begin{tabular}{|| m{1cm} m{2cm} | m{2.5cm} m{2.5cm} | l @{\extracolsep{\fill}} c ||}
	\hline
	\hline
	\multicolumn{6}{|| c ||}{\textsc{Prueba \arabic{prueba}}}\\
	\hline
	\textbf{ID} & PF-\Alph{pfuncion} &
	\textbf{Configuración} & Remoto & 
	\textbf{Resultado} & Válido\\
	\hline
	\hline
	\multicolumn{6}{|| c ||}{\textbf{Descripción}}\\
	\hline
	\multicolumn{6}{|| p{14cm} ||}{
	La última prueba de funciones corresponde a la respuesta de un L 
	cuando está sobrecargado de acciones en ejecución.
	}\\
	\hline
	\multicolumn{6}{|| c ||}{\textbf{Acciones}}\\
	\hline
	\multicolumn{6}{|| p{14cm} ||}{
		\begin{enumerate}
		\item El L \emph{1} se configura para que sólo acepte 4 
		procesos en ejecución.
		\item Los CE \emph{A} y \emph{B} reservan varias herramientas 
		que pertenecen a L \emph{1} y ejecutan todas las acciones 
		correspondientes.
		\item Las ejecuciones se ven bloqueadas con el estado de 
		espera hasta que alguna con estado de ejecución termine.
		\item Las peticiones de ejecución se ejecutan con el orden de 
		llegada.
		\end{enumerate}
	}\\
	\hline
	\hline
\end{tabular}
\end{center}
\end{table}

\section{Pruebas de recuperación de errores}
Las siguientes pruebas comprobaron que bajo ciertas situaciones de 
error o incontroladas, el sistema respondía tal y como corresponde a 
la recuperación de errores implementada.

Muchos de los errores que pueden surgir en una plataforma RLF son 
externos a la misma, por lo que su control se resume en una parada 
controlada del componente o componentes afectados.

\begin{table}[H]
\begin{center}
	\stepcounter{prueba}
	\stepcounter{perror}
\begin{tabular}{|| m{1cm} m{2cm} | m{2.5cm} m{2.5cm} | l @{\extracolsep{\fill}} c ||}
	\hline
	\hline
	\multicolumn{6}{|| c ||}{\textsc{Prueba \arabic{prueba}}}\\
	\hline
	\textbf{ID} & PE-\Alph{perror} &
	\textbf{Configuración} & Remoto & 
	\textbf{Resultado} & Válido\\
	\hline
	\hline
	\multicolumn{6}{|| c ||}{\textbf{Descripción del error}}\\
	\hline
	\multicolumn{6}{|| p{14cm} ||}{
	La herramienta en ejecución sufre un fallo bloqueante.
	}\\
	\hline
	\multicolumn{6}{|| c ||}{\textbf{Respuesta}}\\
	\hline
	\multicolumn{6}{|| p{14cm} ||}{
		\begin{enumerate}
		\item El L encargado de la herramienta espera hasta que se 
		agote el tiempo de la acción. Dado ese caso, se considera un 
		error bloqueante.
		\item L establece el estado de error para esa acción, para por 
		completo la ejecución de la acción y añade una petición para 
		ejecutar el limpiador.
		\item El CE recibe la notificación de que la acción actual ha 
		sido parada por un error surgido.
		\item Se muestra las excepciones enviadas por la herramienta y 
		el estado final de la misma.
		\end{enumerate}
	}\\
	\hline
	\hline
\end{tabular}
\end{center}
\end{table}

\begin{table}[H]
\begin{center}
	\stepcounter{prueba}
	\stepcounter{perror}
\begin{tabular}{|| m{1cm} m{2cm} | m{2.5cm} m{2.5cm} | l @{\extracolsep{\fill}} c ||}
	\hline
	\hline
	\multicolumn{6}{|| c ||}{\textsc{Prueba \arabic{prueba}}}\\
	\hline
	\textbf{ID} & PE-\Alph{perror} &
	\textbf{Configuración} & Remoto & 
	\textbf{Resultado} & Válido\\
	\hline
	\hline
	\multicolumn{6}{|| c ||}{\textbf{Descripción del error}}\\
	\hline
	\multicolumn{6}{|| p{14cm} ||}{
	Uno de los laboratorios es desarmado por el administrador o sufre 
	una parada por un error.
	}\\
	\hline
	\multicolumn{6}{|| c ||}{\textbf{Respuesta}}\\
	\hline
	\multicolumn{6}{|| p{14cm} ||}{
		\begin{enumerate}
		\item El CE recibe el mensaje de error y comprueba las 
		herramientas reservadas de ese laboratorio.
		\item Se muestra el diálogo de aviso por cierre inesperado de 
		L.
		\end{enumerate}
	}\\
	\hline
	\hline
\end{tabular}
\end{center}
\end{table}

\begin{table}[H]
\begin{center}
	\stepcounter{prueba}
	\stepcounter{perror}
\begin{tabular}{|| m{1cm} m{2cm} | m{2.5cm} m{2.5cm} | l @{\extracolsep{\fill}} c ||}
	\hline
	\hline
	\multicolumn{6}{|| c ||}{\textsc{Prueba \arabic{prueba}}}\\
	\hline
	\textbf{ID} & PE-\Alph{perror} &
	\textbf{Configuración} & Remoto & 
	\textbf{Resultado} & Válido\\
	\hline
	\hline
	\multicolumn{6}{|| c ||}{\textbf{Descripción del error}}\\
	\hline
	\multicolumn{6}{|| p{14cm} ||}{
	El P sufre un error o su nodo es desconectado de la red.
	}\\
	\hline
	\multicolumn{6}{|| c ||}{\textbf{Respuesta}}\\
	\hline
	\multicolumn{6}{|| p{14cm} ||}{
		\begin{enumerate}
		\item Cuando los CE o CW se intentan conectar al P, o 
		realizar una acción si ya lo estaban, se muestra el diálogo de 
		problema con el servidor central.
		\end{enumerate}
	}\\
	\hline
	\hline
\end{tabular}
\end{center}
\end{table}

\begin{table}[H]
\begin{center}
	\stepcounter{prueba}
	\stepcounter{perror}
\begin{tabular}{|| m{1cm} m{2cm} | m{2.5cm} m{2.5cm} | l @{\extracolsep{\fill}} c ||}
	\hline
	\hline
	\multicolumn{6}{|| c ||}{\textsc{Prueba \arabic{prueba}}}\\
	\hline
	\textbf{ID} & PE-\Alph{perror} &
	\textbf{Configuración} & Remoto & 
	\textbf{Resultado} & Válido\\
	\hline
	\hline
	\multicolumn{6}{|| c ||}{\textbf{Descripción del error}}\\
	\hline
	\multicolumn{6}{|| p{14cm} ||}{
	El CE sufre un error o una desconexión del sistema.
	}\\
	\hline
	\multicolumn{6}{|| c ||}{\textbf{Respuesta}}\\
	\hline
	\multicolumn{6}{|| p{14cm} ||}{
		\begin{enumerate}
		\item El CE no consigue desconectarse del sistema de manera 
		normal.
		\item El P bloquea la cuenta de usuario hasta que se ponga en 
		contacto con el administrador para que la desbloquee.
		\end{enumerate}
	}\\
	\hline
	\hline
\end{tabular}
\end{center}
\end{table}

\begin{table}[H]
\begin{center}
	\stepcounter{prueba}
	\stepcounter{perror}
\begin{tabular}{|| m{1cm} m{2cm} | m{2.5cm} m{2.5cm} | l @{\extracolsep{\fill}} c ||}
	\hline
	\hline
	\multicolumn{6}{|| c ||}{\textsc{Prueba \arabic{prueba}}}\\
	\hline
	\textbf{ID} & PE-\Alph{perror} &
	\textbf{Configuración} & Remoto & 
	\textbf{Resultado} & Válido\\
	\hline
	\hline
	\multicolumn{6}{|| c ||}{\textbf{Descripción del error}}\\
	\hline
	\multicolumn{6}{|| p{14cm} ||}{
	El CW sufre un error o una desconexión del sistema.
	}\\
	\hline
	\multicolumn{6}{|| c ||}{\textbf{Respuesta}}\\
	\hline
	\multicolumn{6}{|| p{14cm} ||}{
		\begin{enumerate}
		\item El CW no consigue desconectarse del monitor de manera 
		normal.
		\item P no bloquea la cuenta de usuario, por lo que se puede 
		volver a conectar desde un CW o un CE.
		\end{enumerate}
	}\\
	\hline
	\hline
\end{tabular}
\end{center}
\end{table}


\begin{figure}
	\centering
	\includegraphics[scale=0.3]{images/mac.png}
	\caption[RLF en Mac]{En estas pruebas también se han incluido 
	nodos con el sistema Macintosh.}
	\label{fig:mac}
\end{figure}

\section{Pruebas de estrés}
\label{sec:pruebasestres}
Por último, estas pruebas están recogidas para verificar el correcto 
funcionamiento en ambientes con gran carga o baja potencia. El número 
de componentes y clientes es aumentado.

\begin{table}[H]
\begin{center}
	\stepcounter{prueba}
	\stepcounter{pestres}
\begin{tabular}{|| m{1cm} m{2cm} | m{2.5cm} m{2.5cm} | l @{\extracolsep{\fill}} c ||}
	\hline
	\hline
	\multicolumn{6}{|| c ||}{\textsc{Prueba \arabic{prueba}}}\\
	\hline
	\textbf{ID} & PX-\Alph{pestres} &
	\textbf{Configuración} & Básico & 
	\textbf{Resultado} & Válido\\
	\hline
	\hline
	\multicolumn{6}{|| c ||}{\textbf{Descripción del conjunto}}\\
	\hline
	\multicolumn{6}{|| p{14cm} ||}{
	Todos los componentes se encuentran en un \emph{netbook}, con un 
	procesador Atom 450 y 1 GB de memoria RAM. Acceden dos CE a 
	él, uno de ellos ejecuta la herramienta de vídeo por 
	\emph{streaming}. Acceden otros dos CW mediante dispositivos 
	\emph{Android}.
	}\\
	\hline
	\multicolumn{6}{|| c ||}{\textbf{Respuesta}}\\
	\hline
	\multicolumn{6}{|| p{14cm} ||}{
	El sistema responde correctamente ocupando el 95\% del procesador 
	y 150 MBs de RAM. El vídeo se recibe de manera fluida y la 
	interfaz del CE \emph{A} funciona correctamente.
	}\\
	\hline
	\hline
\end{tabular}
\end{center}
\end{table}

\begin{table}[H]
\begin{center}
	\stepcounter{prueba}
	\stepcounter{pestres}
\begin{tabular}{|| m{1cm} m{2cm} | m{2.5cm} m{2.5cm} | l @{\extracolsep{\fill}} c ||}
	\hline
	\hline
	\multicolumn{6}{|| c ||}{\textsc{Prueba \arabic{prueba}}}\\
	\hline
	\textbf{ID} & PX-\Alph{pestres} &
	\textbf{Configuración} & Remoto & 
	\textbf{Resultado} & No válido\\
	\hline
	\hline
	\multicolumn{6}{|| c ||}{\textbf{Descripción del conjunto}}\\
	\hline
	\multicolumn{6}{|| p{14cm} ||}{
	En un entorno donde las conexiones entre las redes son de muy baja 
	velocidad, se prueba el acceso a las diferentes herramientas. El P 
	y L \emph{1} están conectados a la red mediante \emph{tethering} con 
	un teléfono móvil con acceso 3G (alrededor de 200 kB/s de bajada y 
	64 kB/s de subida en Madrid \cite{tresg}), así como los dos CE y 
	otro CW conectados a redes normales.
	}\\
	\hline
	\multicolumn{6}{|| c ||}{\textbf{Respuesta}}\\
	\hline
	\multicolumn{6}{|| p{14cm} ||}{
	Los accesos al proveedor se ven ralentizados. Cuando se ejecuta 
	una acción, a pesar de que la interfaz no se bloquea, el flujo de 
	datos es demasiado lento como para poder tener la continuidad 
	necesaria para un correcto funcionamiento. 
	}\\
	\hline
	\hline
\end{tabular}
\end{center}
\end{table}

\begin{table}[H]
\begin{center}
	\stepcounter{prueba}
	\stepcounter{pestres}
\begin{tabular}{|| m{1cm} m{2cm} | m{2.5cm} m{2.5cm} | l @{\extracolsep{\fill}} c ||}
	\hline
	\hline
	\multicolumn{6}{|| c ||}{\textsc{Prueba \arabic{prueba}}}\\
	\hline
	\textbf{ID} & PX-\Alph{pestres} &
	\textbf{Configuración} & Educativa & 
	\textbf{Resultado} & Válido\\
	\hline
	\hline
	\multicolumn{6}{|| c ||}{\textbf{Descripción del conjunto}}\\
	\hline
	\multicolumn{6}{|| p{14cm} ||}{
	Se accede desde 15 terminales en los laboratorios 
	informáticos de la universidad hacia el sistema que se encuentra 
	en el taller de Automática. Las redes de conexión son las propias 
	de la universidad.
	}\\
	\hline
	\multicolumn{6}{|| c ||}{\textbf{Respuesta}}\\
	\hline
	\multicolumn{6}{|| p{14cm} ||}{
	La plataforma RLF funciona sin ningún retardo.
	}\\
	\hline
	\hline
\end{tabular}
\end{center}
\end{table}

\begin{table}[H]
\begin{center}
	\stepcounter{prueba}
	\stepcounter{pestres}
\begin{tabular}{|| m{1cm} m{2cm} | m{2.5cm} m{2.5cm} | l @{\extracolsep{\fill}} c ||}
	\hline
	\hline
	\multicolumn{6}{|| c ||}{\textsc{Prueba \arabic{prueba}}}\\
	\hline
	\textbf{ID} & PX-\Alph{pestres} &
	\textbf{Configuración} & Industrial & 
	\textbf{Resultado} & Válido\\
	\hline
	\hline
	\multicolumn{6}{|| c ||}{\textbf{Descripción del conjunto}}\\
	\hline
	\multicolumn{6}{|| p{14cm} ||}{
	Los componentes se encuentran distribuidos en distintas 
	localidades separadas por varios kilómetros de distancia. Además,
	uno de los laboratorios y un CE se situaron en Osaka, Japón.
	}\\
	\hline
	\multicolumn{6}{|| c ||}{\textbf{Respuesta}}\\
	\hline
	\multicolumn{6}{|| p{14cm} ||}{
	La plataforma RLF funciona sin ningún retardo.
	}\\
	\hline
	\hline
\end{tabular}
\end{center}
\end{table}

Como se ha comprobado en las numerosas pruebas, la plataforma RLF 
\emph{Prototype 1} ha superado los objetivos de estabilidad y 
seguridad. Por supuesto, para obtener una versión completa para el 
mercado, ha de someterse a pruebas de estrés mucho más severas. Al 
estar tratando con un sistema crítico donde los errores pueden suponer 
un gasto en reparaciones bastante importante, es necesario un control 
de calidad exhaustivo y de gran precisión.

\cleardoublepage



% Problemas y mejoras
% Problemas y mejoras

\capitulo{Problemas y mejoras}{problemas}{
Se recoge en este capítulo los problemas derivados de la 
implementación de la plataforma RLF así como los trabajos futuros que 
se deberán tener en cuenta para posteriores versiones.
}

\section{Problemas encontrados en el desarrollo}
Se listan a continuación los problemas referentes a la implementación 
y a las tecnologías usadas. Sólo han sido incluidos aquellos más importantes 
que han determinado el desarrollo de la plataforma RLF.

\subsection{La sincronización}
Siendo un problema que arrastran muchas de las plataformas que existen 
actualmente en el mercado, fue arrastrado desde el principio de la 
implementación. Se ha invertido mucho tiempo en conseguir un sistema 
que es asíncrono a partir de muchos componentes síncronos.

Los sistemas síncronos \cite{SistemasOperativos} son aquellos que se 
bloquean a la espera de una comunicación concreta por parte del otro 
interlocutor. En cambio, los sistemas asíncronos, pueden realizar 
otras tareas mientras el otro interlocutor genera la información. Los 
distintos componentes síncronos que se aprecian, como la entrada y 
salida estándar de todas las herramientas, han sido modificados para 
poder permitir no bloquear al resto de componentes.

Esto se ha conseguido con la sustitución de los mecanismos de lectura 
y escritura (de \emph{sockets}, ficheros y teclado) tradicionales, que 
se presentan en la máquina virtual de Java como \emph{streams} 
pertenecientes a las librerías ``java.io''. A partir de la versión 
1.4.2 de Java, Sun Microsystems añadió las librerías ``java.nio'' 
\index{Java!nio} \cite{JavaNIO} que aportaban una nueva forma de ver 
la entrada y salida para la máquina virtual. El conjunto fue llamado 
Java New I/O, en referencia al sistema antiguo, Java I/O. La 
estructura de estas librerías es muy parecida a los sistemas Unix, 
puediendo utilizar funciones no bloqueantes, y algunas herramientas 
muy útiles para no malgastar tiempo de cálculo. Además, incluyen 
\emph{buffers} gestionados por el propio sistema operativo anfitrión 
que dotan de una mayor velocidad a la, de por si lenta, máquina de 
Java.

\begin{figure}[h]
	\centering
	\includegraphics[scale=0.9]{images/io-vs-nio.png}
	\caption[Java.io VS Java.nio]{Comparación de las librerías 
	``java.io'' y su nueva versión ``java.nio''.}
	\label{fig:niovsio}
\end{figure}

En la figura \ref{fig:niovsio} se puede comprobar como con las 
distintas versiones de estos dos conjuntos librerías se mejora la 
velocidad de lectura y escritura. Se realizó el estudio con varios 
ficheros de distinto tamaño (coordenada X) y la velocidad de lectura y 
posterior escritura en KB/s (coordenada Y). Las series corresponden a 
dos experimentos, A y B, con la versión 2.2 de ``java.io'' y la 
versión 2.4 de ``java.nio'' \cite{Javaiovsnio}.

\subsection{La portabilidad de Java}
Aunque es una plataforma que puede ejecutarse en múltiples sistemas, 
Java necesita ``una ayuda'' para poder realizar bien su cometido en 
Linux y Windows por igual. El primer problema que se encuentra es en 
el acceso a archivos del sistema de ficheros. El árbol de rutas es 
distinto para \emph{ext4} \footnote{Sistema de ficheros que las distribuciones 
Linux usan en la actualidad}. que para \emph{ntfs} \footnote{Sistema 
de ficheros moderno para las últimas versiones de Windows, como XP, 
Vista y Windows 7.}. Esto conlleva a que el código de implementación 
debe ser lo suficientemente genérico como para que no surjan problemas 
a la hora de portar los distintos componentes.

Aunque en el desarrollo de RLF no se ha optado en ningún momento por 
la división de código, es decir, incluir un código para cada 
sistema operativo, se han realizado múltiples cambios en el diseño 
para adaptarlo a un único código. Se puede ver a continuación, un 
ejemplo de división de código atendiendo al sistema operativo 
\cite{java2}:

\begin{verbatim}
String osName = System.getProperty("os.name");
if ((osName.equals("Windows NT") 
    || osName.equals("Windows 7")
    || osName.equals("Windows XP")) {
        // Do something...
} else if ((osName.equals("Linux")
    || osName.equals("Mac")) {
        // Do another thing...
} else {
        // Cry.
}
\end{verbatim}

\subsection{\emph{Streaming} de vídeo y Java}
Fueron muchos días los que se intentó, sin éxito, utilizar la 
cámara web de la herramienta RLF\_Video a través de Java. Para ello 
se utilizó el \emph{framework} aportado por Sun Microsystems para el 
control de sistemas multimedia llamado JMF (\emph{Java Media 
Framework}).

A pesar de que el nuevo dueño de las tecnologías Java, Oracle, 
indique en su página que sigue en activo y se está desarrollando 
actualmente aplicaciones con él, la última versión data de 2001, con 
la versión de Java anterior a la 1.4.2 (la primera considerada 
``moderna''). Además, la documentación para el desarrollo ya no es 
accesible desde las páginas oficiales.

Se optó por utilizar el servidor de VLC, incluido en las 
distribuciones Linux y adecuarlo mediante BASH para poder ejecutarlo 
desde la plataforma RLF.

\subsection{La tarjeta PCI-1711-BE}
\index{PCI-1711-BE}
Uno de los elementos más importante que se ha aportado a las 
herramientas presentadas, es la interactividad con un \emph{hardware} 
donde su acceso era en el mismo lugar donde se encontraba. Se utilizó 
una tarjeta que provee de entradas y salidas electrónicas, llamada 
Advantech PCI-1711-BE que disponía de un conjunto de librerías para 
interactuar por medio del \emph{software}, que podían ser utilizadas 
en sistemas Windows y Linux.

Todo intento por utilizar la documentación (escrita en 1996) y las 
herramientas aportadas fue un fracaso hasta que se consiguieron unos 
ejemplos que se podían utilizar en el IDE Visual Studio 2005. Siendo 
aún incompatibles con los sistemas actuales, mediante sustitución de 
código antiguo y de librerías que ya no existen en Windows, se pudo 
adecuar a la plataforma .NET, y con ello, a la plataforma RLF.

Se puede ver a continuación el código original de algunos ejemplos de 
dicha tarjeta:

\begin{verbatim}
/*
 (...)
 * Revision       : 1.00                                           *
 * Date           : 7/1/2003                   Advantech Co., Ltd. *
 (...)
 */
 
 (...)
 
// Estos tipos de datos no son compatibles con .NET
DWORD  dwErrCde;
ULONG  lDevNum;
long   lDriverHandle;
USHORT usChan;
 (...)
 
// Funciones no soportadas por Windows
getch();
 (...)


\end{verbatim}

\clearpage

\section{Próximos pasos}
No cabe duda que la plataforma RLF aquí presentada necesita más 
desarrollo para poder afianzarla como un producto comercial. Estando 
aún en la versión \emph{Prototype}, requiere de determinadas tareas 
para poder ser implantada en entornos de trabajo. Se recopilan a 
continuación los próximos trabajos propuestos:

\begin{itemize}
\item Se pueden aplicar pruebas de estrés a la plataforma mayores de 
las que se incluyen en la sección \ref{sec:pruebasestres}, contando 
con varias decedas de laboratorios, y varios cientos de usuarios.
\item Acoplamiento del sistema de cifrado de comunicaciones con 
túneles SSH que aportarían blindaje a la plataforma, de la misma 
forma que se muestra en la figura \ref{fig:ssh}.
\item Diseño de tareas automatizadas para la base de datos del 
proveedor, como comprobación de nodos de la red o de usuarios con 
estados erróneos.
\item Creación de ``paquetes'' de aplicaciones para alumnos, que 
permitan obtener todo el \emph{software} necesario en un solo 
instalador (como por ejemplo, reproductores de video, clientes FTP, 
etc) pudiendo además reservar las herramientas en conjunto con 
anterioridad.
\end{itemize}

\begin{figure}[H]
	\centering
	\includegraphics[scale=0.7]{images/ssh.png}
	\caption[Túnel SSH]{Ejemplo de la arquitectura de un túnel SSH.}
	\label{fig:ssh}
\end{figure}

\subsection{Próximas versiones de RLF}
Dada la base de RLF Prototype, cabe proponer otras funciones que, con unos 
determinados cambios, pueden llevarse a cabo, aprovechando las 
características principales del mismo. Las líneas de desarrollo 
pueden variarse e incluso crear nuevas, todo depende de las necesidades 
del centro que utilice RLF. Se muestra en la figura \ref{fig:versiones} 
continuación una idea de las próximas versiones y sus cambios.

\begin{figure}[H]
	\centering
	\includegraphics[scale=0.7]{images/versiones.png}
	\caption[Próximas líneas RLF]{Próximas posibles líneas RLF.}
	\label{fig:versiones}
\end{figure}

\begin{description}
\item[RLF.org] Corresponde a la evolución natural del proyecto, con 
mejoras de estabilidad y seguridad, y una fase completa de 
\emph{tests} de estrés. Sería la candidata para salir al 
mercado y adecuarse a varias líneas de trabajo. 
\item[RLF.edu] Línea especializada en entornos educativos, donde se 
pueden establecer configuraciones estándar para la realización de 
prácticas, como por ejemplo, que todos los laboratorios contengan una 
herramienta de vídeo, y que obligue a la interfaz a contener un 
reproductor de \emph{streaming} embebido.
\item[RLF@home] Cambiando el sistema de reserva de herramientas, y 
liberando de carga a los laboratorios, se puede convertir en una 
plataforma para edificios o casas inteligentes. Así se podrían 
gestionar distintos dispositivos desde el cliente.
\item[RLF Science] Al igual que algunos centros de investigación, se 
pueden alquilar por determinado tiempo elementos \emph{hardware} a 
usuarios de Internet. Como por ejemplo, en el observatorio astronómico 
de Chile, se puede utilizar el telescopio si se paga una cuota. 
También puede servir como plataforma para sistemas \emph{grid}, 
proporcionando capacidad de cálculo y de almacenamiento sustituyendo las 
herramientas \emph{hardware} por terminales de acceso a distintos 
nodos de una red de computación.
\end{description}




% Conclusión
% Conclusión

\capitulo{Conclusiones}{conclusiones}{
``Sólo es posible avanzar cuando se mira lejos. Solo cabe progresar 
cuando se piensa en grande.'' \emph{José Ortega y Gasset}
}

\begin{figure}
	\centering
	\includegraphics[scale=0.5]{images/logo_rlf.png}
\end{figure}

\clearpage

\vspace*{1.5cm}

El avance de las nuevas tecnologías significa el avance de la 
sociedad, y de todos sus aspectos, desde la forma de relacionarse 
hasta la forma de trabajar. Es indudable que Internet revoluciona los 
sistemas actuales, sin que se pueda ignorar. Las empresas se han visto 
obligadas a cambiar su modelo de negocio para adecuarse a estos 
tiempos. Y por supuesto los centros de enseñanza. Se pueden encontrar 
universidades donde no es necesario ir a clase, ya que mediante 
plataformas colaborativas disponibles en Internet, así como la llegada 
de la emisión de sonido y vídeo a través de red el alumno puede 
aprender casi de la misma manera que si se desplazada hasta el centro 
educativo. Pero poco a poco, ese ``casi'' irá desapareciendo, 
aumentando a su vez la posibilidad de acceso a enseñanzas superiores.

La mayor parte de las retribuciones positivas de este proyecto provienen
de haber desarrollado una plataforma desde la capa más baja hasta la 
que está en contacto con el usuario. El uso de las múltiples 
tecnologías y su unión como un único sistema ha demostrado que es 
necesario evolucionar la tecnología desde diferentes puntos de vista, 
y sin ser guiados por un único objetivo.

Este proyecto no sólo ha sido la culminación de seis años de 
estudio, si no también un aspecto importante en la formación para ser 
ingeniero informático, y más aún si la especialidad son los sistemas 
distribuidos, que ahora tanta importancia han adquirido desde la 
aparición de la ``nube'' y los sistemas portátiles.

Así pues, habiendo cumplido con los objetivos marcados al inicio del 
proyecto, la satisfacción de haber realizado este proyecto es plena, 
ya que no sólo se ha conseguido crear una plataforma distribuida 
totalmente funcional, sino que se ha realizado con las mismas 
tecnologías que utilizan las grandes empresas en el mundo industrial y 
educativo. Muchas de las decisiones tomadas durante el desarrollo de 
este proyecto han venido influenciadas por conceptos que actualmente 
los grandes proyectos también se plantean, como es el tema de la 
seguridad. Se han tenido que sortear dificultades muy presentes a la 
hora de desarrollar \emph{software}, que cada vez se hacen mayores 
por la cantidad de dispositivos y sistemas, creados por diferentes 
empresas y organizaciones, que se encuentran en los hogares y centros.

El que en un proyecto de fin de carrera sea utilizado en un escenario 
práctico, real y donde se espera continuar es algo muy poco común, y 
desde luego inmensamente satisfactorio. Se espera que este trabajo 
sirva para futuras ideas que ayuden a la comunidad educativa a formar 
a mejores profesionales e investigar nuevas tecnologías. El esfuerzo 
se ha visto recompensado en forma de nuevas ideas.

\vspace{1cm}

\begin{flushright}
Madrid, 1 de Octubre de 2011.\\
\vspace{2.5cm}
Fdo: Carlos A. Rodríguez Mecha
\end{flushright}

\cleardoublepage


\appendix

% Apéndice: Presupuesto
% Apéndice: Presupuesto

\capitulo{Presupuesto}{presupuesto}{
La versatilidad aportada a este proyecto reduce en gran medida 
los gastos necesarios para el despliegue del mismo. Se adjunta a 
continuación un presupuesto aproximado dependiendo de las necesidades 
actuales del lugar de trabajo o enseñanza. Todos los precios aquí 
recogidos datan de Septiembre de 2011.
}

\section*{Costes de desarrollo}
No son necesarios para el despliegue pero cuantifican la cantidad de 
trabajo empleada para la creación de la plataforma, así con su 
diseño y documentación. 

\subsection*{Coste de personal}
Este cálculo se realiza atendiendo a la planificación que se incluyó 
en la tabla \ref{tab:planificacion} del capítulo introductorio. El 
precio por hora está estimado según los sueldos en Septiembre de 2011 
en Madrid:

\begin{table}[H]
\begin{center}
\begin{tabular*}{12cm}{|| p{8.5cm} @{\extracolsep{\fill}} | r ||}
	\hline
	\hline
	Concepto & Coste en \euro\\
	\hline
	\hline
	Ing. Informático por hora & 30,00\\
	Corrector de documentación por hora & 17,00\\
	\hline
	\textbf{TOTAL} & 44.210,00\\
	\hline
	\hline
\end{tabular*}
\end{center}
	\caption{Coste de personal}
	\label{coste:personal}
\end{table}


\subsection*{Costes de material para el desarrollo}
Algunos de estos precios no son por la compra del producto (señalados 
con \# ), si no por el gasto aproximado en el uso del material.

\begin{table}[H]
\begin{center}
\begin{tabular*}{12cm}{|| p{8.5cm} @{\extracolsep{\fill}} | r ||}
	\hline
	\hline
	Concepto & Coste en \euro\\
	\hline
	\hline
	Servidor primario \# & 180,00\\
	Servidor secundario \# & 120,00\\
	Ordenador portátil \# & 120,00\\
	Dispositivo móvil & 169,00\\
	Sistema operativo Windows 7 Professional & 0,00\\
	Sistema operativo Windows XP Professional & 0,00\\
	Sistema operativo Unix & 0,00\\
	Base de datos MySQL & 0,00\\
	Base de datos SQLite & 0,00\\
	Servidor de aplicaciones Tomcat o JBoss & 0,00\\
	Cámara web & 13,50\\
	Micrófono & 6,99\\
	Advantech PCI-1711-BE \# & 60,00\\
	Altavoces doble canal & 29,50\\
	\hline
	\textbf{TOTAL} & 698,99\\
	\hline
	\hline
\end{tabular*}
\end{center}
	\caption{Coste material para el desarrollo}
	\label{coste:matdesarrollo}
\end{table}

Deduciéndose como coste total del desarrollo:

\begin{table}[H]
\begin{center}
\begin{tabular*}{12cm}{|| p{8.5cm} @{\extracolsep{\fill}} | r ||}
	\hline
	\hline
	Concepto & Coste en \euro\\
	\hline
	\hline
	Coste personal & 44.210,00\\
	Coste material & 698,99\\
	\hline
	\textbf{TOTAL} & 44.908,99\\
	\hline
	\hline
\end{tabular*}
\end{center}
	\caption{Coste total de desarrollo}
	\label{coste:totaldesarrollo}
\end{table}

\section*{Costes generales de mantenimiento}
Estos costes están asumidos para el despliegue de la plataforma, 
contando que su distribución se encuentre en la misma red local, 
aunque el número de nodos es indiferente.

\begin{table}[H]
\begin{center}
\begin{tabular*}{12cm}{|| p{8.5cm} @{\extracolsep{\fill}} | r ||}
	\hline
	\hline
	Concepto & Coste anual en \euro\\
	\hline
	\hline
	Línea de alta velocidad empresarial, \emph{Movistar} & 840,00\\
	Suministro eléctrico, \emph{Endesa} & 1.200,00\\
	Seguridad física y de datos, \emph{Movistar} & 600,00\\
	Coste estimado en reparaciones & 400,00\\
	\hline
	\textbf{TOTAL} & 3.040,00\\
	\hline
	\hline
\end{tabular*}
\end{center}
	\caption[Costes generales]{Costes aproximados de servicios 
	requeridos.}
	\label{coste:general}
\end{table}

Debido a que los componentes \hardware tienen más uso que aquellos con 
un limitado tiempo de acceso diario que se pueden encontrar en las 
universidades o puestos de trabajo, se asume un incremento en el coste 
de reparación anual.

\section*{Coste material por componentes}
A pesar de que todos los componentes de RLF pueden estar contenidos en 
la misma máquina, se especifican los costes de cada uno por separado, 
después se incluyen configuraciones estándar con su inversión 
correspondiente.

\subsection*{Proveedor y monitor}
Servidor \index{servidor} central de toda la plataforma, es la máquina con más 
prestaciones de todo el conjunto. Sus sistema es Unix y no requiere de 
ningún programa de pago debeido a su configuración estándar. Puede 
desdoblarse si se requiere para obtener una mayor velocidad y 
potencia, quedando un servidor con mayor capacidad para el proveedor y 
uno con menos para el monitor.

\begin{table}[H]
\begin{center}
\begin{tabular*}{12cm}{|| p{8.5cm} @{\extracolsep{\fill}} | r ||}
	\hline
	\hline
	Concepto & Coste en \euro\\
	\hline
	\hline
	PowerEdge T410 Tower Server, \emph{Dell} & 1.217,00\\
	\{Latitude 13 (Terminal portatil de acceso), \emph{Dell}\} & \{619,00\}\\
	Router (Incluido en el contrato de la línea) & 0,00\\
	Sistema de alimentación ininterrumpida (SAI) & 79,90\\
	Sistema operativo Unix & 0,00\\
	Base de datos MySQL & 0,00\\
	Servidor de aplicaciones Tomcat o JBoss & 0,00\\
	\hline
	\textbf{TOTAL} & 1.296,90\\
	\textbf{TOTAL \{con terminal de acceso\}} & 1.915,90\\
	\hline
	\hline
\end{tabular*}
\end{center}
	\caption[Coste del proveedor y monitor]{Costes aproximados de un 
	proveedor con monitor estándar.}
	\label{coste:proveedor}
\end{table}

Si se desea cobrar por los servicios de esta plataforma, se deberá 
pagar la licencia de MySQL y J2EE a Oracle. La terminal de acceso es 
un ordenador portatil de prestaciones medias con sistema operativo 
Unix. Es opcional y sólo se requiere uno por sistema RLF.

\subsection*{Laboratorio}
Por cada laboratorio en la plataforma se tiene que aplicar este coste. 
Dependiendo del sistema se deberá pagar la licencia de Windows.

\begin{table}[H]
\begin{center}
\begin{tabular*}{12cm}{|| p{8.5cm} @{\extracolsep{\fill}} | r ||}
	\hline
	\hline
	Concepto & Coste en \euro\\
	\hline
	\hline
	PowerEdge T110, \emph{Dell} & 378,00\\
	Sistema operativo Unix & 0,00\\
	\{Sistema operativo Windows 7 Professional\} & \{149,99\}\\
	Base de datos SQLite & 0,00\\
	\hline
	\textbf{TOTAL} & 378,00\\
	\textbf{TOTAL \{con Windows\}} & 527,99\\
	\hline
	\hline
\end{tabular*}
\end{center}
	\caption[Coste por laboratorio]{Costes aproximados por cada 
	laboratorio.}
	\label{coste:laboratorio}
\end{table}

\clearpage

\subsection*{Herramientas}

\begin{table}[H]
\begin{center}
\begin{tabular*}{12cm}{|| p{8.5cm} @{\extracolsep{\fill}} | r ||}
	\hline
	\hline
	Concepto & Coste en \euro\\
	\hline
	\hline
	eLight HD 720p Webcam, \emph{Trust} & 79,00\\
	Advantech PCI-1711-BE & 499,00\\
	Altavoces doble canal & 29,50\\
	\hline
	\textbf{TOTAL} & 607,50\\
	\hline
	\hline
\end{tabular*}
\end{center}
	\caption[Costes de herramientas]{Costes aproximados de las 
	herramientas entregadas.}
	\label{coste:herramientas}
\end{table}

\textbf{NOTA:} Aquí sólo se recogen los gastos de las herramientas 
entregadas junto con este proyecto.

\section*{Ejemplo de un sistema estándar}

Se compone de un proveedor central, un laboratorio de Unix y otro de 
Windows. Todos en diferentes máquinas y con todas las herramientas 
entregadas en uso. No es necesario una terminal de acceso, ya que el 
proveedor cuenta con ella.

\begin{table}[H]
\begin{center}
\begin{tabular*}{12cm}{|| p{8.5cm} @{\extracolsep{\fill}} | r ||}
	\hline
	\hline
	Concepto & Coste en \euro\\
	\hline
	\hline
	Gastos generales (al año) & 3.040,00\\
	Proveedor & 1.296,90\\
	Laboratorio Windows & 527,99\\
	Laboratorio Linux & 378,00\\
	Herramientas & 607,50\\
	\hline
	\textbf{TOTAL} & 5850,39\\
	\hline
	\hline
\end{tabular*}
\end{center}
	\caption[Presupuesto de un sistema estándar]{Presupuesto 
	aproximado de un sistema estándar.}
	\label{coste:sistema}
\end{table}

\cleardoublepage


% Apéndice: Manual de despliegue
% Apéndice: Manual de despliegue

\capitulo{Guía de despliegue}{despliegue}{
RLF está compuesto de varios módulos que trabajan conjuntamente como 
una única entidad, pero para eso, se requieren un conjunto de pasos 
que los \emph{orquestan} al inicio.
}

\section*{Plataforma Java}
Debido \index{Java} a que es un requisito en \textbf{cualquier} máquina de la 
plataforma RLF, es necesario instalarla antes que cualquier otro 
componente que se describe a continuación. Se pueden obtener de los 
siguientes enlaces:

\textit{Windows, Java 6 JRE}
\begin{itemize}
\item Descarga: \texttt{http://java.com/es/download/}
\item Manual: \texttt{http://java.com/es/download/help/index\_installing.xml}
\end{itemize}

\textit{Linux, Java 6 JDK}
\begin{itemize}
\item Descarga: \texttt{http://www.oracle.com/technetwork/java/javase/downloads/index.html}
\item Manual: \texttt{http://www.guia-ubuntu.org/index.php?title=Java}
\end{itemize}

\section*{Proveedor y monitor}
Estos dos módulos están originalmente creados para entornos Linux, 
aunque debido a las tecnologías usadas, pueden ser instalados en 
Windows. Sólo se explicará cómo ponerlos en funcionamiento en una 
distribución Linux.

Se componen de una base de datos principal en MySQL y de dos servicios 
web contenidos en el mismo paquete para su acceso externo. A 
continuación se muestran los pasos necesarios para incluirlos en el 
sistema.

\subsection*{Base de datos}
Se requerirá instalar los paquetes que proveen la plataforma MySQL 
completa. Para ello se echará mano del gestor de aplicaciones 
incluidos en las distintas distribuciones Linux. Los paquetes a 
instalar son los siguientes:

\textit{MySQL 5.1}

\begin{itemize}
\item \texttt{mysql-common} Paquete general de la plataforma MySQL.
\item \texttt{mysql-server} Contiene todo lo necesario para contener 
un servidor de base de datos.
\item \texttt{mysql-admin} Aplicación para la cómoda creación de 
usuarios y realización de \emph{backups} entre otros.
\item \texttt{mysql-client} Herramientas necesarias para el acceso a 
bases de datos MySQL.
\item \texttt{mysql-query-browser} La solución más cómoda que aporta 
MySQL para el manejo de los datos y creación de tablas en la base de 
datos.
\end{itemize}

\textbf{NOTA:} Es recomendable que la base de datos esté en la misma 
máquina que el servidor web ya que están configurados por defecto 
para acceder a una base de datos local, aunque se incluye en la 
sección de este mismo manual cómo configurar los servicios web para 
poder acceder a la base de datos de forma desacoplada.

Una vez instalados estos paquetes, será necesario configurar el 
servidor para atender las peticiones, en ello se asignará su IP de 
escucha y un puerto que será necesario para que cualquier parte de RLF 
acceda al proveedor. Véase para realizar esta configuración inicial:

\begin{verbatim}
http://dev.mysql.com/doc/refman/5.0/es/unix-post-installation.html
\end{verbatim}

Una vez el servidor esté completo con la base de datos activa, tocará 
el turno de aportarle información. Mediante \emph{MySQL Query Browser} 
se creará un esquema nuevo llamado ``rlf''. Después se desplegarán 
los \emph{scripts} que crean todas las tablas necesarias, se 
encuentran en la carpeta \emph{rlf/src/sql} y son el conjunto de 
archivos indicados en el fichero \emph{makedb.sql}.

\subsubsection*{Creación de usuarios.}
Cuando se pueda disponer de la base de datos, se recomienda crear los 
usuarios de acceso a la misma. Mínimo se requiere un administrador 
general, uno para el servicio web y uno por cada laboratorio que se 
vaya a disponer. Se podrá realizar mediante el programa instalado 
anteriormente \emph{MySQL Administrator}. Los permisos necesarios para 
el usuario del proveedor y de los laboratorios son los siguientes:

En el esquema ``mysql'':
\begin{itemize}
\item \emph{SELECT}
\end{itemize}

En el esquema ``rlf'':
\begin{itemize}
\item \emph{SELECT}
\item \emph{INSERT}
\item \emph{UPDATE}
\item \emph{DELETE}
\item \emph{REFERENCES}
\item \emph{LOCK\_TABLES}
\item \emph{EXECUTE}
\end{itemize}

Mientras que el usuario administrador tendrá todos los permisos en 
todos los esquemas disponibles.

\subsection*{Los servicios web}
Una vez que la base de datos está creada, es necesario instalar el 
servidor que permite ejecutar aplicaciones web. Se ha seleccionado la 
plataforma Tomcat que viene disponible con el IDE NetBeans de 
desarrollo en J2EE.

Antes de poder desplegar la aplicación, se requiere modificar el 
código de acceso a la base de datos del servicio web. Para ello, se 
utilizará el IDE NetBeans que se puede descargar de 
\texttt{http://netbeans.org/}, y se importará el proyecto contenido en 
\emph{rlf/projects/netbeans/RLF\_Provider}. A continuación se modifica la 
clase ``Database.java'':

\begin{verbatim}
/** Localización de la base de datos del proveedor. */
public final static String DATABASE = "jdbc:mysql://<IP>:<PUERTO>/rlf";
/** Usuario de acceso a la base de datos. */
private static String USER = <USUARIO PROVEEDOR>;
/** Contraseña del usuario. */
private static String PASS = <CONTRASEÑA>;
\end{verbatim}

Una vez modificado esto, se compilará el proyecto obteniendo como 
resultado un archivo ``Provider.war''. Después se instalará el 
programa Tomcat mediante los siguientes paquetes:

\emph{Tomcat 6.0} \index{Tomcat}

\begin{itemize}
\item \texttt{tomcat6} Paquete general de la plataforma Tomcat.
\item \texttt{tomcat6-admin} Aplicación para navegador que permite la 
configuración del sistema de manera muy sencilla.
\end{itemize}

Es necesario asignarle una IP y un puerto de escucha. Serán estos 
parámetros los que luego se configurarán en los clientes. Se puede 
obtener información de como configurar Tomcat en el siguiente enlace 
oficial:

\begin{verbatim}
http://tomcat.apache.org/tomcat-6.0-doc/setup.html
\end{verbatim}

Por último se copiará el archivo generado anteriormente en la carpeta 
\emph{webapps} donde se haya instalado Tomcat. A partir de ese momento 
se puede acceder al monitor y al proveedor por medio de web o de los 
clientes, aunque no habrá aún usuarios RLF introducidos. Las rutas 
típicas de acceso a estos dos servicios pueden ser:

\begin{verbatim}
http://<IP>:<PUERTO>/RLF/Provider
http://<IP>:<PUERTO>/RLF/Monitor
\end{verbatim}

\section*{Laboratorios}
Una vez instalada la máquina virtual de Java, no es necesario instalar 
ningún componente más. Conviene copiar todos los archivos contenidos en 
\emph{rlf/bin/lab} (\emph{rlf\textbackslash bin\textbackslash lab} en 
Windows) a otra carpeta y configurar el laboratorio modificando el 
fichero \emph{res/lab.conf} (o \emph{res\textbackslash lab.conf} en 
Windows) de la siguiente forma:

\begin{verbatim}
lab_name=<NOMBRE ÚNICO DEL LABORATORIO>
user=<USUARIO DE LA BASE DE DATOS>
pass=<CONTRASEÑA>
provider_host=<IP DE LA BASE DE DATOS DEL PROVEEDOR>
provider_port=<PUERTO DE LA BASE DE DATOS DEL PROVEEDOR>
labmanager_request_port=<PUERTO GENERAL DEL LABORATORIO. DLF=6400>
client_request_port=<PUERTO DE COMUNICACIONES 1. DFL=6401>
client_notification_port=<PUERTO DE COMUNICACIONES 2. DFL=6402>
provider_request_port=<PUERTO DE COMUNICACIONES 3. DFL=6403>
max_process=<NÚMERO MÁXIMO DE PROCESOS EN EJECUCIÓN. DFL=4>
\end{verbatim}

Es importante apuntar el puerto general del laboratorio ya que es 
necesario para su configuración mediante LabConsole (véase Manual de 
mantenimiento).

\subsection*{Insertar el laboratorio en el proveedor}
Para que los laboratorios puedan utilizar correctamente el proveedor, 
deben ser introducidos en la base de datos de este. Para ello es 
necesario obtener la IP de la máquina donde se ejecuta. Por cada 
laboratorio, se insertará en la base de datos central la siguiente 
sentencia SQL:

\begin{verbatim}
INSERT INTO lab (name, host, description) VALUES ('nombre', 'ip', 'descripción');
\end{verbatim}

\subsection*{Ejecutar el laboratorio}
Para iniciar el laboratorio sólo es necesario acceder a la carpeta 
contenedora y ejecutar el fichero ``lab.jar'' aunque es recomendable 
utilizar la terminal que disponga el sistema operativo para obtener 
posibles errores (aparte de los que aparezcan en el \emph{log}). Se 
podrá realizar esto mediante el siguiente comando, válido tanto en la 
terminal de Linux como en el CMD de Windows:

\begin{verbatim}
java -jar <ruta>/lab.jar
\end{verbatim}

\section*{Herramientas}
El despliegue de las herramientas sólo es referido a las que han sido 
entregadas con este proyecto. Para crear nuevas consultar el Manual de 
desarrollo y el Manual de mantenimiento.

\subsection*{Configuración inicial}
Para que las herramientas funcionen se ha de modificar las rutas 
contenidas en sus ficheros de configuración XML. También puede ser 
necesario cambiar algún parámetro de estas que dependa del sistema 
(como por ejemplo, en el caso de RLF\_Video la IP de acceso). Después, 
se podrán registrar (ver Manual de administrador) y obtener la clave 
única, que se incluirá en el código y posteriormente se compilarán. 
Se ha añadido los proyectos para los distintos IDEs de las 
herramientas para poder ser compilados (y arregladas las dependencias) 
sin dificultad.

\subsection*{Componentes necesarios}
Algunas herramientas necesitan de componentes instalados externos que 
se citan a continuación.

\begin{itemize}
\item \textbf{RLF\_DummyTool:} Ninguno.
\item \textbf{RLF\_Music:}
	\begin{itemize}
	\item Es necesario el programa para linux MPG123 que se puede 
	obtener del paquete con el mismo nombre.
	\item El paquete de desarrollo \emph{libsqlite-dev}.
	\end{itemize}
\item \textbf{RLF\_Video:}
	\begin{itemize}
	\item Utiliza el programa VLC que se obtiene de los repositorios 
	oficiales de la distribución bajo el paquete con el mismo nombre.
	\item El driver necesario para utilizar la cámara web en Linux.
	\end{itemize}
\item \textbf{RLF\_FreeMem:}
	\begin{itemize}
	\item Requiere la librería System.Data.SQLite instalada en el 
	sistema, que se puede obtener de 
	\texttt{http://sqlite.phxsoftware.com/}.
	\end{itemize}
\item \textbf{RLF\_Board:}
	\begin{itemize}
	\item También requiere la librería System.Data.SQLite.
	\item Necesita los drivers para la tarjeta asociada 
	(\texttt{http://www.advantech.com/}). Es recomendable 
	que se recompile resolviendo las dependencias de estas librerías 
	ya que no pueden incluirse en el propio ejecutable.
	\end{itemize}
\end{itemize}

\section*{Usuarios de RLF}
Para que se pueda usar y administrar la plataforma, es necesario dar 
de alta a los usuarios de la misma.

\subsection*{Administradores}
Serán los encargados de mantener las herramientas, así como 
eliminarlas y registrarlas. Además utilizan el programa LabConsole 
para manejar los distintos laboratorios. Para dar de alta a un 
administrador en el sistema es requerido insertar esta sentencia SQL 
en la base de datos de proveedor:

\begin{verbatim}
INSERT INTO user (user, hash_pass, email) 
         VALUES ('nombre', SHA1('contraseña'), 'email');
INSERT INTO admin (name, tlf)
         VALUES ('nombre', 'teléfono');
\end{verbatim}

A partir de aquí ya podrá realizar dichas tareas.

\subsection*{Clientes}
Podrán usar el monitor web y el cliente de escritorio. Son los 
usuarios finales de RLF. Tienen asociado un tiempo máximo de 
reserva de herramientas (a discreción de los administradores, 
generalmente una hora) y un rol que les permite acceder a herramientas 
de mayor nivel o menos (el rol \index{rol} 0 es el más básico y según aumenta es 
más restrictivo). Se deberá añadir estas sentencias SQL al proveedor:

\begin{verbatim}
INSERT INTO user (user, hash_pass, email)
         VALUES ('nombre', SHA1('contraseña'), 'email');
INSERT INTO client (name, timeout, role)
         VALUES ('nombre', tiempo, rol);
\end{verbatim}

\textbf{NOTA:} Un administrador no incluye el rol de cliente, por lo 
que si se requiere un usuario con ambos perfiles, habrá que darlo de 
alta en cada perfil por separado de esta forma:

\begin{verbatim}
INSERT INTO user (user, hash_pass, email)
         VALUES ('nombre', SHA1('contraseña'), 'email');
INSERT INTO admin (name, tlf)
         VALUES ('nombre', 'teléfono');
INSERT INTO client (name, timeout, role)
         VALUES ('nombre', tiempo, rol);
\end{verbatim}


% Apéndice: Manual de mantenimiento
% Apéndice: Manual de mantenimiento

\capitulo{Guía de mantenimiento}{mantenimiento}{
Este manual está dedicado a los administradores para realizar un 
correcto mantenimiento de la plataforma RLF mientras esta esté en 
ejecución. Además se incluyen algunos errores externos para su 
tratamiento.
}

\section*{Proveedor y monitor}
La única parte del proveedor y monitor que necesita mantenimiento es 
la base de datos global. Aunque la mayoría del tratamiento de datos es 
automático, en la versión \version entregada hay que realizar algunas 
tareas a mano.

\subsection*{Usuarios}
Debido a motivos de seguridad, si un usuario no se desconecta 
correctamente del sistema (en el cliente, en el monitor no ocurre 
esto) su cuenta se queda bloqueada para su revisión. Para permitir 
otra vez el uso de esta cuenta, hay que eliminar el \emph{token} de 
acceso del usuario, es decir, establecer su valor a \emph{NULL}. Se 
puede utilizar un programa de gestión de la base de datos como MySQL 
Browser (descargable en \texttt{http://dev.mysql.com/doc/query-browser/es/index.html}).

\subsection*{\emph{Backups}}
El proceso de duplicar \index{\emph{backup}} la base de datos es necesario 
cada cierto tiempo (a discreción de los administradores) para 
salvaguardar los datos de los usuarios. Realmente, este proceso es 
requerido sólo para estos datos, ya que la forma de desplegar el 
sistema hace que no sea muy complejo volver a añadir los datos. Para 
obtener información de cómo hacer un \emph{backup} en una base de 
datos MySQL consultar \texttt{http://dev.mysql.com/doc/refman/5.1/en/backup-methods.html}.

\subsection*{\emph{Logs}}
Es recomendable que se revisen los \emph{logs} \index{\emph{log}} del 
proveedor y monitor por si ha habido algún problema durante las 
ejecuciones. Se encuentran en la carpeta del mismo nombre y pueden ser 
consultados mediante un visor de texto genérico. También puede ser 
necesario borrarlos cada cierto tiempo para no saturar el disco.

\section*{Laboratorios}
El mantenimiento de los laboratorios se realiza mediante el programa 
\emph{LabConsole} incluido en este proyecto. Los laboratorios tienen 
tres estados principales:

\begin{itemize}
\item \textbf{Iniciado:} Es el estado por defecto de un laboratorio. 
En él se pueden realizar la mayoría de las tareas de mantenimiento, 
como registrar una nueva herramienta y eliminarla, y también armar el 
propio laboratorio. Cuando se desarma el laboratorio o se arranca por 
primera vez se alcanza este estado.
\item \textbf{Armado:} En este estado el laboratorio está preparado 
para escuchar las peticiones de los usuarios y comunicarse con el 
proveedor. No se puede realizar tareas de mantenimiento, sólo obtener 
el estado de cada una de las herramientas en tiempo real.
\item \textbf{Parado:} La ejecución del laboratorio se cierra. Para 
una parada controlada es necesario desarmarlo antes. También se puede 
llegar a este estado con una parada de emergencia.
\end{itemize}

\subsection*{\emph{LabConsole}}
Esta herramienta permite acceder a cualquier laboratorio de la red. 
Para utilizarlo se requiere un nombre de administrador y su 
contraseña. Las acciones que se pueden realizar son las siguientes:

\begin{verbatim}
java -jar LabConsole.jar [-h <IP> -p <Puerto>] -user <Usuario>
                         -pass <Contraseña> <Comando> <Parámetros>
\end{verbatim}

\begin{itemize}
\item \textbf{Armar:} Arma el laboratorio al que se accede. Antes de 
escuchar peticiones, se ejecutarán todos los limpiadores de las 
herramientas.
\begin{verbatim}
<Comando>: arm
<Parámetros>: ninguno
\end{verbatim}
\item \textbf{Desarmar:} Desarma el laboratorio. Es recomendable que 
se compruebe el estado antes de las herramientas, ya que si hay 
usuarios usándolas se desconectarán.
\begin{verbatim}
<Comando>: disarm
<Parámetros>: ninguno
\end{verbatim}
\item \textbf{Parar:} Para por completo el laboratorio. Si está armado 
lanzará un error.
\begin{verbatim}
<Comando>: stop
<Parámetros>: ninguno
\end{verbatim}
\item \textbf{Emergencia:} Para por completo el laboratorio y dejará 
las herramientas en el estado actual. Todos los usuarios conectados 
serán expulsados y no se aceptarán nuevas conexiones. La clave de 
emergencia por defecto es ``Emergency!''.
\begin{verbatim}
<Comando>: emergency
<Parámetros>: <Clave de emergencia>
\end{verbatim}
\item \textbf{Estado:} Obtiene el estado de las herramientas (en 
ejecución o parada) y del propio laboratorio. Esta acción se pude 
realizar cuando el laboratorio está iniciado o armado.
\begin{verbatim}
<Comando>: status
<Parámetros>: ninguno
\end{verbatim}
\item \textbf{Registrar:} Registra una herramienta. Es necesario 
indicar dónde se encuentra el fichero \emph{.xml} de configuración 
(ruta completa local). Cuando se registre se obtendrá el identificador 
único de la herramienta y su clave. Esta acción sólo puede llevarse 
a cabo cuando el laboratorio no está armado. Véase el Manual de Despliegue.
\begin{verbatim}
<Comando>: registry
<Parámetros>: <Ruta local del fichero XML>
\end{verbatim}
\item \textbf{Eliminar:} Elimina una herramienta. Es necesario 
indicar el identificador de la herramienta y su clave. También será 
borrada de la base de datos. Esta acción sólo puede llevarse a cabo 
cuando el laboratorio no está armado.
\begin{verbatim}
<Comando>: drop
<Parámetros>: <ID de la herramienta> <Clave de la herramienta>
\end{verbatim}
\end{itemize}

\subsection*{Limpiadores}
Cada herramienta posee una acción especial para \emph{resetear} el 
\hardware asociado. Esta acción se ejecuta automáticamente cuando se 
arma el laboratorio y cuando ha ocurrido fallo de ejecución. Es 
recomendable que se realice cada día un desarmado y armado de todos 
los laboratorios para que se ejecuten los limpiadores al menos una vez 
cada 24 horas.

\subsection*{\emph{Logs}}
Al igual que el proveedor y el monitor, los laboratorios tienen su 
propio registro de sucesos, que es recomendable observar. También 
se pueden eliminar cada cierto tiempo.

\section*{Herramientas}
No necesitan un mantenimiento concreto, aunque por motivos de 
seguridad, puede ser necesario registrar y eliminar las herramientas 
cada mes o periodo similar y así puedan cambiar de clave.

\subsection*{\emph{Hardware}}
A pesar de que el mantenimiento del \hardware sea automático, el 
administrador debe realizar revisiones periódicas al mismo, 
independientemente del uso que se le ha dado.


% Apéndice: Manual de desarrollo
% Apéndice: Manual de desarrollo

\capitulo{Guía de desarrollo de herramientas}{desarrollo}{
Este proyecto basa su versatilidad en la gran cantidad de soluciones 
que se pueden aportar en forma de conjuntos de aplicaciones (llamadas 
herramientas). Aquí se explican los métodos necesarios para 
realizar dichas herramientas.
}

\section*{Estructura de una herramienta}
Las herramientas son conjuntos de aplicaciones sin interfaz gráfica que se 
comunican con la plataforma RLF mediante una librería llamada 
\emph{libtool}. Se caracterizan por:

\begin{itemize}
\item Cada herramienta puede tener una o varias acciones a realizar 
(es decir, una o varias aplicaciones) que sólo se podrá ejecutar una 
instancia de ellas, siendo imposible ejecutar dos o más a la vez.
\item La herramienta puede tener una salida estándar (textual) y una 
entrada también textual que serán las que el usuario visualice.
\item Generalmente estas están asociadas a un \hardware específico, 
como puede ser una cámara de vídeo.
\item Pueden ofrecer uno o varios servicios externos, los cuales 
requieren que el usuario se conecte a ellos de forma independiente a 
la plataforma, llamados \emph{sockets}. Por ejemplo, pueden ser 
servicios externos un servidor FTP o HTML, \emph{streaming} de 
archivos multimedia, conexión a una base de datos, etc.
\item Las herramientas y sus componentes se definen en un fichero XML 
con un formato determinado, y se usará para darlas de alta en la 
plataforma RLF.
\item Aquellas que poseen el tipo ``Herramienta de datos'' no disponen 
de entrada estándar y sólo contienen una acción (o aplicación). 
Además esa acción no tiene parámetros de salida ni de entrada.
\item Cada acción tiene un número determinado de parámetros de 
entrada, salida o ambos que serán definidos antes de ejecutarla, así 
como unas constantes que no variarán su valor. 
Cuando se termine la ejecución, se almacena un resultado, y las 
excepciones que puedan ocurrir.
\end{itemize}

En cuanto a los ficheros que debe contener una herramienta, posee una 
carpeta principal (llamada \emph{root}) donde será la ruta base para 
acceder a todas las aplicaciones.

\section*{Desarrollar una nueva herramienta}
Lo primero es realizar el fichero de configuración que defina el 
comportamiento de la herramienta. Se deberá decidir si la herramienta 
pasa a ser ``de Datos''. Dependiendo del tipo, se tendrá que rellenar 
un fichero XML u otro. Se muestran a continuación dos ejemplos de 
ambos tipos:

\subsection*{Configuración}

\begin{verbatim}
<!-- EJEMPLO DE UNA HERRAMIENTA -->
<?xml version="1.0" encoding="UTF-8"?>
<!DOCTYPE tool SYSTEM
   "/home/rodriguezmecha/Universidad/Proyecto/rlf/tools/tool.dtd"
>
<tool
   path="/home/rodriguezmecha/Universidad/Proyecto/rlf/tools/linux/RLF_DummyTool"
   data="false"
>
    <in-stream/>
    <out-stream/>
    <attributes>
        <name>RLF DummyTool</name>
        <version>0.1</version>
        <description>
           Ejecuciones básicas para realizar pruebas en Linux. 
           Comprende dos acciones que ejecutan operaciones de entrada 
           y salida por teclado y acceso a ficheros locales.
        </description>
        <admin>carlos</admin>
        <role>0</role>
    </attributes>
    <constants>
    </constants>
    <parameters>
        <parameter name="exit_word" data-type="string">
            <description>Palabra con la que salir del programa.</description>
        </parameter>
        <parameter name="nechos" data-type="int">
            <description>Número de echos realizados.</description>
        </parameter>
    </parameters>
    <actions>
        <action name="echo" timeout="5">
            <value>rlfdummytool --echo</value>
            <description>Repite la entrada por teclado.</description>
            <action_parameter name="exit_word" type="in"/>
            <action_parameter name="nechos" type="out"/>
        </action>
        <action name="cpu-info" timeout="7">
            <value>rlfdummytool --cpu</value>
            <description>
               Obtiene información básica sobre el sistema (nombre de 
               la máquina y memoria libre) bajo petición del usuario.
            </description>
        </action>
        <resetter>rlfdummytool --clean</resetter>
    </actions>
</tool>
\end{verbatim}

Son pocas las diferencias que hay entre las dos configuraciones, sólo 
denotar el atributo \emph{data} y la forma de declarar las acciones. 
Cada tipo de herramienta tiene un DTD para comprobar la estructura de 
la configuración (llamados \emph{tool.dtd} y \emph{data-tool.dtd}).

\begin{verbatim}
<!-- EJEMPLO DE UNA HERRAMIENTA DE DATOS -->
<?xml version="1.0" encoding="UTF-8"?>
<!DOCTYPE tool SYSTEM
   "/home/rodriguezmecha/Universidad/Proyecto/rlf/tools/data-tool.dtd"
>
<tool 
   path="/home/rodriguezmecha/Universidad/Proyecto/rlf/tools/linux/RLF_Video"
   data="true"
>
    <out-stream/>
    <attributes>
        <name>RLF Video</name>
        <version>0.1</version>
        <description>
            Streaming de video y sonido. Incluye 3 segundos de delay 
            en el envío. Es necesario utilizar programas externos para 
            poder obtener el flujo de datos.
        </description>
        <admin>carlos</admin>
        <role>0</role>
    </attributes>
    <constants>
        <constant name="devices" data-type="string">
            <value>
               v4l:///dev/video0:input-slave=alsa://:v4l-norm=0
               :v4l-frequency=0:file-caching=300
            </value>
            <description>Dispositivos de captura.</description>
        </constant>
        <constant name="http_ip" data-type="string">
            <value>192.168.1.128</value>
            <description>IP del host.</description>
        </constant>
        <constant name="http_port" data-type="int">
            <value>64000</value>
            <description>Puerto de acceso.</description>
        </constant>
        <constant name="user" data-type="string">
            <value>video</value>
            <description>Usuario para la conexión.</description>
        </constant>
        <constant name="pass" data-type="string">
            <value>video</value>
            <description>Contraseña del usuario.</description>
        </constant>
    </constants>
    <action name="http-streaming" timeout="60">
        <value>rlfvideo --http</value>
        <description>
           Streaming por video mediante el protocolo HMTL. Utilizar un 
           programa como VLC (con recepción de imágenes y sonido por 
           volcado de red) para conectarse al servidor (http://nombre 
           del servidor:puerto/)
        </description>
        <socket port="64000" protocol="http" type="media" mode="r"/>
    </action>
    <resetter>rlfvideo --clean</resetter>
</tool>
\end{verbatim}

En primer lugar se encuentra la cabecera de la configuración, donde se 
indica el tipo de herramienta, su esquema DTD correspondiente 
(incluido en el proyecto) y su carpeta \emph{root} bajo el nombre de 
\emph{path}. A continuación se especifica si se dispone de entrada y 
salida estándar (\emph{in-stream} y \emph{out-stream}).

Después están los atributos de cada herramienta, donde se definen su 
nombre, descripción, administrador (debe estar registrado en la base 
de datos), versión y rol \index{rol}(número donde el 0 es la base, y es más 
restrictivo según se va añadiendo valores).

Las constantes definen valores fijos que no pueden ser cambiados, pero 
que todas las acciones pueden utilizar y obtener. Se componen de un 
tipo de datos, nombre, valor y descripción.

Los parámetros son definidos sólo en las herramientas normales, y se 
componen de un nombre, descripción, un valor máximo, mínimo y por 
defecto (opcional). Más adelante se les relaciona con las acciones.

Las acciones son diferentes comandos con un nombre, una descripción y 
un tiempo máximo de ejecución. A partir de ahí, se definen los 
parámetros que tienen asociados (indicando si son de entrada, salida o 
ambos. También tienen asociadas los \emph{sockets} que utilizará el 
comando para proveer el servicio externo, indicando su puerto, el 
protocolo, el tipo (media, datos o gráficos) y si es de lectura, 
escritura o ambos. Hay que recordar que esos comandos serán llamados a 
partir del directorio root, es decir, en el primer caso, la acción ``cpu-info'' 
se llamará de la forma \texttt{<path>/rlfdummytool --cpu}.

Como acción especial está el \emph{resetter} o limpiador, que no 
podrá ser ejecutado por el usuario y sirve para establecer a un estado 
seguro el \hardware asociado a la herramienta. No tiene parámetros ni 
servicios externos.

\subsection*{Uso de la librería \emph{libtool}}
Como ya se ha comentado, esta librería sirve para poner en contacto a 
la herramienta con la plataforma, y es necesario utilizarla para poder 
registrarla. Se ha añadido a este proyecto tres librerías escritas en 
C/C++, Bash y .NET, compartiendo todas la misma estructura de funciones:

\begin{enumerate}
\item \textbf{RLF\_ Init:} Inicia la conexión con la plataforma. Es 
necesario introducir la clave que se proporcionó al registrar la 
herramienta. Sin esta función no se puede acceder a ninguna otra. Si 
se está ejecutando una acción la cual no ha sido ordenada por la 
plataforma, no se permitirá el acceso a la misma.
\item \textbf{RLF\_ Finalize:} Desconecta la herramienta, indicando un 
código de finalización (establecido por el administrador) y una 
descripción. Es obligatorio realizarlo antes de cerrar la aplicación 
o antes de que no haya una finalización concreta (por ejemplo, antes 
de un bucle infinito). Hay que tener en cuenta que la aplicación puede 
ser detenida en cualquier momento por varios motivos, por lo que debe 
estar preparada para ello.
\item \textbf{RLF\_ GetConst:} Obtiene la constante indicada por el nombre y 
es almacenada en la estructura correspondiente (ver código para más 
información).
\item \textbf{RLF\_ GetAttribute:} Obtiene el atributo correspondiente con el 
nombre introducido. Estos atributos son gestionados por la plataforma.
\item \textbf{RLF\_GetParameter:} Obtiene el valor y la información de 
cualquier tipo de parámetro relacionado con la acción en ejecución 
actual. Estos valores han sido establecidos por el usuario.
\item \textbf{RLF\_ SetParameter:} Establece el valor textual de un parámetro 
de salida que esté asociado a la misma acción.
\item \textbf{RLF\_ ThrowException:} Lanza una excepción al usuario con un 
nombre y una descripción. No interrumpe el ciclo de ejecución.
\end{enumerate}

\textbf{NOTA:} Para obtener más información sobre las librerías 
específicas en cada lenguaje consultar el código entregado.

A continuación se muestra un ejemplo de una acción programada en 
Visual C++ que obtiene la memoria libre del sistema y la envía a la 
salida estándar cada cierto tiempo. Es parte de una herramienta de 
datos:

\begin{verbatim}
// 1. Inicio RLF
RLF_Manager ^ manager = gcnew RLF_Manager();
try {
    manager->init(TOOLKEY);
} catch (RLF_Exception ^ e){
        Console::WriteLine("Error con la base de datos al iniciar. " + e->getMsg());
        stream->Flush();
        return 1;
}

// 2. Obtención de la constante de tiempo.
try {
    time = Convert::ToInt32(manager->getConst("time")->getValue());
} catch (RLF_Exception ^ e){
    Console::WriteLine("Error con la base de datos al obtener datos." + e->getMsg());
    try {
        manager->finalize(1, "Error.");
    } catch (...){
    }
    return 1;
}

try {
    manager->finalize(1, "Error.");
} catch (...){
    Console::WriteLine("Error con la base de datos al finalizar.");
    return 1;
}

// 3. Ejecución.
while(true){
    GlobalMemoryStatusEx (&statex);
    Console::Write("Memory in use: {0:G}% ({1:D} / {2:D} Kbytes)", 
                statex.dwMemoryLoad, statex.ullAvailPhys/DIV,
                statex.ullTotalPhys/DIV);
    Thread::Sleep(1000 * time);
}
\end{verbatim}


\subsection*{Buenas prácticas}
Debido a que las aplicaciones que se programen deben seguir unos 
estándares, es recomendable seguir estos pasos:

\begin{itemize}
\item Las librerías que se utilicen para ejecutar las acciones, 
incluso la librería \emph{libtool} deben estar en la carpeta 
\emph{lib} dentro del directorio \emph{root} de la propia herramienta.
\item Los usuarios sólo verán los valores de los parámetros de 
salida cuando se termine de ejecutar la acción, por lo que escribir 
varias veces un valor no tiene sentido.
\item Los \emph{buffers} de salida son tratados como en los ficheros, 
por lo que si se quiere mostrar en tiempo real al usuario, será 
necesario vaciarlos o reducir su tamaño a cero. (Esto se puede 
conseguir mediante la función \emph{flush} contenida en muchos 
lenguajes de programación.)
\item Si se usa en una misma herramienta distinto \hardware 
dependiendo de la acción a tomar, es mejor separarlas y dividirlas en 
varias herramientas, para que los usuarios puedan disponerlas de 
manera más efectiva.
\end{itemize}

\section*{Crear una nueva \emph{libtool}}
Puede ser necesario crear nuevas librerías para otros lenguajes, como 
Java, Python e incluso el \emph{script} Bat de Windows. Para ello es 
necesario tener instalado la librería de acceso a la base de datos 
SQLite ya que la comunicación se realiza mediante ese formato (se 
puede obtener más información sobre las APIs de SQLite en la 
siguiente dirección \texttt{http://www.sqlite.org/}).

En el desarrollo de la librería hay que tener en cuenta las consultas 
a la base de datos, y la protección de escritura y lectura. Como 
explicar esto puede conllevar un aumento de la documentación bastante 
grande, se puede consultar las librerías ya hechas que están 
profusamente comentadas para ver cómo desarrollar una nueva, ya que la 
estructura es esencialmente la misma y su aplicación es sencilla.

La figura \ref{fig:erlibtool} muestra el modelo que sigue la base de 
datos asociada a cada herramienta.

\cleardoublepage


% Apéndice: Manual de usuario
% Apéndice: Manual de usuario

\capitulo{Guía de usuario}{usuario}{
Este manual está destinado al usuario final de la plataforma RLF. 
Contiene las acciones a realizar con el cliente de escritorio así como 
la utilización del monitor mediante web.
}

\section*{RLF\_ Client}
Este programa da acceso a todas las herramientas incluidas en el 
sistema RLF al que se va a conectar. Funciona tanto en Windows como en 
plataformas Linux. Aunque oficialmente no está soportado, también 
funciona en plataformas Macintosh que sean compatibles con la máquina 
virtual Java.

\subsection*{Instalación}
Para poder ejecutar el cliente, es necesario tener con anterioridad 
unas herramientas instaladas en el ordenador, independientemente del 
sistema operativo y son gratuitas:

\begin{description}
\item[JRE 1.6 de Java Sun:] Contiene la máquina virtual y las 
librerías necesarias para ejecutar programas en el entorno Java. Se 
puede descargar de \texttt{http://java.com/es/download/} aunque para 
sistemas Linux se puede encontrar en los repositorios oficiales con el 
nombre de \texttt{sun-java6-jre}.
\item[(Opcional) Reproductor media VLC:] Sólo es necesario si se 
requieren usar herramientas de vídeo. Se puede encontrar en la página 
oficial (\texttt{http://www.videolan.org/vlc/}) para cualquier 
plataforma. Al igual que antes, los repositorios de la mayoría de las 
distribuciones Linux bajo el nombre de \texttt{vlc}.
\item[(Opcional) Navegador de Internet:] Sólo es necesario para 
algunas herramientas concretas que utilicen servicios como FTP. Debido 
a que existen multitud de navegadores, se puede escoger el que 
quiera, incluso el que venga instalado por defecto en los sistemas 
operativos.
\end{description}

La instalación de estos programas se puede encontrar en Internet sin 
complicación, por lo que no se dedicará tiempo a explicarlo en este 
manual.

Una vez que esto esté listo, se puede pasar a la instalación del 
cliente. Para ello, es necesario obtener los archivos de la carpeta 
\texttt{rlf/bin/client} (\texttt{rlf\textbackslash bin\textbackslash 
client} en Windows) y copiarlos donde se desee. Es importante que las 
carpetas \texttt{res} y \texttt{lib} se copien en la misma ubicación 
donde está el archivo ejecutable \texttt{RLF\_ Client.jar}.

\subsection*{Configuración}
Para que el cliente se conecte a una dirección específica (que será 
dada por el administrador) donde se encuentra el proveedor es 
necesario modificar el archivo de configuración 
\texttt{res/client.conf} (\texttt{res\textbackslash client.conf} en sistemas 
Windows). Este fichero contiene la siguiente línea:

\begin{verbatim}
provider_url=http://[direccion]:[puerto]/[ruta]
\end{verbatim}

Para configurarlo sólo es necesario modificar los parámetros 
\emph{dirección}, \emph{puerto} y \emph{ruta} (eliminando los corchetes). 
Como por ejemplo:

\begin{verbatim}
provider_url=http://163.117.150.85:8084/RLF/Provider
\end{verbatim}

\subsection*{Ejecución}
Para ejecutar el programa sólo es necesario hacer ``doble click'' en el 
archivo \texttt{RLF\_ Client.jar} en cualquier sistema. Si como 
consecuencia se obtiene un mensaje de ``No se reconoce el archivo 
ejecutable'', la instrucción para la línea de comandos es:

\begin{verbatim}
java -jar RLF_Client.jar
\end{verbatim}

\textbf{NOTA:} En sistemas Linux puede haber errores con fallos de 
conexión o de la interfaz gráfica, son externos al programa y vienen 
dados por la máquina virtual:

\begin{itemize}
\item Hay que asegurar que no se está utilizando la versión OpenJDK 
de Java, que es la que viene por defecto en Linux. Esta versión tiene 
problemas (reconocidos oficialmente) con las interfaces gráficas Swing. 
Para arreglar este problema, seguir la guía que se puede obtener en 
esta dirección:
\begin{verbatim}
http://www.e-capy.com/reemplazar-openjdk-por-el-jdk-de-java-en-ubuntu/
\end{verbatim}

\item Siempre es recomendable utilizar el comando anteriormente 
descrito en vez de ejecutar el programa con ``doble click'' ya que 
Java establece sus rutas dependiendo de dónde haya sido ejecutado. Es 
por esto que puede que no encuentre las librerías o el fichero de 
recursos.
\end{itemize}

\clearpage

\subsection*{Interfaz}

\begin{figure}[H]
	\centering
	\includegraphics[scale=0.45]{images/user/screen.png}
\end{figure}

Aquí se listan todos los elementos que componen la interfaz del 
cliente mostrados en la anterior figura. Puede que en determinados 
momentos algunas de estas opciones no estén activas.

\begin{enumerate}
\item Es la barra de menús donde poder acceder a todas las acciones, 
como ruta alternativa a los otros botones. Además se indican los 
atajos de teclado de estas acciones.
\item Con este botón se puede conectar o desconectar el usuario al 
sistema. 
\item Sirve para obtener el estado de cada herramienta antes de 
haberlas reservado.
\item Reconstruye la información de las herramientas por si ha habido 
cambios estructurales en el servidor.
\item Reserva las herramientas seleccionadas.
\item En esta zona se indican las herramientas que se han 
seleccionado, o el tiempo restante de reserva de las mismas. Si se 
posiciona el puntero del ratón encima se pueden obtener el nombre e 
identificador de cada herramienta seleccionada o reservada.
\item Cada pestaña corresponde a una herramienta distinta controlada 
por el sistema.
\item Estado de la herramienta, ver la sección ``Estados de una 
herramienta'' para más información.
\item Nombre e identificador único de cada herramienta.
\item Atributos de cada herramienta, ver la sección ``Atributos de una 
herramienta'' para más información.
\item Descripción de la herramienta.
\item Acción seleccionada de la herramienta. Sólo se pueden ejecutar 
una vez la herramienta haya sido reservada. Si se despliega la lista 
se podrán obtener todas las acciones asignadas.
\item Cuando la herramienta está reservada, con este botón se ejecuta 
la acción seleccionada.
\item Descripción de la acción seleccionada.
\item Sirve para seleccionar la herramienta de la pestaña actual para 
su posterior reserva.
\end{enumerate}

\subsubsection*{Estados de una herramienta}
Cada herramienta tiene asociado un estado con cuatro valores posibles 
que se corresponden a las cuatro imágenes de la siguiente figura:

\begin{itemize}
\item \textbf{Disponible:} Esta herramienta no está siendo usada por nadie y 
puede reservarse. 
\item \textbf{No disponible:} Esta herramienta está siendo usada por otro 
usuario o está desconectada del sistema.
\item \textbf{Reservada:} La herramienta ya está reservada y lista para su 
ejecución.
\item \textbf{En ejecución:} Alguna acción de esta herramienta está 
actualmente en ejecución.
\end{itemize}

\begin{figure}[h]
	\centering
	\includegraphics[scale=1]{images/user/statustools.png}
\end{figure}

\subsubsection*{Atributos de una herramienta}
Cada herramienta posee tres atributos característicos que determinan 
su comportamiento. Se encuentran indicados en la zona 10 explicada en 
el apartado anterior:

\begin{itemize}
\item \textbf{In:} Si está activo, la herramienta acepta interacción con el 
usuario mediante la entrada por teclado. Cuando se ejecute alguna 
acción, se podrán enviar comandos en el visor de ejecuciones.
\item \textbf{Out:} Si está activo indica que la herramienta dará información 
textual al usuario cuando se ejecute una acción.
\item \textbf{Data:} Indica si la herramienta tiene la categoría de 
``Herramienta de datos''. Este tipo de herramientas no tienen entrada 
por teclado, sólo contienen una acción y permiten que varios usuarios 
la utilicen a la vez, por lo tanto, esta herramienta siempre estará 
libre para su uso.
\end{itemize}

\subsection*{Cómo utilizar las herramientas}
El proceso para obtener unas herramientas y realizar acciones con 
ellas se describe a continuación.

\subsubsection*{Conexión}
En primer lugar se ha de conectar con el servidor, para ello se pulsa 
el boton de \textit{Login} y se introduce el usuario y la contraseña 
asignada en el diálogo que aparecerá como se ve en la figura 
a continuación. A partir de ese momento, el resto de las opciones 
estarán disponibles.

\begin{figure}[h]
	\centering
	\includegraphics[scale=0.8]{images/user/login.png}
\end{figure}



\subsubsection*{Selección de las herramientas}
Cuando las herramientas aparezcan, se podrá seleccionar aquellas que 
dispongan del estado \textbf{Disponible} mediante el \emph{checkbox} 
propio de cada pestaña. Debido a que puede haber varios usuarios 
realizando peticiones, es recomendable actualizar (con el botón 
\textit{Refresh}) los estados de las herramientas antes de 
seleccionarlas, ya que no se actualizan en tiempo real. Cuando se 
hayan seleccionado las herramientas, aparecerán en el indicador 
mostrado en la siguiente figura.

\begin{figure}[h]
	\centering
	\includegraphics[scale=0.8]{images/user/select.png}
\end{figure}

\subsubsection*{Reserva}
Para poder utilizar unas es necesario reservarlas antes. Cuando todas 
las herramientas que se deseen estén seleccionadas, se procederá a 
reservarlas con el botón \textit{Take Tools}. A partir de aquí, el 
cronómetro se pondrá en marcha, ya que cada usuario tiene un tiempo 
máximo de uso de las herramientas.

\begin{figure}[h]
	\centering
	\includegraphics[scale=0.8]{images/user/timer.png}
\end{figure}

\textbf{ATENCIÓN:} Cuando el cronómetro está cerca del límite de 
tiempo lanza un aviso mediante un diálogo, después, cuando se ha 
terminado cierra todas las acciones en ejecución y libera las 
herramientas.

\subsubsection*{Ejecutar una acción}
Cada herramienta puede ejecutar a la vez una sola acción, pero se 
pueden tener varias acciones de distintas herramientas ejecutándose en 
el mismo instante. Para ello, selecciona la acción correspondiente y 
se pulsa el botón \textit{Execute}.

Se mostrará un diálogo con los parámetros de entrada de la acción y 
con los \emph{sockets} para servicios externos. En los parámetros se indica 
el tipo y si se pasa el ratón por encima del nombre, se podrá obtener 
la descripción del mismo.

\begin{figure}[H]
	\centering
	\includegraphics[scale=0.4]{images/user/inparams.png}
\end{figure}

En el caso de esta figura, hay un parámetro de entrada llamado 
``volts'' de tipo \emph{double} y un servicio externo de FTP de 
lectura al que se puede acceder mediante la URL 
\texttt{ftp://163.117.150.95:64001/}. Para ello se utiliza un 
navegador o un cliente FTP.

\textbf{ATENCIÓN:} Los \emph{sockets} representan servicios que provee 
la herramienta pero que deben ser atendidos fuera de la interfaz de 
RLF\_ Client. Esto significa que es necesario utilizar un programa 
externo, como puede ser un navegador o un reproductor de vídeo. La 
información que se muestra es la localización o \emph{URL} de ese 
mismo servicio, así como su protocolo y el tipo de datos que obtiene.

\begin{figure}[h]
	\centering
	\includegraphics[scale=0.4]{images/user/video.png}
\end{figure}

Cuando se tengan configurados los parámetros, se podrá proceder a 
ejecutar la acción. Si la máquina donde se encuentra la herramienta 
tiene mucha carga, puede ser necesario esperar hasta que se pueda 
ejecutar. Se sabe que la ejecución está en marcha cuando el icono 
muestra el título ``Running...''.

\begin{figure}[h]
	\centering
	\includegraphics[scale=0.6]{images/user/running.png}
\end{figure}

\subsubsection*{Interactuar con una acción}
Dependiendo del tipo de herramienta, en el diálogo de ejecución se 
podrá obtener la información de salida de la herramienta e introducir 
comandos mediante el campo de texto y el botón \textit{Send}. Se 
podrá terminar la ejecución abortándola (cerrando la ventana) o 
completando la acción, la cual lanzará otra ventana con los 
resultados de la aplicación, que conforman el estado final, los 
parámetros de salida y las excepciones lanzadas, como se ve en la 
siguiente figura.

\begin{figure}[h]
	\centering
	\includegraphics[scale=0.5]{images/user/stopdialog.png}
\end{figure}

\subsubsection*{Liberar las herramientas}
Tanto si se ha terminado de utilizar las herramientas como si se ha 
acabado el tiempo, se ha de desconectar mediante el botón 
\textit{Logout} o cerrando la aplicación.

\subsubsection*{Errores comunes}
Por motivos de seguridad, si ocurriera algún error grave en la 
aplicación RLF\_ Client y el usuario no se pudiera desconectar con 
normalidad, no podrá volverse a conectar hasta que el administrador 
haya dado su aprobación, para ello, póngase en contacto con él e 
indique el motivo de su error.

\clearpage

\section*{RLF\_ Monitor}
Este servicio está disponible para poder comprobar en tiempo real y 
desde cualquier dispositivo el estado de las herramientas. Aunque la 
interfaz esté diseñada especialmente para dispositivos móviles, 
puede ser consultado desde un navegador de un ordenador personal sin 
problemas. Con ello se pretende ahorrar tiempo al usuario para 
informarse si las herramientas que quiere utilizar están en ese 
momento libres.

Para acceder al servicio, se debe ir al a dirección proporcionada por 
el administrador mediante cualquier tipo de navegador de Internet. 
Como por ejemplo:

\begin{verbatim}
http://163.117.150.85:8080/RLF/Monitor/
\end{verbatim}

\begin{figure}[H]
	\centering
	\includegraphics[scale=0.5]{images/user/loginmonitor.png}
\end{figure}

Esa página pedirá el nombre y contraseña del usuario que se le ha 
asignado, una vez esté conectado, podrá obtener información como esta:

\begin{figure}[H]
	\centering
	\includegraphics[scale=0.5]{images/user/monitor.png}
\end{figure}

\subsection*{Estados de una herramienta en el monitor}
La lista de estados que se pueden obtener mediante el servicio del 
monitor es distinta a la que se puede observar con la aplicación 
cliente RLF\_ Client:

\begin{itemize}
\item \textbf{Disponible:} Esta herramienta no está siendo usada por nadie y 
puede reservarse. 
\item \textbf{No disponible:} Esta herramienta está desconectada del sistema.
\item \textbf{En uso:} Algún usuario tiene reservada esta herramienta.
\end{itemize}

\begin{figure}[h]
	\centering
	\includegraphics[scale=1]{images/user/statustoolsmonitor.png}
\end{figure}

\cleardoublepage


% Apéndice: Material Entregado
% Apéndice: Material Entregado

\capitulo{Material entregado}{material}{
Debido a que este producto contiene múltiples módulos y componentes, 
se listan a continuación todo lo se ha entregado junto con este 
documento, y su disposición.
}

\section*{Componentes}
\subsection*{Proveedor \index{proveedor}}
Compone el servicio web de acceso a la plataforma y los \emph{scripts} de la 
base de datos para poder crearla. Este servicio web debe ser 
desplegado en alguna plataforma como Tomcat o JBoss que permita 
aplicaciones Java. Los \emph{scripts} están realizados para el SGBD MySQL.

\subsection*{Monitor \index{monitor}}
Sirve para obtener información en tiempo real de las herramientas 
registradas en el sistema. Está compuesto por un servicio y página 
web con formato para dispositivos móviles o pantallas pequeñas.

\subsection*{Laboratorio \index{laboratorio}}
Módulo de gestión de los distintos servidores locales que contienen 
herramientas. Puede instalarse en cualquier plataforma y toma la 
función de un proceso \emph{demonio} \index{demonio}.

\subsection*{Gestor de laboratorios}
Es la herramienta para poder administrar todos los laboratorios 
ejecutados en la red. Aplicación por consola que se recomienda usar en 
sistemas Unix.

\subsection*{Cliente \index{cliente}}
Permite acceder a la plataforma RLF. Puede ser utilizado en cualquier 
sistema operativo con soporte para Java. Utiliza el servicio web 
propio del proveedor.

\subsection*{Componentes RLF}
Son un conjunto de herramientas que utilizan todos los módulos. 
Dispone de un cifrador \index{cifrador} de datos y la capa de 
comunicación de toda la plataforma mediante sockets \index{socket}.

\subsection*{Bibliotecas \emph{libtool} \index{\emph{libtool}}}
Conjunto de funciones y métodos específicos para conectar la 
herramienta con el laboratorio. Están hechas en varios lenguajes de 
programación.

\subsection*{Herramienta 1: RLF\_ Video}
Emite video y sonido mediante un servidor HTTP. Está destinada para 
sistemas Unix fue desarrollada en Shell Bash. Su capa base es el 
programa/servidor VLC. Es una herramienta multiusuario.

\subsection*{Herramienta 2: RLF\_ Music}
Reproduce canciones seleccionadas por el usuario en el sistema remoto, 
por lo que no pueden ser oidas desde el cliente. Pensado para sistemas 
Unix con el programa MPG123.

\subsection*{Herramienta 3: RLF\_ DummyTool}
Conjunto de pruebas básicas en consola Shell Bash para Unix. Comprueba 
la información del sistema en tiempo real.

\subsection*{Herramienta 4: RLF\_ FreeMem}
Obtiene la memoria en uso de un sistema Windows. Esta herramienta es 
multiusuario y la información también es en tiempo real.

\subsection*{Herramienta 5: RLF\_ Board}
Acciones para utilizar la tarjeta Advantec PCI1711BE. Desarrollado para 
sistemas Windows. Pensado para realizar prácticas de laboratorio de 
electrónica y automática. Permite enviar un pulso de 5 segundos con 
el voltaje introducido, obteniendo la respuesta mediante un servicio 
FTP.

\subsection*{Documentación}
Todo el código entregado está profusamente comentado para ayudar a 
desarrollar nuevos módulos a partir de código antiguo. Además, para 
el código escrito en Java, se ha entregado la documentación del API 
completa (realizada mediante \emph{javadoc}).

\clearpage

\section*{Distribución}
La disposición de directorios entregados es la siguiente:
\begin{verbatim}
|-- bin
|   |-- client
|   |   |-- lib
|   |   |-- log
|   |   `-- res
|   |-- lab
|   |   |-- log
|   |   `-- res
|   |-- labconsole
|   `-- provider
|-- doc
|   |-- api
|   |   |-- index-files
|   |   |-- org
|   |   `-- resources
|   |-- images
|   |   `-- user
|   |-- include
|   |-- lib
|   `-- template
|-- img
|-- lib
|-- projects
|   |-- eclipse
|   |   |-- RLF_Lab
|   |   |-- RLF_LabConsole
|   |   |-- RLF_LabManager
|   |   |-- RLF_Log
|   |   `-- RLF_Utils
|   |-- netbeans
|   |   |-- RLF_Client
|   |   `-- RLF_Provider
|   `-- visual studio
|       |-- LibTool
|       |-- RLF_Board
|       `-- RLF_FreeMem
|-- src
|   `-- sql
`-- tools
    |-- lib
    |   |-- Bash
    |   |-- C
    |   `-- NET
    |-- linux
    |   |-- RLF_DummyTool
    |   |-- RLF_Music
    |   `-- RLF_Video
    `-- windows
        |-- RLF_Board
        `-- RLF_FreeMem
\end{verbatim}

\clearpage

\subsection*{Directorio \emph{bin}}
Contiene todos los archivos ejecutables de los distintos componentes 
organizados por carpetas. Si se desea ejecutar esos archivos binarios 
en otra localización, es necesario copiar todo lo que contienen sus 
carpetas, ya que es necesario para un correcto funcionamiento.

\subsection*{Directorio \emph{doc}}
En este directiorio está toda la documentación aportada para el 
proyecto, desde la descripción de los APIs hasta este propio 
documento, en \emph{latex} con todos los recursos necesarios para la 
maquetación del mismo.

\subsection*{Directorio \emph{img}}
Se agrupan todas las imágenes que se han utilizado para el desarrollo 
del proyecto. Son las usadas en los componentes aunque no son 
necesarias para su ejecución, ya que han sido introducidas en el mismo 
código.

\subsection*{Directorio \emph{lib}}
Aquí están todas las librerías requeridas para la compilación de 
los distintos módulos. Cada proyecto tiene su referencia en este 
directorio. Además, se puede encontrar la propia librería de RLF.

\subsection*{Directorio \emph{projects}}
Se ha considerado que es más sencillo realizar nuevos módulos o 
modificar los entregados mediante los proyectos originales de los 
distintos IDEs, como son Eclipse, NetBeans y Visual Studio. Aquí está 
contenido de forma organizada haciendo referencia al tipo de IDE.

\subsection*{Directorio \emph{src}}
Se encuentran los archivos fuente de los componentes que no tienen 
establecido su propio proyecto. Es el caso de los \emph{scripts} en 
SQL de las distintas bases de datos.

\subsection*{Directorio \emph{tools}}
Contiene todas las herramientas entregadas (sus archivos ejecutables) 
separadas por plataforma y las propias \emph{libtools} en los 
distintos lenguajes. Antes de mover estos archivos consúltese el 
Manual del Desarrollador.

\cleardoublepage


% Bibliografía
\phantomsection
\addcontentsline{toc}{chapter}{Bibliografía}
\bibliography{include/bibliografia}
\bibliographystyle{plain}

% Índice alfabético
\phantomsection
\addcontentsline{toc}{chapter}{Índice alfabético}
\printindex
\end{document}
